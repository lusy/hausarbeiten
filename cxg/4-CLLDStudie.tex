\section{CLLD und CLD Konstruktionen im Spanischen und Englischen -- ein Beispiel für Transfer und Übergeneralisierung}

%\subsection{Hauptthese}
%Unterkapitel umbenennen!
%Bzw vlt neue Struktur zum gesamten Kapitel überlegen

Der Überblick über die Zweitspracherwerbsforschung hat gezeigt, dass es zwischen der Muttersprache einer Person und einer zu erlernenden Zweitsprache zu diversen Wechselwirkungen kommen kann.
Lernende einer Fremdsprache projizieren manchmal Konstruktionen aus ihrer Muttersprache in die L2.
Manchmal übergeneralisieren sie den Gebrauch einer gelernten Konstruktion aus der L2 und setzen diese in unpassende Kontexte ein.
Beispiele für beide Erscheinungen finden wir in der Verwendung der \textit{contrast left dislocation}-Konstruktionen (CLD-Konstruktionen) und \textit{clitic left dislocation}-Konstruktionen (CLLD-Konstruktionen) im Spanischen und Englischen in der Studie von Elena Valenzuela \cite{Valenzuela05}.
Valenzuela kommt zu der Schlussfolgerung, dass Lernende mit L1 Englisch bevorzugt CLLD Topik-Konstruktionen im Spanischen benutzen, unabhängig von der Spezifität der Topik.
Bei Lernenden mit L1 Spanisch beobachtet sie eine gewisse Tendenz zur Nutzung von CLLD-Konstruktionen im Englischen, wenn die Konstruktion eine spezifische Topik verlangt \cite{Valenzuela05}.


%Topikalisierung im Spanischen und Englischen -- State of the Art
% * specificity
% * CLD vs CLLD
%vlt kurze Beschreibung der Studie
%Kurze Beschreibung der Funde + Wrapping mit der kognitiven Perspektive
%Kritik am Text in Kapitel 5 auslagern?


\subsection{Topik und Spezifität in CLD- und CLLD-Konstruktionen}
Um die Argumentation der Studie \cite{Valenzuela05} verfolgen zu können, wenden wir uns zunächst deren Kernbegriffen zu.

Sowohl in germanischen als auch in romanischen Sprachen wird die Topik einer Äußerung üblicherweise durch Auslagerung einer Phrase realisiert, die dann im Satz wieder aufgegriffen wird \cite{Valenzuela05}.
In germanischen Sprachen wird die Topik durch CLD-Konstruktionen, wie im Beispiel (\ref{shoes}), ausgedrückt.
Die typische Realisierung der Topik in romanischen Sprachen benutzt dagegen die CLLD-Konstruktion (Beispiel (\ref{zapatos})).

\begin{exe}
    \ex \label{shoes} \uline{These shoes}, I bought \textbf{Op} in Madrid.
    \ex \label{zapatos} \uline{Estos zapatos}, \textbf{los} compré en Madrid.
\end{exe}

Also werden Topik und Kommentar auf Englisch mittels eines \textit{null anaphoric operator} (Op),
auf Spanisch mittels eines Klitikons verbunden.
% wie wissen wir dass es da ein ``null anaphoric operator'' gibt? ich meine, da gibts ja nichts..

% specificity in Spanisch
Die Spezifität der Topik im Spanischen ist entscheidend für ihre Realisierung.
Während spezifische Topiks, wie bereits erwähnt, durch eine CLLD-Konstruktion umgesetzt werden (Beispiel (\ref{libro})),
werden unspezifische oder generische Topiks, ähnlich wie im Englischen, durch eine CLD-Konstruktion formuliert (Beispiel (\ref{revistas})) \cite{Valenzuela05}.

\begin{exe}
    \ex \label{libro} Este libro, *(\textbf{lo}) he leído muchas veces.
    \ex \label{revistas} Revistas, (*\textbf{las}) leo a menudo.
\end{exe}

% specificity in Englisch
Im Englischen dagegen wird die Topik, unabhängig von ihrer Spezifität, immer mit einer CLD-Konstruktion realisiert (vgl. Beispiele (\ref{book}), (\ref{magazines})).

\begin{exe}
    \ex \label{book} \uline{This} book, I have (*it) read many times.
    \ex \label{magazines} Magazines, I often read (*them).
\end{exe}

% root vs embedded clauses
Eine finale Unterscheidung beim Ausdrücken der Topik, auf die \cite{Valenzuela05} aufmerksam macht, ist die zwischen Haupt- und Nebensätzen.
Sie hält fest, dass CLLD-Konstruktionen sowohl in Haupt- als auch in Nebensätzen benutzt werden können (Beispiele (\ref{clld-esp})),
CLD-Konstruktionen dagegen nur in Hauptsätzen erlaubt sind (Beispiele (\ref{cld-esp})) \cite{Valenzuela05}.
Daraus lässt sich schlussfolgern, dass nicht spezifische Topiks in spanischen Nebensätzen nicht möglich sind.
Auf English ist gar keine Topikalisierung in Nebensätzen möglich.

%``In CLLD the left-dislocated phrase can occur in either root or embedded clauses, as in (9). The left-dislocated phrase in CLD, on the other hand, can only occur at the left periphery of root clauses, as in (10):''
%--> no non-specific topics allowed in embedded clauses (auf Spanisch)!
\begin{exe}
    \ex \label{clld-esp} \begin{xlist}
        \ex Un libro, \textbf{lo} leí anoche
        \ex Me pregunto que, a María, el libro, quién \textbf{se} lo dio
        \end{xlist}
    \ex \label{cld-esp} \begin{xlist}
        \ex Un libro, leí anoche
        \ex[*] {Me pregunto que, tarjetas, a amigos, quién mandará}
        \end{xlist}
\end{exe}

\subsection{Die Studie}
In ihrer Studie stellt sich Valenzuela die folgenden Fragen:
(1) werden erwachsene L2 Spanisch Lernende mit Muttersprache Englisch in der Lage sein, die Eigenschaften der CLLD-Konstruktionen zu erwerben;
(2) werden erwachsene L2 Englisch Lernende mit Muttersprache Spanisch in der Lage sein, auf den Gebrauch von CLLD-Konstruktionen zu verzichten;
(3) ist es einfacher eine neue Konstruktion zu erwerben oder eine Konstruktion nicht mehr zu benutzen \cite{Valenzuela05}?

%\subsubsection{Research Question}
%\begin{itemize}
%    \item Will adult L2 Spanish learners be able to acquire the syntactic properties associated with CLLD?
%    \item Will adult L2 English learners be able to let go of the syntactic properties associated with CLLD?
%    \item will it be easier to acquire or to ‘let go’ of a property?
%\end{itemize}

Um diese Fragen beantworten zu können, werden zwei Sprecherinnengruppen untersucht --
eine mit Muttersprache Spanisch und sehr guten Englisch-Kenntnissen, deren Mitglieder zur Zeit der Untersuchung in Spanien oder Kanada wohnten,
und eine mit Muttersprache Englisch und sehr guten Spanisch-Kenntnissen, wohnhaft in Spanien zur Zeit der Untersuchung.
Beide Gruppen sollten Satzauswahl- und Satzvervollständigungsaufgaben lösen.
Die Satzauswhalaufgaben sollten testen, in wie fern die Sprecherinnen einen Satz als akzeptabel befanden.
Ziel der Satzvervollständigungsaufgaben war zu überprüfen, welche Konstruktionen die Sprecherinnen aktiv produzieren.
Zusätzlich zu jeder der beiden Probandinnengruppen wurde jeweils eine Kontrolgruppe von Muttersprachlerinnen befragt \cite{Valenzuela05}.

%\subsubsection{Methoden}

%\begin{itemize}
%    \item L1 Spanisch -> L2 Englisch lalal
%    \item L1 Englisch -> L2 Spanisch: 15 English speakers of L2 Spanish; had had their first exposure to Spanish after puberty;
%        from England, Canada and US; living in Spain; work in Spanish, often work + home life in Spanish;
%        25 monolingual Spanish speakers tested as control group;
%    \item Sentence Selection(tests acception) and Sentence Completion(tests production) tasks;
%    \item specific vs non-specific topics in root clauses
%    \item specific vs non-specific topics in embedded clauses
%\end{itemize}

\subsection{Die Funde}

Die Ergebnisse aus Valenzuelas quantitativer Studie zeigen, dass die Gruppe der L2 Spanisch Sprecherinnen erfolgreich die Realisierung des Topiks mittels einer CLLD-Konstruktion erlernt hat.
Spezifische Topiks in Haupt- und Nebensätzen werden durch die Lernenden richtig mit einer CLLD-Konstruktion umgesetzt.
Allerdings haben bei fast allen Aufgaben mehr als die Hälfte der Lernenden ein nicht spezifisches Topik mit einer CLLD-Kostruktion ausgedrückt (vgl. (\ref{agua}), (\ref{cafe})),
obwohl generische Topiks durch Muttersprachlerinnen vorwiegend mit einer CLD-Konstruktion realisiert werden \cite{Valenzuela05}.
Das ist ein klares Indiz für eine Übergeneralisierung des Gebrauchs der CLLD-Konstruktion.

\begin{exe}
    \ex \label{agua} Agua, *\textbf{la} toma todas las mañanas
    \ex \label{cafe}Me parece que, café, *\textbf{lo} debería tomar menos
\end{exe}

Für die Gruppe der Spanisch Muttersprachlerinnen mit L2 Englisch kommt Valenzuela zur Schlussfolgerung,
dass sie die Einschränkungen der CLD-Konstruktionen kennen, nämlich, dass diese nur in Hauptsätzen benutzt werden können.
Jedoch stellt sie bei dieser Sprecherinnengruppe eine Tendenz zur Nutzung von CLLD-ähnlichen Konstruktionen auf Englisch in Äußerungen mit einem spezifischen Topik (vgl. Beispiele (\ref{book2}), (\ref{notes})).

\begin{exe}
    \ex \label{book2} \uline{This} book, I have *\textbf{it} read many times.
    \ex \label{notes} \uline{Those} class notes, she cannot find *\textbf{them} anywhere.
\end{exe}

Diese Erscheinung deutet sie als Transfer aus der L1 Spanisch der Probandinnen.
(vgl. ``However, the near-native group is both accepting and producing pronouns (overt resumptive elements like clitics) with the specific topics there by exhibiting L1 influence.'' \cite{Valenzuela05})

Zusammenfassend lässt sich sagen, dass in der vorliegenden Studie sowohl Übergeneralisirungen von Konstruktionsmerkmalen in der L2 als auch Transfer aus der L1 in die L2 gezeigt wurden.

%\cite{Valenzuela05}
%``When the L2 input data are insufficient and the L2 speaker cannot restructure his/her grammar, the learner is unable to `let go' of
%one of the two forms of the construction.
%The presence of two forms for one interpretation results in a permanent optionality between both variations.
%This permanent optionality is not found in the native grammar of the target language and is therefore a form of incomplete L2 acquisition.''


%Englisch -> Spanisch

%Sentence Selection: Root Topik-Konstruktionen
%\begin{itemize}
%    \item both participant groups (natives and near-natives) correctly selected the specific topic with a clitic
%    \item bei root non-specific Topiks gab es bei near-natives eine höhere Tendenz zur Wahl eines Klitikons (falsch) - etwa 37\% vs 14\% der Kontrolgruppe
%\end{itemize}

%Sentence Selection: Embedded Topik-Konstruktionen
%\begin{itemize}
%    \item ``both groups correctly preferred the CLLD construction with tokens eliciting specific topics in embedded contexts''
%    \item non-specific Topik: gibt es eigentlich nicht in Embedded Konstruktionen: 77\% der Kontrolgruppe bestätigt das; 67\% der near-natives wählen eine klitische Konstruktion (CLLD) und nur 31\% bezeichnen sowohl CLLD als CLD als inkorrekt
%\end{itemize}

%Sentence Completion: Root Topik-Konstruktionen
%\begin{itemize}
%    \item ``Both groups correctly provided clitics with specific left-dislocated topics in root environments''
%    \item ``In contexts forcing non-specific interpretation, however, near-natives completed them with a clitic over 50\% of the time.'' (falsch)
%       -> Sie tun übergeneralisieren!
%\end{itemize}

%Sentence Completion: Embedded Topik-Konstruktionen
%\begin{itemize}
%    \item ``both groups correctly produced sentences with a clitic in contexts where a specific clitic was provided.''
%    \item Non-specific Topik ist nicht so wirklich erlaubt in Embedded Konstruktionen laut der Einführung, deshalb verstehe ich diesen Teil der Studie nicht;
%       es gab wohl keine Option anzugeben, dass der Satz ohnehin komisch ist und den einfach nicht zu vervollständigen;
%        Ergebnisse zeigen: natives entscheiden sich gegen einen Klitikon, near-natives dafür;
%\end{itemize}

%So
%\begin{itemize}
%    \item near-native haben die CLLD Konstruktion gelernt;
%    \item sie scheinen sich aber manchmal, die Bedeutungsunterschiede zwischen Klitikon/kein Klitikon nicht bewusst zu sein
%    \item Also es gibt schon eine gewisse Tendenz dazu, dass near-natives viel zu oft klitische Konstruktionen benutzen. Auch da wo es nicht geht.-->übergeneralisierung!
%\end{itemize}

%Spanisch -> Englisch

%Sentence Selection: Root Topik-Konstruktionen
%\begin{itemize}
%    \item specific context: beide Gruppen entscheiden sich richtig für ``no pronoun'' zu 48\%; bei den L1 Spanisch Leuten gibt es immer noch eine gewisse Tendenz dazu Klitika zu benutzen (30\%)
%    \item non-specific context: die Mehrheit von beiden Gruppen entscheidet sich richtig für ``no pronoun''; allerdings sind die near-natives, die diese Option wählen mehr als die natives (63\% vs 55\%)
%\end{itemize}

%Sentence Selection: Embedded Topik-Konstruktionen --> gibts nicht; keine CLDs in embedded Konstruktionen
%\begin{itemize}
%    \item specific: beide Gruppen tendieren dazu, sich richtig zu entscheiden (weder Klitikon noch keins ist eine Option); allerdings, mehr als 50\% (bei beiden!) wählt die eine oder die andere oder beide Optionen
%    \item non-specific: same here
%\end{itemize}

%Sentence Completion: Root Topik-Konstruktionen
%\begin{itemize}
%    \item die Mehrheit der beiden Gruppen entscheidet sich für kein Pronomen sowohl bei specific als auch bei non-specific contexts
%    \item allerdings beobachtet man bei den Spanisch L1 Sprecherinnen eine gewisse Tendenz zur Wahl eines Klitikons (54\% bei den specific und 36\% bei den non-specifics) --> also Einfluss der L1
%\end{itemize}

%Sentence Completion: Embedded Topik-Konstruktionen
%--> gibts vernünftigerweise nicht, es hieß ja vom Anfang an, es gäbe keine CLD-Konstruktionen in Embedded Clauses..

%So
%\begin{itemize}
%    \item L1 Spanisch Speaker: haben die Unterschiede zwischen CLLD und CLD im großem und ganzem gelernt: ``appear to know the constraints on the contexts in which CLD can appear (only in root contexts).''
%    \item ``However, the near-native group is both accepting and producing pronouns (overt resumptive elements like clitics) with the specific topics there by exhibiting L1 influence.''--> `letting go' die L1-Konstruktion ist schwer
%\end{itemize}



