\section{Hauptteil}

\subsection{Hauptthese}
%\begin{enumerate}
%    \item \sout{Lernerinnen einer Fremdsprache greifen oft auf schematisierte Äußerungen zurück.
%        Schlagwort: formulaic phrases}
%    \item Lernerinnen einer Fremdsprache projizieren Strukturen/Konstruktionen aus ihrer Muttersprache in die L2.
%\end{enumerate}

Lernerinnen einer Fremdsprache projizieren Strukturen/Konstruktionen aus ihrer Muttersprache in die L2.
Oder tendieren zu Übergeneralisierungen einer gelernten Konstruktion aus der L2.
Beispiele wären die CLD-Konstruktionen im Englischen und CLLD-Konstruktionen im Spanischen.
Lernerinnen mit L1 Englisch tendieren dazu die CLLD Topikkonstruktionen im Spanischen zu benutzen, unabhängig von der Spezifität (also empfinden den Bedeutungsunterschied nicht).
Lernerinnen mit L1 Spanisch tendieren dazu, bei spezifische Topiks, eine CLLD Konstruktion einzusetzen?
\cite{Valenzuela05}


Behauptungsgrundlage

%soll das hier nicht zu 2-Grundlagen über? (ergo Zweitspracherwerb Forschung in historischer Perspektive oder so)

\subsection{Daten}
\begin{itemize}
    \item \cite{Valenzuela05} CLD und CLLD: Spanisch -> Englisch, Englisch -> Spanisch
%    \item Kindlichem L2 Erwerb Spanisch -> Englisch: \cite{Wong-Fillmore76} in \cite{Haberzettl06}
%    \item Kindlichem L2 Erwerb Türkisch -> Deutsch: Wegener\cite{} -- leider unveröfftentlichte Schrift
%und \cite{Haberzettl05} in \cite{Haberzettl06}
%
%SOV-Konstruktionen aus dem Türkischen ins Deutsche projiziert
%
%    \item Evidenzialität Quechua -> Spanisch??
\end{itemize}
