\section{Hauptteil}

\subsection{Hauptthesen}
\begin{enumerate}
    \item Lernerinnen einer Fremdsprache greifen oft auf schematisierte Äußerungen zurück.
Schlagwort: formulaic phrases
    \item Lernerinnen einer Fremdsprache projizieren Strukturen/Konstruktionen aus ihrer Muttersprache in die L2.
\end{enumerate}


zu 2)

\cite{Weinreich79} ``... discussed how two language systems relate to each other in the mind of the same individual.
The key concept was interference, defined as ``those instances of deviation from the norms of either language which occur in the speech of bilinguals as a result of ther familiarity with more than one language'' \cite{Weinreich79} '' \cite{Cook93}

can happen on all levels of a language:
\begin{itemize}
    \item phonology: ``Speakers may carry over the L1 phonological system by ignoring distinctions made in the L2 but not in the L1;'' \cite{Cook93}
        ``some French learners fail to distinguish between the two English phonemes /i:/ and /i/ as in ``keen'' /ki:n/ and ``kin'' /kin/ because they are not distinct in their L1.''

        \cite{Weinreich79} `` For example, neither French nor Russian have /\textipa{D}, $\theta$/ phonemes, but in contact
        with English, French speakers tend to render /\textipa{D}, $\theta$/ as [z, s], while Russian
        speakers generally pronounce [d, t].
        In other words, the French perceive the continuance of /\textipa{D}, $\theta$/ which distinguishes them from /d, t/ as most relevant, while
        the Russians consider the mellowness of /\textipa{D}, $\theta$/ which distinguishes them from /dz, ts/ as decisive.''


    \item word order: ``German learners of English produce ``Yesterday came he'' modelled on the equivalent German sentence ``Gestern kam er''.''

        \cite{Weinreich79} ``(1) The replica of the relation of another language explicitly conveys an unintended meaning. 
        Example: A German speaker says in English `this woman loves the man' on the model of German
        `diese Frau liebt der Mann', intending to communicate the message `the man
        loves this woman', but producing the opposite effect.
        (2) The replica of the relation of another language violates an existing relation pattern, producing
        nonsense or a statement which is understandable by implication.
        Example: A German speaker says in English `yesterday came he' on the model of German
        `gestern kam er', meaning `he came yesterday'.''


\end{itemize}

In \cite{Lado71} The ``fundamental assumption'' is transfer;``individuals tend to transfer the forms and meanings, and the distribution of forms and meanings of their native language and culture to the foreign language and culture'' (in der 1. Auflage, s.2)

According to \cite{Cook93}: ``In this view L2 learning consists largely of the projection of the system of the L1 on to the L2.
This will be successful when the two languages are the same -- called ``positive'' transfer by some;
it will be unsuccessful whenever the L2 fails to correspond to the L1 -- ``negative'' transfer.''

``Spanish learners add an ``e'' before English consonant clusters starting with /s/ so that ``school'' /sku:l/ becomes /esku:l/ in order to conform to the syllable structure of Spanish;''

grammatical structure as ``system of habbits''(s. 57) - speakers can produce speech automatically and without thinking -> die Konstruktionen haben einen Eintrag im mentalen Lexikon (check Einführung \cite{Ziem13} Kapitel 8 ``Lexikon-Grammatik Kontinuum'')

\subsection{Daten}
\begin{itemize}
    \item Kindlichem L2 Erwerb Spanisch -> Englisch: \cite{Wong-Fillmore76} in \cite{Haberzettl06}
    \item Kindlichem L2 Erwerb Türkisch -> Deutsch: Wegener\cite{} -- leider unveröfftentlichte Schrift
und \cite{Haberzettl05} in \cite{Haberzettl06}

SOV-Konstruktionen aus dem Türkischen ins Deutsche projiziert

    \item Evidenzialität Quechua -> Spanisch??
\end{itemize}
