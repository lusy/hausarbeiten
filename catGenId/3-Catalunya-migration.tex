\section{Catalunya--una terra d'immigrants.Lectura autobiogràfica}

``L'últim patriarca'' és, entre altres coses, la història d'una família d'immigrants.
Per això, repassarem breument el desenvolupament de la immigració a Catalunya a l'útlim segle.
Hem de notar que Catalunya sempre ha sigut una terra d'acollida d'immigrants.
Només al segle XX hi havia multiples onades d'immigració, començant amb la primera als anys '20, passant per la segona, als anys '50-'60, la tercera a les '80 i arribant a l'actualitat [quelle!],~\autocite{TarPaGa2013}.
Totes aquestes onades tenen trets diferents -- més prominent, per exemple, l'origen de la nova població o les raons per immigrar, però totes tenen també similituds.
Una d'elles és la recepció dels i de les recentment arribats.
Com assenyala Fransesc Candel al seu llibre ``Els altres catalans'', cada onada està rebuda per una mena de desconfiança, prejuicis i franca hostilitat~\autocite[17]{Candel1965}.
La població vella (syn) té por del desconegut i de l'incert i sovint reacciona amb discursos i comportaments racistes:
``\ldots heu vingut a menjar-vos el pa dels catalans\ldots''~\autocite[17]{Candel1965},
``Aquests nou[sic!] catalans no coneixen Catalunya: la seva tradició, la seva història, el seu art, la seva cultura, la seva literatura, els seus costums, el seu folklore\ldots''~\autocite[17]{Candel1965},
``Parlen un català gruixut, groller i vulgar\ldots Desconeixen la gramàtica catalana. No saben llegir en aquest idioma. Escriure'l, encara menys.''~\autocite[18]{Candel1965}

La immigració de la família Driouch descrita a la novel·la té com a fons un fet/periode històric:
les diverses onades migratòries del nord d'Àfrica cap Espanya que es produexien a partir de la dècada de les '80~\autocite{TarPaGa2013}.
Segons aquest mateix informe, a l'any 2013 els marroquins són amb 20.4\% el més gran grup d'immigrants a Catalunya.
Els autors confirmen:
``El col·lectiu procedent del Marroc ha estat el més important des dels inicis de la immigració estrangera recent a Catalunya, tant pel nombre com per la seva distribució en el territori.''~\autocite{TarPaGa2013}

La mateixa autora de ``L'útlim patriarca'' va néixer a Marroc, quan el seu pare ja vivia a Catalunya, i va immigrar a Vic amb la resta de la seva família quan tenia vuit anys~\autocite{Vidal2012}.
Llavors, una lectura de la novel·la com obra amb elements autobiogràfics (llevat del grad de proximitat entre ficció i realitat) ocorre de manera automàtica.



--compare with interviews!!
Com s'adapta la gent?


\begin{comment}

\subsection{Entrevistes: lectura autobiogràfica?}

  http://www.belletrista.com/2011/Issue12/features\_2.php
--> die Fragmente können vlt mit zum Patriarchat?

  --naming things!!!
  "N. By learning the language, by learning the words, and reading the dictionary, she starts to name what happens around her, and the first step to get out of an abusive situation is to know what's happening and have the ability to name it. So that's very important. If you think about therapy, when we go to therapy the only thing we do is name the things that are happening around us and inside us. So, if you don't have enough words to name the things, you are not able to be free. And literature is a way to see different lives—lives that can be deeper and end happily. "

  "We don't talk a lot about violence and sometimes there is something very dangerous about people that are violent, and that is the silence. Silence makes you part of this violence."

   N: Some of the things that happen around him justify what he does. I was very surprised when I was writing the book that women in the family were so important in making him become the monster he becomes.

   L. So their behavior allows him to become that way?

   N. Yes, and you can see that. His sisters seem to coddle and protect him, his mother a little less so, and less so as she gets older. I think that wasn't very helpful for him. It's something that's really shocking because they're women, but I've seen in that area of Morocco that sometimes you receive more compassion from men than from women. Or from the older women in the families because they are becoming more powerful as they get older. 

"L'àvia sempre va justificar el comportament poc usual del seu fill amb aquesta història [la bufetada] [...] Sí, els sobresalts se't fiquen a dins i es van transformant en la pitjor part de tots nosaltres, però ja ho saps, filla, que en el fons el teu pare és de bona fusta i no faria mai mal. És només això, que els espants no li han acabat de marxar mai del cos i això l'ha fet algú diferent." (p.18)
-- justificacions; niemand zieht den zur Rechenschaft; vgl Laurie Penny, the men are never responsible for their bad life: the women or minorities are the
guilty ones; that's what society teaches us

"Mimoun s'havia anat acostumant que, per a ell, les normes s'exceptuaven sempre." (p.51)
Laurie Penny?

"A Mimoun no és que li semblés la conducta més lògica del món, però tenint en compte que en aquell pais les coses funcionaven tant del revés i que els cristians no tenien cap sentit ni de l'honor ni del que ell considerava dignitat, l'explicació podia ser prefectament plausible." (p.89)
-- bei so was rege ich mich extrem auf; diese Figur ist so krass ueberzogen dargestellt, um gewisse Wirkung zu erzielen/Ideen zu vermitteln; welche?

"Era així com Mimoun aconseguia sempre que les dones de la seva vida l'anessin convertint en patriarca." (p.99)
--Kommentar bzgl. die Schwester kuemmert sich um alles, nachdem er von Spanien abgeschoben wird; wird von allen in seinen Schwachsinn bekraeftigt, die ihm alles erlauben und immer andere Schuldige finden

Soumisha sobre M:
"És un bon home, creu-me, tot el que li passa no és voluntat seva. I li explicava allò del filtre d'amor a la manera de Curial e Güelfa." (p.214)
-- es ist bemerkenswert, wie ihn alle entschuldigen; es sind immer die anderen schuld. die frauen und die minderheiten

"Elles semblaven entendre'l. Jo no." (p.250)
!!!!

"Havia estat un gran gest per part seva, sí, senyor, tothom va dir-ho. Un home enganyat per la dona i el seu propi germà tenia el cor tan gran que havia pogut continuar tenint-la a ella d'esposa i al final fins i tot es decidia a perdonar l'altre culpable. Aquesta era la versió que el pare va sentir i la que es va creure. La que corria pel poble, i que em van explicar cosines i tietes, era que la gent no s'havia cregut mai aquella història, que l'actitud de la mare sempre havaia estat impecable i era impossible que fes una cosa com aquella [...]" (p.276)
--ist sogar noch monströser wie sie isoliert wird; obwohl ist unklar, wie sich die menschen ausm dorf zu ihr verhalten, weil sie auch selber nicht aus dem haus geht

 http://www.elcritic.cat/entrevistes/najat-el-hachmi-tothom-esta-sorpres-que-jo-escrigui-sobre-sexe-per-que-perque-soc-marroquina-3627
 P: A ‘L’últim patriarca’ sembla que la responsabilitat de l’honor i de la vergonya recaigui sobre la dona.
A: Això és un dels pilars bàsics del patriarcat. Crec que la majoria de religions monoteistes ho són; el control de la sexualitat femenina és bàsic.

--vgl Laurie Penny

el moral doble és increïble:
"Si ell es divorciés, la seva dona li hauria de continuar sent fidel fins a la mort; per alguna cosa havia estat ell el primer de tenir-la." (p.151)
--> quina intenció té aquest relat? solidifica estereotips..

A banda d’això, sí, la veritat és que al món musulmà segueix molt vigent la idea que l’home té uns instints naturals que no pot controlar i que és la dona qui s’ha d’encarregar de protegir-se.

sobre Jaume:
"No, era un home, n'estava segur, però no patia la disminució natural del seu gènere a l'hora de fer les tasques de la casa." (p.131)

altra violació:
"Però el Manel tenia aquell mena d'instint de caçador que han de tenir per força els que estan destinats  a ser gran patriarques i no entenia que era no." (.91)
-- no means no

Es diu que Catalunya és terra d’acollida i no és veritat: és terra d’immigrants; no existiria sense la immigració. Ara ha sortit un llibre molt interessant que es diu ‘Catalunya al mirall de la immigració’, d’Andreu Domingo, i editat per L’Avenç.


[Vidal2012]

quote El Hachmi, Najat (2004). Jo també sóc catalana. Barcelona: Columna.
"En definitiva: has de fer els malabarismes que calguin per anar decent per a l’estètica nord-africana i alhora no semblar una pobra noia reprimida davant dels originaris de Catalunya."
--> interseccionalitat?
--> o identitat cultural?

"La roba, la roba, sempre discutint amb la mare quina peça era adequada i quina no i jo ja no sabia com conciliar tantes exigències, entre les modes de l'institut, on no volia semblar rara, les del mercat, on no hi cabia, i les seves, la major part del temps senzillament absurdes." (p.285)

"Para ella, hay muchos tipos de discriminación: la del racista que golpea de frente, pero también la del paternalista, que dice cosas como que, aceptando al inmigrante, nuestra cultura se enriquece."
-- identitat cultural?

quote Interview El Hachmi:
"Los pornógrafos de la etnicidad acentúan rasgos de ti que en tu país encontrarías ridículos."
"El inmigrante no quiere pertenecer a una asociación de inmigrantes, sino a una de vecinos..."
"Cada mujer lleva el velo por motivos diferentes. Y no se
puede pretender salvar, de manera paternalista, a las pobres moritas del yugo de
sus maridos.
-El yugo existe.
—Como en otras culturas."
"El problema es que la mayoría tiene permiso de residencia sin permiso de trabajo. La ley de extranjería las condena a la clandestinidad laboral y eso hace que dependan del marido. Les corta la vía de emancipación."

"Treu-te això del cap, que em fas passar vergonya. I ella que no, que em sentiré despullada, que no. Mira que aquí les coses són diferents i a mi em coneix molta gent i tinc una empresa i no hi ha cap nexessitat de portar aquests draps." (p. 183)
-- der stellt sich sehr befreit und so dar; und spaeter findet er gut, dass seine tochter nen hijab traegt..

"Me'l posava per resar, primer. Després per estar per casa. FIns que vaig sentir que era imprescindible, que no podria viure mai més passant amb el cap descobert davant de ningú. Me'l vaig posar per anar a comprar i vaig sentir les mirades estranyades de les botigueres que em coneixien [...] Vaig sortir així un parell de vegades i un dia el pare em va veure. [...] Au, no surtis més amb aquest drap al cap." (p.228)
-- warum gucken sie die Verkaeuferinnen komisch an? ist sie etwa nicht dieselbe person? was sehen sie im hijab? ein symbol der anderen? ein symbol der opression?

"La mare em va fer anar a ca la Soumisha a buscar alguna cosa i jo vaig posar-me el mocador, [...] Sembles un àngel, m'havia dit ella, segur que entraràs al cel directament, per la porta gran. [...] Els nostres ulls es van trobar i allà mateix ho vaig saber, que no hauria hagut de posar-me el mocador. [..] no sé com no vaig caure. Ell no deia res, però jo ja el sentia derrere meu i quan va dir para, para o encara serà pitjor, jo ja no sé si vaig córrer o em vaig aturar, però em recordo a terra, amorrada a la clavaguera i ell vinga donar-me puntades de peu. No recordo els cops, no recordo si em va picar a la cara, a l'estòmac. [...] I llavors vaig mirar tot al meu voltant i vaig veure els clients del bar de davant de casa amb la beguda a la mà que no deien res i els que passaven pel costat que no deien res i els que ens coneixien i no deien res i allò era estar sola." (p.228-229)
--estar sola
-- la indiferència de la gent davant "els\_les altres"

\subsection{Comments}

La població estrangera a Catalunya (APUNTS de prospectiva territorial, numero 2)

"El marroquins, segon grup de la província, presenta una migració més masculina en els grups de més edat (majors de 35 anys), però equilibrada en les edat inferiors. Destaca especialment l'important volum de població infantil: el 14,6\% dels marroquins tenen menys de 5 anys."

------------------------

[Candel1965]

Els altres catalans

"On hi ha feina, hi són ells" (p.17)
"A la terra dels seus pares no hi ha feina" (p.17)
"Se senten conquistats per Catalunya; no del tot, és clar [...] si els diuen murcians o gallecs s'enfaden; si els diuen catalans, no" (p.17)
"Davant la persistent i discutida qüestió: "...heu vingut a menjar-vos el pa dels catalans..." es posen furiosos, naturalmen! i aleshores, només aleshores, malparlen de Catalunya. A ells no els el regala ningú, el pa; ells se'l suen;" (p.17)

"I de vegades ells mateixos, amb la mateixa espasa flamígera esmentada de "heu vingut a menjar-vos el pa.." etcètera, ataquen els darrers immigrants que denigren Catalunya. Punt." (p.17)

""Aquests nou[sic!] catalans parles català, el van aprendre sense adonar-se'n. Molts el parles de manera natural i quotidiana, perquè sí, i altres per afany de sentir-se catalans de debò"" (p.17)

""Tota aquesta gent no s'adona de la seva aclimatació. "Són" catalans fins a cert punt. "No" són catalans, també fins a cert punt. No és una qüestió d'honor ni de principis."" (p.19)

"Aquells immigrants murcians no havien arribat a Catalunya com a colonitzadors. Tampoc, o molt febement, com a invasors o peoners. Havien vingut a treballar i a menjar, senzillament, perquè a la seva terra es morien de gana [...] Avui dia, tots s'han integrat, i alguns, ultrapassant o sobrepujant aquesta integració, s'han tornat furibunds catalanistes. [...] Aquesta esperiència pot demostrar o permetre d'esperar que amb els immigrants d'ara passarà el mateixo poc més o menys." (p.32)

"el que passa és que no acabem de decidir-nos a anomenar catalans els qui han nascut aquí de pares de fora" (p.34)

\end{comment}
