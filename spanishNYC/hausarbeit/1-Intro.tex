\section{Einleitung}

\begin{comment}
* ca 3/4 Seiten
* Untersuchungsgegenstand
* Erkenntnisinteresse
* Forschungsstand
* Vorgehensweise: also Section 2 macht blabla, Section 3 blablup, ....
* Ergebnisse können/sollen angedeuten werden
\end{comment}


%TODO
%Generall Intro fehlt!
% code switching, borrowing als Phänomene, die einen Language Shift andeuten;
% dafür kann es verschiedene Gründe geben
% interessant herauszufinden, welche davon in konkreten Sprachkontaktsituationen zutreffen
% Prognosen darüber, ob eine Sprache (an einem bestimmten Ort/Gemeinschaft) erhalten bleibt oder verloren geht

Sprachwandel ist eins der zentralen Phänomene, die die Soziolinguistik untersucht.
Oft ist Sprachkontakt eine Voraussetzung für Wandel.
In welchen Kontaktsituationen kommt es jedoch zu Wandel und in welchen etabliert sich ein ``stabiler'' Bilinguismus?
Welche Faktoren begünstigen den Wandel?
Welche Indizien weisen auf Wandelprozesse hin?
% Vlt etwa unglücklich, ich spreche halt gar nicht von Wandel im Rest der Arbeit. Alternative überlegen!

%TODO
%Notiz zum generischen Femininum

Das Ziel dieser Arbeit ist eine Analyse der Code-Switches und Entlehnungen zwischen dem Spanischen und dem Englischen in Artikeln der US-amerikanischen Zeitschrift \textit{Siempre mujer} aus einer soziolinguistischen Perspektive.
Es gilt herauszufinden, ob diese Code-Switches in bestimmten Kontexten passieren oder bestimmten rekurenten Mustern unterliegen.
Wenn diese Erwartung erfüllt wird, wird eine sozioindexikalische Erklärung versucht.

Die Ausgangshypothese ist, dass dies sehr wohl der Fall ist.
Im hauptsächlich spanischen Text werden in speziellen Fällen englische Ausdrücke erwartet:
zum Beispiel wenn es um technische Begriffe oder um Idiome/Redewendungen geht, für die keine genaue spanische Übersetzung existiert.
Weitere Fälle, in denen wir Englisch erwarten können, sind region-/kulturspezifische Begriffe: Essen, Tiere und Pflanzen, Bräuche, etc.

Es wäre weiterhin interessant zu untersuchen, in wie fern bewusst ins Englische geswitcht wird, da diese Sprache als prestigeträchtiger wahrgenommen wird oder um einen bestimmten Lifestyle zu inszenieren, wodurch dann auch zum Beispiel versucht wird gewisse Produkte zu verkaufen.

Um diese Analyse zu realisieren, wurden Artikel aus der elektronischen Ausgabe der Zeitschrift heruntergeladen und Code-Switches ins Englische extrahiert.

\subsection{Verwandte Arbeiten}

Das Thema der Zweisprachigkeit Spanisch-Englisch in den USA ist nicht neu.
Es existieren bereits zahlreiche Arbeiten, die sich diesem Thema widmen.
Ana Zentella und Ricardo Otheguy sind nur zwei der Forschenden, die mit ihrer Arbeit zu Zweisprachigkeit und Spracherwerb/Sprachunterricht einen erheblichen Beitrag auf dem Gebiet geleistet haben (konkreter!).

Der Fokus dieser Arbeit liegt jedoch anderswo. (Ja wo denn?)
Medientheorie?

%[Ticknor12]
Kathryn Ticknor von George Town University hat 2012 einen Blogbeitrag veröffentlicht, der sich der Analyse von Code-Switches auf den Umschlägen der \textit{Latina Magazine} im Zeitraum 1996 (das Jahr, in dem die Zeitschrift gegründet wurde) bis 2012 widmet~\cite[]{Ticknor12}.
Ähnlich wie bei \textit{Siempre Mujer}, handelt es sich bei \textit{Latina Magazine} um eine Publikation, die sich an in den USA lebenden Latinas richtet.
Allerdings, während die Texte in \textit{Siempre Mujer} vorwiegend spanisch sind, wird \textit{Latina} in englischer Sprache herausgegeben.
Dementsprechend sind die Switches, die Ticknor analysiert, Spanisch (während wir in der vorliegenden Arbeit englische Switches ins Spanische betrachten).
Ticknor schaut sich 58 Umschläge aus dem bereits erwähnten Zeitraum an und schließt daraus folgendes:
\begin{itemize}
  \item die Benutzung von Code Switches geht mit der Zeit zurück. (Sie hat einen Durchschnitt von acht Switches ins Spanische pro Umschlag für das Jahr 1998 gezählt  und nur ein Switch im Durchschnitt im Jahr 2010.)
  \item Code-Switches nach grammatischer Kategorie:
    \begin{itemize}
      \item es werden vor allem offene gramatische Kategorien wie Substantive und Adjektive geswitched;
      \item es gibt jedoch auch Switches von Elementen geschlossener Kategorien, wenn sie in Kombination mit offenen vorkommen (``Take our test fashions \textit{del futuro}'');
      \item seltener kommen auch Switches innerhalb von Phrasen oder vom selben Wort vor (``\textit{mami}hood'', ``\textit{mami}'s boys''). Voraussetzung hierfür ist die phonologische Ähnlichkeit der Elemente, die geswitchted werden (die spanischen \textit{mami} und \textit{madre} sind ähnlich zu den englischen \textit{mama} und \textit{mother}).
    \end{itemize}
  \item es werden oft Bezeichner von Familienbeziehungen, besonders viel weibliche solche, geswitched (``madre'', ``hija'', ``chica'');
  \item ferner werden weitere Begriffe geswitched, die mit einer Latinakultur in Verbindung gebracht werden (Essen wie ``tortilla'' oder ``flan'' ; ``tango'') und für die es keine adequate Übersetzung auf Englisch gibt;
  \item sowohl solche, die eine Latina-Identität stärken (und dabei sich aber Klischees bedienen und solche auch verfestigen): ``gordita''.
\end{itemize}

Ticknor erklärt die Switches vor allem mit der Motivation, ein Latinapublikum aufgrund von einer gemeinsamen Identität anzusprechen.
Code Switches (bewusste und unbewusste) sind imanent für eine bilinguale Gemeinschaft.
Sie merkt jedoch weiterhin an, dass die Herausgeberinnen der Zeitschrift sich bemühen, nur solche Begriffe zu switchen, die auch einem monolingualen (englisch-sprachigen) Publikum zugänglich sind und dieses nicht abschrecken.


%[Lee99]

%Kürzen!
In ihrer Masterarbeit an der City University Hong Kong untersucht Lee Pui Yin Micky Code-Switching in populären Hong Kongnesischen Zeitschriften (aus soziolinguistischer? und diskursanalytischer Perspektive)~\cite[]{Lee99}.
Sie bespricht wie verschiedene Codes in Medientexten genutzt werden und wie Hong Kong Bürgerinnen versuchen, eine moderne Identität zu konstruieren.
Lee argumentiert, dass Code-Switches in verschiedenen Genren von Medientexten verschiedene Funktionen haben.
%Was ist ihre Forschungsfrage: go over abstract again
%Focus groups
%Interviews
Die Studie wird zusätzlich durch Interviews angereichert, in denen Produzentinnen und Leserinnen/Konsumentinnen der untersuchten Medien zu ihrer Produktion und Konsum/Wahrnehmung/.. befragt werden.

Obwohl die meisten davon ins Englische sind, wird gelegentlich auch in andere Sprachen wie z.B. Französisch oder Japanisch geswitched.
Die Autorin argumentiert, dass bei Code-Switches in Sprachen, die mit dem lateinischen Alphabet verschriftlicht werden, die Leserinnen gar nicht unbedingt die genaue Sprache wahrnehmen, bzw. gar nicht unbedingt erwartet wird, dass sie Französisch (oder auch Englisch) verstehen, sondern eben die Tatsache, dass geswitched wird.
Sie spricht vom ästhetischen/visuellen Wert von lateinischen Buchstaben inmitte von chinesischen Schriftzeichen und attibuiert diesen eine symbolische Bedeutung.

Zwei große Kategorien werden bei der Analyse identifiziert und weiter/tiefer/genauer untersucht: Mode und Technik.
In der Modekategorie wird beobachtet, dass vorwiegend Namen von Produkten, Marken oder Boutiqueläden geswitched werden.
Dies ist anscheinend auch in gesprochener Sprache der Fall.
Die Forscherin? erklärt dies mit dem Konzept ``voice-quoting'': Sprecherinnen werden bei der Wahl der Bezeichnungen (für die oft, wenn auch nicht immer, auch eine chinesische Übersetzung existiert) von der eigenen Peergruppe, von Verkäuferinnen aber auch von Werbung (unter welchem Namen wird das Produkt vermarktet) beeinflusst.
Also alle ``zitieren'' die Namen, die sie von wem anders gehört haben.
Ferner wird beobachtet, dass Kosmetikartikel wissenschaftlichen Berichten ähneln.
Oft werden die Inhaltsstoffe von Kosmetikprodukten auf Englisch aufgelistet, was die Texte technisch/wissenschaftlich erscheinen lässt.
Zugleich sind viele diese Begriffe relativ abstract, sodass selbst wenn die Leserinnen sich ihre wörtliche Bedeutung erschliessen können, sich wahrscheinlich immer noch nicht allzukonkret vorstellen, was sich hinter den Bezeichnungen verbirgt.
Dies sollte den Texten eine mystische/magische Art verleihen.

Eine wissenschaftliche Art haben auch Texte aus der Technikkategorie. ("informative reports about computer hardware")
Da werden Produkt- und Firmennamen, Webseitadressen, sowie andere technische Begriffe geswitched.
Die Autorin argumentiert, dass diese insofern den ``technischen'' Artikeln aus der Kosmetikkategorie ähneln, dass eine Leserin, ohne domäne spezifisches Wissen aus dem Technikbereich Schwierigkeiten haben würden, sich die Bedeutung der Texte zu erschließen.
% marking group membership by using a specific technical jargon
% ``Hong Kong computer users may perceive English (or Roman alphabets, to be precise) as being associated with sciences, advanced technology and objectivity.'' (p.151)


%[Andr07]
Jannis Androutsopoulos nimmt sich das Thema in seinem Überblicksartikel ``Bilingualism in the mass media and on the internet'' vor~\cite[]{Andr07}.
Er betrachtet diverse Mediengenres/arten/formen: Sendungen und gedruckte Texte in öffentlich-rechtliche, privaten kommerziellen und privaten nicht-kommerziellen Medien; Werbung; Songtexte; öffentlichen Diskurs im Internet;
und bestätigt/stellt fest, dass ``linguistic diversity is gaining an unprecedented visibility in the mediascapes of the late twentieth and early twenty-first century''.
Ihm zufolge weicht die Einsprachigkeit der öffentlich-rechtlichen Massenmedien, die unter anderem deren Projekt der Nationsbildung geschuldet ist, diversen (linguistischen) Ausdrucksweisen und -formen.
Eine der Erklärungen, die für diesen Wandel geliefert werden, ist die zunehmende Globalisierung.
Zudem ``The global Anglo-American dominance in science, technology and entertainment is often evoked to account for the use of English in national media (e.g. Phillipson and Skutnabb-Kangas 1999).''
Eine zweite Erklärung ist die Veränderung in den Formen/Voraussetzungen/Umständen der Medienproduktion und -rezeption.
Mit dem Internet wird die klare Abgrenzung zwischen den Beiden immer unschärfer.
Marginalisierte Gruppen und Aktivistinnen haben immer leichter Zugang zu Medienproduktion und auch kommerzielle Anbieterinnen massschneidern ihre Angebote für immer kleinere Zielgruppen, die sie unterm anderen auch durch die sprachliche Ausdrucksweise versuchen anzusprechen.
Jedoch, meint Androutsopoulos, wurde Zweisprachigkeit in den Medien relativ wenig untersucht, mit der Ausnahme der Werbung.
Dies wäre der Tendenz geschuldet ``to view bilingualism in the media as ‘derivative’, ‘artificial’ or ‘inauthentic’''. % sehe ich auch so (also dass es nicht "frei"/"authentisch" entsteht; ich würde das 2./3. Sozioindexikalität nennen.
von der wir uns wegbewegen sollten, um den Phänomen wissenschaftlich untersuchen zu können.
``The aim is rather to examine the
hows and whys of the strategic selection, combination and transformation of
linguistic resources in particular discursive spaces of mediated communication.
Understanding the nature of these spaces can help us comprehend how the
media shape public images of bilingualism (cf. Hill 1999), and how they
might be used to challenge and alter these images.'' %<-- summarize this for the different genres


%[Mahootian05]
Shahrzad Mahootian untersucht die Beziehung zwischen Code Choice (Sprachwahl?), zweisprachiger Identität und Sprachwandel~\cite[]{Mahootian05}.
Sie stellt sich folgende Fragen: In wie fern sind stilistische und soziale Variablen, inklusive Identität, verantwortlich für die Wahl des sprachlichen Codes?
Was ist die Beziehung zwischen Wahl des sprachlichen Codes und Sprachwandel?
In ihrer Arbeit untersucht die Autorin die Benutzung von gemischtem Spanisch-Englisch Code in Druckmedien (genauer gesagt, betrachtet Mahootian auch das US-amerikanische Magazin \textit{Latina}) und interpretiert diese bereits als ein Zeichen von Sprachwandel.
Sie unterstreicht, dass die Codeauswahl in solchem Kontext sehr bewusst passiert, da diese in der Regel einen mehrstuffigen Redigierprozess unterlaufen.
Mahootian bietet/formuliert folgende (formulier anders!) Erklärungsansätze für das Codemixing in Druckmedien an:
\begin{itemize}
  \item Schaffen einer Intimität, Nähe zwischen Herausgeberinnen und Leserinnen, aufgrund von einer geteilten Latina-Identität
  \item ``Linguistic necessity'' (also das genaue Wort existiert auf Englisch nicht) / idiomatischer Gebrauch (DE?): \textit{la musica tejana}, food names, titles of Spanish songs, movies,..
  \item Manche Ausdrücke wurden durch Muttersprachlerinnen als ``emotional stärker'' auf Spanisch beurteilt
  \item Code-Switching wird teilweise benutzt nur um Aufmerksamkeit zu erregen (durch die Tatsache, dass der Code wechselt)
  \item Zugleich Bewusstsein dafür, dass Englisch die ``prestige trächtigere'' Sprache ist, dass sie auf Englisch angewiesen sind, um an der Wirtschaft teilhaben zu können/für ihre Existenzgrundlagen sorgen zu können;
\end{itemize}

\begin{comment}

[Mahootian05]
Die Autorin untersucht die Beziehung zwischen Code Choice (Sprachwahl?), zweisprachige Identität und Sprachwandel.
Sie stellt sich folgende Fragen:
* in wie fern sind stilistische und soziale Variablen, inklusive Identität verantwortlich für die Wahl des sprachlichen Codes?
"The first asks the extent to which stylistic and social variables, including
identity, govern code choice."
* was ist die Beziehung zwischen Wahl des sprachlichen Codes und Sprachwandel?
Dabei interpretiert sie die Benutzung von gemischtem Code in Druckmedien bereits als ein Zeichen von Sprachwandel.

Besonders relevant, dass es sich um Druchmedien handelt: die Codeauswahl hier passiert bewusst, die Artikel werden nicht nur von ihrer Autorin gelesen, sondern unterlaufen weitere Redigierstufen bevor sie abgedruckt werden.
Also einerseits Merkmal der Akzeptanz: Wenn mixed Code Diskurs in den Mainstream Medien angekommen ist (vgl die Debatte wegen Sprachverunglimpfung und bla und die abwertende Wahrnehmung von \textit{pocho} und \textit{caló} Sprecher*innen).
Erklärungsversuche:
* Schaffen einer Intimität, Nähe zwischen Herausgeber*innen und Leserinnen
* "Linguistic necessity" (das genaue Wort existiert auf Englisch nicht)
* Manche Ausdrücke wurden durch Muttersprachler*innen als "emotional stärker" auf Spanisch beurteilt
* Zugleich Bewusstsein dafür, dass Englisch die "prestige trächtigere" Sprache ist, dass sie auf Englisch angewiesen sind, um an der Wirtschaft teilhaben zu können/für ihre Existenzgrundlagen sorgen zu können;
Andererseits auch Versuch, ihre (hispanische?) linguistische und kulturelle Identität zu bewahren.

Ähnlich
Shahrzad Mahootian führt eine soziolinguistische Analyse des Magazins \textit{Latina} durch.
Da liegen die Artikel anscheinend in 3 Versionen vor: Spanisch, Englisch und Englisch mit Codeswitching in Spanisch.
Dabei scheint die spanische Version in der Regel eine kürzere Zusammenfassung der Englischen zu sein.

Bzw. die Autorin macht folgende Kategorien auf:
(1) idiomatic: \textit{la musica tejana}, food names, titles of Spanish songs, movies,..
(2) attention-getting
(3) emotionally/culturally evocative/bonding

Und befragt 3 Muttersprachler*innen dazu, wie sie 56 ausgewählte Beispiele, wo Code-Switches vorkommen, kategorisieren würden.

"Language is both coconstructor and
a reflection of social identity."

"I’ve shown that codemixed discourse is itself a speech variety and that codemixed
discourse and codeswitching may be used intentionally to index the “bilingual identity.”"

"Mixed Spanish-English code is another means for Latinos in the
U.S.A. (a) to show unity among themselves as a subset of all Hispanic-Americans, (b)
to identify themselves as a group separate from their predecessors’ generation, and (c)
to continue to maintain strong emotional ties with their heritage."

-----

[Ticknor12]
  code switching can arise from individual choice or be used as a major identity marker for a group for speakers who must deal with more than one language in their common pursuits. As Gal (1998, p. 247) explains, “code-switching is a conversational strategy used to establish, cross or destroy group boundaries; to create, evoke or change interpersonal relations with their rights and obligations.” The combination of grammatical constraints and social-psychological motivations for a magazine editor’s decision to put code-switch certain words is how a study involving written code-switching must proceed in order to fully account for the factors presented by sociolinguists up to this point.

es again and again Latina magazine is constructing a myth of what it means to be latina.  As Goffman explores in Gender Advertisements, the creation of a culture, be it gender-based or ethnicity-based, is built through the repetition of semiotic and textual messages.

how Latina identity is constructed through a combination of textual and image resources. In order to determine which aspects of identity are being appropriated for Latina identity as differentiated from feminine identity, studies of ‘womens’ magazines would also serve as a point of comparison.

-----

[code switching in popular hk magazines]

"The model proposed in this study emphasizes:
(1) the different functions of code-switching in different genres;
(2) the complex relationships between the participants in printed media discourses;
(3) the graphical values of non-Chinese words in printed media texts; and
(4) the linkage between the micro studies of languages in daily mediated i nteractions and the macro sociocultural context."

"The theoretical argument in a local sense of this study is that by employing two or more languages in media texts, this practice may maintain and reinforce the grand narrative that the government has constructed and propagated for Hong Kong people.
The grand narrative is that Hong Kong people are conventionally described as "Westernized Chinese" and Hong Kong is a place where "east meets west". Code-switching, as a form of language use, may be coherent with the grand narrative of Hong Kong and Hong Kong people. Language use is hence a form of identity constructing tool."

"...foreign words may serve as an eye-catching device in texts written predominantly in Chinese"

Topics:
fashion
computers

\end{comment}

\subsection{Aufbau der Arbeit}

Im Weiteren ist diese Arbeit wie folgt aufgebaut:
Kapitel 2 gibt einen Überblick über die theoretischen Grundlagen, auf denen die vorliegende Analyse aufbaut.
In Kaptel 3 stelle ich das untersuchte Korpus und die benutzte Methodik vor.
Die Ergebnisse der Untersuchung werden in Kapitel 4 diskutiert.
Kapitel 5 gibt einen Ausblick auf Verbesserungen, Anregungen und Fragen für zukünftige Analysen und fasst nochmal abschließend die Arbeit zusammen.

%\cite[vgl.][s.20]{Tomasello06}
