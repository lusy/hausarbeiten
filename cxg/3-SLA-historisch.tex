\section{Die Zweitspracherwerbsforschung in historischer Perspektive}
\cite{Weinreich79} ``... discussed how two language systems relate to each other in the mind of the same individual.
The key concept was interference, defined as ``those instances of deviation from the norms of either language which occur in the speech of bilinguals as a result of ther familiarity with more than one language'' \cite{Weinreich79} '' \cite{Cook93}

can happen on all levels of a language:
\begin{itemize}
    \item phonology: ``Speakers may carry over the L1 phonological system by ignoring distinctions made in the L2 but not in the L1;'' \cite{Cook93}
        ``some French learners fail to distinguish between the two English phonemes /i:/ and /i/ as in ``keen'' /ki:n/ and ``kin'' /kin/ because they are not distinct in their L1.''

        \cite{Weinreich79} `` For example, neither French nor Russian have /\textipa{D}, $\theta$/ phonemes, but in contact
        with English, French speakers tend to render /\textipa{D}, $\theta$/ as [z, s], while Russian
        speakers generally pronounce [d, t].
        In other words, the French perceive the continuance of /\textipa{D}, $\theta$/ which distinguishes them from /d, t/ as most relevant, while
        the Russians consider the mellowness of /\textipa{D}, $\theta$/ which distinguishes them from /dz, ts/ as decisive.''


    \item word order: ``German learners of English produce ``Yesterday came he'' modelled on the equivalent German sentence ``Gestern kam er''.''

        \cite{Weinreich79} ``(1) The replica of the relation of another language explicitly conveys an unintended meaning.
        Example: A German speaker says in English `this woman loves the man' on the model of German
        `diese Frau liebt der Mann', intending to communicate the message `the man
        loves this woman', but producing the opposite effect.
        (2) The replica of the relation of another language violates an existing relation pattern, producing
        nonsense or a statement which is understandable by implication.
        Example: A German speaker says in English `yesterday came he' on the model of German
        `gestern kam er', meaning `he came yesterday'.''


\end{itemize}

In \cite{Lado71} The ``fundamental assumption'' is transfer;``individuals tend to transfer the forms and meanings, and the distribution of forms and meanings of their native language and culture to the foreign language and culture'' (in der 1. Auflage, s.2)

``Robert Lado proposed in his
Contrastive Analysis Hypothesis (CAH) (Lado 1957): ‘We assume that the
student who comes in contact with a foreign language will find some features
of it quite easy and others extremely difficult. Those elements that are similar
to his native language will be simple for him, and those elements that
are different will be difficult’ (Lado 1957: 2)'' (in \cite{Ellis06})


According to \cite{Cook93}: ``In this view L2 learning consists largely of the projection of the system of the L1 on to the L2.
This will be successful when the two languages are the same -- called ``positive'' transfer by some;
it will be unsuccessful whenever the L2 fails to correspond to the L1 -- ``negative'' transfer.''

``Spanish learners add an ``e'' before English consonant clusters starting with /s/ so that ``school'' /sku:l/ becomes /esku:l/ in order to conform to the syllable structure of Spanish;''

grammatical structure as ``system of habits''(s. 57) - speakers can produce speech automatically and without thinking -> die Konstruktionen haben einen Eintrag im mentalen Lexikon (check Einführung \cite{Ziem13} Kapitel 8 ``Lexikon-Grammatik Kontinuum'')

----------------

Effects of L1 on L2-Learners in \cite{Braidi99}


Adoption of a L1 rule in the IL grammar:
``I go not to school'' German-speaking L2 learners of English ``reflecting the post-verbal negation structure
of the German L1 (``Ich gehe nicht in die Schule'')''

``... English-speaking L2 learners of Spanish may overgeneralize the usage of present progressive to indicate a
future event, an inappropriate context for this structure in Spanish (*``Estoy saliendo mañana''/``I am leaving tomorrow''
for ``Voy a salir mañana.'')


Lado according to Braidi:
Similar structures are easily transfered from L1 in L2;
``structures could be similar in 3 ways: they could be signalled by the same formal device, have the same meaning,
or have a similar distribution in the language system'' (Lado 1957: 66)

``The student tends to transfer the sentence forms, modification devices, the number, gender and case patterns of
his native language.'' (Lado 1957: 58)

``Transfer is the influence resulting from similarities and differences between the target language and any other
language that has been previously (and perhaps imperfectly) acquired'' (Odlin 1989: 27 Language Transfer: Cross-linguistic Influence in Language Learning
JFK F 107.4/ O 24)

* use of L1 typological organization and overproduction of structures (Schachter and Rutherford 1979 Discourse function and language transfer; Working Papers on Bilingualism 19 3-12)

\subsection{Kognitivlinguistische/Konstruktionsgrammatische Perspektive}

\cite{Ellis06} ``Interference theory primarily concerned the transfer of the content of
associations. But more recent analyses demonstrate how from content, given
enough of it, emerges principle, how form–meaning mappings conspire in
biasing attention and process. As a ubiquitous process of learning, transfer
pervades all language learning.''

``If a referent already has an associated name, it is harder to
attach a new name to it.''

`` when two cues are presented together and they
jointly predict an outcome [...] the most salient cue becoming associated with the
outcome and the less salient one being overshadowed''

`` Forms that have not
previously cued particular interpretations are harder to learn as cues when
they do become pertinent later.''

``Overshadowing as it plays out over time produces a type of learned
selective attention known as blocking. Chapman and Robbins (1990) showed
how a cue that is experienced in a compound along with a known strong
predictor is blocked from being seen as predictive of the outcome.''

``Not only are many grammatical meaning–form relationships low in salience,
but they can also be redundant in the understanding of the meaning of
an utterance.''

``Schumann (1987) describes how L2 temporal reference is initially made exclusively
by use of devices such as temporal adverbials [...] with the grammatical expression of tense and aspect emerging only
slowly thereafter (Bardovi-Harlig 2000)''

``Thus, another pervasive reason for the non-acquisition of low salience cues
in L2 acquisition is that of blocking, where redundant cues are overshadowed
for the historical reasons that the learners’ first language experience leads
them to look elsewhere for their cues to interpretation.''

``The usual pedagogical reactions to these situations of overshadowing or
blocking involve some means of retuning selective attention, some type of
form-focused instruction or consciousness raising (Sharwood-Smith 1981) to
help the learner to ‘notice’ the cue and to raise its salience.''

``Whereas the associative learning effects detailed above relate to specific cues
or constructions and their interpretations, perceptual learning is more to do
with the organization of the whole system''

``As more and more instances are processed,
so the representations of these exemplars become sorted and positioned in
psychological space so that similar items are close together and dissimilar
ones are far apart.''

``human categorization
research evidencing attention shifts toward the use of dimensions that are
useful for the tasks in hand [...] learners
become more sensitive to these dimensions and are better able to make
discriminations involving them.''

``Early, balanced simultaneous bilingual
simulations showed the clearest evidence of language separation: through
self-organization, the network comes to separate the Chinese lexicon from
the English lexicon, implicating distinct lexical representations for the
two languages. In the DevLex model, less balanced or later L2 input
produce representations whereby L2 is partially parasitic on L1, and adult
second-language simulations show relatively little L1–L2 separation at a local
level and maximal transfer and interference.''

``L2 acquisition is clearly affected by the transfer
of learners’ knowledge of their first language.''

``Taylor investigated
the English of elementary and intermediate native Spanish speaking ESL
students. He analyzed their errors in the Auxiliary and Verb Phrases of eighty
sentences, categorizing them into those errors that resulted from L1 transfer,
and those that resulted from overgeneralization of L2 patterns.
[...] with transfer
errors more prevalent among elementary students (40 percent) than
intermediate students (23 percent), and overgeneralizations more prevalent
in intermediate (77 percent) than elementary students (60 percent).''
