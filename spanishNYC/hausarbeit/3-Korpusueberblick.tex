\section{Korpusüberblick}

%Überblick - Datensatz:
%----------------------

\subsection{Datensatz}

Dieser Untersuchung liegen $1419$ Artikel der Zeitschrift \textit{Siempre Mujer} zu Grunde,
die in der Periode März 2010 bis Februar 2015 in die Online-Ausgabe des Magazins\footnote{http://siempremujer.com/} veröffentlicht wurden.

\textit{Siempre Mujer} wird von der Meredith Corporation USA, California herausgegeben\footnote{http://www.meredith.com}.
%2 Sätze mehr zu Meredith Corporation?
Das Zielpublikum der Publikation sind Latinas, die in den USA leben. %Quelle?
%Read http://siempremujer.com/pagina/quienes-somos/ -> Zielpublikum?
Die Artikel der Online-Ausgabe sind vorwiegend in/auf? spanischer Sprache verfasst,
wobei wir gelegentlich Code-Switches ins Englische beobachten können.
Diese sollten im weiteren Verlauf der Arbeit näher untersucht werden.

Die vorliegende Analyse hat keinen Anspruch auf Vollständigkeit:
es werden weder zwangsläufig alle in der oben genannten Periode veröffentlichte Artikel betrachtet,
noch werden alle Englisch-Vorkommnisse extrahiert und besprochen.


%\begin{itemize}
%  \item Zeitschrift siempremujer --> Herausgeber: meredith corporation USA, California
%  \item 1419 Artikeln von 23.03.2010(?) bis 13.02.2015
%  \item Online-Version (nicht Druckausgabe)
%  \item Kein Anspruch auf Vollständigkeit weder bzgl Anzahl betrachteten Artikeln, noch bzgl *alle* English-Vorkommnissen innerhalb der Artikeln
%  \item Zielpublikum?
%  \item Sprache
%\end{itemize}

\subsection{Methodik}
%Ein bisschen über die Methodik:
\begin{itemize}
  \item runtergeladen
  \item plaintext
  \item open office dicts (LGPL)
  \item in Wörter segmentiert (white spaces delimiters)
  \item Wörter annotiert
  \item nicht perfekt, Wörterbücher sind keine erschöpfende Wortlisten -- verbesserungsbedürftig
  \item alles angeguckt (manuel), was *nur* als Englisch annotiert wurde
  \item auch problematisch beim Ansatz: viele Wörter, die es sowohl im Englischem als auch im Spanischen gibt (Homographe); selbst wenn man nur die als Englisch annotierte Tokens herauspickt, handelt es sich doch öfters um Homographe, da Wörterbücher nicht erschöpfend sind
\end{itemize}
