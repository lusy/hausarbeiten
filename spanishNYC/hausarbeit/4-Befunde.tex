section{Befunde}

% vlt anderer Titel, wenn man genau weiß, was die These ist
%\subsection{Hauptthese}
%Bzw vlt neue Struktur zum gesamten Kapitel überlegen

Die automatisch erkannten Gebrauche von Englischen Floskeln wurden per Hand angeguckt und in Kategorien unterteilt.
Insgesamt wurden in den 1419 Artikeln 3271 unique? Wörter identifiziert.
Das heißt wenn das selbe Wort in einem Artikel öfter vorkam, wurde es hierfür nur einmal gezählt.

\begin{itemize}
  \item Die Artikel: Eine Art Performance, die auf ein Zielpublikum zugeschnitten ist.
  \item Sucht eine bestimmte Wirkung (welche?)
  \item Was hat die Tatsache, dass es sich um Code-Switching in geschriebener Sprache handelt, für eine Bedeutung?
\end{itemize}

%------------------------------------------
\subsection{Ergebnisse beschreiben}
(vlt exemplarisch; wobei es macht evtl sinn, die alle mal zu erwähnen zumindest)

Kategorien:
\begin{itemize}
  \item Named Entities: products (commercials), journals, newspapers, magazines, organizations, books, movies, tv stuff, events, places, music, shops/labels, companies, youtube-Videos, misc, US holidays
  \item English vocab which has more or less entered Spanish (presumably): (unisex, party, piercing, fitness, blazer, shock, fans, club, picnic) % ТОDO: check whether they're listed in spanish dicitonaries (wobei RAE ist relativ konservativ, auch andere instanzen angucken)
  \item Cool/slang/hip... (v.a. ``cool'')
  \item ``look'' % vlt haben die letzten 3 Kategorien was gemeinsames: hippe/internet/jugend sprache? die für flexibilität/verfügbarkeit/bla steht? (kapitalismuskritik? --> TODO was wären theoretische Texte, die sich damit befassen?)
  \item English discourse markers as part of the Spanish text  (``anyway'', ``pros and contras'', ``must'', etc.)% (gehört vlt auch dazu?) oder ist dieses "Oh, guckt ma, ich kann voll Englisch und bin deshalb voll dabei." Wie heißt auf Deutsch "upward mobility"? -- "Sozialer Aufstieg" oder "soziale Aufstiegsfähigkeit"
  \item Further TV/Media/Movies stuff
  \item Music stuff (v.a. Genres)
  \item "One night stand"/sex stuff (#72933) (und Dating)
  \item Essen:
    \begin{itemize}
      \item Kochrezepte bzw. Werbung/Diättips: auf Spanisch, die Englische Übersetzung in Klammern (damit die Menschen es im Supermarkt finden und kaufen können)
presumption: people living in the US would know (specific) foods better by there English name;
people living in the US should be able to buy stuff under their English name, since it's unlikely that the supermarket has Spanish labels
e.g. if the food is not very popular in their country of origin; (couldn't say that for baking soda probably though^^)
      \item lifestyle shit (``baby greens'', ``pavo wild'', ``gluten free'', ``nutritional yeast''..)
Organic/healthy/lifestyle bla
      \item Zeug, das spezifisch US-amerikanisch ist (oder relativ krass mit der Kultur in Verbindung gebracht wird): ``brownies'', ``cheesecake''; (was ist mit ``black licorice''??)
      \item auch lifestyle aber mehr das "immer unterwegs, schwerbeschäftigt, cool, hipster bla": ``coffeehouse'', ``dip'', ``snack'', ``brunch''
      \item Diätkram: v.a. ``light'', aber auch ``thigh gap''
    \end{itemize}
  \item Fashion stuff
    \begin{itemize}
      \item clothing
      \item hair
      \item colors
      \item makeup
      \item misc: ``look'', ``vintage'', ``glamour''
    \end{itemize}
  \item computer/technik/internet/social media/mobile phones stuff
  \item (Half translated) English idiomatic expressions/collocations
    * full: "vivimos el presente a full"
    * off: "la voz en off"
    * trans: "grasas trans" ("trans fets" in English)(does this fit here? or open a diet cathegory?)
    * go shopping: "homres tampoco les gusta ir de shopping" (hereor fashion?)
    * footsie: "jugar footsie" ("to play footsie with sb" -- mit jmd füßeln)
    * happily ever after: "casarse y vivir su happily ever after"
    * in love: "ahora esta in love con este hombre"
    * crush: " Las mujeres tenemos crush en otras chicas "(love cathegory? or half translated stuff?), (to have a crush on sb)
    * tote bag: " un bolso tote " (or fashion?)
    * living: "Living la fiesta "loca"" (not an English expression, but a pun with Ricky Martin's song?/with the Spanish "la vida loca"?)
    * seniority: "damos por sentado que como por seniority, papá y mamá deben guiarnos y enseñarnos y resolvernos la vida."  (English: "according to seniority"?)
    * time out: "date un time out para procesar esos sentimientos"  (to take time out)
    * rush: "los dientes necesitan limpieza rush !"
    * theme: "deberá ajustarse al theme de la fiesta si escoges uno" ("theme party"? half translated stuff?)
    * bye-bye: "Es hora de decir bye-bye" ("It's time to say goodbye")
    * decision makers: "son verdaderas decision makers"
    * behind the scenes: "echale un vistazo al behind the scenes"
    * "Behind the scenes :"
    * best seller: "mi libro se convirtiese en un best seller"
    * big deal: "no es a big deal!"
    * the one: "y tu, ya has encontrado a the one?"
    * the one: "el sentimiento general que debe darte the one es positivo y feliz"
    * mood swings: "como anticipar estos mood swings"
    * "baby showers, BBQ y muchos otros eventos"
    * trick or treat: "celebrar Halloween o a hacer trick or treat por tu barrio"
    * treat: "darle un bocadito (treat) cada vez que..." (not really sure whether it's ok here..)
    * road trip: "que se embarcan en un road trip con.."
    * do like treats like: "lo parecido trata lo pareci- do [ like treats like ] "
    * double check: "sindrome del double check"
    * the sky is the limit: "como dice la frase en inglés, que the sky's the limit! (el cielo es el limite)"
    * snow birds: "está el fenómeno que se llama los “snow birds”" ("snowbird" is listed in dict.cc as "jd., der den Winter im Süden der USA verbringt")
    * "los Credit Unions (cooperativas de ahorro y y crédito)" (is it really a collocation?)
    * business card: " te muestra su business card " (in Spanish: "tarjeta comercial"; not sure why English collocation chosen)
    * green card: ".. la residencia permanente o green card"
    * family room: "pero si la sala o family room es pequeña" (according to dict.cc in US: family room = living room)
    * love seat: " usando un sofá o love seat pequeño"  (love seat = small sofa for 2 persons dict.cc)
    * cat cafe: " pero no en un cat café ", "Hoy día hay cat cafés en España , Inglaterra", " El propósito del cat café no es sólo aumentar ", " La popularidad de los cat cafés es tal "  ("A cat café is a theme café whose attraction is cats that can be watched and played with. Patrons pay a cover fee, generally hourly, and thus cat cafés can be seen as a form of supervised indoor pet rental." wikipedia) (lifestyle/pet stuff?)
    * hit parade: "en todas las listas de hit parade nutricional"
    * let go: "Y libérate . Let go , como se dice en inglés "
    * "Happy party!" (whole phrases?)
    * "Mix and match , la pareja ideal"
    * "Wait , what !"
    * " Let’s read ! A leer ! " (not really a collocation... more an appeal/invocation)
    * "has oído aquello de An apple a day keeps the doctor away ( “A diario una manzana es cosa sana" )"
    * "la que se puede leer : Spanish , please !"
    * "hombres decir “te quiero” o “I love you” es algo que no les cuesta nada"
    * "detente y piensa : Yes You Can !" (gym/training context)
    * "Yes. You Can!" (salud, training)
    * "la dieta Yes You Can!" (sure it should be here? maybe diet?)
    * " recuerda : Yes You Can !" --> diet? something with "dulces" ...
  \item Direct quotes
  \item Personal descriptions (relativ wenige)
  \item Fitness stuff:
  \item cosmetic (nicht zwangsläufig um modeerscheinungen zu beschreiben)
  \item Occupation Positions : ``nanny'', ``babysitter'', ``dog walker'', ``coach de lifestyle''
  \item ?? : "dog walker", "cat cafe", aber auch ``cooler'', ``microwave'', ``juicer''


\end{itemize}



%--------------------------------------
\subsection{Interpretation}

\begin{itemize}
  \item English is the more prestigious language;
  \item the dominant (both in numbers and power) group of people in the USA speaks English
  \item serves to avoid misunderstandings/clarify (zb cooking recipes)
  \item es wird ein bestimmter Lebensstil kreiert/bedient
  \item jung, flexibel, kreativ?, gut ankommend aufm Arbeitsmarkt
\end{itemize}

\begin{comment}
  %Language Shift
Language Shift (Spanish -> English) : Speech community of a language shifts to speaking another language (source? Thomason?)
Can we talk about language shift in this context?
Who is shifting? The editors of the magazine? Or do they induce language shift?
Should that be a focus at all? Or focus more on communities of practice/style projection through language/sozioindexikalität, etc.

Ein Shift Spanisch->Englisch ist nicht nur mit Prestige zu erklären, sondern auch mit allgemeinen Anpassungs/Integrationsprozessen:
Wer in den USA Englisch kann, hat bessere Chancen einen Job zu finden, etc. na, aber das hat wohl auch mit Prestige zu tun gewissermaßen

%Language Prestige
English is the more prestigious language;
the dominant(both in numbers and power) group of people in the USA speaks English
serves to avoid misunderstandings/clarify (zb cooking recipes)
--> das kommt alles zur Interpretation!


\end{comment}

