\section{Marco teórico}

En este capitulo repasaremos brevemente nociones y teorías claves que nos ayudarán a analizar el proyecto de redes comunitarias Altermundi.

\subsection{Bienes comunes}

Elinor Ostrom, una de las científicas más reconocidas por su trabajo en este campo, define los bienes comunes (en inglés ``commons'') como bienes ``que se producen, se heredan o se transmiten en una situación de comunidad''~\autocite{Ostrom1990}.
Los bienes comunes surgen entonces como resultado de procesos comunales y su estatuto puede cambiar en cada momento, tal como cambian las relaciones en la comunidad~\autocite{Harvey2012}.
Los bienes comunes se han de diferenciar de los bienes privados, han de estar fuera de la lógica del mercado.

Según la teoría convencional, lxs consumindorxs de algunos bienes comunes están en concurrencia entre sí.
Si una persona usa un recurso particular, esto disminuye la posibilidad de usar este recurso para otras personas~\autocite[85]{Helfrich2012}
(por ejemplo si alguien se lleva dos kilos de manzanas de un árbol en el parque local, el árbol tiene dos kilo menos para todas las otras personas a que les gustaría coger manzanas).
Eso no está válido para bienes como información o conocimiento: si una persona escucha un programa de radio, esto no disminuye el programa o la experiencia para el resto de sus oyentes.

También, se suele postular que los bienes comunes son non-exclusivos~\autocite[86]{Helfrich2012}.
Eso quiere decir que todas las personas pueden gozar, o bien, tienen el derecho de gozar de ellos de manera libre y no restringida.
Aunque sea justo y lógico desde un punto de vista normativo/moral, este principio se puede violar fácilmente.
Usando los avances tecnológicos, las leyes o otros mecanismos sociales o económicos es bastante fácil (y se pone en práctica muy a menudo) limitar el uso de ciertos recursos o bienes para personas o grupos de personas específicos o excluirlas del uso completamente.
A veces se va aún más allá, creando la impresión de que este orden, esta exclusión, que son productos de procesos sociales, sean naturales e inalterables.

\subsection{Actor Network Theory}

La Teoría del Actor-Red (Actor Network Theory o ANT) se atribuye al antropólogo y filósofo francés Bruno Latour~\autocite{Latour2010}.
Es un concepto bastante abstracto y a veces difícil de captar.
En esencia se trata de un modelo de redistribución de acciones y poder.
Un actor puede ser todo o cada unx que hace algo: actores son no solamente personas, sino también organizaciones, otros seres o incluso objetos inanimados~\autocite{Venturini2010b}.
Todos los actores estan interconectados, no existen actores en aislamiento.
En suma, los actores son tales porque inter-actuan.
Lo interesante y específico es que cada actor se puede descomponer en una red suelta y cada red puede ser comprimida en un actor:
podemos hacer un tipo de liquido zooming in and out, according to nuestro foco de investigación.

La ANT está muy apta para el análisis de fenómenos sociales y controversias ya que explicita dependencias.


\subsection{Instituciones y jerarquías}

Bernhard Rieder habla en su artículo ``Institutionalizing without Institutions'' de dos distintos tipos de poder~\autocite{Rieder2012}.
Los llama ``statutory'' y ``capillary power''.
Lo que llama ``poder legal'' (``statutory power'') es el poder que viene de instituciones formales y sus regulaciones.
En contraste, el poder capilar (``capillary power'') se debe a prácticas y convenciones sociales, a normas no escritas.
Entonces, aun cuando/comoquiera que no existen instituciones en un contexto social, siempre hay prácticas sociales y dinámicas establecidas, entonces un espacio social nunca está vacio de relaciones de poder.

La científica feminista Jo Freeman ha observado ya en el año 1970, en el contexto de la segunda onda del movimiento feminista, un hecho similar:
el movimiento solía ser formado casi exclusivamente por grupos sin estructura formal.
Además, la falta de instituciones o jerarquías formales no significa que tales no existen, sino que operan de modo encubierto y por lo tanto estan aun más difíciles de controlar, desafíar y cambiar~\autocite{Freeman1970}.

Esta aversión hacía las estructuras formales y las jerarquías está muy típica para todos los proyectos de base de la izquierda.
Y aunque Freeman entiende la desconfianza, enuncia una crítica profundamente válida:
``structurelessness'' (la falta de estructura) se convierte en meta en si, también en casos donde estructura formal será necesaria o útil para lograr las metas de los grupos en cuestión.
Para Freeman esto es el caso para todos los objetivos que van más allá del intercambio de experiencias y concientización.
La estructura formal y explícita es necesaria para que todxs puedan participar, también personas en posiciones marginalizadas, con pocos recursos y que no pertenezcan al círculo interno~\autocite{Freeman1970}.

Su postura se puede resumir así: lo que cada grupo de activistas tiene que hacer es pensar en qué grado de estructura y cuál está necesario para el éxito de sus actividades.

\subsection{Ciberoptimismo}

El ciberoptimismo es otro concepto que Rieder aborda~\autocite{Rieder2012}.
Sus defensorxs/favorecedorxs/abogadxs (syn) contemplan el Internet como motor de la democratización.
Ellxs ven la tecnología de las redes como un vehículo de la decentralización que posibilita una revolución capilar y que va a acabar con las jerarquías culturales, con los privilegios, los prejuicios y la discriminación.

\begin{quotation}
We are creating a world that all may enter without privilege or prejudice accorded by race, economic power, military force, or station of birth.\\
We are creating a world where anyone, anywhere may express his or her beliefs, no matter how singular, without fear of being coerced into silence or conformity.
\end{quotation}
proclama John Perry Barlow de la Electronic Frontier Foundation~\footnote{La EFF es una NGO estadounidense que se dedica a la defensa de los derechos digitales.} en la ``Declaration of the Independence of Cyberspace~\autocite{Barlow1996}.

Los valores de esta fracción  son la decentralización, la flexibildad, la complejidad.
En su narrativa, la tecnología ``will unshackle capillary self-organization from the suffocating embrace of statutory institutions''~\autocite{Rieder2012} y el enemigo lógico es el estado con sus instituciones formales.

Sin embargo, Rieder toma una postura crítica antes esta filosofía: sobre todo critica el/desaproba del determinismo tecnológico de sus defensorxs y la postura esencialista hacia la tecnología y la democacria.
Al final, la tecnología no es ni democrática ni antidemocrática, sino se puede utilizar para cualquier fin, dependiente de las ideologías y objetivos de las personas que la empleen.
Las redes, y en particular el Internet, pueden ser tanto herramienta de la decentralización, empoderamiento y autonomía local, como impulsoras de centralización, arquitecturas jerárquicas y promulgar vigilancia.
El argumento central de Rieder es que la tecnología sóla no puede resolver problemas políticos y que, al final, hemos de buscar soluciones políticas y no intentar a ``route around them''~\autocite{Rieder2012}.
Como hemos señalado en la divisa, estados democráticos permiten que el Internet sea democrático y no al revés.

% Otros autores/aportaciones al ciberoptimismo

