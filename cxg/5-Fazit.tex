\section{Fazit}

\begin{comment}
    * Einleitung und Fazit müssen zusammenpassen.
    * sind die Erkenntnisse im Fazit aus der Arbeit ableitbar?
    * Vorgehensweise zusammenfassen

Kritische Zusammenfassung
\begin{itemize}
    \item was war in den Texten nicht so gelungen?
    \item welche Fragen sind offen geblieben?
    \item in welche Richtung kann noch weiter geforscht werden?
    \item was sind gegnerische Meinungen zum Thema?
\end{itemize}
\end{comment}


Das Ziel dieser Arbeit war, die Phänomene des Transfers und der Übergeneralisierung im Zweitspracherwerb aus einer kognitiv-linguistischen Perspektive zu beleuchten.
Dazu wurden diese zunächst in der historischen Entwicklung der Zweitspracherwerbsforschung verfolgt.
Daran anschließend wurden Transfer und Übergeneralisierung am konkreten Beispiel von \textit{contrast left dislocation} und \textit{clitic left dislocation} Konstruktionen in den Zweitsprachen Englisch und Spanisch in \cite{Valenzuela05} untersucht.
Es hat sich gezeigt, dass Merkmale, die nicht auffällig genug sind oder in der L1 auf einer anderen Weise verarbeitet werden als in der L2, schwieriger wahrgenommen werden
und es in solchen Fällen häufig zu einer Übergeneralisierung von erlernten L2-Konstruktionen kommt.
Das war in \cite{Valenzuela05} mit den CLLD-Konstruktionen der L2 Spanisch Sprechenden der Fall.
Wenn wiederum ein Merkmal relevant für eine Konstruktion in der L1 war, ist es schwieriger, dieses bei der Sprachproduktion in der L2 zu ignorieren.
Wenn dieses Merkmal im Sprachgebrauch der L2, trotz seiner Irrelevanz für die L2, berücksichtigt wird, beobachten wir einen Transfer.
In \cite{Valenzuela05} wurde das am Beispiel der Spezifität des syntaktischen Topiks und ihrer Relevanz für die Art der Topik-Konstruktion beobachtet.
Valenzuela hat gezeigt, dass Spanisch L1 Sprecherinnen dieses Merkmal häufig in ihrem Englisch L2 Sprachgebrauch berücksichtigt haben.

\begin{comment}
\subsection{Die Funde}

Die Ergebnisse aus Valenzuelas quantitativer Studie zeigen, dass die Gruppe der L2 Spanisch Sprecherinnen erfolgreich die Realisierung des Topiks mittels einer CLLD-Konstruktion erlernt hat.
Spezifische Topiks in Haupt- und Nebensätzen werden durch die Lernenden richtig mit einer CLLD-Konstruktion umgesetzt.
Allerdings haben bei fast allen Aufgaben mehr als die Hälfte der Lernenden ein nicht spezifisches Topik mit einer CLLD-Kostruktion ausgedrückt (vgl. (\ref{agua}), (\ref{cafe})),
obwohl generische Topiks durch Muttersprachlerinnen vorwiegend mit einer CLD-Konstruktion realisiert werden \cite{Valenzuela05}.
Das ist ein klares Indiz für eine Übergeneralisierung des Gebrauchs der CLLD-Konstruktion.

\begin{exe}
    \ex \label{agua} Agua, *\textbf{la} toma todas las mañanas
    \ex \label{cafe}Me parece que, café, *\textbf{lo} debería tomar menos
\end{exe}

Für die Gruppe der Spanisch Muttersprachlerinnen mit L2 Englisch kommt Valenzuela zur Schlussfolgerung,
dass sie die Einschränkungen der CLD-Konstruktionen kennen, nämlich, dass diese nur in Hauptsätzen benutzt werden können.
Jedoch stellt sie bei dieser Sprecherinnengruppe eine Tendenz zur Nutzung von CLLD-ähnlichen Konstruktionen auf Englisch in Äußerungen mit einem spezifischen Topik (vgl. Beispiele (\ref{book2}), (\ref{notes})).

\begin{exe}
    \ex \label{book2} \uline{This} book, I have *\textbf{it} read many times.
    \ex \label{notes} \uline{Those} class notes, she cannot find *\textbf{them} anywhere.
\end{exe}

Diese Erscheinung deutet sie als Transfer aus der L1 Spanisch der Probandinnen.
(vgl. ``However, the near-native group is both accepting and producing pronouns (overt resumptive elements like clitics) with the specific topics there by exhibiting L1 influence.'' \cite{Valenzuela05})

Zusammenfassend lässt sich sagen, dass in der vorliegenden Studie sowohl Übergeneralisirungen von Konstruktionsmerkmalen in der L2 als auch Transfer aus der L1 in die L2 gezeigt wurden.

\end{comment}
%Es werden 2 Ansätze vorgestellt, was (Zweit)Spracherwerb angeht: sequentieller Spracherwerb und Erwerb eines Netz von Konstruktionen

%Gegnerische Meinungen:

%\cite{Cook93} - generative Perspektive auf Zweitspracherwerb
%``A person who knows two languages has been through the acquisition process twice.
%Second Language research must explain the means by which the mind can acquire more than one grammar.''

%``One goal of second language research is to describe grammars of more than one language simultaneously existing in the same person.''

%``The importance of second language research lise not in its account of the knowledge and acquisition of the L2 in isolation, but its account of the second language present and acquired in a mind that already knows a first [...]''

%aber das sind doch voll peanuts.. ist überhaupt nicht der kernpunkt..
%\cite{Valenzuela05} Kritik:
%bisschen zu großzügige Interpretation der gemessenen Daten zugunsten von der Arbeitshypothese


