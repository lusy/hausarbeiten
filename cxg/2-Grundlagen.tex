\section{Grundlagen}

\subsection{Kognitive Linguistik}
Was ist Kognitive Linguistik?

* Der Kognitionswissenschaften unterzuordnen: es geht um Infromationswahrnehmung und -verarbeitung.
* kein abgegrenztes Sprachmodul im Gehirn. (Ziem08?)

\cite{Ziem13} ``(1) Sprache kein mentales Modul ist, (2) Sprachwissen aus dem Sprachge-
brauch entsteht und (3) grammatische Strukturen Ausdruck menschlicher Konzeptualisierungs-
leistungen sind''

In \cite{Ziem08}: ``Kognitive Modelle bilden nicht Objekte oder Ereignisse außerhalb unseres
Bewusstseins ab, sondern sind Abstraktionsprodukte, die allein aufgrund
unserer kognitiv-konstruktiven Eigenleistungen zustande kommen.''

--> die kognitive Linguistik lehnt eine Unterscheidung vom sprachlichen und Weltwissen ab

``Um spezifisch sprachliche Repräsentationen handelt es sich dann, wenn der
perzeptuelle Input sprachlicher Art ist. Solche Repräsentationen sind indivi-
duell, sofern sie mentale Entitäten bilden, und interindividuell, sofern die Art
der Modellierung selbst das Ergebnis einer kommunikativen Praxis ist..''

--> die Abstraktionen werden aufgrund von wiederkehrenden Erfahrungen gemacht.

--> gebrauchsbasierte Perspektive auf Sprache und Spracherwerb (<--> generative Grammatik)


In \cite{Ellis08}: ``The meaning of the words of a given language, and how they can be used in combination, depends on the perception and categorization of the real world around us.''

``Figure/ground segregation and perspective taking, processes of vision and attention, are mirrored in language and have systematic relations with syntactic structure.''

``In laguage production, what we express reflects which parts of an event attract our attention;
depending on how we direct our attention, we can select and highlight different aspects of the frame, thus arriving at different linguistic expressions.''

``.. language cognition cannot be separated from semantics and the rest of cognition.''

``.. denies clear boundaries between the traditional linguistic separations of syntax, lexicon, phonology, and pragmatics.''
\begin{itemize}
    \item usage based perspective

        der Sprachgebrauch bestimmt die Struktur von Sprache (und deswegen ist es komisch, dass er bei generativen Theorien und Strukturalismus eine untergeordnete Rolle bekommt)

        In \cite{Ellis08} usage based theories ``hold that structural regularities of language emerge from learners' lifetime analysis of the distributional characteristics of the language input and, thus, that the knowledge of a speaker/hearer cannot be understood as an innate grammar, but rather as a statistical ensemble of language experiences that changes slightly every time a new utterance is produced.''

        ``.. the acquisition of grammar is the piecemeal learning of many thousands of constructions and the frequency-biased abstraction of regularities within them.''

        ``frequency of patterns in the input affects acquisition''

        In \cite{Bybee06}: ``high-frequency instances of constructions undergo grammaticization processes''

        ``a usage-based theorist would make the more specific proposal that grammar
        is the cognitive organization of one’s experience with language''

        ``the study of grammaticization has played a central role in emphasizing
        the point that both grammatical meaning and grammatical form come into being through
        repeated instances of language use''

        ``use of language is lexically particular;
        certain words tend to be used in certain collocations or constructions''

        ``A second effect of token frequency (the
        CONSERVING EFFECT
        ) relates to the morpho-
        syntactic structure of a string. High-frequency sequences become more entrenched in
        their morphosyntactic structure and resist restructuring on the basis of productive pat-
        terns that might otherwise occur.  [...]
        This effect applies to syntactic sequences as well, allowing
        higher-frequency exemplars to maintain a more conservative structure''

        ``A theory based on usage, by contrast, which takes grammar
        to be the cognitive organization of language experience, can refer to general cognitive
        abilities: the importance of repetition in the entrenchment of neuromotor patterns, the
        use of similarity in categorization, and the construction of generalizations across similar
        patterns.''
\end{itemize}


\subsection{Konstruktionsgrammatik}
Alle Konstruktionsgrammatiken gehören zum Bereich der Kognitiven Linguistik (vgl Seminarhandouts)

\begin{itemize}
\item Alle sprachlichen Phänomene lassen sich innerhalb derselben Theorie erklären (es gibt keine Kernsprache und Randerscheinungen!)
\item Konstruktion - Form/Bedeutungs-Paar (Goldbergs Definition?)
    In \cite{Lakoff87}: ``Each construction will be a form-meaning pair (F,M), where F is a set of conditions on syntactic
    and phonological form and M is a set of conditions on meaning and use.''

     --> also bringt eine ``Veränderung der syntaktischen Form stets eine Veränderung der Semantik mit sich''

    Syntaktische Konstruktionen bestimmen die allgemeine, grundlegende Bedeutung einer Äußerung und nicht die Verben.

    In \cite{Goldberg95}: `` A construction is posited in the grammar if and only if something about its form, meaning or use is not strictly predictable from other aspects of the grammar, including previously established constructions.''

    In \cite{Goldberg06}: ``[A]t the same time, unpredictability is not a necessary condition for positing a stored construction.
    There is evidence from psycholinguistic processing that patterns are also stored if they are sufficiently frequent, even if they are fully regular instances of other constructions and thus predictable.
    We must recognize that patterns are stored as constructions even when they are fully predictable.''

    ``Im Gegensatz zu strukturalistischen Theorien nimmt sie jedoch nicht an, dass semantisch leere Regeln bedeutungstragende Elemente kombinieren und dass die Ableitbarkeit sich aus der Rekonstruktion der Bedeutung der Lexeme und ihrer Art der Verknüpfung sich ableiten lässt.
    Sie postuliert, dass sowohl die Bedeutung der Wörter als auch die Bedeutung größerer und abstrakterer Konstruktionseinheiten zur Bedeutung eines sprachlichen Ausdrucks beitragen.''

    In \cite{Ellis08}: ``Any construction with unique, idiosyncratic formal or functional properties must be represented independently in order to capture speakers' knowledge of their language.''

    ``Frequency of occurance may lead to independent representation of even ``regular'' constructional patterns.''
\end{itemize}

\subsection{Spracherwerb}
\begin{itemize}
    \item L1: Tomasello - Abstraktion aus konkretem Input -> Chunks -> item-gestützte Konstruktionen -> abstrakte Kategorien

        In \cite{Ellis08} ``children are picking up frequent patterns from what they hear around them, and only slowly making more abstract generalizations as the database of related utterances grows.''
    \item Unterschiede/Gemeinsamkeiten L1/L2

        In \cite{Ellis08}: ``in child language acquisition knowledge of the world and knowledge of language are developing simultaneously whereas adult SLA builds upon pre-existing conceptual knowledge;''

        ``adult learners have sophisticated formal operational means of thinking and can trea language as an object of explicit learning''

        ``typical L1 pattern of acquisition results from naturalistic exposure .... whereas classroom environments for second or foreign language teaching can distort patterns of exposure''

        ``adult SLA builds on pre-existing L1 knowledge... adults have already acquired knowledge of these cathegories and their lexical membership for L1, and this knowledge may guide creative combinations in their L2 interlanguage to variously good and bad effects.''

        ``naturalistic SLA develops in broadly the same fashion as does L1 - from formulae, through low-scope patterns, to constructions''
    \item Gebrauchbasierte/Konstruktionsgrammatische Perspektive auf dem Zweitspracherwerb
\end{itemize}

\subsubsection{Zweitspracherwerb}

Check Grundlagen in \cite{Ziem13}
