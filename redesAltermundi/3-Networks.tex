\section{Redes comunitarias}
\label{cap:redes}

Para reflexionar sobre la infraestructura técnica, hemos primeramente de entender su funcionamiento.

En su forma convencional las redes de comunicación estan, más a menudo que no, organizadas de una manera jerárquica.
Una de las arquitecturas de red más comunes es la así llamada topología estrella (Figura~\vref{fig:top-star}):
todos los nodos estan conectados a través de un nodo central, que tiene un papel clave.
Si este nodo falla, toda la red se desmantela y nadie puede comunicarse~\autocite{Mabb2014},~\autocite{Medosch2004}.

Sin embargo, existen también otras topologías/maneras de organizar una red.
Las redes comunitarias emplean frecuentamente la que se llama mesh (en malla) (Figuras~\vref{fig:top-mesh} y \vref{fig:top-full-mesh}):
cada nodo está conectado con más de un otro, en el caso ideal cada nodo está conectado directamente con cada otro (full mesh) y transmite su tráfico dentro de la red.
No obstante, la full mesh no está adecuada para redes más grandes, sobre todo inalámbricas, donde una directa línea de visión se necesita para establecer la conexión.
En este caso tenemos más bien una mesh entre vecinxs, y, a veces, enlaces punto a punto a distancias más largas.
La mesh tiene la ventaja de estar mucho más perdurable.
Todos los nodos en la mesh son iguales y si uno de ellos se quema, el resto sigue existiendo y puede comunicarse como antes.
Ningún nodo tiene este papel clave que hemos visto en la otra topología y por eso no puede ejercer control sobre la comunicación de los demás~\autocite{FiTre2015},~\autocite{Medosch2004}.

\begin{figure}[b]
\centering
\subfloat[Estrella]{\includegraphics[width=.3\columnwidth]{topology-star}\label{fig:top-star}} \quad
\subfloat[Partial mesh]{\includegraphics[width=.3\columnwidth]{topology-mesh}\label{fig:top-mesh}} \quad
\subfloat[Full mesh]{\includegraphics[width=.3\columnwidth]{topology-full-mesh}\label{fig:top-full-mesh}}
\caption[Topologies]{Topologías de la red} % The text in the square bracket is the caption for the list of figures while the text in the curly brackets is the figure caption
\label{fig:topologies}
\end{figure}

Ya hemos señalado que las redes comunitarias usan muy a menudo redes mesh.
¿Pero cuáles son sus otras características?
En general, se trata de proyectos de base locales con una exigencia política.
La idea clave es construir una infraestructura de comunicación decentralizada, creada, mantenida y poseada por la comunidad misma de lxs usuarixs.
Así, ni el estado, ni empresas grandes tienen el control sobre la red y no pueden cerrarla de manera fácil~\autocite{Medosch2004},~\autocite{FiTre2015}.
% casos extremos
Redes comunitarias son especialmente interesantes en contextos de regimenes políticos autocráticos o en casos de catástrofes naturales~\autocite[12]{Mabb2014},~\autocite{FiTre2015}.
Para los últimos está relevante sobre todo la tecnología de la mesh, que ante otras, garantiza servicio en condiciones extraordinarias.
En AlterMundi distinguen incluso entre redes libres y redes comunitarias (que son un tipo particular de redes libres)~\autocite{Piccoli2015}.
Otros proyectos usan ambos términos de manera intercambiable y se refieren con esto a redes que tengan todas las características que vamos a nombrar.
% otras características/rasgos/particularidades/atributo/cualidad
Nicolás Echaniz de AlterMundi explica en una entrevista publicada por la revista \textit{pillku} los rasgos decisivos de redes libres~\autocite{Piccoli2015}:
\begin{itemize}
  \item ``libre uso: puede ser utilizada por sus participantes para ofrecer y recibir cualquier tipo de servicio que no afecte su buen funcionamiento
  \item neutralidad: no inspecciona ni modifica los flujos de datos dentro de la red más allá de lo necesario para su operación
  \item libre interconexión: permite, de forma libre y gratuita, el flujo de datos con otras redes que respeten las mismas condiciones
  \item libre tránsito: provee a otras redes libres acceso a las redes con las que mantiene acuerdos voluntarios de libre interconexión.''
\end{itemize}

y las particularidades adicionales de las redes comunitarias:
\begin{itemize}
   \item ``propiedad colectiva: su infraestructura es propiedad de la comunidad que la despliega;
   \item gestión social: la red es gestionada por la misma comunidad;
   \item diseño accesible: la información sobre cómo funciona la red y sus componentes es pública y accesible;
   \item participación abierta: cualquiera puede extender la red, respetando su diseño y sus principios''~\autocite{Piccoli2015}.
\end{itemize}

\begin{comment}
%TODO: Vlt die aufzaehlungen oben durch die sachen hier ersetzen
características:
* infraestructura de comunicación decentralizada, creada y mantenida por la comunidad de lxs usuarixs (el estado/empresas grandes no pueden cerrarla tan facilmente) --> interesante sobre todo también para contextos autócratos (regimenes políticos antidemocráticos, vease Iran, China, Cuba, .. )
* garantizar acceso libre a información
* garantizar la libertad de expresión
* ..
* el firmware: software libre

\item infraestructura de comunicación abierta, accesible para tod@s
\item infraestructura creada y mantenida por la comunidad de l@s usuari@s

  \begin{itemize}
    \item conectar a comunidades excluidas por los proveedores convencionales de servicios Internet
    \item garantizar acceso libre a información
    \item garantizar la libertad de expreción
  \end{itemize}
\end{comment}

% motivaciones
Muchos de los proyectos de redes comunitarias, también las redes apoyadas por AlterMundi, han surgido desde la necesidad de conexión, que proveedores convencionales no querían satisfacer ya que no veían oportunidad de lucro~\autocite{Piccoli2015},~\autocite{Vaseva2016a}.
Se trata aquí sobre todo de poblaciones pequeñas, ajenas de ciudades grandes y a menudo con terreno difícil.
Así nacieron, aparte de AlterMundi, por ejemplo los proyectos guifi.net\footnote{\url{https://guifi.net/}} en Cataluña y freifunk\footnote{\url{https://freifunk.net/}} en Berlín (redes comunitarias de wifi), o VillageTelco\footnote{\url{https://villagetelco.org/about/}} en Sudáfrica y rhizomatica\footnote{\url{https://www.rhizomatica.org/}} en Oaxaca, México (redes comunitarias de telefonía celular).
No obstante, hay varias otras motivaciones para la explotación de tales redes~\autocite{Vaseva2016b}:
desde el deseo de tener una conexión al Internet más rápida, hasta el disfruto de areglar problemas técnicos y las ganas de aprender como funciona la tecnología;
desde las aspiraciones a prestigio en la comunidad local o en la comunidad hacker, hasta la identificación ideológica con el valor básico de las redes comunitarias--la obtención de una soberania comunicativa.
Varios proyectos se proponen también la misión del empoderamiento de la comunidad, el despliegue del conocimiento técnico acerca del funcionamiento de las redes~\autocite{Vaseva2016b}.
Para AlterMundi ese aspecto es de relevancia particular~\autocite{Vaseva2016a}.
