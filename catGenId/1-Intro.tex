\section{Introducció}

\epigraph{``Every translation is a misrepresentation.''}{Ilan Stavans in~\autocite{Albin2005}}

Al seu assaig ``Dancing through the minefield'' la crítica literària feminista Annette Kolodny defensa/advoca per a una pluralitat de les lectures~\cite{Kolodny1980}.
Partint d'aquesta idea, intentarem a analitzar l'obra d'El Hachmi des d'una multitud de punts de vista diferents.


\begin{comment}
Fragestellung:
??
verschiedene readings anbieten:
* interseccionalitat
* translation/language
* laurie penny

--> adrienne rich/anette kolodny in die intro nehmen
Annette Kolodny: "In my view, our purpose is not and should not be the formulation of
any single reading method or potentially procrustean set of critical procedures[...] Instead, as I see it, our
task is to initiate nothing less than a playful pluralism, responsice to the possibilities of multiple critical
schools and methods, but captive of none.."
Leitfragen/themen:
* Qui és català? Qui decideix?
* La dona entre dues cultures i dues societats: expectacions, sentiments,... | interseccionalitat
* La dona en una societat machista                                           |
* coming-of-age
* el paper de la llengua

------

Motto candidates:

“The greatest masterpiece in literature is only a dictionary out of order.”
― Jean Cocteau
VA: Jean Cocteau once quipped that even the greatest masterpieces of literature are nothing but a
dictionary out of order.
(Stavans: Dictionaries Interview)

"IS: Every translation is a misrepresentation."
(Stavans: Dictionaries Interview)
\end{comment}
