\section{Feministische Lektüren}

\subsection{Pluralität der Lektüren}

\subsection{Close Reading}

\subsection{Die Rolle des Kanons}

\begin{comment}
3. Feministische Lektüren
   * Pluralität (vlt in die Intro vorziehen)
     Annette Kolodny: "In my view, our purpose is not and should not be the formulation of any single reading method or potentially procrustean set of critical procedures[...] Instead, as I se it, our task is to initiate nothing less than a playful pluralism, responsice to the possibilities of multiple critical schools and methods, but captive of none.."

   * Close Reading
   * Kanon
    ** Was einmal im Kanon ist, ist halt im Kanon, und das "ästhetische" Wert wird nicht hinterfragt (vgl Annette Kolodny;  wir sollten uns dringendst fragen sollten, wie der Kanon zustande gekommen ist, wer das bestimmt und was drin ist und warum)
    ** wir lesen gerne und gut, das was wir gelernt haben zu lesen
    ** Annette Kolodny: "we appropriate meaning from a text according to what we need (or desire) or, in other words, according to the critical assumptions or predispositions (conscious or not) that we bring to it. And we appropriate different meanings, or report different gleanings, at different times-even from the same text-according to our changed assumptions, circumstances, and requirements."
    ** Annette Kolodny: "It can provide that, but, I must add, too often it does not. Frequently our reading habits become fixed"
   * ..
\end{comment}
