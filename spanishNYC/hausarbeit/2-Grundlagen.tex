\section{Grundlagen}

\begin{comment}
Begriffe:
* code switching
(* borrowings)
* language shift
* Sozioindexikalität
* Indexical field?
* Medientheorie?
\end{comment}

%Try to keep short, it should clear the notions, is not the main part of the paper!

Dieser Kapitel stellt kurz die Grundlagen vor, auf denen sich die spätere Analyse stützt.
Im Folgenden werden die Begriffe Code-Switching und Borrowing, Language Shift, %oder??
Sozioindexikalität und indexikalische Felder erläutert.

\subsection{Code-Switching und Borrowing}
%(and borrowings)
Der Begriff Code-Switching beschreibt die alternierende Benutzung von zwei oder mehreren Codes (Sprachen, Varietäten, Dialekten) innerhalb einer und der selben Äußerung. %cite needed vlt Pfaff \cite[]{Pfaff79}
%Thomason
%include intra/inter sentential stuff? kommt mir nur so semi-wichtig vor.aber vlt nicht schlecht
Code Switching kann sogar innerhalb derselben Sprache erfolgen, wenn Menschen sich je nach Situation verschiedener Register bedienen.

Borrowing oder Entlehnung ist die Benutzung der Elemente einer Sprache in einer anderen Sprache.
Meistens meinen wir damit Entlehnungen auf der Lexikon-Ebene, jedoch können auch Elemente anderer Systeme entlehnt werden (was durchaus seltener passiert)~\cite{Thomason03}.% \cite[vgl.][]{Thomason03}.
Wichtig dabei ist, dass Borrowings in einer Sprache A durch Muttersprachlerinnen dieser Sprache eingesetzt werden~\cite{Thomason03}. % \cite[vgl.][]{Thomason03}.
Wenn die Personen, die die Elemente aus Sprache B in Sprache A übertragen, Muttersprachlerinnen der Sprache B sind, die die Sprache A gerade erlernen, reden wir von einer ``shift-induced interference''~\cite{Thomason03}. % \cite[vgl.][]{Thomason03}.


Laut Sarah Thomason ist Code-Switching vermutlich der häufigste Weg, auf dem Entlehnungen Teil einer Sprache werden~\cite{Thomason03}. % \cite[vgl.][]{Thomason03}.
In der Praxis ist es oft schwierig zu bestimmen, ob es sich bei einem konkreten Gebrauch um ein Code-Switch oder um eine Entlehnung handelt.
Code-Switches und Borrowings können nicht scharf von einander abgegrenzt werden und der Unterschied zwischen den Beiden ist eher quantitativer als qualitativer Natur.
Wenn ein Begriff vereinzelt und im Sprachgebrauch einer einzelnen Sprecherin vorkommt, würden wir das als Code-Switching bezeichnen.
Wenn jedoch derselbe Begriff immer wieder und von mehreren Sprecherinnen gebraucht wird, reden wir von einer Entlehnung~\cite{Thomason03}. % \cite[vgl.][]{Thomason03}.

Interessant ist ferner die Motivation für das Vorkommen beider Phänomene.
Ana Zentella spricht von ``powerful socioeconomic and cultural forces'', die Borrowings in Kontaktsituationen zwischen zwei Kulturen fördern~\cite{Zentella90}. %\cite[vgl.][]{Zentella90}.
Laut der Wissenschaftlerin, folgende Faktoren ermöglichen Entlehnungen:
\begin{itemize}
  \item Begriffe, die kulturelle Phänomene beschreiben, die in der Muttersprache und der dazugehörigen Kultur in der Form nicht existieren. Beispiele hierfür wären [kei/keike/keiki] ``cake'', [\^yins/blu\^yines] ``(blue)jeans''~\cite{Zentella90} oder ``la boila'' (boiler). %\cite[vgl.][]{Zentella97} wo kommt dieses letzte bsp her? --> buch wieder ausleihen!
  \item Ähnlichkeit in der phonologischen oder morphologischen Struktur eines Wortes.
    Deshalb wird das spanische Wort ``libreria'' (Buchladen) auf einmal im Spanischen auch für Bibliothek (auf Englisch ``library'', auf Spanisch ``biblioteca'') benutzt. %wo kommt das bsp her? vermutlich aus der selben quelle da oben
  \item einer der universellen Prinzipien des Sprachwandels -- Sparsamkeit: das entlehnte Wort ist kürzer. Deshalb wird zum Beispiel ``florero'' durch ``vase'' ersetzt~\cite{Zentella90}.% \cite[vgl.][]{Zentella90}.
  \item Prestige: die Sprache oder Varietät, aus der entlehnt wird, wird als prestigeträchtiger angesehen.
  \item im konkreten Fall von Entlehnungen zwischen dem Spanischen und dem Englischen in den USA können Anglizismen als neutralisierende Begriffe zwischen den verschiedenen Varietäten des Spanischen dienen.

\end{itemize}

Zentella beobachtet ferner die Tendenz, dass Sprecherinnen der 2. Generation mehr Entlehnungen benutzen als die der 1. Generation~\cite{Zentella90}.% \cite[vgl.][]{Zentella90}.

%TODO:
%some kind of a wrap up?

\subsection{Language Shift}
%TODO: die Frage ist, werde ich in die Richtung argumentieren? wenn nicht, dann kann man evtl den Teil rausnehmen
%kann man die Floskel als ein Shift von Spanisch->Englisch begreifen?
%Warum? Warum nicht?
%Who is shifting? the editors of the magazine?or do they induce language shift?
%should this be a focus at all? or more on style/socioindexicality?

% brauch ich den kapitel überhaupt?
% Potowski schreibt auch noch zu den Faktoren, die Shift begünstigen..

Mit \textit{Language Shift} bezeichnen wir den Prozess, bei dem eine Sprache, die als Kommunikations- und Sozialisationsmittel für eine Gemeinschaft gedient hat, durch eine andere ersetzt wird~\cite{Potowski13}. % \cite[vgl.][]{Potowski13}.
Wenn die Weitergabe einer Sprache von einer Generation an der nächsten aufhört, können wir sagen, dass die Gemeinschaft den Shift vollendet hat.

Mehrere Faktoren können einen Language Shift begünstigen.
Ähnlich wie beim Code-Switching ist eine häufige Motivation hier die Prestige. %cite?
Mit der Zeit wird zu der prestigeträchigere Sprache oder Varietät übergegangen.
Die Personen werden besser angesehen, haben bessere Aussichten auf dem Arbeitsmarkt, etc.

%Fix me: bisschen chaotisch, komisch formuliert und ohne wirklich ein Ziel zu haben;
In Situationen der Migration beobachten wir häufig einen Language Shift.
In der 3. Generation Migrantinnen ist es schwierig, die Herkunftssprache weiter zu behalten, wenn keine starken Netzwerke dafür sorgen, dass sie weiter verwendet wird.
Tendenziell ist der Erwerb der Landessprache ein Zeichen für gute Integration in die neue Gesellschaft und ermöglicht einer den Zugang zu mehreren Ressourcen (Bildung, Arbeitsmarkt, Freizeitangeboten, etc.)

\subsection{Sozioindexikalität, indexikalische Felder und Stil}

Menschen sortieren alle Sachen, die sie umgeben, mehr oder weniger bewusst in Kategorien, damit sie kognitiv damit umgehen können.
%macht dieser Satz als Eröffnung Sinn? es wird nur indirekt wieder aufgegriffen. und ich hab dafür erstmal keine quelle; Sofies Welt^^

Primäre Sozioindexikalität bezeichnet die Zugehörigkeit/Zuweisung einer Person zu bestimmten sozialen/Bevölkerungsgruppen (z.B. Frau, spanisch sprachiger Herkunft, Arbeiterin). %unklar welche die Quelle ist. Zentella97??
Diese passiert nach ``objektiven Kriterien'' und wird nicht durch die Sprecherin selbst bestimmt und gesteuert.
Mit anderen Worten: ``A first-order index simply indexes membership in a population.''~\cite{Eckert08}.% \cite[vgl.][]{Eckert08}

%sekundäre Indexikalität
Jedoch ist für Penelope Eckert diese Kategoriezugehörigkeit nicht statisch (was der klassischen soziolinguistischen Interpretation entsprechen würde). %reformulate sentence.; reference zu classical sociolinguistics?
Sprecherinnen sind Akteurinnen, die entsprechend ihrer Ideologie und Ziele ihre Sprache und ihr allgemeines Auftreten aktiv selbst mitgestalten.
Die Interpretation, bzw. die bewusste Wahl und Aneignung (sprachlicher) Elemente in einem bestimmten Kontext und der damit assoziirten Werte und Eigenschaften werden wir als sekundäre Sozioindexikalität bezeichnen.
Eckert erklärt, dass die linguistischen Merkmale zu sekundären Indizien werden: sie zeigen nicht nur die Zugehörigkeit einer Person zu einer bestimmten Gruppe, sondern verwandeln sich in Indizien/Symbole für bestimmte Charaktereigenschaften, Ideologien, etc.~\cite{Eckert08}. % \cite[vgl.][]{Eckert08}.
Die sekundäre Sozioindexikalität ist somit eine ideologische Analyse (sprachlicher) Merkmale: die sozialen Kategorien und die Zugehörigkeit zu denen werden nicht von außen zugeschrieben sondern durch soziale Akteurinnen geschaffen und interpretiert.
Es ist zudem immer möglich ein Index höherer Ordnung zu erschaffen: in dem Moment, in dem ein Index n-ter Ordnung idelogisch reinterpretiert wird, entsteht ein Index der (n+1)-ter Ordnung.

%Dieser Begriff ist eng an den Begriff ``Stil'' gekoppelt und in mehreren Zusammenhänge/für mehrere Zwecke können Beide synonym verwendet werden.

%Index. Feld?
Eckert bezeichnet ein indexikalisches Feld als ``a constellation of meanings that are ideologically linked''~\cite{Eckert08}. % \cite[vgl.][]{Eckert08}.
Der Wissenschaftlerin nach haben linguistische Variablen / Merkmale keine festen Bedeutungen, sondern viel mehr damit assoziierte indexikalische Felder.
Wenn eine Variable/ein Merkmal benutzt wird, wird dann nicht eine statische Bedeutung aktiviert, sondern, je nach Kontext und Ideologie der Gesprächspartnerin, eine von mehreren möglichen Bedeutungen hervorgerufen, bzw. es wird möglicherweise auch eine komplett neue geschaffen~\cite{Eckert08}. % \cite[vgl.][]{Eckert08}.
Die indexikalischen Felder sind ferner ``fluid'': jede neu geschaffene Bedeutung kann potenziell das gesamte Feld verändern, in dem sie die ideologischen Verbindungen innerhalb des Feldes neu definiert. %vlt nach dem nächsten Satz erst?
%Eckert stellt fest, dass die Bedeutung einer und der selben Variable/linguistischen Merkmals nicht uniform durch die Gesamtbevölkerung sein kann, da diese von verschiedenen Menschen in unterschiedlichen Situationen und zu verschiedenen Zwecken benutzt werden kann~\cite{Eckert08}. % \cite[vgl.][]{Eckert08}.
%``Since the same variable will be used to make ideological moves by different people, in different situations, and to different purposes, its meaning in practice will not be uniform across the population.'' \cite[vgl.][S.466]{Eckert08}

%Stil
Der Begriff ``Stil'' ist eng am Begriff Sozioindexikalität gekoppelt und beide können in mehreren Zusammenhängen synonym verwendet werden.
Eckert erklärt, dass sie, entgegen der weit verbreiteten Ansicht, nicht die Meinung vertritt, ``Stil'' wären verschiedene Weisen, die selben Sachen zu sagen.
Diese verschiedenen Weisen gehen aus den verschiedenen Identitäten und Ideologien der Sprecherinnen hervor und bedeuten, dass die Sprecherinnen, potenziell, auch ganz verschiedene Sachen zu sagen haben können~\cite{Eckert08}. % \cite[vgl.][]{Eckert08}.
``Stil'' ist eigentlich viel breiter gefasst: dadrunter werden nicht nur die gewählten linguistischen Merkmale verstanden, sondern diese werden mit weiteren stilistischen Systemen kombiniert (z.B. Kleidung etc.)
Zugrunde der stilistischen Praxen (es sind sowohl die Produktion als auch die Interpretation von Stils gemeint) liegen Ideologien, erläutert Eckert.
Jede stilistische Entscheidung ist das Ergebnis der Interpretation der sozialen Welt und der Bedeutungen deren Elemente~\cite{Eckert08}. %\cite[vgl.][]{Eckert08}.
Die Auswahl bestimmter Merkmale für die eigene stilistische Representation hängt von der bisherigen Erfahrung mit stilistischen Systemen ab.
Je nach Erfahrung nehmen Menschen gewisse Unterschiede stärker wahr als andere.
Wenn ein Merkmal als solches wahrgenommen und mit Bedeutung ausgestattet wird, können sich Sprecherinnen entscheiden, dieses in die eigene stilistische Representation einzubauen oder eben nicht.
Die Entscheidung für oder gegen bestimmte linguistische Features verläuft in der Regel weniger bewusst als die Entscheidung für oder gegen Elemente anderer stylistischen Systemen (z.B Kleidung).
Jedoch verläuft die Bedeutungszuschreibung an Elementen und die Wahl und Kombinieren dieser Elemente ähnlich für alle stilistischen Systeme.

\begin{comment}

Grundlagen
----------

% Spanglish /Code switching!
``Powerful socioeconomic and cultural forces stimulate borrowing whenever two cultures are in contact;
the borrowing by the subordinate group's language from that of the dominant group is always
significantly greater'' [Zentella90]
Factors, which facilitate borrowings:[Zentella90]
 1) Most items reflect a cultural reality that is new or different, e.g., \[kei/keike/keiki\]
``cake'', \[bobipín\] ``bobby pin'', \[ŷins/bluŷines\] ``(blue)jeans.''
 ``la boila'' (boiler) [Zentella97]
 2) Others may be facilitated or even triggered by a similarity in the phonological and/or morphological
 structure of a word in the dominant language that makes it sound like a possible subordinate word,
 matre(s) ``mattress'' is similar to that of madre(s) ``mother(s)''
 ZB ``libreria'' (Buchladen) für Bibliothek (library)
 3) reduction in the number of syllables, e.g., \[beis\] replaces florero for ``vase''
 4) prestige
 5) Anglicisms can play the role of neutralizer between competing dialectal variants because the
 prestigious outside language acts as the lingua franca that resolves the conflict without favoring one
 group at the expense of the other.

-> Zentellas Konzept ``linguistic insecurity'': ``consider their dialect inferior to others''[Zentella07]

``but it seems that the longer a group has been in the U.S., the less they use the homeland's term''
``second generation Cubans incorporated more loans in their speech than members of the first generation.''
``Language proficiency appears to be a more relevant factor than age'' [Zentella90]

%Borrowings
lexical borrowing - both form and content are new to the borrowing language [Thomason03]

%Media
% erstmal: außen vor lassen; wenn das hier aufgemacht wird, kriegt die arbeit ne völlig neue dimension
* media: promotes wide spread/conservative varieties

* influence of the media: ``they do contribute to popular acceptance and use of some new vocabulary'' [Zentella90]
(siehe oben bei Spanglish)
-> Spielen auch für leveling eine Rolle: versuchen neutrale Varianten zu nutzen, um möglichst mehr Menschen zu erreichen;

This is especially true regarding language, because, as Trudgill notes, «the media... have almost no effect at all in
phonological or grammatical change» (1984:61), although they do contribute to popular acceptance and
use of some new vocabulary. [Zentella90]

Marshall McLuhan: hot and cold media
hot media: ``Hot media do not leave so much to be filled in by audience.
Hot media are, therefore, low in participation or completion by audience.'' [Willie79]

``Communities and individuals are bombarded constantly with messages from a multitude of sources including TV, billboards, and magazines, to name a few. These messages promote not only products, but moods, attitudes, and a sense of what is and is not important. Mass media makes possible the concept of celebrity: without the ability of movies, magazines, and news media to reach across thousands of miles, people could not become famous. In fact, only political and business leaders, as well as the few notorious outlaws, were famous in the past. Only in recent times have actors, singers, and other social elites become celebrities or “stars.” '' (http://www.cliffsnotes.com/sciences/sociology/contemporary-mass-media/the-role-and-influence-of-mass-media)


%http://www.pbs.org/speak/ahead/mediapower/media/#talk
% Jack Chambers
% “Talk the talk?” [author’s title “TV and Your Language.”]. Website “Do You Speak American?”
% McNeil - Lehrer Productions. http://www.pbs.org/speak/ahead/mediapower/media
% 2005

A final common assumption is that the media leads language changes. In fact, it belatedly reflects the changes.

The same fallacy seems to underlie the casual assumption that the mass media drives all kinds of language changes.

If the mass media can popularize words and expressions, then “presumably” it can also spread other kinds of linguistic changes. We generalize from one limited effect to a host of others.


%%%%%%%%%%%%%%%%%%%%%%%%%%%%%%%%%%%%%%%%%%%%%%%
%Belege

[Zentella07]


%Language and Identity
the defining role of language networks in identity, i.e., “identity is defined as the linguistic con-
struction of membership in one or more social groups or categories” (Kroskrity 2001: 107)
Latina/o identity in the USA is often linked to Spanish, presumed
to be the heritage language of more than 40 million people with roots in 20
Spanish-speaking Latin American nations, including Puerto Rico.
[Zentella07]

distinct ways of being Latina/o are shaped by the dominant language ideology
that equates working-class Spanish speakers with poverty and academic failure,
[Zentella07]
--> also eine Erklärung, warum sie zu %Language Shift tendieren würden

The fate and form of the languages spoken by US
Latinas/os will be determined in part by the ways in which they respond to the
construction of their linguistic identities as a group and as members of distinct
speech communities, and those responses in turn can have a significant impact
on Latina/o unity.

%Latina/o Linguistic Capital
%Racism
In
the USA, where race has been remapped from biology onto language because
public racist remarks are censored, comments about the inferiority and/or
unintelligibility of regional, class, and racial dialects of Spanish and English
substitute for abusive remarks about color, hair, lips, noses, and body parts, with
the same effect.

--> noch eine Erklärung für %Language Shift

no one expects you to be able to change your color, but you are expected to
change the way you speak radically to earn respect (Urciuoli 1996).

These attitudes are communicated in everyday conversations and promoted by
the media and public institutions, but some groups of Latinas/os are more affected
than others.

%%%%%%%%%%%%%%%%%%%%%%%%%%%%%%%%%%%%%%%%%%%%%%%%%%%%%%%%%%%%%%%
[Thomason03]

When fluent speakers of language A incorporate features into A
from another language, B, the first and most common interference features
will be non-basic lexical items, followed (if contact is sufficiently intense) by
structural features and perhaps also basic vocabulary.

%borrowing vs shift-induced interference
These two types of interference were characterized in Thomason and
Kaufman (1988) as borrowing, in which features are incorporated into A by
native (L1) speakers of A, versus shift-induced interference, in which a group of
L2 learners of A carry over features from B (their L1) into A during a process
of shift from B to A.

--> i would say, we have to do with borrowings

Linguistic factors in linguistic interference:
* universal markedness
* typological distance between source and recipient language

Social factors:
* imperfect learning
* intensity of contact
* speaker attitudes (prestige, ..)

--> weiß nicht in wie fern all das brauchbar ist, weil ich eigentlich der Meinung bin, dass in diesen Artikeln die Sprache bewusst in der Form eingesetzt wird, also dass es sich um ``speakers' attitudes'' und Sozioindexikalität handelt.
Vlt auch ein bisschen intensity of contact..

Features that
are deeply embedded in elaborate interlocking structures are in general less
likely to be borrowed, because they are less likely to fit into the recipient
language’s structures; that is why the lexicon, which for all its structure is less
highly organized than other grammatical subsystems, is borrowed first, and it
is why inflectional morphology tends to be borrowed last. But highly integrated
features may be borrowed readily between systems that are typologically very
similar;

And when contact is intense enough, there appear to be no absolute linguistic
barriers at all to borrowing

%code switching --> USE!
Code-switching, as used in this section, includes both intrasentential switching
(sometimes called code mixing) and intersentential switching


Code-switching is a (perhaps the) major route by which loanwords enter a  %loanwords
language. It surely plays a role in at least some kinds of structural borrowing
as well, although the more dramatic kinds of structural interference are prob-
ably likely to result from code alternation instead (see section 2.2 below).

I believe that is impossible in principle and in practice to draw an
absolute boundary between code-switching and borrowing. % das ist mir auch zu feine Unterscheidung, ich würde damit nichts anfangen

A code-switched word or other morpheme becomes a borrowing if
it is used more and more frequently – with or without phonological adaptation
– until it is a regular part of the recipient language, learned as such by new
learners.

% 2. Mechanisms of Intereference

Mechanisms of contact-induced change fall into four categories.
* one set of mechanisms comes into play when the implementers of a
change are bilingual in both source and recipient language
* while the other set comprises second language
acquisition strategies
* A third category, “negotiation,” seems to
overlap with both of these types
* and the fourth category has to
do with more or less conscious and deliberate decisions by speakers to imple-
ment language change

%borrowing and new word inventions
The addition by borrowing of a new word for a new
concept, like bok choy in English, must begin with a single use and continue
with increasing usage by the innovating speaker(s) and by other speakers, and
the addition by invention of a new word, like photocopy, must follow the same
path.

These parallels are hardly surprising, given that all speakers draw on a
variety of repertoires, typically characterized as styles, registers, and dialects,
in using a single language. If there is no evidence to the contrary – and certainly
no convincing evidence has been presented – then it is surely most reasonable
to assume that speakers whose repertoires include more than one language
will employ the same strategies in deploying their linguistic resources. Change
resulting from code-switching between different languages does, of course,
differ from change via diffusion from another register and dialect borrowing;
but, as noted at the start of this chapter, the differences are a matter of degree,
not of kind.

%Borrowing vs code-switching: matter of frequency
morphemes, both lexical and grammatical, would be introduced directly via
code-switching, changing from code-switches to borrowings through increas-
ingly frequent usage by code-switching speakers and then (if not all members
of the speech community engage in code-switching) by adoption by other
speakers;

%Deliberate change
types of deliberate decisions:
* two-language mixtures created by bilinguals,
apparently to serve as a symbol of a new ethnic identity; (Chicana Spanish, "Kanakendeutsch"? meine Bsp)
* of speaker groups deliberately withholding their
“real” language from outsiders, using instead a distorted and simplified for-
eigner talk version that, in some cases, forms the basis for a trade pidgin
* creation of a secret language, either by phonological distortion [...] or lexical replacement.
* the motive for making a particular change has to do with emphasizing
in-group status, or differentness from other groups;

%%%%%%%%%%%%%%%%%%%%%%%%%%%%%%%%%%%%%%%%%%%%%%%
[Milroy93]

the strength of weak ties? (ein bisschen abseits)

fokus auf gender der sprecher\_innen;
kann ich nicht wirklich benutzen, weil ich nicht weiß, wer das geschrieben hat; aber die Zielgruppe ist definitiv gegendert..

``what are usually called low-prestige varieties can be maintained over generations as flourishing vernaculars.''
also kann Class und prestige nicht die einzige Erklärung für Language Shift und Loss sein

``strong informal social ties within
communities provide the mechanisms that enable speakers to maintain non-
standard dialects, rural or urban, despite intense pressure from the standard
language through routes such as the educational system and the media.''
kann wohl eher schwierig als Argumentation für mich benutzt werden

%vlt eher nicht; ich kann schwer sagen, was für sozialen Netzwerken die Leser_innen dieses Magazins angehören
``Close-
knit networks, which will of course vary in actual levels of density and multi-
plexity, are assumed to have the capacity to maintain and even enforce local
conventions and norms - including linguistic norms. It is, after all, remarkable
that stigmatized linguistic forms and low-status vernaculars can persist over
centuries in the face of powerful national policies for diffusing and imposing
standard languages,''
spannend, aber vermutlich nicht sehr nützlich in meinem Kontext

``close-knit network structure is associated with
language maintenance; the corollary to this is that a loose-knit network struc-
ture is associated with language change. We have argued in detail elsewhere
that where ties are relatively loose-knit, communities will be susceptible to
change originating from outside localized networks. These changes are not
necessarily in the direction of the standard''

``loose-knit structures with
socially and geographically mobile middle class speakers, and close-knit ties
with lower and upper class speakers.''
-->vlt spannend

``innovators are likely to be persons weakly linked to
such a group. Susceptibility to outside influence is likely to increase in inverse
proportion to the strength of the tie with the group''
vlt nochmal ``The strength of weak ties'' lesen.. wobei bin mir nicht sicher was es
bei der Arbeit helfen würde.. man kann eh nicht zuverlässig behaupten ob es da ein 
Netzwerk gibt und wie stark genau die Connections sind

``A ‘weak tie’ model of change thus seems able to account for some
instances of variation and change which are difficult to explain in terms of the
usual unqualified assumption that linguistic change is encouraged by frequency
of contact and relatively open channels of communication''

%%%%%%%%%%%%%%%%%%%%%%%%%%%%%%%%%%%%%%%%%%%%
[Eckert08]

``Thus variation constitutes an indexical system
that embeds ideology in language and that is in turn part and parcel of the
construction of ideology.''

``the variation (and the entire linguistic) enterprise must
be integrated into a more comprehensive understanding of language as social
practice''

``Peter Trudgill
(1972), for instance, called upon the perceived toughness of working-class men
as a motive for middle-class men to adopt local working-class sound changes,
accounting for the upward spread of change''

agency and ideology: Why do speakers do what they do?

``Speakers’ agency in the use of variables has
been viewed as limited to making claims about their place in social space by
either emphasizing or downplaying their category membership through the
quantitative manipulation of markers.''
--> The old view: speakers' agency is ignored

``This generalization says nothing about the kinds of behaviors and ideologies that
underlie these patterns, what kinds of meaning people attach to the conservative
and innovative variant, who does and does not fit the pattern and why.''

%Sozioindexikalität

%1st order
``A first-order index simply indexes membership in a population''

%2nd order
``But the social evaluation of a population
is always available to become associated with the index and to be internalized
in speakers’ own dialectal variability to index specific elements of character. 2 At
that point, the linguistic form becomes a marker, a second-order index,''

``Participation in discourse involves a continual
interpretation of forms in context, an in-the-moment assigning of indexical
values to linguistic forms.''

``an nth order usage, is always available for reinterpretation – for the
acquisition of an n + 1st value.''

``fluid and ever-changing ideological field.''
``The emergence of an n + 1st indexical value is the result of an ideological move,''

%indexical field
``Variables have indexical fields rather than fixed meanings because speakers use
variables not simply to reflect or reassert their particular pre-ordained place
on the social map but to make ideological moves.''
``The use of a variable is not
simply an invocation of a pre-existing indexical value but an indexical claim
which may either invoke a pre-existing value or stake a claim to a new value.''

``The
pejoration of many English words referring to females is a perfect example of the
systematic absorption of ideology into the lexicon.''

``(1) the term was used
repeatedly in negative utterances about specific women or categories of women;
and (2) the utterances of those who said such negative things were registered
disproportionately.''

und dann kommt Wertung dazu:
``a negative evaluation of a speaker using
the apical variant might be that the speaker is inarticulate or lazy, a favorable
evaluation might be that he or she is unpretentious or easygoing''
--> interpretation depends on:
* perspective of the hearer
* style in which it is embedded

``Since the same variable will be used to make ideological moves by different
people, in different situations, and to different purposes, its meaning in practice
will not be uniform across the population.''


%%%%%%%%%%%%%%%%%%%%%%%%%%%%%%%%%%%%%%%%%%%%
%Passt nicht rein
%oder wo anders

%Communities of practice
can we say that the target reader group of these magazines is a community of practice?
--> bisschen weit hergeholt; passt meiner Meinung nach nicht zu der Definition von Community of Practice

%Language Prestige
English is the more prestigious language;
the dominant(both in numbers and power) group of people in the USA speaks English
serves to avoid misunderstandings/clarify (zb cooking recipes)
--> das kommt alles zur Interpretation!


Mögliche Änderungen:
* feature addition
* feature replacement
* feature loss

Das Problem ist:
* wir wissen nicht wer genau die Artikel schreibt;
* nur wer die Zielgruppe ist (mehr oder weniger)
* wir können schwer Theorien anwenden, die Sprecher\in-zentriert sind, da es unklar ist, wer das geschrieben hat; (und auch irrelevant)
* also eher schwer zb Gender/Social networks mit ein zu beziehen
* Aber was scheint eine Rolle zu spielen sind: Code Switching; Borrowings; Language Shift; Sozioindexikalität (Style)


Allgemeine Theorie

Linguistische Wandelprozesse kann man durch
* Interferenz
* allgemeine Demarkierung (zu erwarten in Kontaktsituationen)
erklären

Sprachliche Prozesse:
* transparenz fördernde --> Hörer orientiert --> bei extremer Mehrsprachigkeit
* Ökonomie fördernde --> Sprecher orientiert --> eher bei größeren Einsprachigkeit

[Klein92]
beim L1-Erwerb muss Kognition + soz. Identität dazu erworben werden
beim L2-Erwerb nicht; aber zieht wohl auch soz. Entwicklung mit sich
%

%%%%%%%%%%%%%%%%%%%%%%%%%%%%%%%%%%%%%
[Thomason03]


feature addition
        replacement
        deletion

--> sind meine Sätze Beispiel für einen dieser Prozesse?
kp, aber ich würd sagen, das ist nicht der Fokus und würd mich nicht explizit mit beschäftigen

(By contrast, if people who are not fluent speakers of A introduce features
into A from another language, B, the first interference features (and usually
the most common ones overall) will not be lexical, but rather phonological and
syntactic.--> curious fact, but not really useful in the context


%code-switching vs shift-induced interference
lexical items predominate in code-switching, while
phonological and syntactic features predominate in shift-induced interference.

%code alternation
%e.g. using one language at home, another at work
%also brauch ich erstmal nicht
streng situationell --> funktional

In many cases the two languages
are used by the same speaker with different interlocutors, often monolinguals;

the unconscious and involuntary incorporation of foreign structural features into
one of a bilingual’s languages fits very well with psycholinguists’ finding that
“bilinguals rarely deactivate the other language totally”

%passive familiarity
Sometimes interference features are introduced by speakers whose competence
in the source language is strictly passive – that is, a speaker may borrow a
feature from a dialect or language that she or he does not speak actively at all.
(e.g. from African American Vernacular English)

%Deliberate change
Theories of language change rarely or never allow for the possibility of deliber-
ate change, except in such trivial cases as the conscious adoption of loanwords
and even new sounds in words of foreign origin

% definitions for language attrition, language death
% not sure it has something to do with the current topic
as the “loans to loss” model, extensive
borrowing leads eventually to language loss;

the adoption of lexical
and structural features from the dominant language – that is, convergence
toward the dominant language

in general a change is more likely to occur if independ-
ent forces are pushing in the same direction.

Bewegende Macht: Sprecher\_innen-Kreativität

%%%%%%%%%%%%%%%%%%%%%%%%%%%%%%%%%%%%%%%%%%%%%%%%%%%%%%%%%%%%%
[Krefeld04]

Modelierung migratorischen Kommunikationsräume

Der Raum ist "Produkt seiner interagierenden Bewohner"

"Welt in aktueller" und "Welt in potentieller Reichweite" --> individuell, bei jeder Person anders

"Identische Räume darf man daher grundsätzlich niemals voraussetzen; vielmehr ergeben sich mehr oder weniger große gemeinschaftliche Teilräume aus dem Gebrauch bzw. aus der Entwicklung gemeinsam verfügbarer Idiome und Varietäten"

"System räumlicher Gliederungen [...] Differenzierung der Intimität und Anonymität, der Fremdheit und Vertrautheit, der sozialen Nähe und Distanz" [Schütz/Luckmann 1979, 68]

Dimensionen des kommunikativen Raums:
* Sprache
* Sprecher\_in
* Sprechen

Räumlichkeit der Sprache:
* Umgebung und umgebungsspezifische Varietäten (Arealität) (z.B. Land/Stadt, West/Ost)
* Territorium und seine Sprache(n) (Territorialität) (Dialekten? Sprachen?)

Arealität: "Bindung sprachlicher, in der Regel dialektarel Merkmale an einen spezifischen Ort"
Territorialität: "staatlich garantierte und nicht selten juristisch sanktionierte Geltung einer Staatssprache in einem administrativ scharf begrenzten Gebiet"

Räumlichkeit der Sprecherin (und der Hörerin):
* Provenienz (wo kommt sie her?)
* Mobilität (wohin/wie hat sie sich bewegt)
* Repertoire

Räumlichkeit des Sprechens:
* situative Positionalität
* Interaktion

"Relationen zwischen Provinenz und sozialer Hierarchie, d.h. die Formen sozialer Mobilität zu beachten"

Positionalität des Sprechers:
relativen Nähe/Distanz --> "Dabei ist die soziale Nähe, d.h. die Vernetzung mit unmittelbar erreichbaren Kommunikationspartnern (aus peer-groups, bzw. Familien), von der pragmatischen Nähe, der Situationalität, zu unterscheiden"

Aufgabe der Raumlinguistik: "wie die Sprecher selbst ihren (gemeinsamen) Raum konstruiert haben und wie sie ihn selbst wahrnehmen"

Glossotop: "Analog dazu bezeichne ich den Ort einer mehrsprachigen Kommunikationsgemeinschaft als 'Glossotop'"
--> "Gesamtheit der Regularitäten (...) die den lokalen Gebrauchder sprachlichen Varietäten in einer bestimmten lebensweltlichen Gruppe (zum Beispiel einer Familie, einer Nachbarschaft, einer peer-group etc.) steuern."

"wer spricht mit wem in welcher Varietät/Sprache?" --> soziale Netze ([Immacolata Tempesta 2000])

%%%%%%%%%%%%%%%%%%%%%%%%%%%%%%%%%%%%%%%%%%%%%%%%%%%%%%%%%%
[Milroy93]

--> introduces the network concept; the strength of weak ties; gender variable

Untersucht: impact of speakers' variables social class, social network and gender on linguistic variability and change
``discuss the way in which these extralinguistic variables are interrelated''

``gender difference is often prior to social class in accounting for sociolinguistic variation''

selten jemand hinterfragt welche Variablen werden bei Untersuchungen gewählt und warum
bis dato (1993): class is primary variable <- critism

``The form which linguistic gender-marking has commonly been interpreted as
taking is for women to approximate more commonly than men of similar s t a t u s
to the (so-called) prestige norm. But as Coates (1986) has pointed out, no satis-
factory explanation has emerged of why women should be more oriented than
men to a prestige norm.''
naja, mir fällt eine ziemlich lange Erklärung dazu ein.
aber die hier argumentieren gleich, dass es anders rum funktioniert...

``and the relation of gender to social class and to prestige patterns is by no means consistent
or predictable from the usual assumptions about females preferring higher social-class norms.''

``The generalization suggested is not that females
favour prestige forms, but that they create them; i.e., if females favour certain
forms, they become prestige forms. In these developments, both class and
gender are implicated, but gender is prior to class.''
eine Erklärung dafür wäre, dass Frauen den Hauptteil der Care-Arbeit leisten,
also sind sie diejenigen, die auch die Sprache weiter geben

``Some
linguists have apparently assumed that ‘social network’ is concerned only with
strong ties and is roughly synonymous with ‘peer-group’,''

``some linguists have seemed to believe that
individuals may o r may not possess a ‘social network’, when i n fact all individ-
uals ar e embedded in networks.''

%%%%%%%%%%%%%%%%%%%%%%%%%%%%%%%%%%%%%%%%%%%%%%%%%%%%%%%%%%%%%
[Eckert08]

field of potential meanings -> indexical field

``Thus variation constitutes an indexical system
that embeds ideology in language and that is in turn part and parcel of the
construction of ideology.''

``the variation (and the entire linguistic) enterprise must
be integrated into a more comprehensive understanding of language as social
practice''

Objective of the paper:
``propose an approach to the study of social meaning in variation that builds
upon linguistic-anthropological theories of indexicality, and most particularly
Michael Silverstein’s (2003) notion of indexical order.''
``the meanings
of variables are not precise or fixed but rather constitute a field of potential
meanings – an indexical field, or constellation of ideologically related meanings,''
``The field is fluid, and each new activation has the potential to change the field by
building on ideological connections''

``This very local construction of meaning in variation, the recruiting of
a vowel as part of a local ideological struggle, suggested that variation can be a
resource for the construction of meaning and an integral part of social change. But
this power of variation was lost in the large-scale survey studies of sound change
in progress in the years that followed, as social meaning came to be confused
with the demographic correlations that point to it.''

``variables index demographic categories not directly but indirectly (Silverstein
1985), through their association with qualities and stances that enter into the
construction of categories.''

schemata of speaking (Piaget 1954): we notice differences and attribute meaning to them

``Style has a similar function in everyday language,
picking out locations in the social landscape such as Valley girls, cholos,''

variables: components of styles

%Persona style
``at this level that we connect linguistic styles with other stylistic systems
such as clothing and other commoditized signs and with the kinds of ideological
constructions that speakers share and interpret''

`The connection between the sound of rhotacization and oiliness and
between oiliness and a specific persona is a particularly striking example
of iconization''

``At the same time,
this distancing process reinscribes the old types by creating a new space
in the social map in opposition to them. Meanwhile, yuppies’ adoption
of a non-Beijing feature, full tone, projects them out into transnational
space.''

%indexical field
``Variables have indexical fields rather than fixed meanings because speakers use
variables not simply to reflect or reassert their particular pre-ordained place
on the social map but to make ideological moves.''
``The use of a variable is not
simply an invocation of a pre-existing indexical value but an indexical claim
which may either invoke a pre-existing value or stake a claim to a new value.''

``The
pejoration of many English words referring to females is a perfect example of the
systematic absorption of ideology into the lexicon.''

``(1) the term was used
repeatedly in negative utterances about specific women or categories of women;
and (2) the utterances of those who said such negative things were registered
disproportionately.''

``listeners develop an impression of a speaker based on general speech
style and the content of the utterance, and interpret the particular use of (ING)
on the basis of that impression.''
``associate the velar variant with education, intelligence, and articulateness.
Central to this perception is a view of the velar form as a full form and therefore
effortful and of the apical form as a reduced form, hence a sign of lack of effort.''

und dann kommt Wertung dazu:
``a negative evaluation of a speaker using
the apical variant might be that the speaker is inarticulate or lazy, a favorable
evaluation might be that he or she is unpretentious or easygoing''
--> interpretation depends on:
* perspective of the hearer
* style in which it is embedded

``Since the same variable will be used to make ideological moves by different
people, in different situations, and to different purposes, its meaning in practice
will not be uniform across the population.''

not answered: is stylistic meaning compositional?

%%%%%%%%%%%%%%%%%%%%%%%%%%%%%%%%%%%%%%%%%%%%%%%%%%%%%%%%%%%%%%%%%%%

Media
-----
This is especially true regarding language, because, as Trudgill notes, «the media... have almost no effect at all in
phonological or grammatical change» (1984:61), although they do contribute to popular acceptance and
use of some new vocabulary. [Zentella90]

Marshall McLuhan: hot and cold media
hot media: ``Hot media do not leave so much to be filled in by audience.
Hot media are, therefore, low in participation or completion by audience.'' [Willie79]

(http://www.cliffsnotes.com/sciences/sociology/contemporary-mass-media/the-role-and-influence-of-mass-media)

``Communities and individuals are bombarded constantly with messages from a multitude of sources including TV, billboards, and magazines, to name a few. These messages promote not only products, but moods, attitudes, and a sense of what is and is not important. Mass media makes possible the concept of celebrity: without the ability of movies, magazines, and news media to reach across thousands of miles, people could not become famous. In fact, only political and business leaders, as well as the few notorious outlaws, were famous in the past. Only in recent times have actors, singers, and other social elites become celebrities or “stars.” '' (http://www.cliffsnotes.com/sciences/sociology/contemporary-mass-media/the-role-and-influence-of-mass-media)

%The limited effects theory
The limited‐effects theory argues that because people generally choose what to watch or read based on what they already believe, media exerts a negligible influence.
%Criticism
First, they claim that limited‐effects theory ignores the media's role in framing and limiting the discussion and debate of issues. How media frames the debate and what questions members of the media ask change the outcome of the discussion and the possible conclusions people may draw. Second, this theory came into existence when the availability and dominance of media was far less widespread.

%The class-dominant theory
The class‐dominant theory argues that the media reflects and projects the view of a minority elite, which controls it.
For example, owners can easily avoid or silence stories that expose unethical corporate behavior or hold corporations responsible for their actions.
The issue of sponsorship adds to this problem. Advertising dollars fund most media.
elevision networks receiving millions of dollars in advertising from companies like Nike and other textile manufacturers were slow to run stories on their news shows about possible human‐rights violations by these companies in foreign countries.

%The culturalist theory
The culturalist theory, developed in the 1980s and 1990s, combines the other two theories and claims that people interact with media to create their own meanings out of the images and messages they receive.
This theory sees audiences as playing an active rather than passive role in relation to mass media.
Therefore, culturalist theorists claim that, while a few elite in large corporations may exert significant control over what information media produces and distributes, personal perspective plays a more powerful role in how the audience members interpret those messages.

%http://www.pbs.org/speak/ahead/mediapower/media/#talk
% Jack Chambers
% “Talk the talk?” [author’s title “TV and Your Language.”]. Website “Do You Speak American?”
% McNeil - Lehrer Productions. http://www.pbs.org/speak/ahead/mediapower/media
% 2005
Second, the lasting power of words that spread via the mass media has nothing to do with the various media themselves. Punk’d lasted just two seasons (2002-03). While it lasted, its name was raised into common parlance. What are the chances a word will persist for another five to 10 years? Not good. In buzzwords as in outré attire, there is a direct relationship between the height of the craze and the decline into oblivion. Fads mark their users as members of an in-group. The faster fads spread, the more pressure there is to find a new marker. Only your mother, if she was a beatnik, thinks rimless specs are groovy. Only your grandmother, if she was a gate, thinks black horn-rims are crazy.

A final common assumption is that the media leads language changes. In fact, it belatedly reflects the changes.

The same fallacy seems to underlie the casual assumption that the mass media drives all kinds of language changes.

If the mass media can popularize words and expressions, then “presumably” it can also spread other kinds of linguistic changes. We generalize from one limited effect to a host of others.

Finally, we should note that high mobility has even greater social significance than the media explosion.

%More stuff zu Media
%\cite[vgl.][s.20]{Tomasello06}
\end{comment}
