\section{El proyecto Altermundi}

\subsection{redes comunitarias}

\begin{comment}
redes convencionales: organizadas de manera jerárquica (Telematik heranziehen?):
topología centralizada: el nodo en el centro tiene control/papel clave: si falla, toda la red se desmantela.

redes mesh: cada nodo está conectado con más de 1 otro: topología más perdurable/estable; todos los nodos son iguales;
full mesh: cada nodo está conectado directamente con cada otro: la red falla si todos los nodos fallan --> grande capacidad de recuperación

redes comunitarias: son en general redes mesh con una pretención política
características:
* infraestructura de comunicación decentralizada, creada y mantenida por la comunidad de lxs usuarixs (el estado/empresas grandes no pueden cerrarla tan facilmente) --> interesante sobre todo también para contextos autócratos (regimenes políticos antidemocráticos, vease Iran, China, Cuba, .. )
* garantizar acceso libre a información
* garantizar la libertad de expresión
* ..
* el firmware: software libre

      \item infraestructura de comunicación abierta, accesible para tod@s
      \item infraestructura creada y mantenida por la comunidad de l@s usuari@s
      \item ejemplos:
        \begin{itemize}
          \item Freifunk (Alemania)
          \item guifi.net (España/Cataluña)
          \item ninux (Italia)
          \item Funkfeuer (Austria)
        \end{itemize}

  \begin{itemize}
    \item conectar a comunidades excluidas por los proveedores convencionales de servicios Internet
    \item garantizar acceso libre a información
    \item garantizar la libertad de expreción
    \item profundizar los propios conocimientos técnicos, experimentar
    \item educar y concienciar a más gente
  \end{itemize}

anti sistemas autocratas:
"Anlässe für dieses neuerliche Interesse sind etwa die Versuche in autoritären
Staaten, das Internet komplett abzuschalten, um den Informationsaustausch
zu verhindern."[p.12][Mabb2014]

auch fuer Katastrophenfaelle relevant
"Ein weiteres Einsatzfeld sind Mesh-Netze im Katastrophenfall, besonders nach
Naturkatastrophen. So nutzten etwa nach dem Hurrikan Sandy Bürger im Brook-
lyner Viertel Red Hook ein solches, bereits vorhandenes und weiter funktionsfä-
higes Netz. Es wurde um einen Dienst erweitert, über den Bewohner Schäden ­
melden und lokale Informationen austauschen konnten. Zusammen mit der
Katastrophenschutzbehörde wurde das lokale Netz provisorisch per Satellit
ans Internet angebunden. Der dort eingesetzte technische Werkzeugkasten
des „Commotion Wireless Project“ teilt viele Komponenten mit dem deutschen
Freifunk."[p.12][Mabb2014]

bridging the digital devide!

surgen de necesidades locales:
por ejemplo en el caso de Altermundi: pequeños pueblos, poblaciones en las altas cumbres; localidades donde los proveedores convencionales no ven oportunidad de lucro y por eso no prestan servicio

se ubican en el contexto del ciberoptimismo (DEF)

reflexión crítica: está la solución fiable en gran escala: vgl papel del estado (IV Gui); última milla
\end{comment}

\subsection{Altermundi: fundación, objetivo, estructura, desarrollo histórico}
\begin{comment}
* existe en su forma actual desde 2012
* objetivo: facilitar la construcción y el mantenimiento de redes comunitarias inalámbricas
* grupo básico de activistas + redes locales autónomas
* educar y concienciar: talleres, empoderamiento, el grupo central en papel consultativo
* la comunidad decide
\end{comment}


\subsection{Análisis}
* community network projects: are to be found in the context of (2nd wave) cyber optimism: "decetralization", "distributed control", "self-governance", "non-hierarchical organization"
an attempt at a capillary revolution -> ist es erfolgreich?

\begin{comment}
  Recepción desde las Redes

% licencia de todos los videos: Creative Commons Attribution license (reuse allowed) /Creative Commons Atribucion-Compartir Obras Derivadas Igual 2.5./CC BY-SA 2.5 AR

%  LaBolsaLibre https://www.youtube.com/watch?v=x6nONJXcUQ8
* somos grupo de vecinos/amigos, que ha surgido desde el despliegue de la red, después de la semana de hackathon de construcción de la red
* nos faltan conocimientos técnicos, somos dependientes de otras redes, si un nodo se quema o la red se cae

% QuintanaLibre https://www.youtube.com/watch?YMKlvUS7B-A&v=hsPjT2R-ToQ
* proyecto grande: diferentes grados de involucración
* mucho trabajo
* cómo atraer miembros nuevos?: reuniones con gente ya involucrada, que expliquen que es lo que les gusta en la red, como es su experiencia, "como cambió su vida por la parte de la red"

% AnisacateLibre https://www.youtube.com/watch?YMKlvUS7B-A&v=iasMfOAaDpk
* 10 familias conectadas
* 1 persona responsable, empezó sola, el único nodo en el barrio; se reunió con gente de Altermundi después, intercambiando experiencia, etc.
* desde 2012-2013, después de José de la Quintana
* comparten datos, tráfico, etc.. --> para el responsable el aspecto social es clave: estar conectado con más personas, que comparten
* red comunitaria != internet gratis, sino se hace entre todosa -> importante que lxs participantes lo entiendan

% LaSerranitaLibre https://www.youtube.com/watch?YMKlvUS7B-A&v=Kv2246EDPPg
* desde fines de 2013-2014
* 2016: 15 familias conectadas?
* tenían la necesidad, los proveedores no querían prestarle servicio y conocieron a QuintanaLibre
* ir de casa a casa, algo comunitario, sin un fin económico que estaba en vista
* la mayor dificultad entre vecinos: entender que es una red libre, sacarte un poco de la lógica del mercado, algo que se hace en la comunidad
* desafío: unir La Serranita con Córdoba, para tener conección; el tereno es duro, habían de subir mucho más alto;

% NonoLibre https://www.youtube.com/watch?YMKlvUS7B-A&v=RaXZqlALRII
* 25 nodos operativos: 18 familias y resto cabañeros?
* comenzó sep? 2014
* Nono: localidad túristica; por lo tanto la Camara de Comercio fue un de los impulsores del proyecto, cuando se enteró de ello
* conectividad importante para el desarollo personal pero también para el desarollo laboral del pueblo
* proyecto previo: Nono digital (tener wifi libre en Nono), se juntó con Altermundi -> NonoLibre;
* tenían 40 nodos, pero solamente 25 operativos, ya que el terreno es difícil
* desafío más grande: montar una antena en la Pampa de Acharla?
* experiencia muy linda, gratificante: trabajar en comunidad


\end{comment}


\begin{comment}
> framing the Internet alternately as lawless, anarchic,
> free, “a world where anyone, anywhere may express his or her beliefs, no matter how singular,
> without fear of being coerced into silence or conformity” (Barlow 1996) (p.1)

* la infraestructura está prerequisito para participación

## Conclusión del texto [Rieder2012]

> If technology won’t deliver us from the conundrums of
> governance, negotiation, and struggle, we
> may as well reengage politics proper[ly].

* sobre todo problemas sociales/políticos: organización de grupos (no tanto técnicos): ¿cómo decidimos como grupo? ¿quién hace qué? ¿quién está responsable?
* soluciones políticas para problemas políticos
\end{comment}

\subsection{Mapeo: área conflictiva}

\begin{comment}
ANT: podemos hacer aquí zooms in y out!

                          * da talleres
                          * apoyo técnico
                          * Bildungsauftrag

            Altermundi <-----------------------> redes
         /   ^                                     ^
individuos   |                                     |
             |                                     |
             |                                     |
             |                                     |
             v                                     v
            estado<-------------------------> infraestructura

Altermundi:
* individuos, gente que se comunica entre si, con todos los problemas y desafíos que surgen de esto
* tiene una forma organizativa legal: asociación? fundación? (siehe IV) para poder entrar en alianzas, cooperaciones (con el estado y otras organizaciones formales, p.e. la Universidad de Córdoba)
* función auxiliar (las redes son autónomas)

Estado:
* está responsable para regulaciones que ayudan o impiden el trabajo de tales proyectos (ley telemediatica)
* la última milla
* solucionar el problema de conexión de manera completa/de gran escala (siehe papel del estado)
* apoyo financial (eher nicht, vgl Entstehungsgeschichte)

Infraestructura:
* ¿donde está?
* quien la mantiene/gestiona/instala
* quien tiene acceso?

Redes
* autónomas
* tensiones: problemas con la comunicación dentro de la comunidad
* falta de conocimientos técnicos

\end{comment}

\subsection{Cronología}

2007 reforma constitucional/de la ley telemediatica: "derecho a comunicación"

2009 ley... ??  Ley de Servicios 
de Comunicación Audiovisual No. 26.522, sancionada el 10 de octubre del 2009.  : deconcentración de la comunicación: empresas comerciales no pueden obtener más freqüencias/licencias: hay un límite de freqüencias para cada sector;
http://servicios.infoleg.gob.ar/infolegInternet/anexos/155000-159999/158649/norma.htm
Grupo Clarín se nieda a implementar la ley

y también:
"ARTICULO 97. — Destino de los fondos recaudados. La Administración Federal de Ingresos Públicos destinará los fondos recaudados de la siguiente forma:
...
f) El diez por ciento (10%) para proyectos especiales de comunicación audiovisual y apoyo a servicios de comunicación audiovisual, comunitarios, de frontera, y de los Pueblos Originarios, con especial atención a la colaboración en los proyectos de digitalización107."

2012
* El gobierno argentino conversa con codigosur con la idea de ofrecer talleres de software libre/.. por Argentina
* nace también la idea de ofrecer talleres de redes
* Altermundi se forma como proyecto desde codigosur
* el gobierno quita el apoyo financial prometido (el contacto ya no está)
* 1. hackathon (la gente ya se había comprado la técnica): QuintanaLibre en José de la Quintana en Córdoba: compartir Internet entre 2-3 familias (llega hasta 60 familias hoy)

...
* Delta Libre (1. Hackathon?)
* Quintana Libre
* Nono libre
* ..

2013
* la Corte Suprema de Justicia confirma la consitucionalidad general de la ley

2015
* talleres educativos con otros proyectos (cooperativa de mujeres en NIC?)
* Abr: LaBolsaLibre va en linea
* Dec: Macri llega al poder en Argentina --> política orientada hacia al mercado: laissez-faire capitalismo, los límites para lincencias cesan


reticulación (vernetzung)/intercambio de experiencias con otros proyectos internacionales
* Rhizomatica
* BattleMesh



2016
* principios: Macri modifica la Ley 26.522 con un decreto: " morigerar el carácter antimonopólico de la ley, beneficiando a los principales medios de comunicación del país. "
* FCC + EU: Firmware lockdown
* Nace el proyecto del Libre router

2017
* Julio: el decreto de Macri llega al Corte Suprema, que aun ha de pronunciarse hacia su legalidad
* Octubre: planificado el lanzamiento del libre router
pliegues de redes comunitarias en Argentina

% Desde la página web
Las redes aquí enumeradas utilizan tecnología desarrollada por AlterMundi y reciben soporte y asistencia de parte de la Asociación.

QuintanaLibre, Córdoba, iniciada en Marzo de 2012 
DeltaLibre, Buenos Aires, iniciada en Abril de 2012 
AnisacateLibre, Córdoba, iniciada en Septiembre de 2012 
LaSerranitaLibre, Córdoba, iniciada en Septiembre de 2013 
NonoLibre, Córdoba, Iniciada en Marzo de 2014 
BoquerónLibre, Santiago del Estero, Abril de 2015 
LaBolsaLibre, Córdoba, iniciada en Mayo de 2015

Despliegues de redes comunitarias en el exterior

Fumaça Online, Río de Janeiro, Brasil, Julio de 2015
MulukukuLibre, Nicaragua, Enero de 2016
Caimito Libre, Esmeraldas, Febrero de 2017

