\section{Marco teórico}

\subsection{Commons/bienes comunes y redes comunitarias}

\begin{comment}
## "community"

> The term “community” has played a central role in American affective politics for a long time,
> and as Cherry Schrecker (2006) argues, this thoroughly positive connotation carries, with ex-
> tremely few exceptions, through most of Anglo-Saxon sociology.

* en alemán: más ambiguo (los nacis y la "Volksgemeinschaft")
* español? ("comunidad", "comunitari@");
\end{comment}

\subsection{Actor Network Theory}

La Teoría del Actor-Red (Actor Network Theory o ANT) se atribuye al antropólogo y filósofo francés Bruno Latour~\autocite{Latour2010}.
Es un concepto bastante abrstracto y a veces difícil de comprender/captar.
En esencia se trata de un modelo de redistribución de acciones y poder.
Un actor puede ser todo o cada unx que hace algo: actores son no solamente personas, sino también organizaciones, otros seres o incluso objetos inanimados~\autocite{Venturini2010b}.
Todos los actores estan interconectados, no existen actores en aislamiento.
En suma, los actores son tales porque inter-actuan.
Cada actor se puede descomponer en una red suelta y cada red puede ser comprimida en un actor.

La ANT está muy apta para el análisis de fenómenos sociales y controversias ya que explicita dependencias.

\begin{comment}
%ANT
[Latour2010] ANT

-- ich weiß noch nicht wie hilfreich das ist, aber paar Gedanken/Kernzitate vom Text

"whenever you wish to define an entity (an agent, an actant, an actor) you have to deploy its attributes, that is, its network"
"the technology.. /digital technology .. makes networks material and explicit" (infrastruktur ist präsent und sichtbar im öffentlichen Raum) --> wem gehört die Infrastruktur? wer kontrolliert die? Wer hat (physischen) Zugang dazu?
"he expansion of digitality has
enormously increased the material dimension of networks: the more digital, the less
virtual and the more material a given activity becomes." <-- das ist das Exaktzitat

"In its simplest but also in its deepest sense, the notion of network is of use
whenever action is to be redistributed."

Eigenschaften eines Netzwerks:
* distribution
* visibility/Explizierung von dependencies

Think about modelling the case study through ANT: kind of a liquid zooming in and out
Kern der Theorie:
"any entity can be seized
either as an actor (a corpuscle) or as a network (a wave). It is in this complete
reversibility—an actor is nothing but a network, except that a network is nothing but
actors—that resides the main originality of this theory. Here again, network is the
concept that helps you redistribute and reallocate action."

"philosophers have been carried out by the verb to be and its
problem of identity and not by the verb to have and the range of its properties and
avidities. But the web is changing all of that and fast: “to have” (friends, relations,
profiles...) is quickly becoming a stronger definition of oneself than “to be."" --> Sachen kriegen Identität durch ihre Beziehung zu einander

"subversion it introduces in the notion of distance (the adjectives “close” and “far” are
made dependant on the presence of conduits, bridges, and hubs)," <-- physical network

"it dissolves entirely the individual versus society conundrum"

Generelle Kritik: "Since the information is here why not use it?" --> und spricht gar nicht ethical concerns an

%------------------------

[Venturini2010b]

"In controversies, any actor can decompose in a loose
network and any network, not matter how heterogeneous, can coagulate to function as
an actor."

"To understand how social phenomena are built it is not
enough to observe the actors alone nor is it enough to observe social networks once they are
stabilized. What should be observed are the actors-networks—that is to say, the fleeting con-
figurations where actors are renegotiating the ties of old networks and the emergence of new
networks is redefining the identity of actors."

"an actor is anything doing something. "

"there is no such thing as an isolated actor. Actors are always composed
by and components of networks."

"Actors are such because they inter-act, shaping relations"

\end{comment}

\subsection{Ciberoptimismo}

\begin{comment}
community network projects: are to be found in the context of (2nd wave) cyber optimism: "decentralization", "distributed control", "self-governance", "non-hierarchical organization"

## técnica y política

> framing the Internet alternately as lawless, anarchic,
> free, “a world where anyone, anywhere may express his or her beliefs, no matter how singular,
> without fear of being coerced into silence or conformity” (Barlow 1996) (p.1)

* la infraestructura está prerequisito para participación
\end{comment}

\subsection{Instituciones y jerarquías}

\begin{comment}
## statutory vs capillary power

formal establishments / social mechanisms

> Even if there are no institutions (as formal establishments) regulating behavior,
> there are always institutions in the sense of mechanisms,
> rules, and established dynamics.
\end{comment}

