\section{Conclusión}

Con este trabajo hemos propuesto una lectura posible de \textit{Evita: Libro de lectura para primer grado inferior}.
Partiendo de nociones teóricas como ``cultura'', ``modernidad'' y ``nación'' hemos intendado a oferir una discusión de dicha publicación.
Esta es solamente una de muchas posibles lecturas que tampoco aborda todo el contenido del libro.
En un discurso más detallado podemos profundizar ciertos aspectos, fijarnos más en los textos o dedicarnos a la discusión de las partes que hemos omitido en este análisis breve:
los capítulos denominados ``Lo que dijo Eva Perón'' que provienen de su autobiografía \textit{La Razón de mi Vida}, por ejemplo.

Lo que podemos afirmar después de las reflexiones hasta este momento es que el peronismo, sí tenía una imagen de la nación argentina particular que promovía con medios y medios diversos e intentaba a indoctrinar aún a lxs más jóvenes.

