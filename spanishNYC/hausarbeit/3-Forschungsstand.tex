\section{Verwandte Arbeiten}
\label{chap:related-works}

Das Thema der Zweisprachigkeit Spanisch-Englisch in den USA ist nicht neu.
Es existieren bereits zahlreiche Arbeiten, die sich diesem Thema widmen.
Ana Zentella und Ricardo Otheguy sind nur zwei der Forschenden, die mit ihrer Arbeit zu Zweisprachigkeit und Spracherwerb und deren politischen Bedeutungen einen erheblichen Beitrag auf dem Gebiet geleistet haben~\cite{OthZen11},~\cite{Zentella90},~\cite{Zentella97},~\cite{Zentella07}.

Der Fokus dieser Arbeit liegt jedoch anderswo.
Sie beschäftigt sich zwar auch mit dem Sprachkontakt zwischen dem Spanischen und dem Englischen, allerdings konzentriert sie sich primär auf eine sozioindexikalische Untersuchung (siehe Kapitel~\ref{chap:sozioindexicality}) des Phänomens in einem schriftlichen Medium.
In dem Sinne werden im Folgenden kurz Arbeiten vorgestellt, die eine ähnliche Analyse von Code-Switches in Medientexten versuchen.

%[Ticknor12]
Kathryn Ticknor von George Town University hat 2012 einen Blogbeitrag veröffentlicht, der sich der Analyse von Code-Switches auf den Umschlägen der \textit{Latina Magazine} im Zeitraum 1996 (das Jahr, in dem die Zeitschrift gegründet wurde) bis 2012 widmet~\cite[]{Ticknor12}.
Ähnlich wie bei \textit{Siempre Mujer}, handelt es sich bei \textit{Latina Magazine} um eine Publikation, die sich an in den USA lebenden Latinas richtet.
Allerdings, während die Texte in \textit{Siempre Mujer} vorwiegend spanisch sind, wird \textit{Latina} in englischer Sprache herausgegeben.
Dementsprechend sind die Switches, die Ticknor analysiert, spanisch (während wir in der vorliegenden Arbeit englische Switches ins Spanische betrachten).
Ticknor schaut sich 58 Umschläge aus dem bereits erwähnten Zeitraum an und schließt daraus folgendes:
\begin{itemize}
  \item Die Benutzung von Code Switches geht mit der Zeit zurück. (Sie hat einen Durchschnitt von acht Switches ins Spanische pro Umschlag für das Jahr 1998 gezählt und nur ein Switch im Durchschnitt im Jahr 2010.)
  \item Code-Switches nach grammatischer Kategorie:
    \begin{itemize}
      \item Es werden vor allem offene grammatische Kategorien wie Substantive und Adjektive geswitched;
      \item Es gibt jedoch auch Switches von Elementen geschlossener Kategorien, wenn sie in Kombination mit offenen vorkommen (``Take our test fashions \textit{del futuro}'');
      \item Seltener kommen auch Switches innerhalb von Phrasen oder vom selben Wort vor (``\textit{mami}hood'', ``\textit{mami}'s boys''). Voraussetzung hierfür ist die phonologische Ähnlichkeit der Elemente, die geswitchted werden (die spanischen \textit{mami} und \textit{madre} sind ähnlich zu den englischen \textit{mama} und \textit{mother}).
    \end{itemize}
  \item Es werden oft Bezeichner von Familienbeziehungen, besonders viel weibliche solche, geswitched (``madre'', ``hija'', ``chica'');
  \item Ferner werden weitere Begriffe geswitched, die mit einer Latina-Kultur in Verbindung gebracht werden (Essen wie ``tortilla'' oder ``flan'' ; ``tango'') und für die es keine adequate Übersetzung auf Englisch gibt;
  \item sowohl solche, die eine Latina-Identität stärken (und dabei sich aber Klischees bedienen und solche auch verfestigen): ``gordita''.
\end{itemize}

Ticknor erklärt die Switches vor allem mit der Motivation, ein Latina-Publikum aufgrund von einer gemeinsamen Identität anzusprechen.
Sie merkt jedoch weiterhin an, dass die Herausgeberinnen der Zeitschrift sich bemühen, nur solche Begriffe zu switchen, die auch einem monolingualen (englischsprachigen) Publikum zugänglich sind und dieses nicht abschrecken.


%[Lee99]

%Kürzen!
In ihrer Masterarbeit an der City University Hong Kong untersucht Lee Pui Yin Micky Code-Switching in populären Hong Kongnesischen Zeitschriften aus einer diskursanalytischen Perspektive~\cite[]{Lee99}.
Sie bespricht wie verschiedene Codes in Medientexten genutzt werden und wie Hong Kong Bürgerinnen versuchen, eine moderne Identität zu konstruieren.
Lee argumentiert, dass Code-Switches in verschiedenen Genres von Medientexten verschiedene Funktionen haben.

Obwohl die meisten beobachteten Code-Switches ins Englische sind, wird gelegentlich auch in andere Sprachen wie z.B. Französisch oder Japanisch geswitched.
Die Autorin erklärt, dass bei Code-Switches in Sprachen, die mit dem lateinischen Alphabet verschriftlicht werden, die Leserinnen gar nicht unbedingt die genaue Sprache wahrnehmen, bzw. gar nicht unbedingt erwartet wird, dass sie Französisch (oder auch Englisch) verstehen, sondern eben die Tatsache, dass geswitched wird.
Sie spricht vom ästhetischen/visuellen Wert von lateinischen Buchstaben inmitten von chinesischen Schriftzeichen und attribuiert diesen eine symbolische Bedeutung.

Zwei große Kategorien werden bei der Analyse identifiziert und genauer untersucht: Mode und Technik.
Bei beiden werden vorwiegend Namen von Produkten, Marken und Läden bzw. Firmen geswitched.
Die Autorin erklärt dies zum einen mit dem Konzept ``voice-quoting'': Sprecherinnen werden bei der Wahl der Bezeichnungen (für die oft, wenn auch nicht immer, auch eine chinesische Übersetzung existiert) von der eigenen Peergruppe, von Verkäuferinnen aber auch von Werbung (unter welchem Namen wird das Produkt vermarktet) beeinflusst.
Also alle ``zitieren'' die Namen, die sie von anderen gehört haben.
Ferner wird beobachtet, dass Beschreibungen von Kosmetikartikeln wissenschaftlichen Berichten ähneln.
Oft werden die Inhaltsstoffe von Kosmetikprodukten auf Englisch aufgelistet, was die Texte technisch/wissenschaftlich erscheinen lässt.
Ähnliches wird über Fachbegriffe aus der Technik-Kategorie berichtet.
So kommt es dazu, dass Hong Kongnesische Leserinnen Texte in lateinischen Schriftzeichen mit Wissenschaft, Objektivität und technologischen Fortschritt assoziieren, erklärt Lee.


%[Andr07]
Jannis Androutsopoulos nimmt sich das Thema Code-Switching in Medien in seinem Überblicksartikel ``Bilingualism in the mass media and on the internet'' vor~\cite[]{Andr07}.
Er betrachtet diverse Medienarten: Sendungen und gedruckte Texte in öffentlich-rechtliche, privaten kommerziellen und privaten nicht-kommerziellen Medien; Werbung; Songtexte; öffentlichen Diskurs im Internet;
und stellt fest, dass ``linguistic diversity is gaining an unprecedented visibility in the mediascapes of the late twentieth and early twenty-first century''.
Ihm zufolge weicht die Einsprachigkeit der öffentlich-rechtlichen Massenmedien, die unter anderem deren Projekt der Nationenbildung geschuldet ist, diversen (linguistischen) Ausdrucksweisen und -formen.
Eine der Erklärungen, die für diesen Wandel geliefert werden, ist die zunehmende Globalisierung.
Zudem ``[t]he global Anglo-American dominance in science, technology and entertainment is often evoked to account for the use of English in national media''.
Eine zweite Erklärung ist die Veränderung in den Umständen der Medienproduktion und -rezeption.
Mit dem Internet wird die klare Abgrenzung zwischen den Beiden immer unschärfer.
Marginalisierte Gruppen und Aktivistinnen haben immer leichter Zugang zu Medienproduktion und auch kommerzielle Anbieterinnen maßschneidern ihre Angebote für immer kleinere Zielgruppen, die sie unter anderem auch durch die sprachliche Ausdrucksweise versuchen anzusprechen.
Jedoch, meint Androutsopoulos, wurde Zweisprachigkeit in den Medien relativ wenig untersucht, mit der Ausnahme der Werbung.
Dies wäre der Tendenz geschuldet ``to view bilingualism in the media as ‘derivative’, ‘artificial’ or ‘inauthentic’'', % sehe ich auch so (also dass es nicht "frei"/"authentisch" entsteht; ich würde das 2./3. Sozioindexikalität nennen.
von der wir uns wegbewegen sollten, um den Phänomen wissenschaftlich untersuchen zu können.


%[Mahootian05]
Shahrzad Mahootian untersucht die Beziehung zwischen Code Choice, zweisprachiger Identität und Sprachwandel~\cite[]{Mahootian05}.
Sie stellt sich folgende Fragen: In wie fern sind stilistische und soziale Variablen, inklusive Identität, verantwortlich für die Wahl des sprachlichen Codes?
Was ist die Beziehung zwischen Wahl des sprachlichen Codes und Sprachwandel?
In ihrer Arbeit untersucht die Autorin die Benutzung von gemischtem Spanisch-Englisch Code in Druckmedien (genauer gesagt, betrachtet Mahootian auch das US-amerikanische Magazin \textit{Latina}) und interpretiert diese bereits als ein Zeichen von Sprachwandel.
Sie unterstreicht, dass die Codeauswahl in solchem Kontext sehr bewusst passiert, da diese in der Regel einen mehrstufigen Redigierprozess unterlaufen.
Mahootian bietet folgende Erklärungsansätze für das Codemixing in Druckmedien an:
\begin{itemize}
  \item Schaffen einer Nähe zwischen Herausgeberinnen und Leserinnen, aufgrund von einer geteilten Latina-Identität;
  \item ``linguistic necessity'' (also das genaue Wort existiert auf Englisch nicht): \textit{la musica tejana}, Bezeichnungen von Essen oder Gerichten, Titeln von spanischsprachigen Liedern oder Filmen u.s.w.;
  \item Manche Ausdrücke wurden durch Muttersprachlerinnen als ``emotional stärker'' auf Spanisch beurteilt;
  \item Code-Switching wird teilweise benutzt nur um Aufmerksamkeit zu erregen (durch die Tatsache, dass der Code wechselt);
  \item Englisch ist die ``prestigeträchtigere'' Sprache, und die Leserinnen sind auf Englisch angewiesen, um an der Wirtschaft erfolgreicher teilhaben zu können.
\end{itemize}



