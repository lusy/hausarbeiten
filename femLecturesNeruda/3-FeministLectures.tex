\section{Feministische Lektüren}

In diesem Kapitel werden kurz die Werkzeuge und Begriffe erläutert, die für die anschließende Diskussion einer Auswahl Nerudas Liebesgedichte relevant sind.

\subsection{Die Rolle des Kanons}
\label{chap:canon}

Einer dieser zentralen Begriffe ist der literarische Kanon.
Der Kanon ist das Konstrukt, mit dem wir uns am ehesten das Phänomen der nicht hinterfragt positiven Rezeption Nerudas Liebeslyrik erklären könnten.
Dieser hat eine ganz besondere Rolle:
Er hilft uns, die Kontinuitäten und Brüche, die Einflüsse und Beziehungen zwischen Werken, Autor*innen und Gattungen zu explizieren und zu ordnen~\cite{Kolodny1980}.
Und wir als Menschen versuchen immer Sachen zu ordnen, um damit kognitiv umgehen zu können~\cite{JorRus1999}.
Allerdings sollen wir dieses Modell nicht als auf ``inherenten ästhetischen Eigenschaften'' der Werke basierend wahrnehmen, sondern müssen uns dabei bewusst sein, welche Relationen durch so eine Konstruktion hervorgehoben werden und welche eben verdeckt bleiben, wie kritische feministische Wissenschaftler*innen hiweisen~\cite{Kolodny1980}.
Wie wird der Kanon überhaupt gebildet?
Welche Werke und Autor(*inn)en (in deren überwiegenden Mehrheit männlich) werden überliefert?
Wer entscheidet das?
Und nach welchen Kriterien?

Die Antworten auf diese Fragen sind intuitiv nahe liegend.
An der Entscheidung sind maßgeblich diejenigen beteiligt, die gesamt gesellschaftlich über Macht verfügen.
Diese Machtbeziehungen spiegeln sich dann auch in Wissensproduktion und Literatur(kritik) wider.
Eine Elite aus westlichen, weißen, oft konservativen älteren Männern versucht ihre priviligierte Stellung dadurch zu sichern, indem sie eine Hierarchie herstellt, einen Mechanismus, der aufzeigen sollte, was ``gute'' und was ``schlechte'' Literatur ist~\cite{North2013}.
Und dieser funktioniert natürlich aufgrund eines komplett ``objektiven'' Kriteriums: des ``inherenten'' ästhetischen Werts der Werke~\cite{Kolodny1980}.
% Objektivität, die uns immer vorgegauckelt wird bei Wissensproduktion! (Quote??)
% ``Instead, we find ourselves endlessly responding to the riposte that the overwhelmingly male presence among canonical authors was only an accident of history and never intentionally sexist-coupled with claims to the "obvious" aesthetic merit of those canonized texts.''~\cite{Kolodny1980}
Dabei bleibt verdeckt, dass die westliche (und nicht nur) Literatur überwiegend eine androzentrische Perspektive vertritt:
Im Kanon sind vorwiegend Werke von Männern zu finden, in denen Männer als handelnde Figuren im Mittelpunkt stehen, und die wiederum an Männern als Rezipienten gerichtet sind.

% elaborieren, Zusammenhang fehlt grad ein bisschen
Wir lesen gerne und gut das, was wir gelernt haben zu lesen.
Das Vergnügen ist oft ein Resultat der erfolgreichen Anwendung bestimmten interpretativen Schemata und nicht Spaß an der Lektüre per se, vor allem wenn es sich um Texte handelt, die wir sonst schwer erfassen können~\cite{Kolodny1980} (weil sie zu weit von uns liegen, sei es zeitlich, räumlich, ideologisch, etc.).
Wie das Annette Kolodny formuliert:
``Radical breaks are tiring, demanding, uncomfortable, and sometimes wholly beyond our comprehension.''~\cite{Kolodny1980}.
Das ist einer der Gründe warum die hervorgehobene Stellung von Kanon-Werken sich immer fester etabliert:
Sie werden immer häufiger gelesen und nach den bekannten Mustern interpretiert und diese Popularität wird dann für einen direkten Indikator für einen hohen literarischen Wert gehalten.
Wenn ein Werk einmal im Kanon angelangt ist, ist es halt im Kanon, und dessen Wert sowie die Art und Weise seiner Rezeption werden kaum mehr hinterfragt~\cite{Kolodny1980}.

% Call to action!
Es gilt also, diesen zeitlosen Kanon, der aufgrund ``intrinsischen ästhetischen Werten'' aufgebaut wurde, zu durchbrechen, etablierte Kriterien und Rezeptionen zu hinterfragen, %<-- vlt auch erst am Ende (zu der Pluralität nehmen)??
sich, in Anlehnung an I.A. Richards und der Cambridge Liberal School, mit Texten zu beschäftigen, um diese in Frage zu stellen, und nicht, um ihre Autorität nochmal zu beteuern~\cite{North2013}.

\begin{comment}
%todo put this somewhere in this chapter
Worüber geschrieben wird und welche Schrifte überliefert werden, prägt auch welche Themen als wichtig angesehen werden.
Was wiederum maßgeblich durch vorherrschende Machtverhältnisse geprägt wird.
\end{comment}


\subsection{\textit{Close Reading}}

Das zentrale Werkzeug, das ich bei der Lektüre von Nerudas Liebeslyrik benutzen werde, ist \textit{close reading}.
Obwohl \textit{close reading} heutzutage vor allem mit der US-amerikanischen literaturtheoretischen Schule des New Criticism assoziiert wird, wo es dafür verwendet wurde, die inherenten ästhetischen Werte und ``richtige'' Botschaft von Texten aufzuzeigen und deren Kanonzugehörigkeit zu bestätigen, sind sich auch heute viele Literaturwissenschaftler*innen einig, dass die Methode wertvoll ist, weil sie dafür benutzt werden kann, um ein Werk durch genaues Lesen kritisch zu hinterfragen und verschiedene Lektüren desselben Textes zu generieren~\cite{Gallop2007},~\cite{Beehler1988}.

In Anlehnung an der Analyse von Joseph North möchte ich die \textit{close reading} Methode in ihrem ursprünglichen Kontext instrumentalisieren, wie sie vom liberalen britischen Literaturkritiker I.A. Richards konzipiert wurde, eben um ``to use literature as a tool of aesthetic education for the improvement of people's lives''~\cite{North2013}
und nicht ind dem Sinn der (konservativen) US-amerikanischen Strömung des New Criticism~\cite{North2013}.

Es geht dabei nicht darum, nur ein Werk für sich zu betrachten und jeglichen historischen und politischen Kontext außer Acht zu lassen.
% "For him, this means shifting the emphasis away from the supposedly “objective” aesthetic or formal qualities of the work of art considered in isolation, and onto the nature of the relationship between the artwork and its most important context—its audience."(I.A. Richards)\cite{North2013}
% "gives us the very misleading impression that it is somehow, at root, a practice of autonomous or idealist aesthetics, and as such originally or even necessarily dehistoricizing or depoliticizing."
Richards setzt sich (laut North) nicht für ``l'art pour l'art'' ein, sondern möchte mit seinem \textit{close reading} vielmehr evaluieren, was der Text sagt, aber in Bezug auf die (Wechsel-)Wirkung(-en) mit den rezipierenden Leser*innen.
Er geht von einem instrumentellen Ästhetik-Begriff aus, von der Annahme, dass Ästhetik ein Mittel und nicht der Zweck ist.
Der Zweck ist, den Rezipirenden zu helfen, die eigene Weltanschauung zu ordnen.
Die lesende Person (mit ihren Erfahrungen, Wissen, Einstellungen, Weltanschuung, Gemütszustand) ist dabei der Kontext, in dem das Werk analysiert wird.
Und in dem Sinne, hier vorweggenommen:
Meine persönlichen Erfahrungen, Wissen, Werte, Gemütszustand etc. werden der Kontext für die angebotenen Lektüren von Nerudas Liebeslyrik sein.

%Weitere Merkmale von Close Reading, die uns helfen werden
Vielleicht ist die Definition Nancy Boyles' greifbarer: 
Diese unterstreicht, dass es beim \textit{close reading} darum geht, Bedeutungsschichten aufzudecken, die ein tieferes Verständnis des Textes ermöglichen werden~\cite[90]{Boyles2016}.

Auch die Professorin für Literaturwissenschaft Jane Gallop hebt das Emanzipatorische am \textit{close reading} hervor:
Es ermöglicht Studierenden (und nicht nur), nicht nur bereits vorhandenes Wissen anzuwenden, das ihnen von Lehrbüchern und Dozierenden angeboten wurde, sondern eher, in einem direkten Kontakt mit Texten, ihr eigenes Wissen zu produzieren~\cite{Gallop2007}.
Dieser Gedanke ähnelt sehr an dem von North festgehaltenen: ``The `practical' in Practical Criticism had, in Richards’s usage, meant something like `directed towards the practical end of training readers';''~\cite{North2013}.


\subsection{Pluralität der Lektüren}

Mit diesem Text möchte ich Leseweisen von Nerudas Liebeslyrik erkunden, die von der etablierten Rezeption abweichen.
Eine Pluralität der Lektüren einzuräumen ist bereits an sich eine feministische Handlung,
da eine inhärente Charakteristik des vorherrschenden patriarchalen Literaturdiskurses ist, die ``richtige'' Interpretation, die ``Essenz'' des Textes zu suchen~\cite{Kolodny1980}.

Kritische feministische Wissenschaftler*innen hingegen gehen davon aus, dass ein Text keine zugrunde liegende Wahrheit offenbart, sondern vielmehr eine Ausgangsfläche bietet, aufgrundderen mit Hilfe verschiedener Methoden und interpretative Strategien eine Mehrzahl an Bedeutungen erzeugt werden können~\cite{Beehler1988}.
Verschiedene Lektüren kommen nicht nur aufgrund unterschiedlicher interpretativen Schemata zustande, sondern auch aufgrund unterschiedlicher Erfahrungen, Gemütszustände und aktuelle Bedürfnisse der Lesenden.
Also ist nicht nur jede Lektüre von einem Werk, die durch verschiedenen Menschen getätigt wird, anders (ein ``Neu-Schreiben'' wie es Terry Eagleton nennt), sondern auch jedes Mal, wenn dieselbe Person einen Text erneut liest, entdeckt sie dadrin potenziell andere Bedeutungen~\cite{Eagleton1997},\cite{Kolodny1980}.

Annette Kolodny warnt jedoch, dass oftmals diese verschiedenen Lektüren gar nicht stattfinden, da sich unsere Lesegewohnheiten verfestigen~\cite{Kolodny1980} und wir bequemerweise ``anerkannte'' Werke lesen und deren Standardinterpretationen übernehmen (was deren ``ästhetischen Wert'' nur noch bestätigt und steigert).
% TODO warum ist das doof?
Das würde zum Beispiel auch erklären warum junge Frauen in Seminaren über Neruda ausschweifig seine Liebeslyrik als ``universell menschlich'' bezeichnen.
% Das wird ganz am Ende im Fazit erwähnt
%Es wäre spannend zu wissen, was die selbe Person heute, sechs Jahre später, in dieselben Gedichten lesen würde.
%Und wenn sich ihre Leseweise nicht geändert hat, ist das ein Zeichen dafür, dass wir mehr kritische feministische Bildung brauchen.

In Anlehnung an dieser kritischen feministischen Tradition, möchte ich bei den Lektüren folgende Fragen stellen: Wie wird der ästhetische Wert der Werke vergeben? Kommt dieser vom Text oder von den Leser*innen? 
Welchem Zweck dienen diese Urteile?
Und welche Wertesysteme helfen sie aufrecht zu erhalten?
%TODO check ob ich auf diese Fragen auf irgendeiner Weise eingehe, ansonsten rausschmeißen

Nicht zuletzt, immer wieder die selbe Sorte von Werken auf der selben Art und Weise zu lesen wird auf Dauer recht langweilig. %versuche weniger lapidär zu formulieren
Rein deswegen ist eine Pluralität der Lektüren erstrebenswert.
