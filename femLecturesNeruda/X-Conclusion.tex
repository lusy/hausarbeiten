\section{Fazit}

In dieser Arbeit habe ich exemplarisch drei Gedichte aus zwei verschiedenen Liebeslyrikbändern Nerudas—\textit{Veinte poemas de amor} und \textit{Los versos del capitán}—gelesen.
Anhand der Texte habe ich aufgezeigt, dass die etablierte Rezeption der Lyrik als Veräußerung ``universeller menschlicher Gefühle'' (oder syn) aus einer männerzentrierten Perspektive heraus passiert.
Wenn wir uns allerdings diese Texte aus einem feministischen Diskurs heraus nähern, entdecken wir darin weitere Leseweisen, die ein völlig anderes Bild zeichnen.
%Ja was denn?
Die passiven, stummen, von einem Mann abhängigen Frauenfiguren sind auf einmal kein Zeichen mehr für eine erstrebenswerte Position/Rolle innerhalb einer romantischen Liebesbeziehung, sondern für tief verwurzelten (syn!) Machtstrukturen, die wir aufgehört haben, als solche wahrzunehmen.

Literatur wird immer im Kontext dessen gelesen und interpretiert was wir kennen und gelernt haben zu lesen.
Nicht nur unsere früheren Lektüre(syn), sondern auch die von der Gesellschaft angesehenen Werte werden in unsere Lektüren reinprojiziert.
Und anders herum: Die Kunst hilft diese Normen zu verfestigen~\cite{Kolodny1980}.
Wenn junge Student*innen grafische Verherrlichung von Gewalt an Frauen als ``universell menschliches fleischliches Begehren'' bezeichnen, ist das ein Zeichen dafür, dass irgendwas fundamental schief läuft.
Um der vorherrschenden \textit{rape culture} etwas entgegen zu setzen, brauchen wir mehr kritische feministische Bildung, mehr Lektüren, die den tief verwurzelten patriarchalen Diskurs in Frage stellen.

\begin{comment}
* "als nah, natürlich wahrgenommen", die Leser*innen können sich damit identifizieren
  ** Duncan spricht von "real", "natural/beautiful expression of male/female relationships", die Menschen können sich damit identifizieren, sie sehen die Gedichte als Reflexion der gesellschaftlichen Idealen, die sie gelernt haben anzustreben
  ** Duncan:  "a sort of manual on the workings of modern romantic love and have come  to be regarded by many, as a standard agains which real life  relationships can be judged"
  ** Duncan: ".. in fact it seems extraordinarily "real" and "natural", as if it were not a literary construct at all, but, rather a reflection of real-life experiences. Readers who turn to these poems to learn "what love is supposed to be like", "what men are like", and "what women are like", ultimately receive a skewed message told from the traditional dominant male position"
\end{comment}

\begin{comment}
[Duncan1992]
"Araya notes that in Veinte poemas, ``la amada es esencialmente carne'' (165), and adds, ``siendo ella objeto pasivo del amor, no manifiesta su identidad ni dentro del libro ni fuera de él''(168).
He also notes, ``Ella no tiene voz propia en este poemario.[...]
Más que un sujeto que actúa y ama, la amada es un objeto al que se dirige la capacidad amatoria del yo'' (168).
Despite these astute observations, however, Araya continues to view the image of woman that emerges from the text as a positive one, for he believes that she is ``sometida a la ley natural'' (182) which conditions her to accept the male speaker as a superior being (159)."

Citing Araya:
``No es difícil entonces que la muchacha y la mujer que leen este libro se identifiquen con la imagen femenina que brota de él''(187)
-> "culturally accepted norm of romantic love"
\end{comment}

\begin{comment}
[Duncan1992]
In Neruda's poems, the male speaker is seen as the voice of authority; his view is rarely, if ever, questioned, since it is presented as the only view possible.
Because he duplicates dominant cultural values in his text, his voice has the ringof truth to it.
\end{comment}

\begin{comment}
    [Duncan1992]
Vlt move this to 4
"Neruda's women possess the qualities we have been taught to covet: they are beautiful, sensual, desirable, and eminently agreeable."
"In short, they strike us as being ``real'', rathre than the carefully chosen literary constructs they are."
"They want what all women have been conditione to want in the way of self-fulfillment: romantic love"
-> vgl Laurie Penny; Liv Stromquist
"They may have no identity, no voice, no sense of puprose, but, in exchange, they are promised the reward of man's eternal devotion if they agree to play their role properly."

"The message to female readers is clear: woman bears the responsibility of atttacting, nurturing, and keeping man's love alive. He is not obliged to love her and, as he often reminds her, he can fare better without her than she can without him. Without love, he is still a man, but she, without love, is nothing."
"Neruda's women are familiar images, women ``as they should be''. His men are ``real men''. Together, they function to reassure us that everything is as it has always been, and that all is well with the world. But, is it? Are men and women destined to repeat, generation after generation, the same patterns? Are they eternally locked in their gender-specific roles?"
    --> Vlt als Fazit: beauty ins not in subjugation of women; learn to view beauty (and love, vgl bell hooks) in new/different ways;
--> learn to question the familiar
\end{comment}
