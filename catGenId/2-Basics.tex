\section{Fonaments teòrics}

\subsection{La immigració a Catalunya al llarg dels anys}

``L'últim patriarca'' és, entre altres coses, la història d'una família d'immigrants.
Per això, repassarem breument el desenvolupament de la immigració a Catalunya a l'útlim segle.
Hem de notar que Catalunya sempre ha sigut una terra d'acollida d'immigrants.
Només al segle XX hi havia multiples onades d'immigració, començant amb la primera als anys '20, passant per la segona, als anys '50-'60, la tercera a les '80 i arribant a l'actualitat [quelle!],~\autocite{TarPaGa2013}.
Totes aquestes onades tenen trets diferents -- més prominent, per exemple, l'origen de la nova població o les raons per immigrar, però totes tenen també similituds.
Una d'elles és la recepció dels i de les recentment arribats.
Com assenyala Fransesc Candel al seu llibre ``Els altres catalans'', cada onada està rebuda per una mena de desconfiança, prejuicis i franca hostilitat~\autocite[17]{Candel1965}.
La població vella (syn) té por del desconegut i de l'incert i sovint reacciona amb discursos i comportaments racistes:
``\ldots heu vingut a menjar-vos el pa dels catalans\ldots''~\autocite[17]{Candel1965},
``Aquests nou[sic!] catalans no coneixen Catalunya: la seva tradició, la seva història, el seu art, la seva cultura, la seva literatura, els seus costums, el seu folklore\ldots''~\autocite[17]{Candel1965},
``Parlen un català gruixut, groller i vulgar\ldots Desconeixen la gramàtica catalana. No saben llegir en aquest idioma. Escriure'l, encara menys.''~\autocite[18]{Candel1965}

La immigració de la família Driouch descrita a la novel·la té com a fons un fet/periode històric:
les diverses onades migratòries del nord d'Àfrica cap Espanya que es produexien a partir de la dècada de les '80~\autocite{TarPaGa2013}.
Segons aquest mateix informe, a l'any 2013 els marroquins són amb 20.4\% el més gran grup d'immigrants a Catalunya.
Els autors confirmen:
``El col·lectiu procedent del Marroc ha estat el més important des dels inicis de la immigració estrangera recent a Catalunya, tant pel nombre com per la seva distribució en el territori.''~\autocite{TarPaGa2013}

La mateixa autora de ``L'útlim patriarca'' va néixer a Marroc, quan el seu pare ja vivia a Catalunya, i va immigrar a Vic amb la resta de la seva família quan tenia vuit anys~\autocite{Vidal2012}.
Llavors, una lectura de la novel·la com obra amb elements autobiogràfics (llevat del grad de proximitat entre ficció i realitat) ocorre de manera automàtica.

Com s'adapta la gent?


\begin{comment}

La població estrangera a Catalunya (APUNTS de prospectiva territorial, numero 2)

--> discuteix la població estrangera i el seu desenvolupament en el periode de 2000-2013
analitza la distribució territorial

"a partir de la dècada de 1980 es produexien diverses onades migratòries del nord d'Àfrica"
--> ich glaube, mein Roman faellt hier rein; nach meinen Berechnungen sind sie Anfang der 90ern nach Catalunya gekommen

"A Catalunya [...] El primer contigent són els africans (27.5\%), la gran majoria marroquins (20.4\% del total d'immigrants a Catalunya)."

"El col·lectiu procedent del Marroc ha estat el més important des dels inicis de la immigració estrangera recent a Catalunya, tant pel nombre com per la seva distribució en el territori."

"El marroquins, segon grup de la província, presenta una migració més masculina en els grups de més edat (majors de 35 anys), però equilibrada en les edat inferiors. Destaca especialment l'important volum de població infantil: el 14,6\% dels marroquins tenen menys de 5 anys."

relevància per a mi:
trets/dades històrics serveixen com a base per la història

------------------------

[Candel1965]

Els altres catalans

"On hi ha feina, hi són ells" (p.17)
"A la terra dels seus pares no hi ha feina" (p.17)
"Se senten conquistats per Catalunya; no del tot, és clar [...] si els diuen murcians o gallecs s'enfaden; si els diuen catalans, no" (p.17)
"Davant la persistent i discutida qüestió: "...heu vingut a menjar-vos el pa dels catalans..." es posen furiosos, naturalmen! i aleshores, només aleshores, malparlen de Catalunya. A ells no els el regala ningú, el pa; ells se'l suen;" (p.17)

"I de vegades ells mateixos, amb la mateixa espasa flamígera esmentada de "heu vingut a menjar-vos el pa.." etcètera, ataquen els darrers immigrants que denigren Catalunya. Punt." (p.17)

""Aquests nou[sic!] catalans no coneixen Catalunya: la seva tradició, la seva història, el seu art, la seva cultura, la seva literatura, els seus costums, el seu folklore... Tampoc en aquest aspece no coneicen la regió d'on procedeixen. Però jo pregutno: quants catalans hi ha que coneguin tot això que he dit? Del poble, cap. De les classes privilegiades, alguns. Però tot això ja és qüestió de cultura"" (p.17)

""Aquests nou[sic!] catalans parles català, el van aprendre sense adonar-se'n. Molts el parles de manera natural i quotidiana, perquè sí, i altres per afany de sentir-se catalans de debò"" (p.17)

""Parlen un català gruixut, groller i vulgar[...] Desconeixen la gramàtica catalana. No saben llegir en aquest idioma. Escriure'l, encara menys. Però no n'hi ha per a escandaltzar-se'n. Infinitat de catalans d'origen, que s'expressen en català, que parlen en català i que viuen en català, llegiexen en castellà i escriuen les seves cartes en castellà."" (p.18)

""Tota aquesta gent no s'adona de la seva aclimatació. "Són" catalans fins a cert punt. "No" són catalans, també fins a cert punt. No és una qüestió d'honor ni de principis."" (p.19)

"Aquells immigrants murcians no havien arribat a Catalunya com a colonitzadors. Tampoc, o molt febement, com a invasors o peoners. Havien vingut a treballar i a menjar, senzillament, perquè a la seva terra es morien de gana [...] Avui dia, tots s'han integrat, i alguns, ultrapassant o sobrepujant aquesta integració, s'han tornat furibunds catalanistes. [...] Aquesta esperiència pot demostrar o permetre d'esperar que amb els immigrants d'ara passarà el mateixo poc més o menys." (p.32)

"el que passa és que no acabem de decidir-nos a anomenar catalans els qui han nascut aquí de pares de fora" (p.34)

\end{comment}

\subsection{Interseccionalitat}
En l'any 1989 la professora de dret i advogada de drets civils Kimberle Crenshaw fa servir el terme ``interseccionalitat'' per primera vedaga~\autocite{Crenshaw1989}.
Amb aquesta noció l'acadèmica vol descriure l'opressió multiple que sofreixen persones que pertanyen a la vegada a diversos grups marginalitzats.
Ella critica que fins aquest moment els discursos acadèmics, legals i activistes analitzen la discriminació al llarg d'un únic eix,
tractant l'opressions de gènere i racistes com dos fenòmens mútuament excloents,
concentrant-se en els membres privilegiats de cada grup i d'aquesta manera amaguen la multidimensionalitat d'experiències de les persones més vulnerables, com per exemple les dones negres.
Segons Crenshaw, la interseccionalitat denomina un tipus de discriminació que consta de més que la suma de les seves parts.
Això vol dir que les dones negres per exemple experimenten una discriminació molt particular que és diferent de la combinació d'experiències de les persones negres i les dones en general.

\subsection{Cultura, llengua, traducció i identitat}
Com afirma el fundador de les ciències postcolonials Edward Said en el seu llibre ``Culture and Imperialism'', avui dia cap persona pot ser definida amb una sola etiqueta
i identitats (atribuïdes o seleccionades per a una mateixa) com ``dona'', o ``musulmà'', o ``americà'' serveixen només per a una primera orientació~\autocite{Vidal2012}.
Ell planteja també la idea que tothom construeix la seva pròpia identitat cultural i ètnica.
Ja no tenim cultures i identitats ``pures'' i ``estables''~\autocite{Vidal2012}.
Però encara, amb la mateixa força, l'altre, el diferent, l'estrany està percebut com una amenaça per l'ordre establert i per això es rep amb por i prejuicis.


Un aspecte molt important/crucial de la nostra identitat i de la nostra cultura (llevat de que entenem sota aquests dos termes) és/són la/les llengüa/es a través de les quals ens comuniquem.
A més, a través del llenguatge les cultures diferents es poden entendre~\autocite{Vidal2012}.


En el seu treball sobre l'obra de Najat El Hachmi María Carmen África Vidal Claramonte destaca la traducció com l'element bàsic de ``L'últim patriarca''.
Explica que la traducció té lloc a nivels multiples: podem llegir no només una traducció entre dues llengües, sinó també entre dues cultures i fins i tot entre dues generacions~\autocite{Vidal2012}.
Exemple!

\begin{comment}
[Vidal2012]

Quintessenz:
Language, translation between languages, between cultures, between generations

quote Edward Said (Culture and Imperialism, 1993):
"No one today is purely one thing. Labels like Indian, or woman, or Muslim, or American are no more than starting-points"
"Imperialism consolidated the mixture of cultures and identities on a global scale."
"Yet just as human beings make their own history, they also make their cultures and ethnic identities."
"Survival in fact is about the connections between things..."
"But this also means not trying to rule others, not trying to classify them or put them in hierarchies, above all, not constantly reiterating how «our» culture or country is number one (or not number one, for that matter)."
--> "describen a la perfección los miedos, los prejuicios, los estereotipos, las clasificaciones, las jerarquías, las tradiciones, las fronteras, las limitaciones, tan presentes en el mundo occidental contemporáneo, aparentemente global, libre, democrático y tolerante."

"El Extraño nos amenaza de forma indirecta"

quote Bauman, Zygmunt (2002 [1999]). La cultura como praxis.:
"empañe la codiciada claridad de la división entre ellos y nosotros"
"ya que tiende a desafiar todas las distinciones que soportan el mundo inteligible."
--> die etablierte Weltordnung wird in Frage gestellt

"la idea de una cultura coherente y «pura», el sueño de una identidad estable, pertenece al pasado"

quote Simon, Sherry (1999). «Translating and Intercultural Creation in the Contact Zone: Border Writing in Quebec». En: Bassnett, Susan y H. Trivedi (eds.), Postcolonial Translation: 58-74.
"The great migrations of post-colonialism have produced a new socio-demographic situation: all Western nations now have increasingly mixed populations. The ease and rapidity of global communication have created an international mass culture,"
"so the idea of culture as a set of unchanging and coherent values, behaviours or attitudes, has given way to the idea of culture as negotiation, symbolic competition or «performance»."

lenguaje!!
"Y es que el lenguaje es el medio a través del cual las culturas tienen que llegar a entenderse. Tal vez por eso la literatura de los últimos años refleja, cada vez con más fuerza, la situación híbrida que caracteriza al ser humano, la idea de que vivimos en las intersecciones de historias, experiencias, lenguas y traducciones:"


quote Chambers, Ian (1994). Migrancy, Culture, Identity. Londres y Nueva York: Routledge:
"This drama, rarely freely chosen, is also the drama of the stranger. Cut off from the homelands of traditions, experiencing a constantly challenged identity, the stranger is perpetually required to make herself at home in an interminable discussion between a scattered historical inheritance and a heterogeneous present."

"Sin embargo, en este artículo me quiero centrar en una autora que me llama especialmente la atención porque, por un lado, plantea la cuestión de la diferencia a partir del lenguaje y, por otro, porque en ese planteamiento de la diferencia aparece explícitamente la cuestión de la traducción como un elemento fundamental."
--> lenguaje y traducción

biografía:
"Najat El Hachmi es una escritora que nació en Marruecos cuando su padre ya
había emigrado a Cataluña."
"se traslada a los ocho años a vivir a Vic."

"Para ella, hay muchos tipos de discriminación: la del racista que golpea de frente, pero también la del paternalista, que dice cosas como que, aceptando al inmigrante, nuestra cultura se enriquece."
"El Hachmi llama «pornografía étnica»,"

quote Interview El Hachmi:
"Los pornógrafos de la etnicidad acentúan rasgos de ti que en tu país encontrarías ridículos."
"El inmigrante no quiere pertenecer a una asociación de inmigrantes, sino a una de vecinos..."
"Cada mujer lleva el velo por motivos diferentes. Y no se
puede pretender salvar, de manera paternalista, a las pobres moritas del yugo de
sus maridos.
-El yugo existe.
—Como en otras culturas."
"El problema es que la mayoría tiene permiso de residencia sin permiso de trabajo. La ley de extranjería las condena a la clandestinidad laboral y eso hace que dependan del marido. Les corta la vía de emancipación."

"Se puede estar o no de acuerdo con alguna de estas afirmaciones, pero de lo que no hay duda es de que lo que esta autora plantea es la vida como traducción."

"al abordar en una lengua que no es la materna, le permite plantearse en profundidad la experiencia de ser el Otro, o, más complicado, la Otra,"

Sobre L'últim patriarca:
"le han permitido convertir en uno solo los fragmentos de los diferentes mundos que la han acompañado desde siempre."

"L’últim patriarca es un ejemplo a la vez conmovedor, tierno y cruel de que la literatura no puede ser nunca un refugio, sino una vía para poner de manifiesto las relaciones entre lenguaje y poder y para hacernos reflexionar sobre las fronteras físicas y metafóricas"

"El Hachmi nos cuenta la historia de un inmigrante marroquí, padre déspota y marido maltratador, celoso, inmisericorde, promiscuo y machista hasta la médula."

la hija es una traductora

"Es la vida como traducción, la traducción como reescritura entre culturas, la (im)posibilidad de traducir determinadas situaciones porque nunca, en ninguna lengua, significarán lo mismo que en la original."

quote Deleuze, Gilles y Guattari, Felix ([1975] 1986). Kafka: Toward a Minor Literature. Trans. Dana Polan. Minneapolis y Londres: University of Minnesota Press.
"How many people today live in a language that is not their own? Or no longer, or not yet, even know their own and now poorly the major language that they are forced to serve? This is the problem of immigrants, and especially of their children, the problem of minorities, the problem of a minor literature,"

"Efectivamente, tal vez hoy más que nunca necesitemos la traducción, porque son muchas las personas que no viven en su propia lengua."

"La última frase, «Nos casamos cuando yo tenía dieciocho años», es una traducción de una traducción. Es el resumen del paso de una cultura a otra, y no sólo eso, de una generación a otra."

quote El Hachmi, Najat (2004). Jo també sóc catalana. Barcelona: Columna.
"En definitiva: has de fer els malabarismes que calguin per anar decent per a l’estètica nord-africana i alhora no semblar una pobra noia reprimida davant dels originaris de Catalunya."

"El Hachmi plantea a la mujer como un campo para la reescritura, como una verdadera traducción y como el espacio para la traducción (¿de otros?)."

"En un mundo como el actual en el que lo que prevalece es el conflicto, el
comportamiento de cada uno de nosotros, y tal vez todavía más de cada una de
nosotras, está guiado por las historias que nos hemos ido construyendo, por la
relación que entablamos entre las palabras y las cosas,"

Como advierte Mona Baker, «Catego-
ries... do not exist outside the narrative within which they are constituted.
Moreover, the process of (narrative) categorization is far from disinterested, even
in the most abstract and apparently “objective” of sciences, such as statistics»
(Baker 2006: 10). Baker, Mona (ed.) (2006). Translation and Conflict. Londres: Routledge.

"que hablan de mujeres que de un día para otro se ven transportadas a un espacio
social y cultural muy diferente «and find themselves having to negotiate a major
conflict between their personal narratives and those in circulation in their new
environment», lo cual «could lead to significant trauma» (Baker 2006: 31)."

quote baker:
"...a concrete personal story told in one language cannot necessarily be retold or
translated into another language unproblematically."

"Fins que la mare es va cansar de tot allò i va dir, aquest any, les notes les vaig a
buscar jo, fins i tot les teves... Jo feia de traductora, com sempre. La mare deia
digues-li que és una mala puta i que deixi estar el meu marit d’una vegada, i jo
somreia i deia la mare diu que com que és ella qui passa tant de temps amb els
fills, que és millor que sigui qui et vingui a buscar les notes i, a més, que ja tenia
moltes ganes de conèixer-te. Doncs jo preferiria parlar amb el teu pare directa-
ment, que és una mica estrany que tu tradueixis l’informe a la teva mare, no et
sembla? Ja t’agradaria, ja, que hagués vingut ell, deia la mare sense haver esperat
la meva traducció, malparida, no et molestes ni a dissimular-ho. Diu que el pare té
molta feina i no li anava bé de venir, però que ella ja es refia de mi. Notable,
excel·lent, notable, excel·lent, mostra interès, tot allò no tenia traducció i jo deia
res, que diu que tot ha anat bé. Només un bé de gimnàstica i li aniria bé de fer
alguna activitat fora de l’escola, sobretot anglès, que aquí no en fem i ella té faci-
litat. La mare va dir val, val, i volia dir que ni pensar-ho només perquè era ella qui
ho havia proposat. (El Hachmi 2008: 263)"
"De nuevo, la vida cotidiana como traducción. La madre es aquí doblemente
subalterna: con respecto al esposo y fuera del espacio privado, en el público, por
ser diferente."
"Gayatri Spivak publi-
có en 1988 en el que denuncia que la subalterna, no es que no pueda hablar, sino
que no se la escucha, que su discurso no está validado por las instituciones."

quote Spivak in Segarra
"cualquier intento de ayudar a los subalternos tropieza con problemas éticos impo-
sibles de soslayar: la tendencia a considerarlos como una masa homogénea, en
lugar de fijarse en su singularidad heterogénea y, en especial, la intención benevo-
lente de querer hablar por ellos, lo cual significa un acto de apropiación, y no con
ellos. (Segarra 2006)"
Segarra, Marta (2006). «Más allá del poscolonialismo. Contra la subalternidad», La Van-
guardia, 1-3-2006.

«All speaking, even seemingly the most
immediate, entails a distanced decipherment by another, which is, at best,
an interception.» (Spivak 1999: 309).

"El lenguaje se utiliza en este
ejemplo como un instrumento de poder que ya no se basa en un ideal armónico
de comunidad en el que prevalece el principio de cooperación."

"Las identidades que aparecen en L’últim patriarca son sujetos que desde
luego no son puros sino que están en permanente traducción"

"Son identidades
nunca esencialistas (Spivak 1993: 4), nunca unificadas (Hall 1996: 3-4) sino
cada vez más caleidoscópicas y múltiples,"

"Son más «identity as becoming» que «identity as being» (Hall 1993b: 394)."

"Y el mundo de El Hachmi no es sino un
palimpsesto, por eso también lo son sus libros, tablillas en la que se aprecian hue-
llas de escrituras/culturas anteriores y en las que leemos la última, que se nos apa-
rece más perceptible. Sin embargo, las huellas persisten."

---------

[Albin2005]
On Dictionaries

"VA: Indeed, Henri Meschonnic argues that "[Les] Dictionnaires [...] sont donc à merveille les lieux
où lire entre lignes, où reconnaître, plus facilement qu'ailleurs, les conflits, les masquages des
conflits, les clichés qui font l'album de famille d'une culture" [Dictionaries, [...] are the best examples
of texts that one should read between the lines, where the conflicts, the hidden and ignored
oppositions, the clichés that make up the family album of a culture can be detected more easily than
anywhere else]."

"VA: In the early 19 th century, Constantin François de Chasseboeuf, Comte de Volney, said that the
first book of a nation is a dictionary of its language,"

"IS: Lexicons aren't only reductivistic. They are also outright xenophobic. Still, they serve a purpose:
to define a people's universe."

prescriptivism vs descriptivism

"VA: Regarding "love," you mentioned in Dictionary Days that Acadians, Caldeans, Phoenicians,
Sumerians, Babylonians, Egyptians, Normans, Toltecs, Vikings, and Quechuas didn't have a word for
it. Knowing that images are an important part of how you see the world, what would you have done
had you been born speaking Latin, that according to linguists doesn't have a lexeme for gray or
brown, or born to that of the Dani of New Guinea, whose only color words are for black and white, or
speaking a 4-color language like Hanunóo that has words only for black, white, green and red?

IS: The limits of our language are the limits of our worldview. "

"VA: Living in two or more cultures, two or more languages, produces some rifts and upheavals; it
requires a constant rearranging of schemata."

"IS: Furthermore, translation always involves wonderment and surprise: what is the speaker really saying?
Is there a way to convey the message in my own language? Is it possible to avoid becoming a
falsifier? The answer to the last question, obviously, is no. Every translation is a misrepresentation."
\end{comment}

\subsection{Patriarcat}
L'última noció teòrica que definirem abans de començar l'anàlisi de l'obra de Najat El Hachmi és el patriarcat.
Com senyala la periodista i activista Laurie Penny, aquest terme és refereix al comandament/regiment d'una elit d'homes sobre la resta de la societat.
Els homes que no disposen de poder polític, tenen al menys la satisfacció de manar els membres de la seva família com a compensació per la falta de control sobre la resta de les seves vides~\autocite[69-70]{Penny2014}.


\begin{comment}
[Penny2014]

"How are men supposed to cope with this loss of power in a society that still insists that the only way to be a man is to grab as much power as possible, to be rich, to be capable of extreme violence, to dominate other men physically and to dominate women sexually and emotionally? The received wisdom is that they're not supposed to cope. Without power over others, particularly over women, men are supposed to crumble, to lash out, to collapse in an extravagant welter of identity implosion that leaves a suspicious mess on the carpet." (p.64)

"'Patriarchy' does not mean 'the rule of men'. It means 'the rule of fathers' - literally, the rule of powerful heads of household over everybody else in society. Men further down the social chain were expected to be content with having power over women in order to make up for their lack of control over the rest of their lives.
[...]
Most individual men do not rule very much, and they never have. Most individual men don't have a lot of power, and now the small amount of social and sexual superiority they held over women is being questioned." (p.69-70)

"There are two big secrets about 'traditional masculine power' that mainstream culture does not want us to discuss, and it is imperative that we discuss them honestly [...]
The first big secret is this: most men have never really been powerful. Throughout human history, the vast majority of men have had almost no structural power, except over women and children. In fact, the power over women and children - technical and physical dominance within the sphere of one's own home - has been the sop offered to men who had almost no power outside of it." (p.75)

"Thus, a poor man working a job he hated could once expect to feel, at the very least, superior to his wife and children, to be master of his home even if he was treated like a slave outside it." (p.76)
\end{comment}
