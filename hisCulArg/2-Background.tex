\section{Marco histórico}

situar el periodo histórico:

* en el Mundo? - finales de la Segunda Guerra Mundial; principios de la Guerra Fría; división en los bloques capitalista y comunista
  * no sé si es relevante, pero: Argentina también acojió a una multitud de los nazis alemanes después de la Segunda Guerra.

* en América Latina

* que pasa an Argentina?
  * primer Peronismo!:1946-1955 --> las dos primeras presidencias de Perón
    * la clase trabajadora como base del gobierno y poder de Perón; discurso basado en clase, no en "raza";
      * aumenta sus derechos, sindicalismo
    * intención de unir más seguidorxs;
    * "descamisados", "cabecitas negras"
    * pero todavía un discurso hegemonial; la diversidad étnica y cultural no está reflejada de manera relevante; la herencia afroargentina está completamente ignorada; si está reflejada la gente no está en posiciones activa, no está representada como actorxs;
    * signos propios; no se puede presionar/apretar en el continuo tradicional de derecha-izquierda; todavía hoy muy polarizante
      * se habla de una "tercera posición"
    * pero el aparato propagandista parece ser inspirado por el Zeitgeist (vgl propaganda comunista)
    * papel de la iglesia? --> discurso inequívocamente católico; apoyandose en valores católicos; hispanista;
    * amplió los derechos de la ciudadanía
      * se consiguió el voto femenino : 1947
    * control estatal de la economía y los sectores más importantes: ferrocarriles, energía, bancos,..
      * se promueve la industria.. (vgl página sobre deudas en el Libro)

    * creación de Fundación Eva Perón (1948): facilitó ayudas estatales sociales, sobre todo para mujeres, niñxs y ancianxs

* la figura histórica de Eva Perón:
  * migrante del interior, de la clase trabajadora (por eso tiene tanta fuerza reunidora, la gente se identifica con ella)



