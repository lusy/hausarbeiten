\section{Fazit}

% Umgangsstrategie: Re-writing?
% subversive Strategien, um Probleme/Absurditäten, die sich in der Gesellschaft verfestigt haben aufzuzeigen
% diversion bzw diversifizierung: für andere Bilder, Diskurse, Modelle sorgen
\begin{comment}
  [Rich1972]
  ``We need to know the writing of the past, and know it differently than we have ever known it; not to pass on
  a tradition but to break its hold over us.''
\end{comment}
\begin{comment}
1. Intro
  * Ziel von Feministischen Lektüren: zugrunde liegende Machtstrukturen in Werken und deren Rezeption aufzudecken
    ** androzentrische Perspektive der Literatur:
       *** Männer in Mittelpunkt (als Figuren)
       *** von Männern gemacht
       *** an Männer gerichtet
  * Wie erreicht? Durch eine Pluralität der Lektüren und Close Reading
ohne das ouevre Nerudas nicht als ganzes in Frage stellen
exemplarische Lektüren: Poetisierung bestimmter Heteronormativen Perspektiven
gehört historisiert; nicht als zeitlos darzustellen

* durch Sprache (und Literatur dann) bestimmte identitäre Zuschreibung geprägt: Sprache formt unser Gehirn, Gedanken, Weltanschauung, Denkweise, Kathegorien --> im Kopf behalten wenn man mit "Lektüren" umgeht

Worüber geschrieben wird und welche Schrifte überliefert werden, prägt auch welche Themen als wichtig angesehen werden.
\end{comment}
