\section{Einleitung}

\begin{comment}
* ca 3/4 Seiten
* Untersuchungsgegenstand
* Erkenntnisinteresse
* Forschungsstand
* Vorgehensweise: also Section 2 macht blabla, Section 3 blablup, ....
* Ergebnisse können/sollen angedeuten werden
\end{comment}

Die kognitive Linguistik und die Konstruktionsgrammatik haben wesentlich zu der Weiterentwicklung der Spracherwerbsforschung beigetragen.
%Die Untersuchung des Spracherwerbs im theoretischen Rahmen der kognitiven Linguistik hat bedeutende Erkenntnisse erzielt.
Alle Theorien, die unter den Namen subsumiert werden, nehmen an, dass Sprache nicht angeboren, sondern auf Basis von sprachlichem Input stückweise erlernt wird.
Sie gehen von allgemeinen sozio-kognitiven Fähigkeiten der Menschen aus, die es ihnen ermöglichen, Sprache zu erlernen.

Mit dem Überbegriff ``Spracherwerb'' werden viele unterschiedliche Entwicklungsschemata bezeichnet, die Menschen beim Erwerben von Sprache(n) durchlaufen.
Beispiele hierfür sind der frühe monolinguale Erstspracherwerb, der bilinguale Erstspracherwerb, der frühe Zweitspracherwerb, der Zweitspracherwerb im Erwachsenenalter, der Schriftspracherwerb usw.
Bis dato gibt es etliche Untersuchungen, die sich dem frühen Erstspracherwerb aus kognitiv linguistischer/konstruktionsgrammatischer Perspektive widmen.
Nicht so zahlreich sind die in der Konstruktionsgrammatik verankerten Studien des Zweitspracherwerbs.
Diese nehmen an, dass sich Erst- und Zweitspracherwerb durch die selben kognitiven Prozesse erklären lassen, was manche zu den Schlussfolgerung veranlässt, dass beide Szenarien grob die selben Lernphasen durchlaufen.

Die vorliegende Arbeit konzentriert sich auf dem (erwachsenen) Zweitspracherwerb (L2-Erwerb).
Es werden dabei Phänomene wie Transfer aus der Erstsprache (L1) und Übergeneralisierung bestimmter L2-Konstruktionen beobachtet.
Diese lassen sich in einem kognitiv linguistischen/konstruktionsgrammatischen theoretischen Rahmen erklären.
Beispielhaft wird das auf eine Studie von Elena Valenzuela \cite{Valenzuela05} über die \textit{contrast left dislocation} (CLD) und \textit{clitic left dislocation} (CLLD) Konstruktionen im Spanischen und Englischen demonstriert.

Im Weiteren ist diese Arbeit wie folgt aufgebaut:
Kapitel 2 erläutert die theoretischen Grundlagen -- es wird eine kurze Übersicht über ausschlaggebende Annahmen der kognitiven Linguistik, der Konstruktionsgrammatik und der Erst- und Zweitspracherwerbsforschung im Rahmen dieser Theorien gegeben.
Kapitel 3 beschreibt die historische Entwicklung der Spracherwerbsforschung mit Schwerpunkt Wechselwirkungen (Übergeneralisierung und Transfer) zwischen L1 und L2 aus einer konstruktionsgrammatischen gebrauchsgestützten Perspektive.
In Kapitel 4 werden die Phänomene der Übergeneralisierung und Transfer auf dem Beispiel von CLD und CLLD-Konstruktionen im Spanischen und Englischen untersucht.
Die Ergebnisse werden dann in Kapitel 5 zusammengefasst. % Ausblick + Kritik!
