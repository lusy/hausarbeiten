\section{Llengua i traducció. Lectura llingüística}

Una lectura llingüística del llibre és especialment interessant i possble en nivels multiples.

%Primer, el llibre està escrit en un llenguatge molt particular./
Primer, es nota directament el llenguatge particular en el que està escrit el llibre.
Tenim tota la història explicada des de la perspectiva de la filla i encara una instància narrativa auctorial.
Aquesta decisió ens permet sentir-nos molt propers al personatge de la filla i, al mateix temps, tenir la sensació que llegim tot el que succeeix, sense filtres.
L'estil directe al llarg del llibre no està marcat de manera gràfica en ningún lloc; el discurs s'assembla més a un corrent de consciència, un fet que de nou afirma la proximitat amb la filla.
És curiós també, que el nom de la filla no es menciona ni una sola vegada, un fet que crea atmosfera de universalitat.

%naming things here?

A més a més, a través del llenguatge les cultures diferents es poden entendre, que segons María Carmen África Vidal Claramonte és crucial, ja que avui dia més que mai gran quantitats de persones no viuen en la seva llengua materna.~\autocite{Vidal2012}.
O, més específicament, viuen en més que una llengua i això produeix constantement conflictes i trastorns i fa necessari la reorganització de les esquemes que organitzan la seva vida~\autocite{Albin2005}.

Exemple!

%Deleuze i Guattari, citats per Vidal Claramonte, posen una pregunta interessant:
%``How many people today live in a language that is not their own? Or no longer, or not yet, even know their own and now poorly the major language that they are forced to serve?''~\autocite{Vidal2012}
%i elaboren que aquest problema sigui molt propi dels i de les immigrants i els/les seves filles.

%traducció
En el seu treball sobre l'obra de Najat El Hachmi Vidal Claramonte destaca la traducció com l'element bàsic de ``L'últim patriarca''.
Explica que la traducció té lloc a nivels multiples: podem llegir no només una traducció entre dues llengües, sinó també entre dues cultures i fins i tot entre dues generacions~\autocite{Vidal2012}.
Tenim per això exemples diversos:
la filla qui tradueix els preus dels aliments a la seva mare, no només canviant la llengüa, sinó també convertint les pesetes en duros;
la filla qui ajuda a la seva mare embarassada de parlar amb la ginecòloga:
\begin{quote}
``Hi havia coses que no sabia passar d'un idioma a l'altre, que no volia passar d'un idioma a l'altre. Contiuanva sense entendre per què tantes dones per tot el món m'explicaven coses d'aquelles a mi. Quina va ser l'última vegada que li va venir la regla a la mare? I jo ja sabia què era allò de la regla, però no n'havia parlat mai amb ella. La primera vegada que li va venir? [\ldots] La primera vegada que va tenir relacions sexuals? Déu, Déu, volia fugir corrents de tot allò, jo no les vull saber, totes aquestes coses, i encara menys traduir-les a un idioma on no hi havia cap paraula que jo conegués per a relacions sexuals que no fossin paraulotes [\ldots] au, va, pregunta-li-ho. La mare em mirava i deia què, què t'ha preguntat, i jo hauria volgut desaparèixer així, de cop, i que elles mateixes s'entenguessin. No podia dir follar, no [\ldots] Vaig intentar de buscar un eufemisme. Quants anys tenies quan vas dormir amb el pare per primera vegada? I no la vaig mirar als ulls mentre li ho deia; ella va dir, també ben de pressa, ens vam casar que jo tenia divuit anys. Això és tot.''~\autocite[217]{ElHachmi2008}
\end{quote}
En aquest fragment la traducció multiple és especialment interessant.
Tenim les dificultats de la filla de trobar paraules adequades per relacions sexuals en una llengua, en la qual encara existeix el tabu de parlar lliurement sobre aquestes coses,
i en la qual ningú mai li ha parlat d'això d'una manera que no sigui vulgar.
Tenim també la seva assumpció ``natural'' qua el pare sigui el primer home amb qui la seva mare n'ha tingut--una traducció de la pregunta de la metgessa.
I tenim encara la resposta de la mare, que és una nova traducció: ``ens vam casar que jo tenia divuit anys''~\autocite[217]{ElHachmi2008}--traducció generacional, perquè ``casar-se'' substitueix ``dormir amb el pare'' i traducció llingüística, ja que la filla passa la frase a català~\autocite{Vidal2012}.

Els nens són també traductors tàctics i creatius en la seva intenció de preservar la família:
``Sobretot quan el pare va tornar a parlar i a nosaltres ens deia allò de digueu a la puta de la vostra mare que\ldots digueu a la porca de la vostra mare que\ldots digueu a aquella gossa que\ldots Nosaltres només li deiem mama ell diu que\ldots Allà vam començar a fer de traductors.''~\autocite[177]{ElHachmi2008}
i ``Llàgrimes sense soroll, i jo vaig acabar a traduir: vés-te'n  no tornis mai més. I vaig interpretar una mica el paper.''~\autocite[223--224]{ElHachmi2008}
%"No devien entendre gaire res i les filles del veí de davant deien a la mare, au, fes-lo fora de casa, si vols, nosaltres t'ajudarem. Ella no les entenia, somreia i deia sí, sí, però elles ja devien veure que no en traurien l'aigua clara. Llavors es psssaven els dits entre els cabells de permament i em feien servir de traductora. Em feien sentir frases que jo no volia sentir i em feien dir coses que jo no volia dir. Què dieun?, deia la mare, què diu?, deien elles. Jo hauria cridat res, res, res, calleu totes si no us enteneu." (p.196)
Com menciona l'intelectual Ilan Stavans davant Verónica Albin en una entrevista, aquestes traduccions són, naturalment, totes tergiversacions fins a un cert punt~\autocite{Albin2005}.
Cada llengua i cada situació tenen el seu propi context i diferències semàntiques subtils, que fan la transferència idèntica quasi impossible.

Seguint el tema de la traducció, hem de notar que
un objecte important per a la filla, al qual es fa referència al llarg del llibre, és el diccionari de la llengua catalana.
El diccionari i, en conseqüència, la llengua, són un refugi per a la noia de les contradiccions de la vida quotidiana.
Les consultacions amb el/les lectures del diccionari serveixen també com un fil que ens ajuda seguir la història.

Hem de recordar el valor simbólic dels diccionaris d'una llengua.
El diccionari és el primer llibre d'una nació, cita Verónica Albin en la seva entrevista amb Stavans~\autocite{Albin2005}.
També, és important notar que el mític objecte ``el diccionari'' no existeix.
Cada llengua disposa de diccionaris multiples, cada un d'aquests té lxs sevxs redactorxs, persones amb ideologies i circunstàncies concretes que naturalment influeixen les seves obres~\autocite{Albin2005}.
Com ho formula Henri Meschonnic, citat per Albin, els diccionaris reflecteixen ``les conflits, les masquages des
conflits, les clichés qui font l'album de famille d'une culture''~\autocite{Albin2005}.
Stavans es posa d'acord amb aquesta noció i subratlla que encara que siguin xenòfobs, els diccionaris serveixen per definir l'univers d'una persona,
perquè ``The limits of our language are the limits of our worldview.''~\autocite{Albin2005}.
Llavors, el diccionari marcaria també els límits de l'univers de la filla.
%prescriptivism vs descriptivism -- key characteristic of dictionaries; important? warum?

%Damunt, un aspecte fundamental de la nostra identitat i de la nostra cultura (llevat de que entenem sota aquests dos termes) és/són la/les llengüa/es a través de les quals ens comuniquem.
%naming things
En una entrevista Najat El Hachmi aborda un altre aspecte clau del llenguatge--la possibilitat d'anomenar persones, emocions, esdeveniments i relacions al voltant nostre.
Clau, perquè si no podem parlar dels fets i els problemes que ens envolten de manera adequada, tampoc podem elaborar solucions.
En particular, l'autora es refereix aquí a la violència extrema que comet el personatge de Mimoun.
El Hachmi assenyala que si no parlem de la violència, ens convertim en seus còmplices~\autocite{HaAM2011}.
%Per tant, destaca ella, les lectures del diccionari per part de la filla són el primer pas cap a estant en control de la situació, cap a un alliberament.
Les lectures del diccionari per part de la filla són el primer pas cap a estant en control de la situació, destaca ella.
Per tant, el diccionari no només marca els límits d'un univers, sinó també el camí cap a l'alliberament.

D'altra banda, la literatura és una manera d'ampliar els propis horitzons, d'experimentar vides diferents, de descobrir que altres maneres de viure són possibles~\autocite{HaAM2011}.
%literature!
Tenim al llarg del llibre multiples referències a les lectures de la protagonista que aparentament exerceixen un paper important per a la seva vida.
Algunes d'elles són directes:
\begin{quote}
  Escrivia molt en aquelles planes que m'havia regalat la mestra que era amiga i que no vaig veure durant anys, hi escrivia cent vegades em vull morir, em vull morir, em vull morir\ldots però no era cert. Sort en vaig tenir, de \textit{Mirall trencat}, de l'\textit{Ariadna al laberint grotesc}, de les memòries del Tísner, de Faulkner, de Goethe, de totes les lectures que passaven per les meves mans. Que el diccionari ja se m'acabava i jo havia de créixer del tot, però em resistia i pensava que tot era una fase, que aquella obsessió que tenia per mi aviat li passaria.~\autocite[286]{ElHachmi2008}
\end{quote}

Altres són referències intertextuals indirectes.
Sobre tot tres obres semblen bastant importants per a la novel·la, ja que es mencionen per raons diferents repetidament: ``La plaça del diamant'' de la Mercè Rodoreda, ``El petit princep'' d'Antoine de Saint-Exupéry i ``The House on Mango Street'' de Sandra Cisneros.
``El petit princep'', similar al diccionari, forma un dels fils que ens condueixen per la història.
La domesticació (sobre tot de la mare) per part de Mimoun i la força dels seus lligams són elements repetitius:
``\ldots però Mimoun s'havia fet gran, havia començat a anar a escola i, el més important, havia començat a exercitar-se en el difícil art de domesticar les persones que l'envoltaven, de crear lligams, que deia la guineu''~\autocite[24]{ElHachmi2008},
``Mimoun pensava que aquella separació posaria a prova els lligams que havia creat amb la seva dona i es veuria si ja l'havia domesticada prou.''~\autocite[124]{ElHachmi2008}, etc.

``La plaça del diamant'' té un paper important per el personatge de la mare.
La seva protagonista principal, la Colometa, serveix com a una descripció de la mare, com a una comparació amb algú familiar per la majoria de les lectores:
``La mare a vegades semblava la Colometa en comptes de la Mila, de tant com havia netejat els excrements secs de damunt els taulons de fusta sota les deules d'uralita.''~\autocite[199]{ElHachmi2008}.
I, de fet, trobarem trets similars entre les dues (apart de netejar els excrements de coloms): el patiment en silenci, la submissió femenina davan un marit abussiu, el moral doble al que estan sotmeses totes dues.
Demés, aquestes línies contenen les poques vegades quan el nom de la mare es menciona en absolut:
``Una estona i jo vaig admirar la mare per ser més que una Mila, més que una Colometa, per ser de debò.''~\autocite[223]{ElHachmi2008}.
Damunt, les referències de Rodoreda també ens fan sentir la proximitat entre la protagonista i la cultura catalana.
Ella mateixa es compara amb Mercè Rodoreda en els seus experiments literàris: ``Jo no era Mercè Rodoreda, però havia d'acabar amb l'ordre que ja feia temps que em perseguia.''~\autocite[331]{ElHachmi2008}.

\begin{comment}
  Quotes (moved)
  ---------------
"De vegades passa allò que la mort et fa pensar en la vida i aquell primer estiu al nostre particular Mango Street, el pare va tenir un instant d'aquells de lucidesa."(p.235)
* al Petit Princep (els lligams)
Referències al "Petit Princep":
"... però Mimoun s'havia fet gran, havia començat a anar a escola i, el més important, havia començat a exercitar-se en el difícil art de domesticar les persones que l'envoltaven, de crear lligams, que deia la guineu." (p.24)
--> einer der roten Faeden der Geschichte


Qui es català?
Entre (dues)/multiples cultures (wenn wir auch den Kleinen Prinz hier reinschieben)

* referències a la Mercè Rodoreda: vinculació amb la cultura/tradició literària catalana --> soll das hier hin?
"La mare a vegades semblava la Colometa en comptes de la Mila, de tant com havia netejat els excrements secs de damunt els taulons de fusta sota les deules d'uralita." (p.199)
--refèrencia Mercè Rodoreda
--també la primera i una de les poques vegades on surt el nom de la mare

"Jo no era Mercè Rodoreda, però havia d'acabar amn l'ordre que ja feia temps que em perseguia." (p.331)
"Va dir o la deixes, o et deixo. Jo no em vaig creure el que sentia, però era la meva mare que parlava, era la Mila que s'havia afartat de netejar capelles i relíquies, la Colometa que fugia de tot per trobar-se." (p.222)
"Una estona i jo vaig admirar la mare per ser més que una Mila, més que una Colometa, per ser de debò." (p.223)

* Sandra Cisneros: "The house on Mango Street"
cap 15 (segona part): "Una casa en un passatge, no pas a Mango Street"
-- referència a Sandra Cisneros! (look up)
-- vinculació a altres obres romàniques
"Tot i que mudar és canviar o transformar, el que nosaltres vam fer va ser mudar-nos, canviar de casa sense transformar-nos gaire." (p.230)


"Escrivia molt en aquelles planes que m'havia regalat la mestra que era amiga i que no vaig veure durant anys, hi escrivia cent vegades em vull morir, em vull morir, em vull morir... però no era cert. Sort en vaig tenir, de <textit>Mirall trencat</>, de l'<textit>Ariadna al laberint grotesc</>, de les memòries del Tísner, de Faulkner, de Goethe, de totes les lectures que passaven per les meves mans. Que el diccionari ja se m'acabava i jo havia de créixer del tot, però em resistia i pensava que tot era una fase, que aquella obsessió que tenia per mi aviat li passaria." (p.286)
--referècies a Rodoreda i altres clàssics
-- oder Kap LLengua: die Literatur (nicht nur das Wörterbuch) spielt eine wichtige Rolle in ihrem Leben

\end{comment}
