\section{El proyecto Altermundi}

\subsection{Desarollo histórico}
% oder \subsection{Altermundi: fundación, objetivo, estructura, desarrollo histórico}

El proyecto Altermundi existe como tal desde el año 2012.
Se fundó desde la cooperativa CodigoSur (...), que hace ... con el objetivo de facilitar talleres de redes libres dentro del proyecto Arraigo Digital (fuente), iniciado por el gobierno argentino.

% add cronología
\begin{comment}
2007 reforma constitucional/de la ley telemediatica: "derecho a comunicación"

2009 ley... ??  Ley de Servicios 
de Comunicación Audiovisual No. 26.522, sancionada el 10 de octubre del 2009.  : deconcentración de la comunicación: empresas comerciales no pueden obtener más freqüencias/licencias: hay un límite de freqüencias para cada sector;
http://servicios.infoleg.gob.ar/infolegInternet/anexos/155000-159999/158649/norma.htm
Grupo Clarín se nieda a implementar la ley

y también:
"ARTICULO 97. — Destino de los fondos recaudados. La Administración Federal de Ingresos Públicos destinará los fondos recaudados de la siguiente forma:
...
f) El diez por ciento (10%) para proyectos especiales de comunicación audiovisual y apoyo a servicios de comunicación audiovisual, comunitarios, de frontera, y de los Pueblos Originarios, con especial atención a la colaboración en los proyectos de digitalización107."

2012
* El gobierno argentino conversa con codigosur con la idea de ofrecer talleres de software libre/.. por Argentina
* nace también la idea de ofrecer talleres de redes
* Altermundi se forma como proyecto desde codigosur
* el gobierno quita el apoyo financial prometido (el contacto ya no está)
* 1. hackathon (la gente ya se había comprado la técnica): QuintanaLibre en José de la Quintana en Córdoba: compartir Internet entre 2-3 familias (llega hasta 60 familias hoy)

...
* Delta Libre (1. Hackathon?)
* Quintana Libre
* Nono libre
* ..

2013
* la Corte Suprema de Justicia confirma la consitucionalidad general de la ley

2015
* talleres educativos con otros proyectos (cooperativa de mujeres en NIC?)
* Abr: LaBolsaLibre va en linea
* Dec: Macri llega al poder en Argentina --> política orientada hacia al mercado: laissez-faire capitalismo, los límites para lincencias cesan


reticulación (vernetzung)/intercambio de experiencias con otros proyectos internacionales
* Rhizomatica
* BattleMesh



2016
* principios: Macri modifica la Ley 26.522 con un decreto: " morigerar el carácter antimonopólico de la ley, beneficiando a los principales medios de comunicación del país. "
* FCC + EU: Firmware lockdown
* Nace el proyecto del Libre router
"This virtuous relation between hardware vendors and the community has been threatened by new regulation from the Federal Communitations Commision (FCC) – U.S.A. – which has led vendors to globally close up their routers to third party modifications, hindering open innovation and effectively closing the door to Community Networks in terms of access to the hardware they depend on. https://librerouter.org/what-and-why"

2017
* Julio: el decreto de Macri llega al Corte Suprema, que aun ha de pronunciarse hacia su legalidad
* Octubre: planificado el lanzamiento del libre router
pliegues de redes comunitarias en Argentina

% Desde la página web
Las redes aquí enumeradas utilizan tecnología desarrollada por AlterMundi y reciben soporte y asistencia de parte de la Asociación.

QuintanaLibre, Córdoba, iniciada en Marzo de 2012 
DeltaLibre, Buenos Aires, iniciada en Abril de 2012 
AnisacateLibre, Córdoba, iniciada en Septiembre de 2012 
LaSerranitaLibre, Córdoba, iniciada en Septiembre de 2013 
NonoLibre, Córdoba, Iniciada en Marzo de 2014 
BoquerónLibre, Santiago del Estero, Abril de 2015 
LaBolsaLibre, Córdoba, iniciada en Mayo de 2015

Despliegues de redes comunitarias en el exterior

Fumaça Online, Río de Janeiro, Brasil, Julio de 2015
MulukukuLibre, Nicaragua, Enero de 2016
Caimito Libre, Esmeraldas, Febrero de 2017
\end{comment}

Al final, aquell proyecto no se realizó en la forma prevista~\autocite{Vaseva2016a},~\autocite{iv-nico-pillku}, pero la primer red local, QuintanaLibre, ha ido en linea igualmente.
Desde entonces el grupo básico de activistas, que es Altermundi, ha ayudado a diversas comunidades a construir redes comunitarias locales:
hasta hoy existen ....

\subsection{La red como infraestructura técnica}
% oder La red como infraestructura física

Bastante intrigante, aunque en un primer momento quizás un poco contralógica, resulta la observación de Latour que ``the expansion of digitality has enormously increased the material dimension of networks''~\autocite{Latour2010}.
Sin embargo, si reflejionamos un poco sobre el asunto, nos damos cuenta que sí, la tecnología digital deja muchas huellas y artefactos en el mundo físico: si nos fijamos, podemos descubrir las antenas, los routers y los cables per tot arreu(esp??).
Gulli deckel marcan los lugares de los cables transatlánticos~\autocite{video internet landscapes}.
Si seguimos las antenas y los routers, podemos reconstruir hasta un cierto grado las redes físicas de proyectos comunitarios.

\begin{comment}
%TODO
[Latour2010]
"the technology.. /digital technology .. makes networks material and explicit" (infrastruktur ist präsent und sichtbar im öffentlichen Raum) --> wem gehört die Infrastruktur? wer kontrolliert die? Wer hat (physischen) Zugang dazu?
  geografía: terreno difícil; quien controla la infraestructura?

"subversion it introduces in the notion of distance (the adjectives “close” and “far” are
made dependant on the presence of conduits, bridges, and hubs)," <-- physical network
\end{comment}

% Bild: artefactos físicos en las redes facilitadas por Altermundi
Reference the figure composed of multiple subfigures as Figure~\vref{fig:esempio}. Reference one of the subfigures as Figure~\vref{fig:ipsum}. % The \vref command specifies the location of the reference

\begin{figure}[tb]
\centering
\subfloat[Outdoor router en la red DeltaLibre.]{\includegraphics[width=.45\columnwidth]{deltalibre8}} \quad
\subfloat[Antenas en DeltaLibre.]{\includegraphics[width=.45\columnwidth]{deltalibre10}\label{fig:ipsum}} \\
\subfloat[Instalación de antenas en NonoLibre.]{\includegraphics[width=.45\columnwidth]{nonolibre7}} \quad
\subfloat[Autoproducción de antenas.]{\includegraphics[width=.45\columnwidth]{nonolibre8}} \\
\subfloat[Jesi en QuintanaLibre]{\includegraphics[width=.45\columnwidth]{2013_quintanalibre2}} \quad
\subfloat[Antena en las altes cumbres con autonomía energética solar]{\includegraphics[width=.45\columnwidth]{16-todos_trabajando}}
\caption[A number of pictures.]{Altermundi: artefactos físicos.} % The text in the square bracket is the caption for the list of figures while the text in the curly brackets is the figure caption
\label{fig:esempio}
\end{figure}

% eher zum kapitel gemeinschaftsnetzwerke; vlt den kapitel mit dem hier verschmelzen
Proyectos de redes comunitarias como Altermundi se ubican claramente en el contexto de la segunda onda del ciberoptimismo, que Rieder esboza~\autocite{Rieder2012}.
Sus valores claves (core values) son la decentralización, la distribución del control, la autogovernanza y la organización sin jerarquías~\autocite{FiTre2015}.
Estos proyectos de base (grassroots projects) intentan a revolucionar la comunicación desde abajo, a empoderar a lxs usuarixs a tomar su comunicación en sus propias manos.

\begin{comment}
* community network projects: are to be found in the context of (2nd wave) cyber optimism: "decentralization", "distributed control", "self-governance", "non-hierarchical organization"
an attempt at a capillary revolution -> ist es erfolgreich?
\end{comment}


\subsection{La red social de Altermundi}
% \subsection{La red como tejido social}

Como ya hemos señalado, Altermundi es una asociación social que ayuda a comunidades locales en la construcción de redes inalámbricas.
En cierto sentido, podemos entender a Altermundi y las redes juntos como un único actor en la ANT.
No obstante, si hacemos un zoom in, este se descompone en el grupo básico de activistas y redes autónomas locales.
Y un zoom más allás nos deja aun con personas con necesidades y opiniones individuales.

En esta constelación el papel de Altermundi es sobre todo educar y concienciar.
A través de talleres de formación el grupo intenta a empoderar a las comunidades locales de manejar sus propias necesidades comunicativas.
La comunidad toma las decisiones relevantes de forma autónoma y Altermundi tiene un papel(syn) consultativo.


Bastante interesante es también la recepción del proyecto por parte de las redes.
Algun de los problemas mencionados más frecuentemente es la falta de conocimientos técnicos en la comunidad local.
Entonces, observamos que aunque el proyecto intenta a facilitar la construcción de redes para personas non-técnicas, esto funciona solamente hasta un cierto grado.
\begin{comment}
  Recepción desde las Redes

% licencia de todos los videos: Creative Commons Attribution license (reuse allowed) /Creative Commons Atribucion-Compartir Obras Derivadas Igual 2.5./CC BY-SA 2.5 AR

%  LaBolsaLibre https://www.youtube.com/watch?v=x6nONJXcUQ8
* somos grupo de vecinos/amigos, que ha surgido desde el despliegue de la red, después de la semana de hackathon de construcción de la red
* nos faltan conocimientos técnicos, somos dependientes de otras redes, si un nodo se quema o la red se cae

% QuintanaLibre https://www.youtube.com/watch?YMKlvUS7B-A&v=hsPjT2R-ToQ
* proyecto grande: diferentes grados de involucración
* mucho trabajo
* cómo atraer miembros nuevos?: reuniones con gente ya involucrada, que expliquen que es lo que les gusta en la red, como es su experiencia, "como cambió su vida por la parte de la red"

% AnisacateLibre https://www.youtube.com/watch?YMKlvUS7B-A&v=iasMfOAaDpk
* 10 familias conectadas
* 1 persona responsable, empezó sola, el único nodo en el barrio; se reunió con gente de Altermundi después, intercambiando experiencia, etc.
* desde 2012-2013, después de José de la Quintana
* comparten datos, tráfico, etc.. --> para el responsable el aspecto social es clave: estar conectado con más personas, que comparten
* red comunitaria != internet gratis, sino se hace entre todosa -> importante que lxs participantes lo entiendan

% LaSerranitaLibre https://www.youtube.com/watch?YMKlvUS7B-A&v=Kv2246EDPPg
* desde fines de 2013-2014
* 2016: 15 familias conectadas?
* tenían la necesidad, los proveedores no querían prestarle servicio y conocieron a QuintanaLibre
* ir de casa a casa, algo comunitario, sin un fin económico que estaba en vista
* la mayor dificultad entre vecinos: entender que es una red libre, sacarte un poco de la lógica del mercado, algo que se hace en la comunidad
* desafío: unir La Serranita con Córdoba, para tener conección; el tereno es duro, habían de subir mucho más alto;

% NonoLibre https://www.youtube.com/watch?YMKlvUS7B-A&v=RaXZqlALRII
* 25 nodos operativos: 18 familias y resto cabañeros?
* comenzó sep? 2014
* Nono: localidad túristica; por lo tanto la Camara de Comercio fue un de los impulsores del proyecto, cuando se enteró de ello
* conectividad importante para el desarollo personal pero también para el desarollo laboral del pueblo
* proyecto previo: Nono digital (tener wifi libre en Nono), se juntó con Altermundi -> NonoLibre;
* tenían 40 nodos, pero solamente 25 operativos, ya que el terreno es difícil
* desafío más grande: montar una antena en la Pampa de Acharla?
* experiencia muy linda, gratificante: trabajar en comunidad


\end{comment}
\subsection{Zusammenfuegen}

Al final, la separación aguda entre capa técnica y capa social es difícil y no necesariamente útil.
% mach das sinn?
% vlt ciberoptimismo hier?

\begin{comment}
> framing the Internet alternately as lawless, anarchic,
> free, “a world where anyone, anywhere may express his or her beliefs, no matter how singular,
> without fear of being coerced into silence or conformity” (Barlow 1996) (p.1)

* la infraestructura está prerequisito para participación

## Conclusión del texto [Rieder2012]

> If technology won’t deliver us from the conundrums of
> governance, negotiation, and struggle, we
> may as well reengage politics proper[ly].

* sobre todo problemas sociales/políticos: organización de grupos (no tanto técnicos): ¿cómo decidimos como grupo? ¿quién hace qué? ¿quién está responsable?
* soluciones políticas para problemas políticos
\end{comment}

\subsection{Mapeo: área conflictiva}

\begin{comment}
ANT: podemos hacer aquí zooms in y out!

                          * da talleres
                          * apoyo técnico
                          * Bildungsauftrag

            Altermundi <-----------------------> redes
         /   ^                                     ^
individuos   |                                     |
             |                                     |
             |                                     |
             |                                     |
             v                                     v
            estado<-------------------------> infraestructura

Altermundi:
* individuos, gente que se comunica entre si, con todos los problemas y desafíos que surgen de esto
* tiene una forma organizativa legal: asociación civil (siehe IV) para poder entrar en alianzas, cooperaciones (con el estado y otras organizaciones formales, p.e. la Universidad de Córdoba)
* función auxiliar (las redes son autónomas)

Estado:
* está responsable para regulaciones que ayudan o impiden el trabajo de tales proyectos (ley telemediatica)
* la última milla
* solucionar el problema de conexión de manera completa/de gran escala (siehe papel del estado)
* apoyo financial (eher nicht, vgl Entstehungsgeschichte)

Infraestructura:
* ¿donde está?
* quien la mantiene/gestiona/instala
* quien tiene acceso?

Redes
* autónomas
* tensiones: problemas con la comunicación dentro de la comunidad
* falta de conocimientos técnicos

\end{comment}


\subsection{Ciberoptimismo}
\begin{comment}
% Analyse
  community network projects: are to be found in the context of (2nd wave) cyber optimism: "decentralization", "distributed control", "self-governance", "non-hierarchical organization"
  "the second wave of Inter-
  net enthusiasm was able to transpose key terms such as “decentralization”, “distributed control”,
  “self-governance”, or “non-hierarchical organization”, from the language of countercultural
  community-building into the realm of entrepreneurial cyber-capitalism where scarcity doesn’t
  exist, without them losing their anti-establishment ring and affective value."

## técnica y política

> framing the Internet alternately as lawless, anarchic,
> free, “a world where anyone, anywhere may express his or her beliefs, no matter how singular,
> without fear of being coerced into silence or conformity” (Barlow 1996) (p.1)

"framing the Internet alternately as lawless, anarchic,
free, “a world where anyone, anywhere may express his or her beliefs, no matter how singular,
without fear of being coerced into silence or conformity” (Barlow 1996) or, more recently, as a
space of surveillance, commercial manipulation, and sweeping monopoly."
* la infraestructura está prerequisito para participación
* Community networks operieren in diesem Spannungsfeld

"My first line of critique will therefore try to show that digital networks may very well
produce effects of centralization as much as decentralization, and give rise to new mechanisms of
power that imply new vectors of domination and abuse." (anderer Pol: google & co)
decentralización, pluralidad, democracia --> diese ganzen ciberoptimismo Zeugs in Basics:
Community Networks: die werden idealerweise angestrebt; aber gelingt das?

%ANT:
[Rieder2012]
"Software now habitually provides specific answers to questions that do not seem technical at all: What is communication? What is cooperation? Which information is valuable? What is decision-making?"
"The idea that technology implements values and should be seen as an “actor” in the shaping of
social processes is of course not new; from actor-network theory"
"In many of the “self-organized” sys-
tems that make up Web 2.0, we find that a small group dominates structures of visibility."
Zoom out: big internet providers + social media corporations (Google, Facebook, Cisco, AT&T)
Zoom in to community networks: you'd see above all Altermundi, but not the redes

"The hope for magical technological
solutions to the messy realities is counterproductive if it leads to an attitude that disengages tradi-
tional political process to simply “route around it” ." --> LibreRouter routs around the firmware lockdown quite litereally

\end{comment}

