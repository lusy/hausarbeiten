\section{Marco teórico}

\subsection{Commons/bienes comunes y redes comunitarias}

\begin{comment}
## "community"

> The term “community” has played a central role in American affective politics for a long time,
> and as Cherry Schrecker (2006) argues, this thoroughly positive connotation carries, with ex-
> tremely few exceptions, through most of Anglo-Saxon sociology.

* en alemán: más ambiguo (los nacis y la "Volksgemeinschaft")
* español? ("comunidad", "comunitari@");
\end{comment}

\subsection{Actor Network Theory}

\begin{comment}
[Venturini2010a]

"In controversies, any actor can decompose in a loose
network and any network, not matter how heterogeneous, can coagulate to function as
an actor."

"To understand how social phenomena are built it is not
enough to observe the actors alone nor is it enough to observe social networks once they are
stabilized. What should be observed are the actors-networks—that is to say, the fleeting con-
figurations where actors are renegotiating the ties of old networks and the emergence of new
networks is redefining the identity of actors."

"an actor is anything doing something. "

"there is no such thing as an isolated actor. Actors are always composed
by and components of networks."

"Actors are such because they inter-act, shaping relations"

\end{comment}

\subsection{Ciberoptimismo}

\begin{comment}
community network projects: are to be found in the context of (2nd wave) cyber optimism: "decentralization", "distributed control", "self-governance", "non-hierarchical organization"

## técnica y política

> framing the Internet alternately as lawless, anarchic,
> free, “a world where anyone, anywhere may express his or her beliefs, no matter how singular,
> without fear of being coerced into silence or conformity” (Barlow 1996) (p.1)

* la infraestructura está prerequisito para participación
\end{comment}

\subsection{Instituciones y jerarquías}

\begin{comment}
## statutory vs capillary power

formal establishments / social mechanisms

> Even if there are no institutions (as formal establishments) regulating behavior,
> there are always institutions in the sense of mechanisms,
> rules, and established dynamics.
\end{comment}

