\section{Befunde}
\label{chap:results}

% vlt anderer Titel, wenn man genau weiß, was die These ist
%\subsection{Hauptthese}

Die automatisch erkannten Vorkommnisse von englischen Floskeln wurden einzeln gesichtet und in Kategorien unterteilt.
Insgesamt wurden in den 1419 Artikeln 3271 eindeutige englische Tokens identifiziert. %TODO Tokens oder Vorkommnisse?
Das heißt, wenn das selbe Token in einem Artikel öfter vorkam, wurde es hierfür nur einmal gezählt.
Im Kapitel~\ref{chap:results-desc} werden diese Kategorien beschrieben und exemplarisch dargestellt.
Anschließend werden in Kapitel~\ref{chap:results-analysis} verschiedene Erklärungsversuche dafür angeboten.


\subsection{Ergebnisse beschreiben}
\label{chap:results-desc}

Im Folgenden wird ein qualitativer Überblick über die identifizierten Kategorien von Code-Switches und dazugehörige Beispiele gegeben (die angeführten Beispiele erschöpfen die Kategorie nicht, sondern sind als exemplarische Illustration/Veranschaulichung zu betrachten).

\subsubsection{Named Entities/Eigennamen}
Eigennamen bilden die erste und zugleich umfangreichste Kategorie von gefundenen Code-Switches.
Es handelt sich dabei um Namen von Organisationen (z.B. ``Sony Music'', ``Washington State University'', ``Adventure Cycling Association''),
Produkten (``Kindle'', ``Maybeline Fit Me Foundation'', ``Converse All-Star''),
Medien ( ``American Journal of Clinical Nutrition'', ``Daily Mail'', ``The Huffington Post'', ``Fitness Magazine''),
Filmen (``Letters to Juliet'', ``Frozen''),
Büchern (``The Girl with the Dragon Tattoo'', ``Fifty Shades of Grey'')
und Fernsehsendungen (``American Idol'', ``Animal Planet'', ``BBC News'')
oder Veranstaltungen (``New York Fashion Week'', ``Super Bowl'').

\subsubsection{Nahrungsmittel}
Eine andere umfangreichere Kategorie, die vielleicht nicht vom Anfang an erwartet wurde, formieren Nahrungsmittel bzw. Begriffe, die mit Essenszubereitung und -konsum zusammenhängen.
Zunächst haben wir es hier mit Lebensmitteln in Rezepten oder Dietprogrammen zu tun, die oft auf Englisch, oder auf Spanisch mit der englischen Übersetzung in Klammern erscheinen.
Manchmal sind es spezifischere Lebensmittel, die in diese Kategorie auftauchen wie z.B. ``kale'' (Grünkohl), oft lesen wir aber auch von ``banana'', ``steak'' oder ``baking soda''.

%% Lifestyle Essen
Manche der Switches in dieser Kategorie haben etwas mit einem gesunden Lebensstil zu tun.
Es handelt sich dabei um Sachen wie ``baby greens'', ``pavo wild'', ``gluten free'', ``smoothie'', ``olive oil'', ``light'' etc.
Andere wiederum werden eher mit einer Atmosphäre von ``Coolness'' oder ``Hippness'' assoziiert:
``brunch'', ``coffeehouse'', ``dip'', ``snack'', ``steak'', ``bacon'', ``mint julep'', ``Dirty Martini''

%% US Zeug
Nicht zuletzt gibt es unter den Nahrungsmitteln auch welche, die vielleicht als spezifisch US-amerikanisch identifiziert werden können (oder zumindest sehr stark mit der US-amerikanischen Kultur in Verbindung gebracht werden) und für die es dementsprechend vielleicht auch gar keine spanischen Begriffe existieren.
Beispiele hierfür sind ``brownies'', ``cheesecake'' oder ``peanut butter''.

\subsubsection{``Gesundheit''}
Verwandt mit der vorherigen Kategorie wurde auch die Gruppe der Switches, die unter ``Gesundheit'' grupiert (syn!) werden können, identifiziert.
Nicht nur Nahrungsmittel, die einen gesunden Lebensstil suggerieren werden geswitched, sondern auch Begriffe, die mit Diät oder Fitness zu tun haben.
Dazu gehören z.B. Namen von Diät-Programmen (``la dieta Yes You Can!'', ``the Ultimate Diet'' oder ``la dieta Fast Diet''),
Körperbeschreibungen (syn?) wie ``water weight'', ``thigh gap'' oder ``love handles'',
sowie Fitness und Training-Sachen (syn) wie z.B. ``spinning'', ``gym'', ``workout'', ``bicycle crunches''.

\subsubsection{Technik}
Eine andere sehr umfangreiche Gruppe von Code-Switches wurden in die Technik-Kategorie grupiert.
Diese wurde von mir und den Leserinnen vermutlich am ehesten erwartet.
Beispiele hieraus umfassen allgemeine Techologiebegriffe wie ``chip'', ``laptop'', ``upgrade'', ``software'';
solche, die mit sozialen Medien verwandt sind wie ``hastag'', ``trending topic'' oder ``likes'';
solche, die mit dem Internet allgemein (syn!) zu tun haben: ``wifi'', ``live stream'', ``online'', ``chat'', ``email'';
und welche, die mit Smartphones zusammenhängen: ``app'', ``smartphone'', ``selfie''.
Diese wirken wenig überraschend und die allermeisten davon werden auch auf Deutsch verwendet.

% solche, die schon ``angespanischt'' wurden: ``clic'' und ``postear''

\subsubsection{Unterhaltungsindustrie}
Neben den Titeln von Filmen, Büchern oder Fernsehprogrammen wurden weitere Switches identifiziert (syn!), die mit Fernsehen und Medien sowie mit Musik zu tun haben.
Zu den ersteren gehören Vokabeln wie ``reality'', ``show'' und ``TV'',
bei der Musik handelt es sich vor allem um Genres, die bereits auf vielen Sprachen entlehnt worden sind: ``country'', ``pop'', ``folk'', ``jazz'', etc.

\subsubsection{Liebe und Sex}
Eine andere relativ große Kategorie bilden Switches, die mit den Themenfeldern ``Liebe'' und ``Sex'' verbunden sind.
Beispiele daraus sind ``one night stand'', ``dating'', ``sexy'', ``sex appeal'', ``hot'', ``crush'' oder ``the one'' (``ya has encontrado a the one'').

\subsubsection{Mode}
Außerdem gab es einige Switches, die dem Modebereich zuzuordnen sind:
manche, wie z.B. ``outfit'' waren auf Kleidung bezogen, andere auf Frisur: ``bob'' und ``pixie'';
wiederum andere hatten mit Makeup zu tun ``smoky eye'', ``cat eye'', ganz prominent dadrunter waren Bezeichnungen für Farbtönen: ``nude'' ``coral reef'', ``snow shadow'', ``cherry red''.
Und nochmal andere wiederum allgemein mit einem hippen Life Style:
``fashion'', ``trendy'', ``look'', ``vintage'', ``glamour''.

\subsubsection{Entlehnungen aus dem Englischen}
Ferner wurden auch einige allgemeine/ungeordnete Entlehunungen aus dem Englischen identifiziert.
Beispiele hierfür wären die Vokabeln ``unisex'', ``party'', ``piercing'', ``fitness'', ``blazer'', ``fan'', ``club'', ``shock''\footnote{All dieser Wörter werden beim Online-Wörterbuch \url{https://pons.eu} als spanisch aufgelistet. \url{https://leo.org} führt alle bis auf ``party'' und ``fitness'' als spanisch. Und die Online-Version des Wörterbuchs der konservativen Real Academia Española \url{https://dle.rae.es} erkennt alle davon bis auf ``piercing'' und ``fitness'' als spanisch an.}.
Manche davon, wie ``piercing'' oder ``blazer'', beschreiben Gegenstände, für die es kein einzelnes Wort auf Spanisch existiert.
Bei anderen, wie ``party'', ``fan'' oder ``shock'',  ist die Entlehnung vielleicht weniger einfach zu erklären, da es entsprechende Äquivalenten auf Spanisch gibt (nämlich ``fiesta'', ``aficionadx'' und ``choque''). % also warum werden sie dann benutzt??

\subsubsection{Diskurspartikel und (idiomatische) Redewendungen}
Zuletzt kann man einige der Switches als Diskurspartikel oder (Teile von idiomatischen) Redewendungen bezeichnen.
In den Artikeln werden sowohl ``anyway'' und ``must'' gebraucht, als auch Phrasen wie ``los pros y los contras'', ``vivimos el presente a full'', ``la voz en off'', ``casarse y vivir su happily ever after'', ``que se embarcan en un road trip'', ``Mix and match , la pareja ideal'', ``son verdaderas decision makers'', ``echale un vistazo al behind the scenes'', ``no es a big deal'', ``Yes You Can!'', ``Wait, what!''

\begin{comment}
    %TODO tue ich die hier noch irgendwie rein?
\begin{enumerate}
  \item English interjections as part of the Spanish text:
      ``me di quenta que wow!'', ``pense wow esto es lo que quiro''
  \item Whole phrases in English, not necessarily idiomatic % todo: check: what's idiomatic anyway?
  \item Misc: one English word without category at the moment
  \item Cool/slang/hip/life-style
      ``cool''
  \item Personal descriptions
      needy, creepy, charming
  \item Berufe/Stellenbezeichnungen: ``nanny'', ``babysitter'', ``dog walker'', ``coach de lifestyle''
  \item Electrodomestics
      ``juicer'' (in Klammern, next to the spanisch text), ``microwave''
\end{enumerate}
\end{comment}

\subsection{Interpretation}
\label{chap:results-analysis}

Die vorgefundenen Code-Switches lassen die Vermutung aufstellen, dass es sich bei den Artikeln um eine Art Performance handelt, die auf ein bestimmtes Zielpublikum (bilinguale Latinas, die einen sozialen Aufstieg und Anpassung an die dominante Leitkultur anstreben?) zugeschnitten ist.
Da es dabei um Kommunikation in geschriebener Sprache geht, sind die Code-Switches ganz bewusst ausgewählt und benutzt worden (vgl.~\cite{Mahootian05}) und sind nicht das Produkt einer spontanen Sprachproduktion, wo der Sprecherin ein Begriff nicht eingefallen ist.

Es können für die vorgefundenen Kategorien von Code-Switches verschiedene Erklärungen geliefert werden.

\subsubsection{Leichtere Wiedererkennung}
Der erste Erklärungsversuch geht zunächst von der Gruppe der Named Entities aus.
Diese hat man vielleicht bei Code-Switching am Anfang vielleicht nicht mitgedacht, sie sind jedoch kaum überraschend.
Es existieren gewiss für einige davon auch spanische Übersetzungen (zum Beispiel für die Film- und Buchtitel, sowie einige Produkte).
Die Publikation richtet sich allerdings an ein Publikum, das diese Produkte auf dem US-amerikanischen Markt erlangen sollte, bzw. in einem dominierend englischsprachigen Kontext lebt und deshalb auch mit höherer Wahrscheinlichkeit einfach die englischen Namen der Filme oder Bücher kennen würde.
Deshalb erscheint es nur logisch, dass sie auf Englisch benannt werden.

Die leichtere Wiedererkennung kann auch hinter den Code-Switches bei bestimmten (vielleicht seltener vorkommenden) Nahrungsmitteln, z.B. in Rezeptzutaten oder Diätanweisungen, vermutet werden, da die Leserinnen diese auf einem englischsprachigen Markt konsumieren (und finden) sollten.
Es handelt sich bei manchen vielleicht um Nahrungsmitteln, die in den Ursprungsländern der Leserinnen nicht besonders populär sind und demzufolge die Menschen kaum die spanischen Namen kennen würden.
Wenn frau in der Lage sein sollte in den Geschäft zu gehen und Produkte auf Englisch zu kaufen, ist es durchaus sinnvoll zu wissen, dass ``col risada'' auf Englisch ``kale'' heißt.
Insgesamt kann man sagen, dass es sich hier um ein Versuch handelt, Zutaten wirklich klar zu stellen und Missverständnisse zu vermeiden.

\subsubsection{``Linguistic necessity''}
% Entlehnungen für Begriffe, für die es auf Spanisch keinen genauen Äquivalent existiert
% alt. Titel: Der Begriff existiert auf Spanisch nicht.
Wie wir bereits gesehen haben, handelt es sich bei manchen Code-Switches um englische Vokabeln, die größtenteils ins Spanische eingedrungen sind und mehr als Entlehnungen funktionieren.
Manche davon werden wir nach Mahootian mit dem Fehlen eines exakten Äquivalents auf Spanisch erklären können (was sie mit ``linguistic necessity'' bezeichnet)~\cite{Mahootian05}.
%, andere vielleicht durch linguistische Sparsamkeit. %eigentlich ist linguistische Sparsamkeit bereits ein separates Argument
Einige davon kommen aus dem Feld der neuen Technologien und die entsprechenden Geräte oder Phänomene sind ursprünglich in einer englischsprachigen Umgebung (z.B. in der Sillicon Valley) entstanden und demzufolge wurden sie auch zunächst (nur) auf Englisch benannt.
Die Konzepte/Gegenstände wurden von da aus in anderen Teilen der Welt übernommen, und da sie vorher nicht existiert haben, werden in den allermeisten Fällen auch gleich ihre Bezeichnungen mitentlehnt.
Manche davon werden schlussendlich auf der lokalen Sprache übersetzt (z.B. ``computadora'' oder ``en nube'' auf Spanisch), bei anderen wird die Benutzung der ursprünglichen Bezeichnung unverändert verstetigt (z.B. ``internet'', ``software'', ``wifi'').
Dieses Phänomen (oder diese Kategorie) können wir vermutlich bei jeder beliebigen Sprache finden und nicht nur im Spanisch-Englisch Kontext der Zeitschrift \textit{Siempre mujer}.
Selbst wenn ein Wort für ein Gegenstand oder Phänomen auf der Lokalsprache (hier Spanisch) existiert und sich etabliert, werden die englischen Bezeichnungen trotzdem oft wiedererkannt und parallel weiterverwendet.

\subsubsection{Zugehörigkeit zu einer bestimmten Gruppe/Lifestyle}
%TODO vlt zugehörigkeit und Lifestyle auseinander pflücken
Eine Zugehörigkeit zu einer bestimmten Gruppe bzw. Lifestyle werden bereits auf mehreren Ebenen evoziiert/inszeniert.

% 1st order indexicality: Latinas
Zunächst wurde die Benutzung vom bilingualen Spanisch-Englisch Code durch die Herausgeberinnen der Zeitschrift als ein Merkmal der Latina-Identität identifiziert und, wie bereits von~\cite{Ticknor12} und~\cite{Mahootian05} gefunden, bewusst eingesetzt, um eine Art Komplizenschaft zwischen den Herausgeberinnen der Zeitschrift und den Leserinnen herzustellen.
Für solche Art Switches wäre es im Grunde egal, was genau geswitcht wird, Hauptsache, die kommen vor, und unterstreichen dabei die Gemeinsamkeit mit den Leserinnen.

\begin{comment}
% Alte Version; das mit der Abgrenzung von der Eltern-Generation ist etwa too much glaub ich und außerdem nimmt die neue Formulierung klarer Bezug auf Sozioindexikalität
Wie von~\cite{Ticknor12} und~\cite{Mahootian05} gefunden, wird durch eine Benutzung des bilingualen Codes vermutlich versucht, eine Art Komplizenschaft zwischen den Herausgeberinnen der Zeitschrift und den Leserinnen herzustellen.
Schließlich gehört Code-Switching eben zu den Charakteristiken einer bilingualen Latina-Identität.
Für solche Art Switches wäre es im Grunde egal, was genau geswitcht wird, Hauptsache, die kommen vor, und unterstreichen dabei die Gemeinsamkeit mit den Leserinnen.
%Zu einer Identitätsbildung gehört oft ``to identify themselves as a group separate from their predecessors’ generation''~\cite[]{Mahootian05}.
Und grad bei den Englisch-Switches in dem vorwiegend spanischen Text kann man wieder meinen, dass sich die Sprecherinnen (bzw. in dem Fall die Zeitschriftredakteurinnen, die diese 2.generation Latina-Sprecherinnen erreichen möchten) sich von der monolingualen Elterngeneration abgrenzen wollen.
\end{comment}

Außerdem wird Englisch auch gezielt für bestimmte Begriffe benutzt, um einen gewissen Lebensstil zu inszenieren.
Zu diesem trägt nicht nur die Sprachwahl/die Wahl des linguistischen Codes bei, sondern auch die Benutzung/der Konsum gewissen Produkte, suggeriert \textit{Siempre mujer}.
% elaborate on the Produkte

% 2nd order: healthy life-style
Es ist auffällig, dass gerade bei Vokabeln, die mit Fitness und Gesundheit zu tun haben (``fitness'', ``light'', ``baby carrots'', ``workout'', ``bicycle crunches''), einige Switches ins Englische vorgefunden wurden.
Damit signalisieren die Herausgeberinnen, dass ...?

% 3rd order: hip life-style
Darüber hinaus wird eine nächste sozioindexikalische Ebene aufgemacht, auf der Englisch nicht nur mit einem gesunden, sondern auch mit einem modernen, hippen und erfolgreichen Lebensstil gleichgesetzt wird.
Hier können Begriffe wie ``coffeehouse'', ``Dirty Martini'', ``trendy'', ``vintage'' oder ``glamour'' verortet werden.
%TODO nochmal gesondert auf Produkte eingehen?

% Look
% TODO vlt nochmal klarer stellen, wieso das zum Lifestyle gehört
% vlt hiermit: moderne Superfrauen, die ihr Leben unter Kontrolle haben?
Auffällig oft kommt in den Artikeln auch das Wort ``look'' vor.
``Look'', also das äußere Aussehen, wird besonders bei Frauen oft mit Erfolg (oder seinem Fehlen) gleichgestellt, wie die britische Periodistin und Autorin Laurie Penny argumentiert:

\begin{quote}
For modern women in this axious age, the makeover is a ritual of health and devotion and social conformity.
[\ldots]
Cosmetic surgery companies plaster public transport with promises to deliver not just physical changes, but emotional ones like `confidence'.
Fashion editorials advise us to spend money we don't have on skirt suits and handbags as `investment pieces'; you're not supposed to dress and style your body simply to please yourself but with one eye on your financial future.
That skirt suit really is an `investment' in a one-woman business whose product is you, only glossier.
This is what power, health and success means to the modern, emancipated woman: terminal exhaustion and a wardrobe full of expensive disguises.~\cite[p.41-42]{Penny14}
\end{quote}


%TODO Laurie Penny + Kapitalismuskritik ins Fazit auslagern?
%Otherwise: re-write rest of this subchapter

%Kapitalismuskritik
Man kann die oben anlautende Kritik erweitern/weiter denken/verfolgen/... und ... daran anknüpfen:
Schlussendlich sind die hier erweckten/angedeuteten/.. Werte wie Flexibilität, (hip, jung), Mobilität, Kreativität, .. vom Kapitalismus aufgegriffen und angeeignet worden.
Wie Luc Boltanski und Ève Chiapello schreiben: .. ~\cite[]{BolChi07}.

% Sex
% TODO Wie kann das angegliedert werden?
Eine weitere spannende Kategorie habe ich ``one night stand'' bzw Sachen, die mit Sex zu tun haben benannt.
Wir kennen den platten Ausdruck ``sex sells'' und die kapitalistische Gesellschaft tut Frauen gnadenlos als sexualisierte Objekte ausbeuten (besser ausführen, nach Kapitalismuskritik).
Warum aber werden Wörter und Ausdrücke aus diesem semantischen Feld auf Englisch gebraucht?
Ich würde hier wieder das Argument mit der Prestige, bzw coolness heranziehen. (vgl auch Laurie Penny!)



% Eckert: ``we connect linguistic styles with other stylistic systemssuch as clothing and other commoditized signs and with the kinds of ideological constructions that speakers share and interpret and thereby populate the social imagination''
% -> also language choice + choice of particular products create meaning

\begin{comment}
Eckert:

``Once the agent isolates and attributes significance to a feature, that feautre becomes a resource that he or she can incorporate or not into his or her own style.''
--> Switching is identified as a resource by the publishers of the magazine

1st order indexicality/Indizes
Latinas

2nd order
bewusste Nutzung von bilingualen Codes von den Herausgeberinnen um sich als Latinas zu positionieren und diese anzusprechen; eine Sisterhood zu simulieren

3rd order
social mobility/prestige?
\end{comment}


\subsubsection{Prestige}
Ohne Zweifel ist in den USA Englisch die angesehenere Sprache/ist besser angesehen als Spanisch.
Sie ist die offizielle Sprache im Land und wird von der Mehrheit der US Bevölkerung gesprochen.
Spanischsprachige Bürgerinnen und solche, die als Latinas identifiziert werden, werden öfter auf dem Arbeitsmarkt aber auch im alltäglichen Leben systematisch diskriminiert.
Wer gut Englisch beherrscht (und am besten noch relativ weiß aussieht und einen angelsächsischen Namen hat), hat bessere Chancen, einen Job zu finden und sozial aufzusteigen.
Wiederum wird Spanisch mit Marginalisierung, Armut, Rückstand, fehlender Bildung und dem nicht-weißen Anderen asoziiert~\cite{Zentella07}. % so für das alles brauchen wir eine Quelle.
Zentella spricht von einem ``remapp[ing of race] from biology onto language''~\cite{Zentella07}
und macht klar, dass die generelle Erwartungshaltung vorherrscht, von Medien und Institutionen forciert, dass sich Menschen anstrengen und gut Englisch lernen, um respektiert zu werden.
Diese Tatsache lässt uns die Vermutung aufstellen, dass öftere Benutzung englischer Vokabeln (egal aus welchem Gebiet) eine gewisse Überlegenheit in der sozialen Hierarchie signalisieren/andeuten sollte, bzw. dass Menschen, die auch so reden, zu verstehen geben, dass sie Englisch beherrschen und somit denen ein sozialer Aufstieg offen steht.

\begin{comment}
% Ich finde die Erklärung für die Discourse markers kommt etwas zu kurz und würd die rausnehmen
% English discourse markers;
Genau damit würde ich die Kategorie der englischen Discourse Markers (DE?) erklären.
Sonst ist es kaum nachvollziehbar, warum im Text ``anyway'', ``must'' oder ``pros and contras'' erscheinen sollten, anstatt auf diese komplett zu verzichten, bzw. die spanischen Übersetzungen zu benutzen. % oder mit Attention devices!
Bei gesprochener Sprache kann man sich diese Gebrauche noch mit spontanem Ausdruck erklären, bei dem Medium Zeitschrift jedoch, wo die Autorinnen der Artikel Zeit zum Nachdenken hatten und der Text vermutlich einen (mehrstuffigen) Redigierprozess durchlaufen hat, kann von Spontanietät kaum die Rede sein.
\end{comment}

\subsubsection{Aufmerksamkeit erregen durch den Code-Wechsel}
Zu guter Letzt wird hier auch eine andere mögliche Erklärung für die Code-Switches erwähnt:
Sie erregen Aufmerksamkeit einfach weil sie nicht spanisch sind~\cite[]{Mahootian05},~\cite{Lee99}.

Möglicherweise können damit die (zur Hälfte übersetzten oder komplett) englischen Redewendungen erklärt werden.


\begin{comment}
%TODO Ist es nötig, dass konkret die Kategorien erwähnt werden?

% (halb)übersetzte englische Idiome/Redewendungen (iwie mit der oberen Kategorie verschmelzen)
Vielleicht können damit auch einige (zur Hälfte übersetzt oder komplett auf Englisch) englische Idiome und Redewendungen erklärt werden.
(Bsp!)
Eventuell werden manche davon benutzt, weil sie sich in dem konkreten Kontext besonders gut dafür eignen, um einen Sachverhalt zu veranschaulichen und keine genaue Übersetzung auf Spanisch existiert.
Das ist jedoch nicht immer der Fall und Redewendungen, die zur Hälfte auf Spanisch und zur Hälfte auf Englisch abgedruckt sind (``vivir su happily ever after'', ``la voz en off''), dienen höchstwahrscheinlich einfach als ein Mechasnisum zum Erregen der Aufmerksamkeit.
\end{comment}

\begin{comment}

\cite{Beer2002}
"the magazines define Latinas primarily as consumers, they are a limited forum for women to explore noncosumerist identities, challenge hegemony, or express oppositional points of view."
"In addition, contemporarary consumer magazines offer advertisers a variety of "value-added" promotions, such as covert ads in the form of product mentions, and, although they are loathe to admit it, tolerate advertiser input into editorial content."


"It might be argued that, precisely because sexual imagery is designed to make the sale, it must, in some way, fulfill the passions it promises."
"We have noticed recently a trend toward a particular type of ad, increasingly more focused on sexuality, yet done with a tone wholly devoid of affect. Looking at these images one might correctly observe that after almost a hundred years of selling sex, the thrill seems to be gone."

"With the commodification of sex, the basic proposition is unatteinable. In terms of human passions, sex ads fail to satisfy because they confuse sexual gratification with the possession of objects. They attempt to substitute a state of *being* with the promise of *having*."

"The lack of affirmative portrayals of women's sexual passions leaves representations of female fulfillment impossible. [...] At worst, these images reinforce a patriarchal predisposition to disrespect women and do violence against them."
\end{comment}

