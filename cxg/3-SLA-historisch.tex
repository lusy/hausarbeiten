\section{Die Zweitspracherwerbsforschung in historischem Wandel: Übergeneralisierung und Transfer}

Zwei der Wissenschaftlerinnen, die den Anfang der Forschung über die Wechselwirkungen zwischen Erst- und Zweitsprache gesetzt haben, sind Uriel Weinreich und Robert Lado.

Weinreich hat den Begriff der Interferenz geprägt --
diese wird von ihm als ``those instances of deviation from the norms of either language which occur in the speech of bilinguals as a result of their familiarity with more than one language'' \cite{Weinreich79}
definiert\footnote{Erste Veröffentlichung dieser Schrift erfolgt bereits in 1953.}.
Er führt aus, dass Interferenz auf verschiedenen sprachlichen Ebenen vorkommen kann.
Ein Beispiel für eine phonologische Interferenz wäre:
\begin{quote}
``For example, neither French nor Russian have /\textipa{D}, $\theta$/ phonemes, but in contact
    with English, French speakers tend to render /\textipa{D}, $\theta$/ as [z, s], while Russian
    speakers generally pronounce [d, t].
    In other words, the French perceive the continuance of /\textipa{D}, $\theta$/ which distinguishes them from /d, t/ as most relevant, while
    the Russians consider the mellowness of /\textipa{D}, $\theta$/ which distinguishes them from /dz, ts/ as decisive.''
\cite{Weinreich79},
\end{quote}
ein Beispiel für morphosyntaktische Interferenz :
\begin{quote}
``A German speaker says in English `yesterday came he' on the model of German
`gestern kam er', meaning `he came yesterday'.''
\cite{Weinreich79}
\end{quote}

% Interference bei Weinreich


 %       \cite{Weinreich79} ``(1) The replica of the relation of another language explicitly conveys an unintended meaning.
 %       Example: A German speaker says in English `this woman loves the man' on the model of German
 %       `diese Frau liebt der Mann', intending to communicate the message `the man
 %       loves this woman', but producing the opposite effect.
 %       (2) The replica of the relation of another language violates an existing relation pattern, producing
 %       nonsense or a statement which is understandable by implication.
 %       Example: A German speaker says in English `yesterday came he' on the model of German
 %       `gestern kam er', meaning `he came yesterday'.''


% Transfer bei Lado
Der Forschungsschwerpunkt bei Lado ist der Transfer --
``individuals tend to transfer the forms and meanings, and the distribution of forms and meanings of their native language and culture to the foreign language and culture''\cite{Lado71}
\footnote{Das Buch \textit{Linguistics across cultures : Applied linguistics
for language teachers} wurde zum ersten Mal in 1957 veröffentlicht.}.
Laut \cite{Lado71} kann der Transfer sowohl positive als auch negative Auswirkungen haben.
Wenn ein sprachliches Muster in der L1 sich auf einer bestimmten Weise analog zu dem entsprechenden Muster in der L2 verhält (bezüglich seiner Form, Bedeutung oder Gebrauchs), wird dieses leicht erworben.
Die Elemente, die sich in beiden Sprachen unterschiedlich verhalten, werden schwieriger erworben.

% Übergeneralisierung
%Das Phänomen der Übergeneralisierung scheint weniger wissenschaftliches Interesse erweckt zu haben.
\cite{Braidi99} veranschaulicht das Phänomen der Übergeneralisierung sprachlicher Muster in der L2 exemplarisch auf dem Beispiel der L2 Spanisch Lernenden mit Muttersprache Englisch.
Es ist möglich, dass diese das Present Progressive im Spanischen übergeneralisiert mit einer zukunftstrageden Bedeutung benutzen und Sätze wie ``*Estoy saliendo mañana'' anstatt ``Voy a salir mañana'' produzieren.
Eine detailliertere Untersuchung der Übergeneralisierung sprachlicher Muster in der L2
%und des Transfers aus der L1 in die L2
finden wir bei \cite{Taylor75}.
Taylor definiert Übergeneralisierung als ``process in which a language learner uses a syntactic rule of the
target language inappropriately when he attempts to generate a novel target language utterance'' \cite{Taylor75}.
Er führt noch aus, dass Übergeneralisierungen bei unregelmäßigen Einheiten häufiger auftreten als sonst.


%Transfer + Übergeneralisierung
%``Taylor investigated
%the English of elementary and intermediate native Spanish speaking ESL
%students. He analyzed their errors in the Auxiliary and Verb Phrases of eighty
%sentences, categorizing them into those errors that resulted from L1 transfer,
%and those that resulted from overgeneralization of L2 patterns.
%[...] with transfer
%errors more prevalent among elementary students (40 percent) than
%intermediate students (23 percent), and overgeneralizations more prevalent
%in intermediate (77 percent) than elementary students (60 percent).''


\subsection{Die kognitiv linguistische Perspektive}
Der Transfer sprachlicher Strukturen aus der L1 in die L2 wird in \cite{Ellis06} aus kognitiv linguistischer Perspektive beleuchtet.
Es gibt verschiedene Gründe, die diesen begünstigen.
Zum Ersten, wenn ein bestimmtes Merkmal in der Erstsprache nicht identische Eigenschaften mit dem selben Merkmal in der Zweitsprache aufweist, ist schon die Wahrnehmung dieses verfälscht -- das Merkmal wird an den Prototyp angepasst und es findet dann bei der Reproduktion dieses Transfer statt.
Des Weiteren, ``[i]f a referent already has an associated name, it is harder to attach a new name to it'' und
``when multiple traces are associated to the same cue, they tend to compete for access to conscious awareness'' \cite{Ellis06}.
Es ist also schwieriger, sich eine neue Repräsentation für ein Konzept anzueignen, wenn eine andere bereits existiert.
Deshalb kommt es beim Zweitspracherwerb öfters zu Transfer aus der L1.

\cite{Ellis06} führt noch aus, dass wenn aus den oben genannten Gründen Transfer derselben Struktur öfters passiert, diese einen Prozess der Einschleifunge (entrenchment) durchläuft und sich als dauerhafte Repräsentation etabliert.



%aber auch : wenn die cues nicht auffällig sind und es sich mehrere Form-Repräsentationen für die selbe Bedeutung etabliert haben;
%transfer
%In \cite{Ellis06}: ``when multiple
%traces are associated to the same cue, they tend to compete for access to
%conscious awareness'' --> reason for transfer

%oder wenn nicht genug Input vorhanden und wegen mangelndem Input die L1 Variante übernommen wird.


%zum Erklären vom Transfer
\cite{Ellis06} ``Interference theory primarily concerned the transfer of the content of
associations. But more recent analyses demonstrate how from content, given
enough of it, emerges principle, how form–meaning mappings conspire in
biasing attention and process. As a ubiquitous process of learning, transfer
pervades all language learning.''

--> es bildet sich also bei wiederholtem Transfer eine Gewohnheit heraus -> wird dann schwieriger die tatsächlichen Cues zu merken und die ``falsche'' Variante bürgert sich ein;
der Transfer erschwert also das Lernen

%``If a referent already has an associated name, it is harder to
%attach a new name to it.''
%-- bezieht sich auch wenn das Wort in der L1 schon existiert, ist es schwer das mit einer neuen Bedeutung in der L2 zu erlernen; bzw meistens ist die L1 Bedeutung, die als erste aufgerufen wird

`` when two cues are presented together and they
jointly predict an outcome [...] the most salient cue becoming associated with the
outcome and the less salient one being overshadowed''

`` Forms that have not
previously cued particular interpretations are harder to learn as cues when
they do become pertinent later.''


%Schlussfolgerung
``L2 acquisition is clearly affected by the transfer
of learners’ knowledge of their first language.''

%Transfer + Übergeneralisierung : kognitive Erklärung
\cite{Taylor75}
``The overgeneralization and transfer learning strategies
appear to be two distinctly different linguistic manifestations of
one psychological process: reliance on prior learning to facilitate
new learning.''

``Errors which seem to reflect an
overgeneralization strategy suggest three important facts about a
learner’s knowledge of the syntax of the target language:
1. The learner has mastered the mechanics of a particular
syntactic rule of the target language.
2. The learner does not know how to use the rule appropri-
ately; i.e., he has not learned the distribution of the rule or the
exceptional cases where the rule does not apply.
3. The learner is an active participant in the language
acquisition process and is exercising his already acquired knowledge
of the target language in a creative way; he is neither operating
under a repetition or imitation strategy, nor transferring native
language structures in his target language attempts.''


%Overshadowing
%``Overshadowing as it plays out over time produces a type of learned
%selective attention known as blocking. Chapman and Robbins (1990) showed
%how a cue that is experienced in a compound along with a known strong
%predictor is blocked from being seen as predictive of the outcome.''

%``Not only are many grammatical meaning–form relationships low in salience,
%but they can also be redundant in the understanding of the meaning of
%an utterance.''

%``Schumann (1987) describes how L2 temporal reference is initially made exclusively
%by use of devices such as temporal adverbials [...] with the grammatical expression of tense and aspect emerging only
%slowly thereafter (Bardovi-Harlig 2000)''

%Blocking
%``Thus, another pervasive reason for the non-acquisition of low salience cues
%in L2 acquisition is that of blocking, where redundant cues are overshadowed
%for the historical reasons that the learners’ first language experience leads
%them to look elsewhere for their cues to interpretation.''

%``The usual pedagogical reactions to these situations of overshadowing or
%blocking involve some means of retuning selective attention, some type of
%form-focused instruction or consciousness raising (Sharwood-Smith 1981) to
%help the learner to ‘notice’ the cue and to raise its salience.''

%Categorization

%``Whereas the associative learning effects detailed above relate to specific cues
%or constructions and their interpretations, perceptual learning is more to do
%with the organization of the whole system''

%``Early, balanced simultaneous bilingual
%simulations showed the clearest evidence of language separation: through
%self-organization, the network comes to separate the Chinese lexicon from
%the English lexicon, implicating distinct lexical representations for the
%two languages. In the DevLex model, less balanced or later L2 input
%produce representations whereby L2 is partially parasitic on L1, and adult
%second-language simulations show relatively little L1–L2 separation at a local
%level and maximal transfer and interference.''


