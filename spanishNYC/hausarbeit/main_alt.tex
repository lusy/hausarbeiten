%%%%%%%%%%%%%%%%%%%%%%%%%%%%%%%%%%%%%%%%%
% Arsclassica Article
% LaTeX Template
% Version 1.1 (10/6/14)
%
% This template has been downloaded from:
% http://www.LaTeXTemplates.com
%
% Original author:
% Lorenzo Pantieri (http://www.lorenzopantieri.net) with extensive modifications by:
% Vel (vel@latextemplates.com)
%
% License:
% CC BY-NC-SA 3.0 (http://creativecommons.org/licenses/by-nc-sa/3.0/)
%
%%%%%%%%%%%%%%%%%%%%%%%%%%%%%%%%%%%%%%%%%

%----------------------------------------------------------------------------------------
%	PACKAGES AND OTHER DOCUMENT CONFIGURATIONS
%----------------------------------------------------------------------------------------

\documentclass[
10pt, % Main document font size
a4paper, % Paper type, use 'letterpaper' for US Letter paper
oneside, % One page layout (no page indentation)
%twoside, % Two page layout (page indentation for binding and different headers)
headinclude,footinclude, % Extra spacing for the header and footer
%BCOR5mm, % Binding correction
]{scrartcl}

%%%%%%%%%%%%%%%%%%%%%%%%%%%%%%%%%%%%%%%%%
% Arsclassica Article
% Structure Specification File
%
% This file has been downloaded from:
% http://www.LaTeXTemplates.com
%
% Original author:
% Lorenzo Pantieri (http://www.lorenzopantieri.net) with extensive modifications by:
% Vel (vel@latextemplates.com)
%
% License:
% CC BY-NC-SA 3.0 (http://creativecommons.org/licenses/by-nc-sa/3.0/)
%
%%%%%%%%%%%%%%%%%%%%%%%%%%%%%%%%%%%%%%%%%

%----------------------------------------------------------------------------------------
%	REQUIRED PACKAGES
%----------------------------------------------------------------------------------------

\usepackage[
nochapters, % Turn off chapters since this is an article
beramono, % Use the Bera Mono font for monospaced text (\texttt)
eulermath,% Use the Euler font for mathematics
pdfspacing, % Makes use of pdftex’ letter spacing capabilities via the microtype package
dottedtoc % Dotted lines leading to the page numbers in the table of contents
]{classicthesis} % The layout is based on the Classic Thesis style

\usepackage{arsclassica} % Modifies the Classic Thesis package

\usepackage[T1]{fontenc} % Use 8-bit encoding that has 256 glyphs

\usepackage[utf8]{inputenc} % Required for including letters with accents

\usepackage{graphicx} % Required for including images
% add subdirectory for images
\graphicspath{{figures/}} % Set the default folder for images

\usepackage{enumitem} % Required for manipulating the whitespace between and within lists

\usepackage{lipsum} % Used for inserting dummy 'Lorem ipsum' text into the template

%\usepackage{subfig} % Required for creating figures with multiple parts (subfigures)

\usepackage{amsmath,amssymb,amsthm} % For including math equations, theorems, symbols, etc

\usepackage{varioref} % More descriptive referencing

\usepackage[normalem]{ulem}
\usepackage[spanish]{babel}
%\usepackage{lmodern} % bessere Schrift für PDFs
\usepackage{caption} % following 2 needed for subfigures
\usepackage{subcaption}
\usepackage{listings}
\usepackage{color}
%\usepackage{natbib}
\usepackage[style=mla, backend=biber, citestyle=authoryear, showmedium=false]{biblatex}
\addbibresource{literature.bib}
\usepackage{comment}
\usepackage{tipa}
\usepackage{hyperref} % allow for including clickable urls in bib and text
\usepackage{footnote} % following 2 allow footnotes in tables
\usepackage{epigraph} % use for include mottos
\makesavenoteenv{tabular}
\makesavenoteenv{table}

\clubpenalty=10000
\widowpenalty=10000

% neue Absatz wird mit Leerzeile begonnen:
\parindent 0pt
\parskip 12pt

%----------------------------------------------------------------------------------------
%	THEOREM STYLES
%---------------------------------------------------------------------------------------

\theoremstyle{definition} % Define theorem styles here based on the definition style (used for definitions and examples)
\newtheorem{definition}{Definition}

\theoremstyle{plain} % Define theorem styles here based on the plain style (used for theorems, lemmas, propositions)
\newtheorem{theorem}{Theorem}

\theoremstyle{remark} % Define theorem styles here based on the remark style (used for remarks and notes)

%----------------------------------------------------------------------------------------
%	HYPERLINKS
%---------------------------------------------------------------------------------------

\hypersetup{
%draft, % Uncomment to remove all links (useful for printing in black and white)
colorlinks=true, breaklinks=true, bookmarks=true,bookmarksnumbered,
urlcolor=webbrown, linkcolor=RoyalBlue, citecolor=webgreen, % Link colors
pdftitle={}, % PDF title
pdfauthor={\textcopyright}, % PDF Author
pdfsubject={}, % PDF Subject
pdfkeywords={}, % PDF Keywords
pdfcreator={pdfLaTeX}, % PDF Creator
pdfproducer={LaTeX with hyperref and ClassicThesis} % PDF producer
}
 % Include the structure.tex file which specified the document structure and layout

\hyphenation{Fortran hy-phen-ation} % Specify custom hyphenation points in words with dashes where you would like hyphenation to occur, or alternatively, don't put any dashes in a word to stop hyphenation altogether

%----------------------------------------------------------------------------------------
%	TITLE AND AUTHOR(S)
%----------------------------------------------------------------------------------------

\title{Es hora de decir bye-bye\footnote{Quelle: http://siempremujer.com/estilo/8-senales-amiga-toxica-en-redes-sociales/72941/ (zuletzt abgerufen 26.08.2015)}.
Sozioindexikalität und Code-Switching in US-amerikanischen ``Frauenzeitschriften''\footnote{Publikationen, die unter dem Namen ``Frauenzeitschriften'' bekannt sind; bestehen aus Werbung, Fotos von (Möchtegern)-Berühmtheiten und frauenfeindlichen Tipps, die Frauen auf Schönheitsobjekte und Konsumentinnen der Kleidungs- und Kosmetikindustrie reduzieren} gerichtet an Latinas}
%TODO: find a way to hard code footnotes symbol

\author{%
\spacedlowsmallcaps{Lyudmila Vaseva}\\
\normalsize{Seminar Urbane Prozesse: Mehrsprachigkeit und soziale Binnendifferenzierung in den Metropolen der Romania}
\normalsize{Sommersemester 2014}\\
\normalsize{Seminarleiter: Uli Reich}
}

\date{\today}

%----------------------------------------------------------------------------------------

\begin{document}

%----------------------------------------------------------------------------------------
%	HEADERS
%----------------------------------------------------------------------------------------

\renewcommand{\sectionmark}[1]{\markright{\spacedlowsmallcaps{#1}}} % The header for all pages (oneside) or for even pages (twoside)
%\renewcommand{\subsectionmark}[1]{\markright{\thesubsection~#1}} % Uncomment when using the twoside option - this modifies the header on odd pages
\lehead{\mbox{\llap{\small\thepage\kern1em\color{halfgray} \vline}\color{halfgray}\hspace{0.5em}\rightmark\hfil}} % The header style

\pagestyle{scrheadings} % Enable the headers specified in this block

%----------------------------------------------------------------------------------------
%	TABLE OF CONTENTS & LISTS OF FIGURES AND TABLES
%----------------------------------------------------------------------------------------

\maketitle % Print the title/author/date block

\setcounter{tocdepth}{2} % Set the depth of the table of contents to show sections and subsections only

\tableofcontents % Print the table of contents

%\listoffigures % Print the list of figures

%\listoftables % Print the list of tables


%----------------------------------------------------------------------------------------
%	AUTHOR AFFILIATIONS
%----------------------------------------------------------------------------------------

%{\let\thefootnote\relax\footnotetext{* \textit{Department of Biology, University of Examples, London, United Kingdom}}}

%{\let\thefootnote\relax\footnotetext{\textsuperscript{1} \textit{Department of Chemistry, University of Examples, London, United Kingdom}}}

%----------------------------------------------------------------------------------------

\newpage % Start the article content on the second page, remove this if you have a longer abstract that goes onto the second page

%----------------------------------------------------------------------------------------
%	MAIN PART
%----------------------------------------------------------------------------------------

\section{Einleitung}

Pablo Neruda, der Nationalpoet Chiles, ist einer der bekanntesten und beliebtesten, wenn nicht der bekannteste überhaupt, Lyriker*innen Lateinamerikas.
Seine Werke werden heute noch rege rezipiert und sind ausgiebig von der Literaturkritik kommentiert worden.

% Liebeslyrik

Nerudas Werk ist umfangreich und vielfältig, der Poet hat sich mit vielen Aspekten des menschlichen Lebens und dem aktuellen Tagesgeschehnis befasst.
Er bearbeitet in seinen Gedichte seine Erlebnisse während des Spanischen Bürgerkriegs, seine Rezeption anderer Poeten, seine Begeisterung für den Kommunismus, ... sowie ``universellere'' Themen wie Liebe, ...
Unter den vielen prägenden Werken sind auch einige, die uns heutzutage stutzen lassen.
Dabei beziehe ich mich nicht nur auf die Oda an Stalin oder ???, die aus heutiger Sicht zweifelslos umstritten sind, sondern auch teilweise auf Nerudas Liebeslyrik.
Von vielen sehr positiv wahrgenommen und als Wiedergabe universellen menschlichen Gefühle angepriesen, %TODO quote
können wir in dieser auch beunruhigende Ideologien und Frauenbilder entdecken.

% Ziel der Arbeit

Ausgehend von der Annahme, dass Sprache und Literatur nicht nur die Welt widerspiegeln, sondern unsere Weltanschauung, Gedanken und Umgang mit der uns umgebenden Welt aktiv mitgestalten~\cite{Kolodny1980},~\cite{North2013},
möchte ich in dieser Arbeit Nerudas Liebeslyrik aus einer feministischen Perspektive lesen.
Das Ziel dieser Arbeit(syn!)/Aufsatzes? ist es nicht, die ``richtige'' Interpretation Nerudas Liebeslyrik (syn!) auszuarbeiten, sondern, in Anlehnung an Annette Kolodny, viel mehr eine Pluralität der Lektüren anzuregen~\cite{Kolodny1980}, Leseweisen ermöglichen, die im androzentrischen Literaturkritikdiskurs verdeckt bleiben.
Durch das Close Reading exemplarisch ausgewählten Gedichte aus zwei verschiedenen Gedichtsbändern soll die Poetisierung und Universalisierung bestimmten heteronormativen und männerzentrierten Perspektiven aufgezeigt und in Frage gestellt werden.

  % iwie ist mein Projekt mehr oder weniger das gleiche wie das von Frau Duncan: ``There is no doubt that Neruda is a gifted poet; whether or not he is guilty of sexism as a writer is not a point I wish to debate here. But, I do take issue with the critical readings of Neruda which have drawn attention toth repression, subjugation, and silencing of women in his poetry only to dismiss these factors as natural conditions which are in and of themselves praiseworthy.''\cite{Duncan1992}

\begin{comment}
``In my view, our purpose is not and should not be the formula-
tion of any single reading method or potentially procrustean set
of critical procedures nor, even less, the generation of prescriptive
categories for some dreamed-of nonsexist literary canon.52 Instead,
as I see it, our task is to initiate nothing less than a playful pluralism,
responsive to the possibilities of multiple critical schools and meth-
ods, but captive of none, recognizing that the many tools needed
for our analysis will necessarily be largely inherited and only partly
of our own making. Only by employing a plurality of methods
will we protect ourselves from the temptation of so oversimplifying
any text''~\cite{Kolodny1980}
\end{comment}


% Aufbau der Arbeit!
Im Weiteren ist diese Arbeit wie folgt aufgebaut:
Im zweiten Kapitel werden zunächst etablierte Rezeptionen Nerudas Liebeslyrik dargestellt/zusammengefasst.
In Kapitel 3 gehe ich auf einige Begriffe bzw. Methoden ein, die für die weitere Analyse eine bedeutende Rolle spielen werden.
Darauf aufbauend biete ich in Kapitel 4 examplarisch (syn!) alternative Leseweisen von den Liebesgedichten ``Tu risa'' (\textit{Los versos del capitán}) und ``Poema XV'' (\textit{Veinte poemas de amor}) an.
Abschließend wird die Arbeit nochmal zusammengefasst und ein Ausblick wird vorgeschlagen/umrissen.


\section{Grundlagen}

%Try to keep short, it should clear the notions, is not the main part of the paper!

Dieser Kapitel stellt kurz die Grundlagen vor, auf denen sich die spätere Analyse stützen wird.
Im Folgenden werden die Begriffe Code-Switching und Borrowing, Language Shift, %oder??
Sozioindexikalität und indexikalische Felder sowie die derzeitige soziale Realität des Spanischen in den USA erläutert.

\subsection{Code-Switching und Borrowing}
%(and borrowings)
Der Begriff Code-Switching beschreibt die alternierende Benutzung von zwei oder mehreren Codes (Sprachen, Varietäten, Dialekten) innerhalb einer und der selben Äußerung. %cite needed
%Thomason
%include intra/inter sentential stuff? kommt mir nur so semi-wichtig vor.aber vlt nicht schlecht

Borrowing oder Entlehnung ist die Benutzung der Elemente einer Sprache in einer anderen Sprache.
Meistens meinen wir damit Entlehnungen auf der Lexikon-Ebene, jedoch können auch Elemente anderer Systeme entlehnt werden (was durchaus seltener passiert) \cite[vgl.][]{Thomason03}.
Wichtig dabei ist, dass Borrowings in einer Sprache A durch Muttersprachlerinnen dieser Sprache inkorporiert werden \cite[vgl.][]{Thomason03}.
Wenn die Personen, die die Elemente aus Sprache B in Sprache A übertragen, Muttersprachlerinnen der Sprache B sind, die die Sprache A gerade erlernen, reden wir von einer ``shift-induced interference'' \cite[vgl.][]{Thomason03}.


Laut Sarah Thomason ist Code-Switching vermutlich der häufigste Weg, auf dem Entlehnungen Teil einer Sprache werden \cite[vgl.][]{Thomason03}.
In der Praxis ist es oft schwierig zu bestimmen, ob es sich bei einem konkreten Gebrauch um ein Code-Switch oder um eine Entlehnung handelt.
Es ist schwierig Code-Switching und Borrowings scharf von einander abzugrenzen und der Unterschied zwischen den Beiden ist eher quantitativer und nicht qualitativer Natur.
Wenn ein Begriff vereinzelt und im Sprachgebrauch einer einzelnen Sprecherin vorkommt, würden wir das als Code-Switching bezeichnen.
Wenn jedoch derselbe Begriff immer wieder und von mehreren Sprecherinnen gebraucht wird, reden wir von einer Entlehnung \cite[vgl.][]{Thomason03}.

Interessant ist ferner die Motivation für das Vorkommen beider Phänomene.
Ana Zentella spricht von ``powerful socioeconomic and cultural forces'', die Borrowings in Kontaktsituationen zwischen zwei Kulturen fördern\cite[vgl.][]{Zentella90}.
Laut der Wissenschaftlerin, folgende Faktoren ermöglichen Entlehnungen:
\begin{itemize}
  \item Begriffe, die kulturelle Phänomene beschreiben, die in der Muttersprache und der dazugehörigen Kultur in der Form nicht existieren. Beispiele hierfür wären [kei/keike/keiki] ``cake'', [\^yins/blu\^yines] ``(blue)jeans'' \cite[vgl.][]{Zentella90} oder ``la boila'' (boiler). %\cite[vgl.][]{Zentella97} wo kommt dieses letzte bsp her? --> buch wieder ausleihen!
  \item Ähnlichkeit in der phonologischen oder morphologischen Struktur eines Wortes.
    Deshalb wird das spanische Wort ``libreria'' (Buchladen) auf einmal im Spanischen auch für Bibliothek (auf Englisch ``library'', auf Spanisch ``biblioteca'') benutzt. %wo kommt das bsp her? vermutlich aus der selben quelle da oben
  \item einer der universellen Prinzipien des Sprachwandels -- Sparsamkeit: das entlehnte Wort ist kürzer. Deshalb wird zum Beispiel ``florero'' durch ``vase'' ersetzt \cite[vgl.][]{Zentella90}.
  \item Prestige: die Sprache oder Varietät, aus der entlehnt wird, wird als prestigeträchtiger angesehen.
  \item im konkreten Fall ``Entlehnungen zwischen dem Spanischen und dem Englischen in den USA'' können Anglizismen als neutralisierende Begriffe zwischen den verschiedenen Varietäten des Spanischen dienen.

\end{itemize}

Zentella beobachtet ferner die Tendenz, dass Sprecherinnen der 2. Generation mehr Entlehnungen benutzen als die der 1. Generation \cite[vgl.][]{Zentella90}.

%TODO:
%some kind of a wrap up?

\subsection{Language Shift}
%die Frage ist, werde ich in die Richtung argumentieren?
%kann man die Floskel als ein Shift von Spanisch->Englisch begreifen?
%Warum? Warum nicht?
%Who is shifting? the editors of the magazine?or do they induce language shift?
%should this be a focus at all? or more on style/socioindexicality?

% brauch ich den kapitel überhaupt?
% Powoski schreibt auch noch zu den Faktoren, die Shift begünstigen..

Mit \textit{Language Shift} bezeichnen wir den Prozess, bei dem eine Sprache, die als Kommunikations- und Sozialisationsmittel für eine Gemeinschaft gedient hat, durch eine Andere ersetzt wird \cite[vgl.][]{Potowski13}.
Wenn die Weitergabe einer Sprache von einer Generation an der Nächsten aufhört, können wir sagen, dass die Gemeinschaft einen Shift abgeschlossen hat /durchlaufen hat.

Mehrere Faktoren können einen Language Shift begünstigen.
Ähnlich wie beim Code-Switching ist eine häufige Motivation hier die Prestige.
Mit der Zeit wird zu der prestigeträchigere Sprache oder Varietät übergegangen.
Die Personen werden besser angesehen, haben bessere Aussichten auf dem Arbeitsmarkt, etc.

%Fix me: bisschen chaotisch, komisch formuliert und ohne wirklich ein Ziel zu haben;
In Situationen der Migration beobachten wir häufig einen Language Shift.
In der 3. Generation Migrantinnen ist es schwierig, die Herkunftssprache weiter zu behalten, wenn nicht starke Netzwerke dafür sorgen, dass sie weiter verwendet wird.
Tendenziell ist der Erwerb der Landessprache ein Zeichen für gute Integration in die neue Gesellschaft und ermöglicht einer den Zugang zu mehreren Ressourcen (Bildung, Arbeitsmarkt, Freizeitangeboten, etc.)

\subsection{Sozioindexikalität, indexikalische Felder und Stil}

Menschen sortieren alle Sachen, die sie umgeben, mehr oder weniger bewusst in Kategorien, damit sie kognitiv damit umgehen können.
%macht dieser Satz als Eröffnung Sinn? es wird nur indirekt wieder aufgegriffen. und ich hab dafür erstmal keine quelle

Primäre Sozioindexikalität bezeichnet die Zugehörigkeit/Zuweisung einer Person zu bestimmten sozialen/Bevölkerungsgruppen (z.B. Frau, spanisch sprachiger Herkunft, Arbeiterin). %unklar welche die Quelle ist. Zentella97??
Diese passiert nach ``objektiven Kriterien'' und wird nicht durch die Sprecherin selbst bestimmt und gesteuert.
Mit anderen Worten: ``A first-order index simply indexes membership in a population.'' \cite[vgl.][]{Eckert08}

%sekundäre Indexikalität
Jedoch ist für Penelope Eckert diese Kategoriezugehörigkeit nicht statisch (was der klassischen soziolinguistischen Interpretation entsprechen würde).%reformulate sentence.; reference zu classical sociolinguistics?
Sprecherinnen sind Akteurinnen, die entsprechend ihrer Ideologie und Ziele ihre Sprache und ihr allgemeines Auftreten aktiv selbst mitgestalten.
Die bewusste Wahl und Aneignung (sprachlicher) Elemente in einem bestimmten Kontext und der damit assoziirten Werte und Eigenschaften werden wir als sekundäre Sozioindexikalität bezeichnen.
Eckert erklärt, dass die linguistischen Merkmale zu sekundären Indizien werden: sie zeigen nicht nur die Zugehörigkeit einer Person zu einer bestimmten Gruppe, sondern verwandeln sich in Indizien/Symbole für bestimmte Charaktereigenschaften, Ideologien, etc. \cite[vgl.][]{Eckert08}.
Die sekundäre Sozioindexikalität ist somit eine ideologische Analyse (sprachlicher) Merkmale: die sozialen Kategorien und die Zugehörigkeit zu denen werden nicht von außen zugeschrieben sondern durch soziale Akteurinnen geschaffen und interpretiert.
Es ist zudem immer möglich ein Index höherer Ordnung zu erschaffen: in dem Moment, in dem ein Index n-ter Ordnung idelogisch reinterpretiert wird, entsteht ein Index der (n+1)-ter Ordnung.

%Dieser Begriff ist eng an den Begriff ``Stil'' gekoppelt und in mehreren Zusammenhänge/für mehrere Zwecke können Beide synonym verwendet werden.

%Index. Feld?
Eckert bezeichnet ein indexikalisches Feld als ``a constellation of meanings that are ideologically linked'' \cite[vgl.][]{Eckert08}.
Der Wissenschaftlerin nach haben linguistische Variablen/Merkmale nicht feste Bedeutungen, sondern viel mehr damit assoziierte indexikalische Felder.
Wenn eine Variable/ein Merkmal benutzt wird, wird dann nicht eine statische Bedeutung aktiviert, sondern, je nach Kontext und Ideologie der Gesprächspartnerin, eine von mehreren möglichen Bedeutungen hervorgerufen(?), bzw. es wird, möglicherweise, auch eine komplett neue geschaffen \cite[vgl.][]{Eckert08}.
Die indexikalischen Felder sind ``fluid'': jede neu geschaffene Bedeutung kann potenziell das gesamte Feld verändern, in dem sie die ideologischen Verbindungen innerhalb des Feldes neu definiert. %vlt nach dem nächsten Satz erst?
Eckert stellt fest, dass die Bedeutung einer und der selben Variable/linguistischen Merkmals nicht uniform durch die Gesamtbevölkerung sein kann, da diese von verschiedenen Menschen in unterschiedlichen Situationen und zu verschiedenen Zwecken benutzt werden kann \cite[vgl.][]{Eckert08}.
%``Since the same variable will be used to make ideological moves by different people, in different situations, and to different purposes, its meaning in practice will not be uniform across the population.'' \cite[vgl.][S.466]{Eckert08}

%Stil
Der Begriff ``Stil'' ist eng an den Begriff Sozioindexikalität gekoppelt und in mehreren Zusammenhänge/für mehrere Zwecke können Beide synonym verwendet werden.
Eckert erklärt, dass sie, entgegen der weit verbreiteten Ansicht, nicht die Meinung vertritt, ``Stil'' wären verschiedene Weisen, die selben Sachen zu sagen.
Diese verschiedenen Weisen gehen aus den verschiedenen Identitäten und Ideologien der Sprecherinnen hervor und bedeuten, dass die Sprecherinnen, potenziell, auch ganz verschiedene Sachen zu sagen haben können \cite[vgl.][]{Eckert08}.
``Stil'' ist eigentlich viel breiter gefasst: dadrunter werden nicht nur die gewählten linguistischen Merkmale verstanden, sondern diese werden mit weiteren stilistischen Systemen kombiniert (z.B. Kleidung etc.)
Zugrunde der stilistischen Praxen (es sind sowohl die Produktion als auch die Interpretation von Stils gemeint) liegen Ideologien, erläutert Eckert.
Jede stilistische Entscheidung ist das Ergebnis der Interpretation der sozialen Welt und der Bedeutungen deren Elemente\cite[vgl.][]{Eckert08}.
Die Auswahl bestimmter Merkmale/Features für die eigene stilistische Representation hängt von der bisherigen Erfahrung mit stilistischen Systemen ab.
Je nach Erfahrung nehmen Menschen gewisse Unterschiede stärker wahr als andere.
Wenn ein Merkmal als solches wahrgenommen und mit Bedeutung ausgestattet wird, können sich Sprecherinnen entscheiden, dieses in die eigene stilistische Representation einzubauen oder eben nicht.
Die Entscheidung für oder gegen bestimmte linguistische Features verläuft in der Regel weniger bewusst als die Entscheidung für oder gegen Elemente anderer stylistischen Systemen (z.B Kleidung).
Jedoch verläuft die Bedeutungszuschreibung an Elementen und die Wahl und Kombinieren dieser Elemente ähnlich für alle stilistischen Systeme.

\subsection{Spanisch in den USA}

Speech patterns (Vortrag Spanish in NYC):
\begin{itemize}
  \item bilingualism: 63\% census respondents (2007) who speak Spanish at home also report that they speak English well or very well
  \item shift from Spanish to English after 2nd generation migrants
  \item Spanish speaking community survives due to new migrations
\end{itemize}


\begin{comment}
Begriffe:
* code switching
(* borrowings)
* language shift
* Sozioindexikalität
* Indexical field?
* Medientheorie?
\end{comment}

\begin{comment}

Grundlagen
----------
%Language Shift
Language Shift (Spanish -> English) : Speech community of a language shifts to speaking another language (source? Thomason?)
Can we talk about language shift in this context?
Who is shifting? The editors of the magazine? Or do they induce language shift?
Should that be a focus at all? Or focus more on communities of practice/style projection through language/sozioindexikalität, etc.

Ein Shift Spanisch->Englisch ist nicht nur mit Prestige zu erklären, sondern auch mit allgemeinen Anpassungs/Integrationsprozessen:
Wer in den USA Englisch kann, hat bessere Chancen einen Job zu finden, etc.


%Style/Socioindexicality
*style/socioindexicality: adopt linguistic forms of groups one sympathises with/has intense
relationships with
*``Although overt norms favor standard speech, powerful covert norms encourage
group members to remain faithful to group codes, linguistic and otherwise.'' (Zentella 2007)
** Dominicans keep their low prestige variety, because ``overcorrectness'' is associated with
``feminine''

Stichwörter:
Ideologie, Sprecher\_innenidentität

Stil (Eckert?)

Kategorien: ``Ich räume alles auf''
Menschen sortieren Sachen in Kategorien, damit sie kognitiv damit umgehen können
Kleidungsstil: indexikalische Bedeutung -> vermittelt einen bestimmten Hintergrund (zB Sakko->Prof)

[Eckert08]
``Sociolinguists generally think of styles as different ways of saying the same thing.''
``style is not a surface manifestation, but originates in content.''
``Different ways of saying things are intended to signal different ways of being,
which includes different potential things to say.''

``Ideology is at the center of stylistic practice:''
``every stylistic move is the result of an interpretation of the social
world and of the meanings of elements within it''

``styles and stylistic moves can be quite local, ultimately they connect the
linguistic sign systematically to the political economy and more specifically to
the demographic categories''

``By stylistic practice, I mean both the interpretation and the production of
styles, for the two take place constantly and iteratively''

``the noticing of the style and the noticing of the group or individual that
uses it are mutually reinforcing, and the meaning of the style and its users
are reciprocal.''

``an ideological link is constructed between the linguistic and the social.''

``This process of selection is made
against a background of previous experience of styles and features; a stylistic
agent may be more attuned to particular kinds of differences''

``Once the agent
isolates and attributes significance to a feature, that feature becomes a resource
that he or she can incorporate or not into his or her own style.''
``The occurrence
of that resource in a new style will change the meaning both of the resource and
of the original style''

``they segmented the new wave style into meaningful elements''
``these girls’ stylistic moves are local,
they deal with fundamental issues related to gender and adolescence: innocence
and independence.''

``making pegging available for segmentation and (re)interpretation.''
``the resources they were using
were available and salient because they had been established at a much more
generalized cultural level.''

``linguistic style is rarely constructed in as intentional a fashion
as clothing style, it is similarly a process of bricolage.''

``distinct styles associated with different communities of practice''
--> aber nach der Definition der Communities of Practice würde ich sagen, sind sie für mich eher unbrauchbar..

``But just as women are not making direct gender claims when they use female-
led changes, burnouts are not making direct urban claims when they use
urban-led changes.''
``what they associated with urban life and urban kids.''
-> indirect connections between social categories and linguistic features

``The urban kids that they identified with were white kids who knew
how to cope in the dangerous urban environment – kids they saw as autonomous,
tough, and street-smart. Presumably in adopting urban forms, suburban kids
were affiliating with those qualities, not claiming to be urban.''

```‘acts of identity’ (Le Page and Tabouret-Keller 
1985) are not primarily a matter of claiming membership in this or that group
or category as opposed to another, but smaller acts that involve perceptions of
individuals or categories''

``systematically related to the macrosociologist’s categories
and embedded in the practices that produce and reproduce them. It is in the
links between the individual and the macrosociological category that we must
seek the social practices in which people fashion their ways of speaking, moving
their styles''


%Sozioindexikalität
1st order
gegeben: Gruppenzuweisung
nicht durch die Sprecher\_innen selbst bestimmt
``association by social actors of a linguistic form/variety with some meaningful social group (female, spanish, working class..)''
-> Manche features eignet man sich an oder gewöhnt sich ab aus ideologischen Gründen

2nd order
describing, noticing.. of the 1st order indexicality;
variaties are differently notices, rationalized, ... from land to land and community to community
ideological analysis: social categories are locally created by social actors and discoverable by analysis rather than given;
wie sich Menschen projizieren, durch Stil selber kreieren, Performance, bewusster, intenationaler Einsatz/Gebrauch

Kurios: Theaterspieler\_innen, Schriftsteller\_innen führen Figuren durch sekundäre Indexikalität ein -> die werden mit Attributen ausgestattet

früher: alle möglichen Kategorien erstellt: black/white/old/young/hispanic/...
aber nicht berücksichtigt, was die Individuen machen, um sich selber darzustellen
Individuen nicht als Akteure berücksichtigt


%indexical field - definition
``a constellation of meanings that are ideologically linked''
``an embodiment of ideology in linguistic form''
``not a static structure''



% Spanglish /Code switching!
``Powerful socioeconomic and cultural forces stimulate borrowing whenever two cultures are in contact;
the borrowing by the subordinate group's language from that of the dominant group is always
significantly greater'' [Zentella90]
Factors, which facilitate borrowings:[Zentella90]
 1) Most items reflect a cultural reality that is new or different, e.g., \[kei/keike/keiki\]
``cake'', \[bobipín\] ``bobby pin'', \[ŷins/bluŷines\] ``(blue)jeans.''
 ``la boila'' (boiler) [Zentella97]
 2) Others may be facilitated or even triggered by a similarity in the phonological and/or morphological
 structure of a word in the dominant language that makes it sound like a possible subordinate word,
 matre(s) ``mattress'' is similar to that of madre(s) ``mother(s)''
 ZB ``libreria'' (Buchladen) für Bibliothek (library)
 3) reduction in the number of syllables, e.g., \[beis\] replaces florero for ``vase''
 4) prestige
 5) Anglicisms can play the role of neutralizer between competing dialectal variants because the
 prestigious outside language acts as the lingua franca that resolves the conflict without favoring one
 group at the expense of the other.

-> Zentellas Konzept ``linguistic insecurity'': ``consider their dialect inferior to others''[Zentella07]

``but it seems that the longer a group has been in the U.S., the less they use the homeland's term''
``second generation Cubans incorporated more loans in their speech than members of the first generation.''
``Language proficiency appears to be a more relevant factor than age'' [Zentella90]

%Borrowings
lexical borrowing - both form and content are new to the borrowing language [Thomason03]

%Media
% erstmal: außen vor lassen; wenn das hier aufgemacht wird, kriegt die arbeit ne völlig neue dimension
* media: promotes wide spread/conservative varieties

* influence of the media: ``they do contribute to popular acceptance and use of some new vocabulary'' [Zentella90]
(siehe oben bei Spanglish)
-> Spielen auch für leveling eine Rolle: versuchen neutrale Varianten zu nutzen, um möglichst mehr Menschen zu erreichen;

This is especially true regarding language, because, as Trudgill notes, «the media... have almost no effect at all in
phonological or grammatical change» (1984:61), although they do contribute to popular acceptance and
use of some new vocabulary. [Zentella90]

Marshall McLuhan: hot and cold media
hot media: ``Hot media do not leave so much to be filled in by audience.
Hot media are, therefore, low in participation or completion by audience.'' [Willie79]

``Communities and individuals are bombarded constantly with messages from a multitude of sources including TV, billboards, and magazines, to name a few. These messages promote not only products, but moods, attitudes, and a sense of what is and is not important. Mass media makes possible the concept of celebrity: without the ability of movies, magazines, and news media to reach across thousands of miles, people could not become famous. In fact, only political and business leaders, as well as the few notorious outlaws, were famous in the past. Only in recent times have actors, singers, and other social elites become celebrities or “stars.” '' (http://www.cliffsnotes.com/sciences/sociology/contemporary-mass-media/the-role-and-influence-of-mass-media)


%http://www.pbs.org/speak/ahead/mediapower/media/#talk
% Jack Chambers
% “Talk the talk?” [author’s title “TV and Your Language.”]. Website “Do You Speak American?”
% McNeil - Lehrer Productions. http://www.pbs.org/speak/ahead/mediapower/media
% 2005

A final common assumption is that the media leads language changes. In fact, it belatedly reflects the changes.

The same fallacy seems to underlie the casual assumption that the mass media drives all kinds of language changes.

If the mass media can popularize words and expressions, then “presumably” it can also spread other kinds of linguistic changes. We generalize from one limited effect to a host of others.


%%%%%%%%%%%%%%%%%%%%%%%%%%%%%%%%%%%%%%%%%%%%%%%
%Belege

[Zentella07]


%Language and Identity
the defining role of language networks in identity, i.e., “identity is defined as the linguistic con-
struction of membership in one or more social groups or categories” (Kroskrity 2001: 107)
Latina/o identity in the USA is often linked to Spanish, presumed
to be the heritage language of more than 40 million people with roots in 20
Spanish-speaking Latin American nations, including Puerto Rico.
[Zentella07]

distinct ways of being Latina/o are shaped by the dominant language ideology
that equates working-class Spanish speakers with poverty and academic failure,
[Zentella07]
--> also eine Erklärung, warum sie zu %Language Shift tendieren würden

The fate and form of the languages spoken by US
Latinas/os will be determined in part by the ways in which they respond to the
construction of their linguistic identities as a group and as members of distinct
speech communities, and those responses in turn can have a significant impact
on Latina/o unity.

%Latina/o Linguistic Capital
%Racism
In
the USA, where race has been remapped from biology onto language because
public racist remarks are censored, comments about the inferiority and/or
unintelligibility of regional, class, and racial dialects of Spanish and English
substitute for abusive remarks about color, hair, lips, noses, and body parts, with
the same effect.

--> noch eine Erklärung für %Language Shift

no one expects you to be able to change your color, but you are expected to
change the way you speak radically to earn respect (Urciuoli 1996).

These attitudes are communicated in everyday conversations and promoted by
the media and public institutions, but some groups of Latinas/os are more affected
than others.

%%%%%%%%%%%%%%%%%%%%%%%%%%%%%%%%%%%%%%%%%%%%%%%%%%%%%%%%%%%%%%%
[Thomason03]

When fluent speakers of language A incorporate features into A
from another language, B, the first and most common interference features
will be non-basic lexical items, followed (if contact is sufficiently intense) by
structural features and perhaps also basic vocabulary.

%borrowing vs shift-induced interference
These two types of interference were characterized in Thomason and
Kaufman (1988) as borrowing, in which features are incorporated into A by
native (L1) speakers of A, versus shift-induced interference, in which a group of
L2 learners of A carry over features from B (their L1) into A during a process
of shift from B to A.

--> i would say, we have to do with borrowings

Linguistic factors in linguistic interference:
* universal markedness
* typological distance between source and recipient language

Social factors:
* imperfect learning
* intensity of contact
* speaker attitudes (prestige, ..)

--> weiß nicht in wie fern all das brauchbar ist, weil ich eigentlich der Meinung bin, dass in diesen Artikeln die Sprache bewusst in der Form eingesetzt wird, also dass es sich um ``speakers' attitudes'' und Sozioindexikalität handelt.
Vlt auch ein bisschen intensity of contact..

Features that
are deeply embedded in elaborate interlocking structures are in general less
likely to be borrowed, because they are less likely to fit into the recipient
language’s structures; that is why the lexicon, which for all its structure is less
highly organized than other grammatical subsystems, is borrowed first, and it
is why inflectional morphology tends to be borrowed last. But highly integrated
features may be borrowed readily between systems that are typologically very
similar;

And when contact is intense enough, there appear to be no absolute linguistic
barriers at all to borrowing

%code switching --> USE!
Code-switching, as used in this section, includes both intrasentential switching
(sometimes called code mixing) and intersentential switching


Code-switching is a (perhaps the) major route by which loanwords enter a  %loanwords
language. It surely plays a role in at least some kinds of structural borrowing
as well, although the more dramatic kinds of structural interference are prob-
ably likely to result from code alternation instead (see section 2.2 below).

I believe that is impossible in principle and in practice to draw an
absolute boundary between code-switching and borrowing. % das ist mir auch zu feine Unterscheidung, ich würde damit nichts anfangen

A code-switched word or other morpheme becomes a borrowing if
it is used more and more frequently – with or without phonological adaptation
– until it is a regular part of the recipient language, learned as such by new
learners.

% 2. Mechanisms of Intereference

Mechanisms of contact-induced change fall into four categories.
* one set of mechanisms comes into play when the implementers of a
change are bilingual in both source and recipient language
* while the other set comprises second language
acquisition strategies
* A third category, “negotiation,” seems to
overlap with both of these types
* and the fourth category has to
do with more or less conscious and deliberate decisions by speakers to imple-
ment language change

%borrowing and new word inventions
The addition by borrowing of a new word for a new
concept, like bok choy in English, must begin with a single use and continue
with increasing usage by the innovating speaker(s) and by other speakers, and
the addition by invention of a new word, like photocopy, must follow the same
path.

These parallels are hardly surprising, given that all speakers draw on a
variety of repertoires, typically characterized as styles, registers, and dialects,
in using a single language. If there is no evidence to the contrary – and certainly
no convincing evidence has been presented – then it is surely most reasonable
to assume that speakers whose repertoires include more than one language
will employ the same strategies in deploying their linguistic resources. Change
resulting from code-switching between different languages does, of course,
differ from change via diffusion from another register and dialect borrowing;
but, as noted at the start of this chapter, the differences are a matter of degree,
not of kind.

%Borrowing vs code-switching: matter of frequency
morphemes, both lexical and grammatical, would be introduced directly via
code-switching, changing from code-switches to borrowings through increas-
ingly frequent usage by code-switching speakers and then (if not all members
of the speech community engage in code-switching) by adoption by other
speakers;

%Deliberate change
types of deliberate decisions:
* two-language mixtures created by bilinguals,
apparently to serve as a symbol of a new ethnic identity; (Chicana Spanish, "Kanakendeutsch"? meine Bsp)
* of speaker groups deliberately withholding their
“real” language from outsiders, using instead a distorted and simplified for-
eigner talk version that, in some cases, forms the basis for a trade pidgin
* creation of a secret language, either by phonological distortion [...] or lexical replacement.
* the motive for making a particular change has to do with emphasizing
in-group status, or differentness from other groups;

%%%%%%%%%%%%%%%%%%%%%%%%%%%%%%%%%%%%%%%%%%%%%%%
[Milroy93]

the strength of weak ties? (ein bisschen abseits)

fokus auf gender der sprecher\_innen;
kann ich nicht wirklich benutzen, weil ich nicht weiß, wer das geschrieben hat; aber die Zielgruppe ist definitiv gegendert..

``what are usually called low-prestige varieties can be maintained over generations as flourishing vernaculars.''
also kann Class und prestige nicht die einzige Erklärung für Language Shift und Loss sein

``strong informal social ties within
communities provide the mechanisms that enable speakers to maintain non-
standard dialects, rural or urban, despite intense pressure from the standard
language through routes such as the educational system and the media.''
kann wohl eher schwierig als Argumentation für mich benutzt werden

%vlt eher nicht; ich kann schwer sagen, was für sozialen Netzwerken die Leser_innen dieses Magazins angehören
``Close-
knit networks, which will of course vary in actual levels of density and multi-
plexity, are assumed to have the capacity to maintain and even enforce local
conventions and norms - including linguistic norms. It is, after all, remarkable
that stigmatized linguistic forms and low-status vernaculars can persist over
centuries in the face of powerful national policies for diffusing and imposing
standard languages,''
spannend, aber vermutlich nicht sehr nützlich in meinem Kontext

``close-knit network structure is associated with
language maintenance; the corollary to this is that a loose-knit network struc-
ture is associated with language change. We have argued in detail elsewhere
that where ties are relatively loose-knit, communities will be susceptible to
change originating from outside localized networks. These changes are not
necessarily in the direction of the standard''

``loose-knit structures with
socially and geographically mobile middle class speakers, and close-knit ties
with lower and upper class speakers.''
-->vlt spannend

``innovators are likely to be persons weakly linked to
such a group. Susceptibility to outside influence is likely to increase in inverse
proportion to the strength of the tie with the group''
vlt nochmal ``The strength of weak ties'' lesen.. wobei bin mir nicht sicher was es
bei der Arbeit helfen würde.. man kann eh nicht zuverlässig behaupten ob es da ein 
Netzwerk gibt und wie stark genau die Connections sind

``A ‘weak tie’ model of change thus seems able to account for some
instances of variation and change which are difficult to explain in terms of the
usual unqualified assumption that linguistic change is encouraged by frequency
of contact and relatively open channels of communication''

%%%%%%%%%%%%%%%%%%%%%%%%%%%%%%%%%%%%%%%%%%%%
[Eckert08]

``Thus variation constitutes an indexical system
that embeds ideology in language and that is in turn part and parcel of the
construction of ideology.''

``the variation (and the entire linguistic) enterprise must
be integrated into a more comprehensive understanding of language as social
practice''

``Peter Trudgill
(1972), for instance, called upon the perceived toughness of working-class men
as a motive for middle-class men to adopt local working-class sound changes,
accounting for the upward spread of change''

agency and ideology: Why do speakers do what they do?

``Speakers’ agency in the use of variables has
been viewed as limited to making claims about their place in social space by
either emphasizing or downplaying their category membership through the
quantitative manipulation of markers.''
--> The old view: speakers' agency is ignored

``This generalization says nothing about the kinds of behaviors and ideologies that
underlie these patterns, what kinds of meaning people attach to the conservative
and innovative variant, who does and does not fit the pattern and why.''

%Sozioindexikalität

%1st order
``A first-order index simply indexes membership in a population''

%2nd order
``But the social evaluation of a population
is always available to become associated with the index and to be internalized
in speakers’ own dialectal variability to index specific elements of character. 2 At
that point, the linguistic form becomes a marker, a second-order index,''

``Participation in discourse involves a continual
interpretation of forms in context, an in-the-moment assigning of indexical
values to linguistic forms.''

``an nth order usage, is always available for reinterpretation – for the
acquisition of an n + 1st value.''

``fluid and ever-changing ideological field.''
``The emergence of an n + 1st indexical value is the result of an ideological move,''

%indexical field
``Variables have indexical fields rather than fixed meanings because speakers use
variables not simply to reflect or reassert their particular pre-ordained place
on the social map but to make ideological moves.''
``The use of a variable is not
simply an invocation of a pre-existing indexical value but an indexical claim
which may either invoke a pre-existing value or stake a claim to a new value.''

``The
pejoration of many English words referring to females is a perfect example of the
systematic absorption of ideology into the lexicon.''

``(1) the term was used
repeatedly in negative utterances about specific women or categories of women;
and (2) the utterances of those who said such negative things were registered
disproportionately.''

und dann kommt Wertung dazu:
``a negative evaluation of a speaker using
the apical variant might be that the speaker is inarticulate or lazy, a favorable
evaluation might be that he or she is unpretentious or easygoing''
--> interpretation depends on:
* perspective of the hearer
* style in which it is embedded

``Since the same variable will be used to make ideological moves by different
people, in different situations, and to different purposes, its meaning in practice
will not be uniform across the population.''

%%%%%%%%%%%%%%%%%%%%%%%%%%%%%%%%%%%%%%%%%%%%
[PloogReich06]

--> Koineisierung: Prinzipien
``1. Transparenz: Im Kontakt zwischen zwei Sprachen/Varietäten setzt sich innerhalb einer Variable meistens diejenige Variante durch, die transparenter ist''
``2. Strukturelle Harmonie: Im Kontakt zwischen zwei Sprachen/Varietäten setzt sich innerhalb einer Variablen meistens diejenige Variante durch, die in beiden Systemen verwendet wird''



%%%%%%%%%%%%%%%%%%%%%%%%%%%%%%%%%%%%%%%%%%%%
%Passt nicht rein
%oder wo anders

%Communities of practice
can we say that the target reader group of these magazines is a community of practice?
--> bisschen weit hergeholt; passt meiner Meinung nach nicht zu der Definition von Community of Practice

%Language Prestige
English is the more prestigious language;
the dominant(both in numbers and power) group of people in the USA speaks English
serves to avoid misunderstandings/clarify (zb cooking recipes)
--> das kommt alles zur Interpretation!


Mögliche Änderungen:
* feature addition
* feature replacement
* feature loss

Das Problem ist:
* wir wissen nicht wer genau die Artikel schreibt;
* nur wer die Zielgruppe ist (mehr oder weniger)
* wir können schwer Theorien anwenden, die Sprecher\in-zentriert sind, da es unklar ist, wer das geschrieben hat; (und auch irrelevant)
* also eher schwer zb Gender/Social networks mit ein zu beziehen
* Aber was scheint eine Rolle zu spielen sind: Code Switching; Borrowings; Language Shift; Sozioindexikalität (Style)


Allgemeine Theorie

Linguistische Wandelprozesse kann man durch
* Interferenz
* allgemeine Demarkierung (zu erwarten in Kontaktsituationen)
erklären

Sprachliche Prozesse:
* transparenz fördernde --> Hörer orientiert --> bei extremer Mehrsprachigkeit
* Ökonomie fördernde --> Sprecher orientiert --> eher bei größeren Einsprachigkeit

[Klein92]
beim L1-Erwerb muss Kognition + soz. Identität dazu erworben werden
beim L2-Erwerb nicht; aber zieht wohl auch soz. Entwicklung mit sich
%

%%%%%%%%%%%%%%%%%%%%%%%%%%%%%%%%%%%%%
[Thomason03]


feature addition
        replacement
        deletion

--> sind meine Sätze Beispiel für einen dieser Prozesse?
kp, aber ich würd sagen, das ist nicht der Fokus und würd mich nicht explizit mit beschäftigen

(By contrast, if people who are not fluent speakers of A introduce features
into A from another language, B, the first interference features (and usually
the most common ones overall) will not be lexical, but rather phonological and
syntactic.--> curious fact, but not really useful in the context


%code-switching vs shift-induced interference
lexical items predominate in code-switching, while
phonological and syntactic features predominate in shift-induced interference.

%code alternation
%e.g. using one language at home, another at work
%also brauch ich erstmal nicht
streng situationell --> funktional

In many cases the two languages
are used by the same speaker with different interlocutors, often monolinguals;

the unconscious and involuntary incorporation of foreign structural features into
one of a bilingual’s languages fits very well with psycholinguists’ finding that
“bilinguals rarely deactivate the other language totally”

%passive familiarity
Sometimes interference features are introduced by speakers whose competence
in the source language is strictly passive – that is, a speaker may borrow a
feature from a dialect or language that she or he does not speak actively at all.
(e.g. from African American Vernacular English)

%Deliberate change
Theories of language change rarely or never allow for the possibility of deliber-
ate change, except in such trivial cases as the conscious adoption of loanwords
and even new sounds in words of foreign origin

% definitions for language attrition, language death
% not sure it has something to do with the current topic
as the “loans to loss” model, extensive
borrowing leads eventually to language loss;

the adoption of lexical
and structural features from the dominant language – that is, convergence
toward the dominant language

in general a change is more likely to occur if independ-
ent forces are pushing in the same direction.

Bewegende Macht: Sprecher\_innen-Kreativität

%%%%%%%%%%%%%%%%%%%%%%%%%%%%%%%%%%%%%%%%%%%%%%%%%%%%%%%%%%%%%
[Krefeld04]

Modelierung migratorischen Kommunikationsräume

Der Raum ist "Produkt seiner interagierenden Bewohner"

"Welt in aktueller" und "Welt in potentieller Reichweite" --> individuell, bei jeder Person anders

"Identische Räume darf man daher grundsätzlich niemals voraussetzen; vielmehr ergeben sich mehr oder weniger große gemeinschaftliche Teilräume aus dem Gebrauch bzw. aus der Entwicklung gemeinsam verfügbarer Idiome und Varietäten"

"System räumlicher Gliederungen [...] Differenzierung der Intimität und Anonymität, der Fremdheit und Vertrautheit, der sozialen Nähe und Distanz" [Schütz/Luckmann 1979, 68]

Dimensionen des kommunikativen Raums:
* Sprache
* Sprecher\_in
* Sprechen

Räumlichkeit der Sprache:
* Umgebung und umgebungsspezifische Varietäten (Arealität) (z.B. Land/Stadt, West/Ost)
* Territorium und seine Sprache(n) (Territorialität) (Dialekten? Sprachen?)

Arealität: "Bindung sprachlicher, in der Regel dialektarel Merkmale an einen spezifischen Ort"
Territorialität: "staatlich garantierte und nicht selten juristisch sanktionierte Geltung einer Staatssprache in einem administrativ scharf begrenzten Gebiet"

Räumlichkeit der Sprecherin (und der Hörerin):
* Provenienz (wo kommt sie her?)
* Mobilität (wohin/wie hat sie sich bewegt)
* Repertoire

Räumlichkeit des Sprechens:
* situative Positionalität
* Interaktion

"Relationen zwischen Provinenz und sozialer Hierarchie, d.h. die Formen sozialer Mobilität zu beachten"

Positionalität des Sprechers:
relativen Nähe/Distanz --> "Dabei ist die soziale Nähe, d.h. die Vernetzung mit unmittelbar erreichbaren Kommunikationspartnern (aus peer-groups, bzw. Familien), von der pragmatischen Nähe, der Situationalität, zu unterscheiden"

Aufgabe der Raumlinguistik: "wie die Sprecher selbst ihren (gemeinsamen) Raum konstruiert haben und wie sie ihn selbst wahrnehmen"

Glossotop: "Analog dazu bezeichne ich den Ort einer mehrsprachigen Kommunikationsgemeinschaft als 'Glossotop'"
--> "Gesamtheit der Regularitäten (...) die den lokalen Gebrauchder sprachlichen Varietäten in einer bestimmten lebensweltlichen Gruppe (zum Beispiel einer Familie, einer Nachbarschaft, einer peer-group etc.) steuern."

"wer spricht mit wem in welcher Varietät/Sprache?" --> soziale Netze ([Immacolata Tempesta 2000])

%%%%%%%%%%%%%%%%%%%%%%%%%%%%%%%%%%%%%%%%%%%%%%%%%%%%%%%%%%
[Milroy93]

--> introduces the network concept; the strength of weak ties; gender variable

Untersucht: impact of speakers' variables social class, social network and gender on linguistic variability and change
``discuss the way in which these extralinguistic variables are interrelated''

``gender difference is often prior to social class in accounting for sociolinguistic variation''

selten jemand hinterfragt welche Variablen werden bei Untersuchungen gewählt und warum
bis dato (1993): class is primary variable <- critism

``The form which linguistic gender-marking has commonly been interpreted as
taking is for women to approximate more commonly than men of similar s t a t u s
to the (so-called) prestige norm. But as Coates (1986) has pointed out, no satis-
factory explanation has emerged of why women should be more oriented than
men to a prestige norm.''
naja, mir fällt eine ziemlich lange Erklärung dazu ein.
aber die hier argumentieren gleich, dass es anders rum funktioniert...

``and the relation of gender to social class and to prestige patterns is by no means consistent
or predictable from the usual assumptions about females preferring higher social-class norms.''

``The generalization suggested is not that females
favour prestige forms, but that they create them; i.e., if females favour certain
forms, they become prestige forms. In these developments, both class and
gender are implicated, but gender is prior to class.''
eine Erklärung dafür wäre, dass Frauen den Hauptteil der Care-Arbeit leisten,
also sind sie diejenigen, die auch die Sprache weiter geben

``Some
linguists have apparently assumed that ‘social network’ is concerned only with
strong ties and is roughly synonymous with ‘peer-group’,''

``some linguists have seemed to believe that
individuals may o r may not possess a ‘social network’, when i n fact all individ-
uals ar e embedded in networks.''

%%%%%%%%%%%%%%%%%%%%%%%%%%%%%%%%%%%%%%%%%%%%%%%%%%%%%%%%%%%%%
[Eckert08]

field of potential meanings -> indexical field

``Thus variation constitutes an indexical system
that embeds ideology in language and that is in turn part and parcel of the
construction of ideology.''

``the variation (and the entire linguistic) enterprise must
be integrated into a more comprehensive understanding of language as social
practice''

Objective of the paper:
``propose an approach to the study of social meaning in variation that builds
upon linguistic-anthropological theories of indexicality, and most particularly
Michael Silverstein’s (2003) notion of indexical order.''
``the meanings
of variables are not precise or fixed but rather constitute a field of potential
meanings – an indexical field, or constellation of ideologically related meanings,''
``The field is fluid, and each new activation has the potential to change the field by
building on ideological connections''

``This very local construction of meaning in variation, the recruiting of
a vowel as part of a local ideological struggle, suggested that variation can be a
resource for the construction of meaning and an integral part of social change. But
this power of variation was lost in the large-scale survey studies of sound change
in progress in the years that followed, as social meaning came to be confused
with the demographic correlations that point to it.''

``variables index demographic categories not directly but indirectly (Silverstein
1985), through their association with qualities and stances that enter into the
construction of categories.''

schemata of speaking (Piaget 1954): we notice differences and attribute meaning to them

``Style has a similar function in everyday language,
picking out locations in the social landscape such as Valley girls, cholos,''

variables: components of styles

%Persona style
``at this level that we connect linguistic styles with other stylistic systems
such as clothing and other commoditized signs and with the kinds of ideological
constructions that speakers share and interpret''

`The connection between the sound of rhotacization and oiliness and
between oiliness and a specific persona is a particularly striking example
of iconization''

``At the same time,
this distancing process reinscribes the old types by creating a new space
in the social map in opposition to them. Meanwhile, yuppies’ adoption
of a non-Beijing feature, full tone, projects them out into transnational
space.''

%indexical field
``Variables have indexical fields rather than fixed meanings because speakers use
variables not simply to reflect or reassert their particular pre-ordained place
on the social map but to make ideological moves.''
``The use of a variable is not
simply an invocation of a pre-existing indexical value but an indexical claim
which may either invoke a pre-existing value or stake a claim to a new value.''

``The
pejoration of many English words referring to females is a perfect example of the
systematic absorption of ideology into the lexicon.''

``(1) the term was used
repeatedly in negative utterances about specific women or categories of women;
and (2) the utterances of those who said such negative things were registered
disproportionately.''

``listeners develop an impression of a speaker based on general speech
style and the content of the utterance, and interpret the particular use of (ING)
on the basis of that impression.''
``associate the velar variant with education, intelligence, and articulateness.
Central to this perception is a view of the velar form as a full form and therefore
effortful and of the apical form as a reduced form, hence a sign of lack of effort.''

und dann kommt Wertung dazu:
``a negative evaluation of a speaker using
the apical variant might be that the speaker is inarticulate or lazy, a favorable
evaluation might be that he or she is unpretentious or easygoing''
--> interpretation depends on:
* perspective of the hearer
* style in which it is embedded

``Since the same variable will be used to make ideological moves by different
people, in different situations, and to different purposes, its meaning in practice
will not be uniform across the population.''

not answered: is stylistic meaning compositional?

%%%%%%%%%%%%%%%%%%%%%%%%%%%%%%%%%%%%%%%%%%%%%%%%%%%%%%%%%%%%%%%%%%
[PloogReich06]

Heterogenität: ``prinzipiell die Möglichkeit einer funktionalen Mehrsprachigkeit beinhaltet''

%Theorie
%Beschreibungsdimensionen sprachlicher Räume
* Strukturquellen
* kommunikative Bereiche
* pragmatische Anforderungen
* sozioindexikalische Einstellungen

%Strukturquellen
``default-Fall'':``die Tradierung eines weitgehend fixierten Systems''
-> Transferenzprozesse -> Innovation (``je nach theoretischem Format als Analogiebildung, Drift oder pragmatischer Wandel'')
``Innovation erfolgt aber auch ohne typologische oder einzelsprach-strukturelle Anleitung als Grammatikalisierung weitgehend universeller kognitiver Relationen''
``einen exklusiven Prozeß der Kreolisierung gar nicht gibt, die historischen Kreolsprachen vielmehr von Prozessn gebildet werden, die auch in anderen Sprachen beobachtbar sind''(s.227)

Interferenz:
* ``zwischen genetisch und typologisch meist völlig verschiedenen Sprachen''
* ``zwischen Sprachen derselben Familie (Portugiesisch, Italienisch und Spanisch in Sao Paolo)''
* ``zwischen Varietäten derselben Sprache (zwischen Español Andino und Español Limeño, zwischen Nordestino und Paulistano, zwischen FPA und Standardfranzösisch)'' (s.228)
--> Koineisierung: Prinzipien
``1. Transparenz: Im Kontakt zwischen zwei Sprachen/Varietäten setzt sich innerhalb einer Variable meistens diejenige Variante durch, die transparenter ist''
``2. Strukturelle Harmonie: Im Kontakt zwischen zwei Sprachen/Varietäten setzt sich innerhalb einer Variablen meistens diejenige Variante durch, die in beiden Systemen verwendet wird''

Transparenz ist hörerorientiert

syntaktische Transparenz: typologische Konsistenz -> ``die Rekurenz eines bestimmten Konstruktionsprinzips in unterschiedlichen syntaktischen Domänen''
``funktionellen Vorteil der Einfachheit'' (s.228-229)
morphologische Transparenz: ``sind also freie Formen und Affixe in Silbengröße transparenter als segmentale und subsegmentale Marke oder modifikatorische Kodierung'' (``Portmanteaumorphe sind also ebenso ausgeschlossen wie Polysemie'') (s.230)
``The syllable is an important perceptual unit. Thus, morphotactic transparency is enhanced if syllable boundaries coincide with morphological boundaries (Dressler et al. 1987)'' (Fußnote s.230)

phonologische Transparenz: absolute und relative Merkmalhaftigkeit
``So wird eine phologische Opposition von einem Lerner dann sofort wahrgenommen, wenn sie auch in seiner Sprache angelegt ist.''(s.232)

Pendel zwischen Transparenz und Ökonomie: ``Keine natürliche Sprache realisiert einen der beiden Idealtypen vollständig'' (s.233)

``Im Sinne einer Sprachökologie (Haugen 1971, 325-339, Mufwene 2001) lassen sich Kommunikationsräume in Einzelbereiche differenzieren, die kommunikative Handlungen für ganz bestimmte soziale Funktionen spezifizieren.'' (s.234)

``Dichte fördert das Zusammentreffen und den Austausch unter Stadtbewohnern, indem sie das tatsächliche, geographische Territorium des Einzelnen einschränkt und Mobilität bewirkt. Vom Einzelnen erfordert sie Flexibilität und provoziert somit kurzfristig den Anstieg der individuellen Mehrsprachigkeit'' (s.237)

``Sozioindexikalische Einstellungen beeinflussen den Koinisierungsprozess insofern, als Strukturmerkmale grundsätzlich gewertet werden..'' (s.241)

%%%%%%%%%%%%%%%%%%%%%%%%%%%%%%%%%%%%%%%%%%%%%%%%%%%%%%%%%%%%%%%%%%%

Media
-----
This is especially true regarding language, because, as Trudgill notes, «the media... have almost no effect at all in
phonological or grammatical change» (1984:61), although they do contribute to popular acceptance and
use of some new vocabulary. [Zentella90]

Marshall McLuhan: hot and cold media
hot media: ``Hot media do not leave so much to be filled in by audience.
Hot media are, therefore, low in participation or completion by audience.'' [Willie79]

(http://www.cliffsnotes.com/sciences/sociology/contemporary-mass-media/the-role-and-influence-of-mass-media)

``Communities and individuals are bombarded constantly with messages from a multitude of sources including TV, billboards, and magazines, to name a few. These messages promote not only products, but moods, attitudes, and a sense of what is and is not important. Mass media makes possible the concept of celebrity: without the ability of movies, magazines, and news media to reach across thousands of miles, people could not become famous. In fact, only political and business leaders, as well as the few notorious outlaws, were famous in the past. Only in recent times have actors, singers, and other social elites become celebrities or “stars.” '' (http://www.cliffsnotes.com/sciences/sociology/contemporary-mass-media/the-role-and-influence-of-mass-media)

%The limited effects theory
The limited‐effects theory argues that because people generally choose what to watch or read based on what they already believe, media exerts a negligible influence.
%Criticism
First, they claim that limited‐effects theory ignores the media's role in framing and limiting the discussion and debate of issues. How media frames the debate and what questions members of the media ask change the outcome of the discussion and the possible conclusions people may draw. Second, this theory came into existence when the availability and dominance of media was far less widespread.

%The class-dominant theory
The class‐dominant theory argues that the media reflects and projects the view of a minority elite, which controls it.
For example, owners can easily avoid or silence stories that expose unethical corporate behavior or hold corporations responsible for their actions.
The issue of sponsorship adds to this problem. Advertising dollars fund most media.
elevision networks receiving millions of dollars in advertising from companies like Nike and other textile manufacturers were slow to run stories on their news shows about possible human‐rights violations by these companies in foreign countries.

%The culturalist theory
The culturalist theory, developed in the 1980s and 1990s, combines the other two theories and claims that people interact with media to create their own meanings out of the images and messages they receive.
This theory sees audiences as playing an active rather than passive role in relation to mass media.
Therefore, culturalist theorists claim that, while a few elite in large corporations may exert significant control over what information media produces and distributes, personal perspective plays a more powerful role in how the audience members interpret those messages.

%http://www.pbs.org/speak/ahead/mediapower/media/#talk
% Jack Chambers
% “Talk the talk?” [author’s title “TV and Your Language.”]. Website “Do You Speak American?”
% McNeil - Lehrer Productions. http://www.pbs.org/speak/ahead/mediapower/media
% 2005
Second, the lasting power of words that spread via the mass media has nothing to do with the various media themselves. Punk’d lasted just two seasons (2002-03). While it lasted, its name was raised into common parlance. What are the chances a word will persist for another five to 10 years? Not good. In buzzwords as in outré attire, there is a direct relationship between the height of the craze and the decline into oblivion. Fads mark their users as members of an in-group. The faster fads spread, the more pressure there is to find a new marker. Only your mother, if she was a beatnik, thinks rimless specs are groovy. Only your grandmother, if she was a gate, thinks black horn-rims are crazy.

A final common assumption is that the media leads language changes. In fact, it belatedly reflects the changes.

The same fallacy seems to underlie the casual assumption that the mass media drives all kinds of language changes.

If the mass media can popularize words and expressions, then “presumably” it can also spread other kinds of linguistic changes. We generalize from one limited effect to a host of others.

Finally, we should note that high mobility has even greater social significance than the media explosion.

%More stuff zu Media
%\cite[vgl.][s.20]{Tomasello06}
\end{comment}

\section{Korpusüberblick}

Überblick - Datensatz:
----------------------
Zeitschrift siempremujer --> Herausgeber: meredith corporation USA, California
1419 Artikeln von 23.03.2010(?) bis 13.02.2015
Online-Version (nicht Druckausgabe)
Kein Anspruch auf Vollständigkeit weder bzgl Anzahl betrachteten Artikeln, noch bzgl *alle* English-Vorkommnissen innerhalb der Artikeln
Zielpublikum?

Ein bisschen über die Methodik:
* runtergeladen
* plaintext
* open office dicts (LGPL)
* in Wörter segmentiert (white spaces delimiters)
* Wörter annotiert
* nicht perfekt, Wörterbücher sind keine erschöpfende Wortlisten - verbesserungsbedürftig
* alles angeguckt (manuel), was *nur* als Englisch annotiert wurde
* auch problematisch beim Ansatz: viele Wörter, die es sowohl im Englischem als auch im Spanischen gibt (Homographe); selbst wenn man nur die als Englisch annotierte Tokens herauspickt, handelt es sich doch öfters um Homographe, da Wörterbücher nicht erschöpfend sind

\section{Befunde}

% vlt anderer Titel, wenn man genau weiß, was die These ist
%\subsection{Hauptthese}

Die automatisch erkannten Vorkommnisse von englischen Floskeln wurden einzeln gesichtet und in Kategorien unterteilt.
Insgesamt wurden in den 1419 Artikeln 3271 eindeutige englische Tokens identifiziert. %TODO Tokens oder Vorkommnisse?
Das heißt, wenn das selbe Token in einem Artikel öfter vorkam, wurde es hierfür nur einmal gezählt.
Im Kapitel~\ref{chap:results-desc} werden diese Kategorien beschrieben und exemplarisch dargestellt.
Anschließend werden in Kapitel~\ref{chap:results-analysis} verschiedene Erklärungsversuche dafür angeboten.


\subsection{Ergebnisse beschreiben}
\label{chap:results-desc}

Im Folgenden wird ein (quantitativer/qualitativer?) Überblick über die identifizierten Kategorien von Code-Switches und dazugehörige Beispiele gegeben.

%TODO:
% Bsp für jede Kategorie aufzählen
% auf Deutsch übersetzen

\subsubsection{Named Entities/Eigennamen}
Eigennamen bilden die erste und zugleich umfangreichste Kategorie von gefundenen Code-Switches.
Es handelt sich dabei um Namen von Organisationen, Produkten, Medien, Veranstaltungen oder US-amerikanischen Feiertagen.

\subsubsection{Entlehnungen aus dem Englischen}
Beispiele hierfür wären die Vokabeln ``unisex'', ``party'', ``piercing'', ``fitness'', ``blazer'', ``fan'', ``club'', ``shock''\footnote{All dieser Wörter werden beim Online-Wörterbuch \url{https://pons.eu} als Spanisch aufgelistet. \url{https://leo.org} führt alle bis auf ``party'' und ``fitness'' als Spanisch. Und die Online-Version des Wörterbuchs der konservativen Real Academia Española \url{https://dle.rae.es} erkennt alle davon bis auf ``piercing'' und ``fitness'' als Spanisch an.}.
Manche davon, wie ``piercing'' oder ``blazer'', beschreiben Gegenstände, für die es kein einzelnes Wort auf Spanisch existiert.
Bei anderen, wie ``party'', ``fan'' oder ``shock'',  ist die Entlehnung vielleicht weniger einfach zu erklären, da es entsprechende Äquivalenten auf Spanisch gibt (nämlich ``fiesta'', ``aficionadx'' und ``choque''). % also warum werden sie dann benutzt??


\begin{enumerate}
  \item Named Entities, dadrunter products (heap of commercials); journals, magazines, newspapers; organizations; book and movie titles; TV-Sendungen; Events; Musik; Orte; Läden/Firmen; Youtube Videos; US-amerikanische Feiertage;
  \item Direct quotes:
  \item English vocab which has more or less entered Spanish (presumably): (unisex, piercing, party)
  \item English discourse markers as part of the Spanish text:
  \item English interjections as part of the Spanish text:
  \item Half translated English idiomatic expressions
  \item English Idiomatic expressions / collocations
    * "Mix and match , la pareja ideal"
  \item Whole phrases in English, not necessarily idiomatic % todo: check: what's idiomatic anyway?
  \item Misc: one English word without category at the moment
  \item "One night stand"/sex stuff (72933)
  \item Dating
  \item Cool/slang/hip/life-style
  \item Further TV/Media/Movies stuff
  \item Music stuff
  \item Food/cooking stuff
    \begin{itemize}
      \item Organic/healthy/lifestyle bla
      \item und dann meat; fish; fruit; vegetables; candy; junkfood; drinks; herbs and spices; misc
    \end{itemize}
  \item Diet
    \begin{itemize}
      \item Products
      \item Diet names
    \end{itemize}
  \item Fashion stuff
    \begin{itemize}
      \item Clothing
      \item Hair
      \item Colors
      \item Makeup
      \item misc: trendy; look; vintage
    \end{itemize}
  \item Cosmetic stuff:
  \item Personal descriptions
  \item Tech/Internet stuff (strange, would've expected more of this to have appeared by now; possibly tokens cathegorized as UNK-> have a look)
    \begin{itemize}
      \item general
      \item Social media:
      \item Internet
      \item Smartphone
    \end{itemize}
  \item Fitness stuff:
  \item Berufe/Stellenbezeichnungen: ``nanny'', ``babysitter'', ``dog walker'', ``coach de lifestyle''
  \item Pets
  \item Electrodomestics
\end{enumerate}

\subsection{Interpretation}
\label{chap:results-analysis}

Es können für die vorgefundenen Kategorien von Code-Switches verschiedene Erklärungen geliefert werden.

%TODo eher nach Erklärungen aufteilen, nicht die Kategorien von oben wiederholen:
\begin{comment}
% Leichtere Wiederekennung
Bei den Named Entities aber auch bei bestimmten andere Sachen, wie z.B. Rezeptzutaten
% Entlehnungen für Begriffe, für die es auf Spanisch keinen genauen Äquivalent existiert
% Stil, Zugehörigkeit zu einer bestimmten Gruppe/Lifestyle
Hier kann man die Sozioindexikalität und die indexikalischen Felder wieder hervorrufen;
Zugehörigkeit auf verschiedenen Ebenen:
- Evokation(ist das ein Deutsches Wort?) einer bilinguale Identität/Komplizenschaft zwischen Herausgeberinnen und Leserinnen
- Zugehörigkeit zu einem hippen erfolgreichen Lifestyle (vlt geht das bereits auch in Prestige über?)

% Prestige

% Aufmerksamkeit erregen durch den Code-Wechsel
\end{comment}

\subsubsection{Leichtere Wiedererkennung}
% Named Entities; aber auch gewisse Nahrungsmitteln, oder andere Produkte, die die Leserinnen auf einem Englisch-sprachigen Markt konsumieren sollten.
Der erste Erklärungsversuch geht zunächst von der Gruppe der Named Entities aus.
Diese hat man vielleicht am Anfang nicht mitgedacht, sie sind jedoch kaum überraschend.
Es existieren gewiss für einige davon auch spanische Übersetzungen (zum Beispiel für die Film- und Buchtiteln, einige Produkte, ...).
Die Publikation richtet sich allerdings an ein Publikum, das diese Produkte auf dem US-amerikanischen Markt erlangen sollte, bzw. in einem dominierend englischsprachigen Kontext lebt und deshalb auch mit höherer Wahrscheinlichkeit einfach die englischen Namen der Filme oder Bücher kennen würde.
Deshalb erscheint es nur logisch, dass sie auf Englisch benannt werden.

Die leichtere Wiedererkennung kann auch hinter den Code-Switches bei bestimmten (vielleicht seltener vorkommenden) Nahrungsmitteln oder.... vermutet werden, da die Leserinnen diese auf einem Englisch-sprachigen Markt konsumieren (und finden) sollten.

\subsubsection{Entlehnungen/``Linguistic necessity''}
% Entlehnungen für Begriffe, für die es auf Spanisch keinen genauen Äquivalent existiert
% alt. Titel: Der Begriff existiert auf Spanisch nicht.
Wie wir bereits gesehen haben, handelt es sich bei manchen Code-Switches um englische Vokabeln, die bereits größtenteils ins Spanische eingedrungen sind und mehr als Entlehnungen funktionieren.
Manche davon werden nach Zentella (cite!!) wir mit dem Fehlen eines exakten Äquivalents auf Spanisch erklären können, andere vielleicht durch linguistische Sparsamkeit. %eigentlich ist linguistische Sparsamkeit bereits ein separates Argument
Beispiele dafür sind...

\subsubsection{Zugehörigkeit zu einer bestimmten Gruppe/Lifestyle}
Eine Zugehörigkeit zu einer bestimmten Gruppe bzw. Lifestyle werden bereits auf mehreren Ebenen evoziiert/inszeniert.
Wie von~\cite{Ticknor2012} gefunden, wird durch eine Benutzung des bilingualen Codes vermutlich versucht, eine Art gemeinsame Identität/Komplizenschaft zwischen der Herausgeberinnen der Zeitschrift und der Leserinnen herzustellen.

Darüber hinaus wird die nächste Ebene von Sozioindexikalität aufgemacht, in dem gewisse (englische Begriffe) für einen (erfolgreichen), hippen, gesunden Lebensstil stehen. % (vlt geht das bereits auch in Prestige über?)
Die Benutzung von Vokabeln wie ``party'' oder ``fitness'' kann vielleicht auch mit der Inszenierung eines gewissen Lebensstils/Life Styles erklärt werden (siehe unten).

\subsubsection{Prestige}
Ohne Zweifel ist in den USA Englisch die prestigevollere/angesehenere Sprache.
Sie ist die offizielle Sprache im Land und wird von der Mehrheit der US Bevölkerung gesprochen.
Spanischsprachige Bürgerinnen und solche, die als Latinas identifiziert werden, werden öfter auf dem Arbeitsmarkt aber auch im alltäglichen Leben systematisch diskriminiert.
Wer gut Englisch beherrscht hat bessere Chancen, einen Job zu finden und sozial aufzusteigen.
Wiederum wird Spanisch mit Marginalisierung, ... asoziiert. % so für das alles brauchen wir eine Quelle.
Diese Tatsache lässt uns die Vermutung aufstellen, dass öftere Benutzung Englischer Vokabeln eine gewisse Überlegenheit in der sozialen Hierarchie signalisieren/andeuten sollte, bzw. dass Menschen, die auch so reden, signalisieren(syn!)/darauf hinweisen/zu verstehen geben, dass sie Englisch beherrschen und somit denen ein sozialer Aufstieg offen steht.
\begin{comment}
[Zentella07]

distinct ways of being Latina/o are shaped by the dominant language ideology
that equates working-class Spanish speakers with poverty and academic failure,
--> also eine Erklärung, warum sie zu %Language Shift tendieren würden

%Latina/o Linguistic Capital
%Racism
In
the USA, where race has been remapped from biology onto language because
public racist remarks are censored, comments about the inferiority and/or
unintelligibility of regional, class, and racial dialects of Spanish and English
substitute for abusive remarks about color, hair, lips, noses, and body parts, with
the same effect.

--> noch eine Erklärung für %Language Shift

no one expects you to be able to change your color, but you are expected to
change the way you speak radically to earn respect (Urciuoli 1996).

These attitudes are communicated in everyday conversations and promoted by
the media and public institutions, but some groups of Latinas/os are more affected
than others.
\end{comment}

% English discourse markers;
Genau damit würde ich die Kategorie der englischen Discourse Markers (DE?) erklären.
Sonst ist es kaum nachvollziehbar, warum im Text ``anyway'', ``must'' oder ``pros and contras'' erscheinen sollten, anstatt auf diese komplett zu verzichten, bzw. die spanischen Übersetzungen zu benutzen. % oder mit Attention devices!
Bei gesprochener Sprache kann man sich diese Gebrauche noch mit spontanem Ausdruck (besser erklären) erklären, bei dem Medium Zeitschrift jedoch, wo die Autorinnen der Artikel Zeit zum Nachdenken hatten und der Text vermutlich einen (mehrstuffigen) Redigierprozess durchlaufen hat, kann von Spontanietät kaum die Rede sein.

% (halb)übersetzte englische Idiome/Redewendungen (iwie mit der oberen Kategorie verschmelzen)
Wir finden in den Artikeln auch einige (zur Hälfte übersetzt oder komplett auf Englisch) englische Idiome und Redewendungen.
(Bsp!)
Vielleicht werden diese teilweise deswegen benutzt, weil sie sich in dem konkreten Kontext besonders gut dafür eignen, um einen Sachverhalt zu veranschaulichen und keine genaue Übersetzung auf Spanisch existiert.
Also der englische Ausdruck ist viel präziser.
Das ist jedoch nicht immer der Fall und erklärt(syn!) nicht, warum wir manchmal Redewendungen haben, die zur Hälfte auf Spanisch und zur Hälfte auf Englisch abgedruckt sind (``vivir su happily ever after'', ``la voz en off'').
Eine weitere (neben der Prestige-Erklärung) Erklärung (besonders für diese Kategorie aber nicht ausschliesslich) ist auch denkbar:
durch die ``lässige'' Benutzung von englischen Ausdrücken im vorwiegend spanischen Text, versuchen die Redakteurinnen Intimität mit den Leserinnen herzustellen. %[Mahootian05]
Schließlich gehört Code-Switching eben zu den Merkmalen(syn?/anderes Wort) der bilingualen Latina-Identität.
Für solche Art Switches wäre es im Grunde egal, was genau geswitcht wird, Hauptsache, die kommen vor, und unterstreichen dabei die Gemeinsamkeit mit den Leserinnen.
Zu einer Identitätsbildung gehört oft ``to identify themselves as a group separate from their predecessors’ generation''~\cite[]{Mahootian05}.
Und grad bei den Englisch Switches in dem vorwiegend Spanischen Text kann man wieder meinen, dass sich die Sprecherinnen (bzw. in dem Fall die Zeitschriftredakteurinnen, die diese 2.generation Latina-Sprecherinnen erreichen möchten) sich als modern, hip und erfolgreich projizieren.
Eine andere Erklärung, zB auch für Englische Interjektionen u.ä. ist, dass sie Aufmerksamkeit erregen einfach weil sie nicht Spanisch nicht~\cite[]{Mahootian05}. % vgl auch hk magazines
\begin{comment}
→ style/socioindexicality: adopt linguistic forms of groups one sympathises with/has intense
  relationships with (Eckert08 ?)
→ "Although overt norms favor standard speech, powerful covert norms encourage
group members to remain faithful to group codes, linguistic and otherwise." (Zentella 2007)
\end{comment}

% Look
Auffällig oft kommt in den Artikeln das Wort ``look'' vor.
Zusammen mit anderen Vokabeln (Bsp!) habe ich dieses in die Kategorie ``Cool/hip/Life-style'' einsortiert.
(Bzw auch life-style bei der Kategorie ``Essen'') würde dazugehören.
Ich würde argumentieren, dass es sich dabei um den Versuch handelt, das Publikum/die Leserschaft der \textit{Siempre Mujer} als junge, coole, hippe, gesund lebende, moderne, schöne, erfolgreiche ... Superfrauen zu inszenieren (syn!)/darzustellen.
``Look'', also das äußere Aussehen, wird besonders bei Frauen oft mit Erfolg (oder sein Fehlen) gleichgestellt, wie die britische Periodistin und Autorin Laurie Penny argumentiert:

\begin{quote}
For modern women in this axious age, the makeover is a ritual of health and devotion and social conformity.
[\ldots]
Cosmetic surgery companies plaster public transport with promises to deliver not just physical changes, but emotional ones like `confidence'.
Fashion editorials advise us to spend money we don't have on skirt suits and handbags as `investment pieces'; you're not supposed to dress and style your body simply to please yourself but with one eye on your financial future.
That skirt suit really is an `investment' in a one-woman business whose product is you, only glossier.
This is what power, health and success means to the modern, emancipated woman: terminal exhaustion and a wardrobe full of expensive disuises.~\cite[p.41-42]{Penny14}
\end{quote}
% aber auch stil!

%Kapitalismuskritik
Man kann die oben anlautende Kritik erweitern/weiter denken/verfolgen/... und ... daran anknüpfen:
Schlussendlich sind die hier erweckten/angedeuteten/.. Werte wie Flexibilität, (hip, jung), Mobilität, Kreativität, .. vom Kapitalismus aufgegriffen und angeeignet worden.
Wie Luc Boltanski und Ève Chiapello schreiben: .. ~\cite[]{BolChi07}.

    \begin{comment}
auch ``The Rise of the Creative Class'' Richard Florida
``They do not consciously think of themselces as a class. Yet they share a common ethos that values creativity, individuality, difference and merit.''
``Everywhere we look, creativity is increasingly valued. Firms and organizations value it for the results that it can produce and individuals value it as a route to self-expression and job satisfaction.''

[Peck05]
he dawn of a ‘new
kind of capitalism based on human creativity’ calls for funky forms of supply-side
intervention, since cities now find themselves in a high-stakes ‘war for talent’, one that
can only be won by developing the kind of ‘people climates’ valued by creatives —
      urban environments that are open, diverse, dynamic and cool (Florida, 2003c: 27). --> (2003c) The new American dream. Washington Monthly March, 26–33.

In the field of urban policy, which has
hardly been cluttered with new and innovative ideas lately, creativity strategies have
quickly become the policies of choice, since they license both a discursively distinctive
and an ostensibly deliverable development agenda. No less significantly, though, they
also work quietly with the grain of extant ‘neoliberal’ development agendas, framed
around interurban competition, gentrification, middle-class consumption and place-marketing — quietly, in the sense that the banal nature of urban creativity strategies in
practice is drowned out by the hyperbolic and overstated character of Florida’s sales
pitch, in which the arrival of the Creative Age takes the form of an unstoppable social
revolution.

      Kapitalismuskritik
      [BolChi07]
      "Finally, capitalist restructuring over the last two decades -which, as we have seen, occurred around financial markets and merger-acquisition activities in a context of favourable government policies as regards taxation, social security and wages - was also accopmanied by significant inventives to greater labour flexibility. Opportunities for hiring on a temporary basis, using a temporary workforce, flexible hours, and a reduction in the costs of layoffs, have developed considerably in all the OECD countries, gradually whittling down the social security systems established during a century of social struggles." (p.xxxviii, Prologue) <-- damit sind die 70er und 80er gemeint
      "This process was widely encouraged by a significant number of the protesters of the era, who were especially sensitive to the themes of the artistic critique - that is to say the everyday oppression and sterilization of each person's creative, unique powers produced by industrial, bourgeois society. The tranformation in working methods was thus effected in large part to respond to their aspirations, and they themselves contributed to it, especially after the left's accession to government in the 1980s. Once again, one cannot fail to stress the fact that critique was effective." (p.199, 1968: The Crisis and Revival of Capitalism)
      "Correlatively, however, at the level of security and wages various gains of the previous period were clawed back - not directly, but via new mechanisms that were much less supervised and protective than the old full-time permanent contact which was the standard norm in the 1960s. Autonomy was exchanged for security, opening the way for a new spirit of capitalism extolling the virtues of mobility and adaptability." (p.199)
      "The displacements operated by capitalism allowed it to escape the constraints that had gradually been constructed in response to the social critique, and were possible without provoking large-scale resistance because they seemed to satisfy the demands issuing from a different critical current." (p.200)
      "What we have observed of the role of critique in the .. also the displacements and transformations, of capitalism .. always conductive to greater social well-being - leads us to underscore the inadequacies of critical activity, as well as the incredible flexibility of the capitalist process. This process is capable of conforming to societies with aspirations that vary greatly over time (but also in space, though that is not our subject), and of recuperatig the ideas of those who were its enemies in a previous phase." (p.200-201)
      "By contrast, it was by opposing a social capitalism planned and supervised by the state - treated as obsolete, cramped and constraining - and leaning on the artistic critique (autonomy and creativity) that the new spirit of capitalism gradually took shape at the end of the crisis of the 1960s and 1970s, and undertook to restore the prestige of capitalism. Turning its back on the social demands that had dominated the first half of the 1970s, the new spirit was receptive to the critiques of the period that denounced the mechanization of the world (post-industrial society against industrial society) - the destruction of forms of life conductive to the fulfilment of specifically human potential and, in particular, creativity - and stressed the intolerable character of the .. of oppression which, without necessarily deriving directly from historical capitalism, had been exploited by capitalis mechanisms for organizing .." (p.201)
      "By adapting these sets of demands to the description of a new, liberated and even libertarian way of making profit - which was also said to allow for realization of the self and its most personal aspirations.." (p.201)
      "By helping to overthrow the conventions bound up with the old domestic world, and also to overcome the inflexibilities of the industrial order - bureaucratic hierarchies and standardized production - the artistic critique opened up an opportunity for capitalism to base itself on new forms of control and commodify new, more individualized and 'authentic' goods." (p.467, The Test of the Artistic Critique)
\end{comment}

% Sex
Eine weitere spannende Kategorie habe ich ``one night stand'' bzw Sachen, die mit Sex zu tun haben benannt.
Wir kennen den platten Ausdruck ``sex sells'' und die kapitalistische Gesellschaft tut Frauen gnadenlos als sexualisierte Objekte ausbeuten (besser ausführen, nach Kapitalismuskritik).
Warum aber werden Wörter und Ausdrücke aus diesem semantischen Feld auf Englisch gebraucht?
Ich würde hier wieder das Argument mit der Prestige, bzw coolness heranziehen. (vgl auch Laurie Penny!)

% Essen
Eine andere umfangreichere Kategorie, die vielleicht nicht vom Anfang an erwartet wurde, ist ``Essen''.
Diese habe ich wiederum in mehrere Unterkategorien aufgeteilt.
Zunächst haben wir es mit Lebensmitteln in Rezepten oder Dietprogrammen zu tun, die oft auf Englisch, oder auf Spanisch mit der Englischen Übersetzung in Klammern erscheinen.
Es handelt sich bei manchen dabei vielleicht um Nahrungsmitteln, die in den Ursprungsländern der Leserinnen nicht besonders populär sind und demzufolge die Menschen kaum die spanischen Namen kennen würden.
Oft sind es unkonventionellere/spezifischere Lebensmittel, die in diese Kategorie auftauchen: ``kale'' (Grünkohl), ... (aber auch nicht ausschließlich, wir haben hier auch ``banana'', ``steak'' oder ``baking soda''). % zu prüfen: eigentlich sind gar nicht so viele weirde sachen am start!
Es kann also auch eine einfachere Erklärung für deren Vorkommen auf Englisch gefunden werden:
nämlich die Leserinnen, die ein Rezept oder Diät ausprobieren möchten, leben schlussendlich in den USA und sollten in der Lage sein im Geschäft die entsprechenden Produkte auf Englisch einzukaufen.
Dann ist es durchaus sinnvoll zu wissen, dass ``col risada'' auf Englisch ``kale'' heißt.
Insgesamt kann man sagen, dass es sich hier um ein Versuch handelt, Zutaten wirklich klar zu stellen und Missverständnisse zu vermeiden.

%% Lifestyle Essen
Eine weitere Unterkategorie, die ich hier identifiziert habe, hat auch mit ``life style'' zu tun.
Es handelt sich dabei um Sachen wie ``baby greens'', ``pavo wild'', ``gluten free'' etc.
Diese würde ich, wie bereits angedeutet, mit dieser Prestigeinszenierung und das Schaffen einer gewissen Atmosphäre von Coolness, Hippness und dadurch von Erfolg und sozialen Aufstieg. % na bzw grad hier wird ein gesunder Lebensstil inszeniert
% cool, flexibel, schwer beschäftigt wird eher durch ``coffeehouse'', ``dip'', ``snack'' evoziert.

%% US Zeug
Nicht zuletzt gibt es unter den Nahrungsmitteln auch welche, die vielleicht als spezifisch US-amerikanisch identifiziert werden können (oder zumindest sehr start mit der US-amerikanischen Kultur in Verbindung gebracht werden) und für die es dementsprechend vielleicht auch gar keine spanischen Begriffe existieren.
Beispiele hierfür sind ``brownies'', ``cheesecake'' oder ``peanut butter''.

% Technik
Eine der gefundenen Kategorien, die auch vom Anfang an vermutet wurde, ist Technik (bzw. Computer/Internet/Smartphonekram).
Hier ist ein erster Erklärungsversuch:
einerseits entstehen ganz viele der Geräte/Phänomenen in einem englischsprachigen Kontext (zb Silicon Valley);
und werden von da aus in anderen Teilen der Welt übernommen, und da die Konzepte/Gegenstände vorher nicht existiert haben, werden auch gleich ihre Bezeichnungen mitentlehnt.
Manche davon werden schlussendlich auf der lokalen Sprache übersetzt/(``angedeutscht'' für die entsprechende Sprache) (zb ``computadora'' oder ``en nube'' auf Spanisch), bei anderen wird die Benutzung der ursprünglichen Bezeichnung unverändert verstättigt (zb ``internet'', ``software'', ``wifi'').
Dieses Phänomen (oder diese Kategorie) können wir vermutlich bei jeder beliebigen Sprache finden und nicht nur im Spanisch-Englisch Kontext der Zeitschrift \textit{Siempre mujer}.
Selbst wenn ein Wort für ein Gegenstand oder Phänomen auf der Lokalsprache (hier Spanisch) existiert und sich etabliert, werden die englischen Bezeichnungen trotzdem oft wiedererkannt und parallel weiterverwendet.

\subsubsection{Aufmerksamkeit erregen durch den Code-Wechsel}

\begin{comment}

  %Language Shift
Language Shift (Spanish -> English) : Speech community of a language shifts to speaking another language (source? Thomason?)
Can we talk about language shift in this context?
Who is shifting? The editors of the magazine? Or do they induce language shift?
Should that be a focus at all? Or focus more on communities of practice/style projection through language/sozioindexikalität, etc.

Ein Shift Spanisch->Englisch ist nicht nur mit Prestige zu erklären, sondern auch mit allgemeinen Anpassungs/Integrationsprozessen:
Wer in den USA Englisch kann, hat bessere Chancen einen Job zu finden, etc. na, aber das hat wohl auch mit Prestige zu tun gewissermaßen

* Belege bei der Analyse, woher kommen die Hypothesen?
  ** Zentella für:
      "Powerful socioeconomic and cultural forces stimulate borrowing whenever two cultures are in contact;
      the borrowing by the subordinate group's language from that of the dominant group is always
      significantly greater" [Zentella90]
      Factors, which facilitate borrowings:[Zentella90]
       1) Most items reflect a cultural reality that is new or different, e.g., [kei/keike/keiki]
       «cake», [bobipín] «bobby pin», [ŷins/bluŷines] «(blue)jeans.»
       "la boila" (boiler) [Zentella97]
       2) Others may be facilitated or even triggered by a similarity in the phonological and/or morphological
       structure of a word in the dominant language that makes it sound like a possible subordinate word,
       matre(s) «mattress» is similar to that of madre(s) «mother(s).
       ZB "libreria" (Buchladen) für Bibliothek (library)
       3) reduction in the number of syllables, e.g., [beis] replaces florero for «vase»
       4) prestige
       5) Anglicisms can play the role of neutralizer between competing dialectal variants because the
       prestigious outside language acts as the lingua franca that resolves the conflict without favoring one
       group at the expense of the other.
      * influence of the media: "they do contribute to popular acceptance and use of some new vocabulary" [Zentella90]
        (siehe oben bei Spanglish)
        -> Spielen auch für leveling eine Rolle: versuchen neutrale Varianten zu nutzen, um möglichst mehr Menschen zu erreichen; und die ist manchmal eben auf Englisch?
      * distinct ways of being Latina/o are shaped by the dominant language ideology
        that equates working-class Spanish speakers with poverty and academic failure,
        [Zentella07]
        --> also eine Erklärung, warum sie zu Language Shift tendieren würden

------------------
{Andr07}
"Language mixing is no doubt part of the symbolic capital that
  lifestyle magazines like Latina (the ‘Magazine for Hispanic Women’) and rap
  stars like N.O.R.E. (‘Oye Mi Canto’) sell to their audiences."

"Conversationalization, i.e.
  the tendency to incorporate conversational speech styles in public discourse
  (Fairclough 1995), may also foster the public visibility of code-switching, for
  instance in host-caller radio talk, which is a commonly examined genre in
  the literature discussed below."

"A few media genres seem to approach this
  authenticity, e.g. live radio talk or computer-mediated interaction; but in
  advertising, television shows, song lyrics, movies or fashion magazines, the
  decision to use two or more languages is subject to careful planning, editing
  and staging. What could be less authentic than that?"

According to his classification of media, mein Fall fällt hier rein:
"Private commercial media conceive of their
  audience as consumers rather than citizens. Broadcasting a meaningful
  programme is far less important than attracting the attention of potential
  consumers of the media output and of the products and services advertised
  therein."
anders als non-commercial community media, die ein emanzipatorisches Bildungsprojekt haben

"Impersonal bilingualism proliferates in
both the commercial and the non-profi t sector, its most prominent forms being
commercially framed (e.g. popular music, advertising, lifestyle magazines)."

and according to media genres classification:
"The fourth group encompasses various non-fi ctional genres of written
discourse in e.g. ethnic, fan or fashion magazines as well as mainstream
newspapers (section 3.6);"

"Minimal bilingualism
in media discourse often responds to (factual or assumed) limited language
competence on the part of the audience, and exploits the symbolic, rather
than the referential, function of language (cf. sections 3.2, 3.6). This is
sometimes achieved by its use as a framing device (Coupland et al. 2003:
167): tiny amounts of a second language are positioned at the margins of
text and talk units, and thereby evoke social identities and relationships
associated with the minimally used language."

"Advertising discourse has a long tradition of ‘language display’, i.e. the
appropriation of out-group language ‘to attract potential customers by
appealing to their sense of what is modern, sophisticated, elegant, etc.’
(Eastman and Stein 1993: 198)"

English in ads:
"English is ‘the single most favored language selected for global mixing’
in advertising (Bhatia 1992)."
"What sets English apart
is the range of values it can be associated with, and the range of commodities
it promotes. It has been attributed symbolic values such as novelty, modernity,
internationalism, technological excellence, hedonism and fun, as opposed
to the stereotypical restriction of French to elegance and eroticism, Italian to
food, German to technology (cf. Piller 2001)."
"English, by contrast, illustrates how ethnosymbolism is left behind,
as its distribution to types of commodities is more signifi cant than the origin
of the commodities themselves."
"Cheshire and Moser (1994) suggest that the type of product is a better predictor
of language choice than the type of media that hosts the advertisement."

\cite{Beer2002}
"the magazines define Latinas primarily as consumers, they are a limited forum for women to explore noncosumerist identities, challenge hegemony, or express oppositional points of view."
"In addition, contemporarary consumer magazines offer advertisers a variety of "value-added" promotions, such as covert ads in the form of product mentions, and, although they are loathe to admit it, tolerate advertiser input into editorial content."


"It might be argued that, precisely because sexual imagery is designed to make the sale, it must, in some way, fulfill the passions it promises."
"We have noticed recently a trend toward a particular type of ad, increasingly more focused on sexuality, yet done with a tone wholly devoid of affect. Looking at these images one might correctly observe that after almost a hundred years of selling sex, the thrill seems to be gone."

"With the commodification of sex, the basic proposition is unatteinable. In terms of human passions, sex ads fail to satisfy because they confuse sexual gratification with the possession of objects. They attempt to substitute a state of *being* with the promise of *having*."

"The lack of affirmative portrayals of women's sexual passions leaves representations of female fulfillment impossible. [...] At worst, these images reinforce a patriarchal predisposition to disrespect women and do violence against them."
\end{comment}


% vlt kommt die Beschreibung der Ergebnisse auch hier rein

Die vorgefundenen Code-Switches lassen die Vermutung aufstellen, dass es sich bei den Artikeln um eine Art Performance handelt, die auf ein bestimmtes Zielpublikum (welches?) zugeschnitten ist.
%Welche Wirkung versuchen diese zu erzielen
Da es dabei um Kommunikation in geschriebener Sprache geht, sind die Code-Switches ganz bewusst ausgewählt und benutzt worden (vgl~\cite{Mahootian2005}) und sind nicht das Produkt einer spontanen Sprachproduktion, wo der Sprecherin ein Begriff nicht eingefallen ist.

% Evtl hier Medientheorie und wie Zeitschriften die Gesellschaft (mit)gestalten heranziehen


\section{Fazit}

\begin{comment}
    * Einleitung und Fazit müssen zusammenpassen.
    * sind die Erkenntnisse im Fazit aus der Arbeit ableitbar?
    * Vorgehensweise zusammenfassen

Kritische Zusammenfassung
\begin{itemize}
    \item was war in den Texten nicht so gelungen?
    \item welche Fragen sind offen geblieben?
    \item in welche Richtung kann noch weiter geforscht werden?
    \item was sind gegnerische Meinungen zum Thema?
\end{itemize}
\end{comment}


Das Ziel dieser Arbeit war, die Phänomene des Transfers und der Übergeneralisierung im Zweitspracherwerb aus einer kognitiv-linguistischen Perspektive zu beleuchten.
Dazu wurden diese zunächst in der historischen Entwicklung der Zweitspracherwerbsforschung verfolgt.
Daran anschließend wurden Transfer und Übergeneralisierung am konkreten Beispiel von \textit{contrast left dislocation} und \textit{clitic left dislocation} Konstruktionen in den Zweitsprachen Englisch und Spanisch in \cite[][]{Valenzuela05} untersucht.
Es hat sich gezeigt, dass Merkmale, die nicht auffällig genug sind oder in der L1 auf einer anderen Weise verarbeitet werden als in der L2, schwieriger wahrgenommen werden
und es in solchen Fällen häufig zu einer Übergeneralisierung von erlernten L2-Konstruktionen kommt.
Das war in \cite[][]{Valenzuela05} mit den CLLD-Konstruktionen der L2 Spanisch Sprechenden der Fall.
Wenn wiederum ein Merkmal relevant für eine Konstruktion in der L1 war, ist es schwieriger, dieses bei der Sprachproduktion in der L2 zu ignorieren.
Wenn dieses Merkmal im Sprachgebrauch der L2, trotz seiner Irrelevanz für die L2, berücksichtigt wird, beobachten wir einen Transfer.
In \cite[][]{Valenzuela05} wurde das am Beispiel der Spezifität des syntaktischen Topiks und ihrer Relevanz für die Art der Topik-Konstruktion realisiert.
Valenzuela hat gezeigt, dass Spanisch L1 Sprecherinnen dieses Merkmal häufig in ihrem Englisch L2 Sprachgebrauch berücksichtigt haben.

Während die Argumentation zu den Voraussetzungen der Erscheinungen Übergeneralisierung und Transfer plausibel und schlüssig erscheint, sind jedoch die Ergebnisse Valenzuelas Studie \cite[][]{Valenzuela05} mit gewisser Vorsicht zu genießen.
Zum Einen fallen die Antworten der Kontrolgruppe aus Muttersprachlerinnen bei manchen Aufgaben ziemlich heterogen aus.
Zum Anderen fragt es sich, ob die Schlussfolgerungen, die bei zwei kleinen Gruppen von Probandinnen erzielt wurden, universell verallgemeinbar sind.

Für zukünftige Studien des Transfers und der Übergeneralisierung im Zweitspracherwerb wären folgende Fragen interessant:
bei welchen anderen Konstruktionen treten diese häufig auf?;
sind Übergeneralisierung und Transfer eher bei abstrakten Konstruktionen zu beobachten oder kommen sie auch in lexikalisch konkreten Konstruktionen vor?;
hängt das Vorkommen dieser Phänomene vom Niveau der L2 Sprachbeherrschung der Lernenden?

% wenig Menschen -- sind die Ergebnisse einfach so zu verallgemeinern?
% Kontrolgruppe manchmal uneindeutig/anders als für "richtig" gehalten aber in der Deutung der Ergebnisse doch zurecht gebogen


%Es werden 2 Ansätze vorgestellt, was (Zweit)Spracherwerb angeht: sequentieller Spracherwerb und Erwerb eines Netz von Konstruktionen

%Gegnerische Meinungen:

%\cite{Cook93} - generative Perspektive auf Zweitspracherwerb
%``A person who knows two languages has been through the acquisition process twice.
%Second Language research must explain the means by which the mind can acquire more than one grammar.''

%``One goal of second language research is to describe grammars of more than one language simultaneously existing in the same person.''

%``The importance of second language research lise not in its account of the knowledge and acquisition of the L2 in isolation, but its account of the second language present and acquired in a mind that already knows a first [...]''

%aber das sind doch voll peanuts.. ist überhaupt nicht der kernpunkt..
%\cite{Valenzuela05} Kritik:
%bisschen zu großzügige Interpretation der gemessenen Daten zugunsten von der Arbeitshypothese




%----------------------------------------------------------------------------------------
%	BIBLIOGRAPHY
%----------------------------------------------------------------------------------------

\renewcommand{\refname}{\spacedlowsmallcaps{Literatur}} % For modifying the bibliography heading

\nocite{*}
\addcontentsline{toc}{section}{Literatur}
%\bibliographystyle{unsrt}
%\bibliographystyle{alpha}
%\bibliographystyle{natdin}
%\bibliography{literature} % The file containing the bibliography

\printbibliography

%----------------------------------------------------------------------------------------
%   APPENDIX
%----------------------------------------------------------------------------------------

%\newpage
%\addcontentsline{toc}{section}{Apéndice}
%\section*{Apéndice}

\subsection*{Cronología}

En el sentido del mapeo de controversias~\autocite{Venturini2010a} resumimos aquí de manera cronológica los eventos más importantes para el desarrollo de la asociación civil AlterMundi y la facilitación de sus redes.
Se trata tanto de sucesos propios del proyecto, como de acontecimientos políticos relevantes de escala local, nacional o mundial.
La información se basa en las fuentes siguientes: página web de AlterMundi\footnote{\url{http://altermundi.net/curriculum-institucional}}, el portal oficial del gobierno argentino\footnote{\url{http://servicios.infoleg.gob.ar}}, la página web del proyecto LibreRouter\footnote{\url{https://librerouter.org/what-and-why}},~\autocite{Piccoli2015},~\autocite{Brock2016},~\autocite{Vaseva2016a} y~\autocite{Wunderlich2017}.

\begin{longtable}{ r | p{.8\textwidth}}
\textbf{2009} & \\
octubre & Ley de Servicios de Comunicación Audiovisual No. 26.522
  \begin{itemize}
    \item desconcentración de la comunicación: empresas comerciales no pueden obtener más frecuencias/licencias; hay un límite de frecuencias para cada sector;
    \item suporte para proyectos comunitarios definido específicamente:
    \begin{quotation}
``ARTICULO 97. — Destino de los fondos recaudados. La Administración Federal de Ingresos Públicos destinará los fondos recaudados de la siguiente forma:
...
f) El diez por ciento (10\%) para proyectos especiales de comunicación audiovisual y apoyo a servicios de comunicación audiovisual, comunitarios, de frontera, y de los Pueblos Originarios, con especial atención a la colaboración en los proyectos de digitalización.''\footnote{\url{http://servicios.infoleg.gob.ar/infolegInternet/anexos/155000-159999/158649/norma.htm}}
\end{quotation}
  \end{itemize}\\
 ? & Grupo \textit{Clarín} se niega a implementar la ley y privarse de algunas de sus frecuencias\\
  & \\
\textbf{2012} & \\
  ? & El Ministerio de Educación planea el proyecto Arraigo Digital con la participación de la tech cooperativa CodigoSur. La idea es, ofrecer talleres de capacitación de software libre en escuelas por toda Argentina. \\
  ? & nace también la idea de ofrecer talleres de redes \\
  ? & AlterMundi se forma como grupo/proyecto desde CodigoSur \\
  ? & el gobierno quita el apoyo económico prometido \\
  marzo & primer hackathon\footnote{Desde ``hackear'' y ``marathon'': un evento de desarrollo de hardware/software, por lo general durante varias horas/días a la vez} facilitado por AlterMundi (la gente ya se había comprado la técnica): arma la red QuintanaLibre en José de la Quintana en Córdoba. El objetivo era compartir Internet entre 2-3 familias (llega hasta 60 familias hoy) \\
  abril & inicio de la red DeltaLibre en Buenos Aires \\
  septiembre & inicio de la red AnisacateLibre en Córdoba \\
  & \\
 \textbf{2013} & \\
 ? & la Corte Suprema de Justicia confirma la constitucionalidad general de la Ley 26.522 \\
septiembre & inicio de LaSerranitaLibre, Córdoba \\
 & \\
 \textbf{2014} & \\
marzo & inicio de NonoLibre, Córdoba \\
abril & Radio Equipment Directive se propone en la EU (``Radio Lockdown directive'') \\
diciembre & Ley Argentina Digital (Ley 27.078)\footnote{\url{https://www.enacom.gob.ar/ley-27-078_p2707}}:
\begin{quotation}
``Artículo 82:

fomento y resguardo de las denominadas redes comunitarias, garantizando que las condiciones de su explotación respondan a las necesidades técnicas, económicas y sociales de la comunidad en particular''
\end{quotation} \\
 & \\
\textbf{2015} & \\
marzo & la FCC (Federal Communications Commission, el órgano de regulación de la comunicación en los EEUU) adopta nuevas normas de seguredad\footnote{\url{https://assets.documentcloud.org/documents/2339685/fcc-software-security-requirements.pdf}} (similar a la Radio Equipment Directive), con el resultado de no deliberado lockdown del hardware de sistemas operativos alternativos (que están en la base de proyectos de redes comunitarias) \\
abril & inicio de la red BoquerónLibre, Santiago del Estero \\
mayo & LaBolsaLibre, Córdoba, va en linea \\
julio & inicio de la Red Fumaça Online, Río de Janeiro, Brasil con la apoyo de AlterMundi \\
diciembre & Mauricio Macri asuma la presidencia en Argentina; adopta política orientada hacia el mercado: laissez-faire capitalismo; deroga del Artículo 82 de la Ley Argentina Digital \\
 & \\
\textbf{2016} & \\
enero & inicio de la red MulukukuLibre, Nicaragua con el apoyo de AlterMundi \\
principios & Macri modifica la Ley 26.522 con un decreto: "morigerar el carácter antimonopólico de la ley, beneficiando a los principales medios de comunicación del país." (por ejemplo, los límites para licencias introducidas por la Ley 26.522 cesan) \\
? & nace el proyecto del LibreRouter \\
 & \\
\textbf{2017} & \\
febrero & inicio de la red CaimitoLibre, Esmeraldas, Ecuador con el apoyo de AlterMundi \\
julio & el decreto de Macri llega a la Corte Suprema, que aun ha de pronunciarse sobre su legalidad \\
octubre & planificado el lanzamiento del LibreRouter
\end{longtable}


\subsection*{Mapeo de la controversia}

En la Figura~\vref{fig:mapeo} tenemos una intento de mapear los Actores-Red en la controversia de la soberanía comunicativa, en acuerdo con~\autocite{Venturini2010a}.

\begin{sidewaysfigure}
  \centering
  \includegraphics[width=\columnwidth]{mapeo-controversia}
  \caption[Mapeo]{Controversia de la soberanía comunicativa utilizando ANT} % The text in the square bracket is the caption for the list of figures while the text in the curly brackets is the figure caption
  \label{fig:mapeo}
\end{sidewaysfigure}






%----------------------------------------------------------------------------------------
\end{document}
