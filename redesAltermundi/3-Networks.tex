\section{Redes comunitarias}

Para las metas del trabajo actual, cuando hablamos de redes comunitarias vamos a entedender redes comunitarias inalámbricas (computer networks).


Para reflexionar sobre la infraestructura técnica, hemos primeramente de entender su funcionamiento.

En su forma convencional las redes de comunicación estan, más a menudo que no, organizadas de una manera jerárquica.
En la informática hablamos de una topología estrella (Bild!): todos los nodos estan conectados a través de un nodo central,
que tiene un papel clave.
Si este nodo falla, toda la red se desmantela y nadie puede comunicarse.

Sin embargo, existen también otras topologías/maneras de organizar una red.
Las redes comunitarias emplean frecuentamente la que se llama mesh (en malla) (Bild):
cada nodo está conectado con más de un otro, en el caso ideal cada nodo está conectado directamente con cada otro (full mesh).
No obstante la full mesh no está adecuada para redes más grandes, sobre todo inalámbricas, donde una directa linea de vista(?) se necesita para establecer la conexión.
En este caso tenemos más bien una mesh entre vecinxs, y, a veces, enlaces punto a punto a distancias más largas.
La mesh tiene la ventaja de estar mucho más perdurable.
Todos los nodos en la mesh son iguales y si uno de ellos se quema, el resto sigue existiendo y puede comunicarse como antes.
Ningún nodo tiene este papel clave que hemos visto en la otra topología y por eso no puede ejercer control sobre la comunicación de los demás~\autocite{FiTre2015}.

Ya hemos señalado que las redes comunitarias usan muy a menudo redes mesh.
¿Pero cuáles son sus otras características?
En general, se trata de proyectos de base locales con una exigencia política.
La idea clave es construir una infraestructura de comunicación decentralizada, creada, mantenida y poseada por la comunidad misma de lxs usuarixs.
Así, ni el estado, ni empresas grandes tienen el control sobre la red y pueden cerrarla de manera fácil.
% casos extremos
Redes comunitarias son especialmente interesantes en contextos de régimenes políticos autocráticos o en casos de catástrofes naturales~\autocite{Mabb2014}, ~\autocite{FiTre2015}.
Para los últimos está relevante sobre todo la tecnología de la mesh, que ante otras, garantiza servicio en condiciones extraordinarias.
% otras características

% motivaciones
No obstante, hay varias otras motivaciones para la construcción(syn) de tales redes~\autocite{Vaseva2016b}.
Desde ... , hasta el disfruto de areglar problemas técnicos y aprender como funciona la tecnología.
...
Muchos de estos proyectos han surgido desde la necesidad de conección, que proveedores convencionales no querían satisfacer ya que no veían oportunidad de lucro.
Se trata aquí sobre todo de poblaciones pequeñas, ajenas de ciudades grandes y a menudo con terreno difícil.
Así nacieron los proyectos Altermundi, guifi.net, freifunk..



\begin{comment}

"
https://pillku.org/article/altermundi-y-las-redes-comunitarias-digitales/
AlterMundi y las redes comunitarias, historia y perspectivas
Entrevista a Nicolás Echániz sobre el trabajo de AlterMundi en relación a las redes comunitarias digitales.Definiciones, desafíos, estrategias y propuestas
7 de febrero de 2015

Decimos que una Red Libre tiene estas características:

* libre uso: puede ser utilizada por sus participantes para ofrecer y recibir cualquier tipo de servicio que no afecte su buen funcionamiento
* neutralidad: no inspecciona ni modifica los flujos de datos dentro de la red más allá de lo necesario para su operación
* libre interconexión: permite, de forma libre y gratuita, el flujo de datos con otras redes que respeten las mismas condiciones
* libre tránsito: provee a otras redes libres acceso a las redes con las que mantiene acuerdos voluntarios de libre interconexión.

Una Red Comunitaria es una Red Libre en la que, además de las características ya enunciadas, se presentan otras:

* propiedad colectiva: su infraestructura es propiedad de la comunidad que la despliega;
* gestión social: la red es gestionada por la misma comunidad;
* diseño accesible: la información sobre cómo funciona la red y sus componentes es pública y accesible;
* participación abierta: cualquiera puede extender la red, respetando su diseño y sus principios.

Nuestro trabajo se centra principalmente en el desarrollo de un modelo de red específicamente adaptado a la realidad de pequeñas poblaciones digitalmente excluidas por el modelo comercial tradicional.
La misión de las redes comunitarias puede variar de un sitio a otro, pero en general tienden a empoderar al pueblo en el uso y comprensión de la tecnología y a promover un acceso más inclusivo a los recursos digitales, locales y externos.


El modelo de red MiniMaxi se compone de:

* una referencia de hardware que comprende tanto el router y las antenas a utilizar como los materiales para la adaptación a la intemperie y la instalación
* un firmware (sistema operativo) para instalar en los routers, que permite su auto-configuración
* un número de herramientas de software necesarias para la personalización y el monitoreo de estado de la red 

Otra experiencia que resulta digna de mención es la interacción que se ha conseguido entre las redes del Valle de Paravachasca con la Universidad Nacional de Córdoba, a través del Laboratorio de Redes y Comunicación y de la Prosecretaría Informática. Desde este año, gracias a un enlace propio de 50Km, las redes comunitarias han establecido interconexión con la red de la Universidad. La UNC, aparte de comenzar a fomentar la investigación y el estudio académico de este tipo de redes, también se ha ido estableciendo como un importante centro de convergencia de las redes regionales, alojando tanto al NAP de CABASE en Córdoba como al futuro nodo de Arsat.
Para nuestras redes comunitarias, lograr este tipo de interconexiones representa un grado de madurez que no se había conseguido previamente en el país y me animaría a decir en el continente.

sigue: descripción de QuintanaLibre, más sobre la infraestructura física

"

redes convencionales: organizadas de manera jerárquica (Telematik heranziehen?):
topología centralizada: el nodo en el centro tiene control/papel clave: si falla, toda la red se desmantela.

redes mesh: cada nodo está conectado con más de 1 otro: topología más perdurable/estable; todos los nodos son iguales;
full mesh: cada nodo está conectado directamente con cada otro: la red falla si todos los nodos fallan --> grande capacidad de recuperación

redes comunitarias: son en general redes mesh con una pretención política
características:
* infraestructura de comunicación decentralizada, creada y mantenida por la comunidad de lxs usuarixs (el estado/empresas grandes no pueden cerrarla tan facilmente) --> interesante sobre todo también para contextos autócratos (regimenes políticos antidemocráticos, vease Iran, China, Cuba, .. )
* garantizar acceso libre a información
* garantizar la libertad de expresión
* ..
* el firmware: software libre

      \item infraestructura de comunicación abierta, accesible para tod@s
      \item infraestructura creada y mantenida por la comunidad de l@s usuari@s
      \item ejemplos:
        \begin{itemize}
          \item Freifunk (Alemania)
          \item guifi.net (España/Cataluña)
          \item ninux (Italia)
          \item Funkfeuer (Austria)
        \end{itemize}

  \begin{itemize}
    \item conectar a comunidades excluidas por los proveedores convencionales de servicios Internet
    \item garantizar acceso libre a información
    \item garantizar la libertad de expreción
    \item profundizar los propios conocimientos técnicos, experimentar
    \item educar y concienciar a más gente
  \end{itemize}

anti sistemas autocratas:
"Anlässe für dieses neuerliche Interesse sind etwa die Versuche in autoritären
Staaten, das Internet komplett abzuschalten, um den Informationsaustausch
zu verhindern."[p.12][Mabb2014]

auch fuer Katastrophenfaelle relevant
"Ein weiteres Einsatzfeld sind Mesh-Netze im Katastrophenfall, besonders nach
Naturkatastrophen. So nutzten etwa nach dem Hurrikan Sandy Bürger im Brook-
lyner Viertel Red Hook ein solches, bereits vorhandenes und weiter funktionsfä-
higes Netz. Es wurde um einen Dienst erweitert, über den Bewohner Schäden ­
melden und lokale Informationen austauschen konnten. Zusammen mit der
Katastrophenschutzbehörde wurde das lokale Netz provisorisch per Satellit
ans Internet angebunden. Der dort eingesetzte technische Werkzeugkasten
des „Commotion Wireless Project“ teilt viele Komponenten mit dem deutschen
Freifunk."[p.12][Mabb2014]

bridging the digital devide!

surgen de necesidades locales:
por ejemplo en el caso de Altermundi: pequeños pueblos, poblaciones en las altas cumbres; localidades donde los proveedores convencionales no ven oportunidad de lucro y por eso no prestan servicio

se ubican en el contexto del ciberoptimismo (DEF)

reflexión crítica: está la solución fiable en gran escala: vgl papel del estado (IV Gui); última milla

[FiTre2015]
"rather binary antagonism between decentralization/freedom and centralization/control in
communications resources."
-- vlt bisschen uebertrieben, das so schwarz-weiss zu sehen, gewisser grad an zentralitaet ist unabdingbar, e.g. wenn das lokale Netzwerk ans Internet angeschlossen werden sollte

\end{comment}


