\section{Análisis}

descripción/análisis de los libros/del libro:
* quienes actores/protagonistas están dibujad@s?
  * está reflejada la realidad en su complejidad? la heterogeneidad cultural?
* a quién se dirige?
  * a la clase de l@s campesin@s (descamisad@s?)

* cuál es la imagen de Argentina que pinta el libro?
  * modernidad!
    * revolución industrial: caminos, ferrocarriles, ..
    * los libres como artefacto material son representaciones de la modernidad (alfabetismo, la circulación masiva de medios impresos?)
  * los espacios que sirven para crear Argentina como nación (vgl las pampas y los parques naturales): página con sehenswuedrigkeiten
    * campañas del desierto, reimaginar las pampas; el estado reconfigura ele espacio, "civilizar" las pampas

* Eva y Juán Perón en el centro
  * cómo están dibujad@s:
    * Personenkult, Gottesgleichheit
    * padres figurativos de la nación: comparación con padres
    * papeles tradicionales de género, pero no solo: Eva Perón tiene un papel público, no está relegada al privado
  * nominación: "Evita y Perón" : la mujer es la que está denombrada por un deminutivo: cute, no tomada en serio? pequeña... (podemos notar la tendencia que sigue... "Dilma vs Temer", "Hilary and Trump")

* wiederkehrende Motive
  * "juguetes para mí" (p.76)
  * Evita: "madre espiritual" (p.75)



# libros infantiles

## libro de primer grado

* in Kontext setzen: wann erschienen? Verlag, Eckdaten? -- obviamente después del fallacimiento de Eva Perón, se le dedica una página; la muerte ha convertido el culto de persona más fuerte? quizá no, se nota lo mismo también para Juán Perón

* stilisiert als ein Buch zum Lesen und Schreiben lernen
* Personenkult --> Gleichstellung mit einer Goettlichkeit?
* "jefa espiritual de la república"
* "¡Es linda mi escuela! Se llama Eva Perón"
* keine Doppelseite, wo "Evita" nicht dran kommt?:
  * eine, da steht "Nueva Argentina" dran -- das ist auch ein Trop, wa
  * 2.: "General Perón"
  * 3. "Viva la patria": wie ist diese dargestellt -- die Laender des Suedens, vgl die Aneignung der Pampas
  * 4. "el niño jesús" : also Katolizismus ist mitreingestrickt
  * 5. "General Juan Perón" e "independencia económica"
  * weitere...
  * "todos deben producir"
  * weitere historische Figuren rein: Sarmiento.. wie sind sie portraertiert?
  * Zielpublikum? "Los derechos del trabajador"
  * Kapitel ueber infrastruktur: "Caminos", "Recuperación nacional", "El buey"; dibujos de ferrocarriles, carros, coches, cabales <--- modernidad inszeniert!/Industrialización: auch teléfono, avión, ferrocarril, barcos (p.72)
  * "Lo que dijo Eva Perón, para comentar en clase"
  * los descamisados (== trabajador?), die basis popular del gobierno?
* "Mamá y papa me aman" - "Perón y Evita, nos aman." <-- gleichgestellt mit Mutter und Vater, die Eltern aller Kinder in Argenitinien
* die immer Geschenke machen
* eine stilisierte Welt: alle laecheln
* eigentlich ein rundum Portraet Argentiniens
  * "bellezas de la patria": sehenswuerdigkeiten --> mito de la estado-nación
* "Policlínico Evita" (p.43), "Fundación Eva Perón"(p.43), "clubes infantiles Evita" (donde se practica futbol^^)(p.55)
* adjetivos para describir Eva Perón: buena (p.15), linda, madre espiritual de la república, "amó a todos" (s.13), cariñosa(p.15), "Nuestra patria es un nidito... y es Evita la mamá"(p.18), "Mártir del trabajo" (s.22), "querida" (p.67), "jefa espiritual de la nación" (p.71), "madre espiritual de los niños"(p.75)
-- papel "femenino" pero todavía demasiado público; no está relegada al espacio privado
* adjetivos para Juán Perón: líder, general, libertador de la república, buen gobernante, presidente de los argentinos, "Primer Trabajador Argentino"(p.35)
-- papel más "masculino", "fuerte"
--> papeles estereotípos de género
