\section{Análisis}

descripción/análisis de los libros/del libro:
* quienes actores/protagonistas están dibujad@s?
  * está reflejada la realidad en su complejidad? la heterogeneidad cultural?
* a quién se dirige?
  * a la clase de l@s campesin@s (descamisad@s?)

* cuál es la imagen de Argentina que pinta el libro?
  * modernidad!
    * revolución industrial: caminos, ferrocarriles, ..
    * los libres como artefacto material son representaciones de la modernidad (alfabetismo, la circulación masiva de medios impresos?)
  * los espacios que sirven para crear Argentina como nación (vgl las pampas y los parques naturales): página con sehenswuedrigkeiten

* Eva y Juán Perón en el centro
  * cómo están dibujad@s:
    * Personenkult, Gottesgleichheit
    * padres figurativos de la nación: comparación con padres
    * papeles tradicionales de género, pero no solo: Eva Perón tiene un papel público, no está relegada al privado
  * nominación: "Evita y Perón" : la mujer es la que está denombrada por un deminutivo: cute, no tomada en serio? pequeña... (podemos notar la tendencia que sigue... "Dilma vs Temer", "Hilary and Trump")
