\section*{Apéndice}

\subsection*{Cronología}

En el sentido del mapeo de controversias~\autocite{Venturini2010a} resumimos aquí de manera cronológica los eventos más importantes para el desarrollo de la asociación civil AlterMundi y la facilitación de sus redes.
Se trata tanto de sucesos propios del proyecto, como de acontecimientos políticos relevantes de escala local, nacional o mundial.
La información se basa en las fuentes siguientes: página web de AlterMundi\footnote{\url{http://altermundi.net/curriculum-institucional}}, el portal oficial del gobierno argentino\footnote{\url{http://servicios.infoleg.gob.ar}}, la página web del LibreRouter proyecto\footnote{\url{https://librerouter.org/what-and-why}},~\autocite{Piccoli2015},~\autocite{Brock2016},~\autocite{Vaseva2016a}.

\begin{longtable}{ r | p{.8\textwidth}}
\textbf{2009} & \\
octubre & Ley de Servicios de Comunicación Audiovisual No. 26.522
  \begin{itemize}
    \item deconcentración de la comunicación: empresas comerciales no pueden obtener más freqüencias/licencias; hay un límite de freqüencias para cada sector;
    \item suporte para proyectos comunitarios definido especifícamente:
    \begin{quotation}
``ARTICULO 97. — Destino de los fondos recaudados. La Administración Federal de Ingresos Públicos destinará los fondos recaudados de la siguiente forma:
...
f) El diez por ciento (10\%) para proyectos especiales de comunicación audiovisual y apoyo a servicios de comunicación audiovisual, comunitarios, de frontera, y de los Pueblos Originarios, con especial atención a la colaboración en los proyectos de digitalización.''\footnote{\url{http://servicios.infoleg.gob.ar/infolegInternet/anexos/155000-159999/158649/norma.htm}}
\end{quotation}
  \end{itemize}\\
 ? & Grupo \textit{Clarín} se niega a implementar la ley y privarse de algunas de sus freqüencias\\
  & \\
\textbf{2012} & \\
  ? & El Minitsterio de Educación planea el proyecto Arraigo Digital con la participación de la tech cooperativa CodigoSur. La idea es, ofrecer talleres de capacitación de software libre en escuelas por toda Argentina. \\
  ? & nace también la idea de ofrecer talleres de redes \\
  ? & AlterMundi se forma como grupo/proyecto desde CodigoSur \\
  ? & el gobierno quita el apoyo financial prometido \\
  marzo & primer hackathon\footnote{Desde ``hackear'' y ``marathon'': un evento de desarrollo de hardware/software, por lo general durante varias horas/días a la vez} facilitado por AlterMundi (la gente ya se había comprado la técnica): arma la red QuintanaLibre en José de la Quintana en Córdoba. El objetivo era compartir Internet entre 2-3 familias (llega hasta 60 familias hoy) \\
  abril & inicio de la red DeltaLibre en Buenos Aires \\
  septiembre & inicio de la red AnisacateLibre en Córdoba \\
  & \\
 \textbf{2013} & \\
 ? & la Corte Suprema de Justicia confirma la consitucionalidad general de la Ley 26.522 \\
septiembre & inicio de LaSerranitaLibre, Córdoba \\
 & \\
 \textbf{2014} & \\
marzo & inicio de NonoLibre, Córdoba \\
abril & Radio Equipment Directive se propone en la EU (``Radio Lockdown directive'') \\
deciembre & Ley Argentina Digital (Ley 27.078)\footnote{\url{https://www.enacom.gob.ar/ley-27-078_p2707}}:
\begin{quotation}
``Artículo 82:

fomento y resguardo de las denominadas redes comunitarias, garantizando que las condiciones de su explotación respondan a las necesidades técnicas, económicas y sociales de la comunidad en particular''
\end{quotation} \\
 & \\
\textbf{2015} & \\
marzo & la FCC (Federal Communications Commission, el orgán de regulación de la comunicación en los EEUU) adopta nuevas normas de seguredad\footnote{\url{https://assets.documentcloud.org/documents/2339685/fcc-software-security-requirements.pdf}} (similar a la Radio Equipment Directive), con el resultado de no deliberado lockdown del hardware de sistemas operativos alternativos (que están en la base de proyectos de redes comunitarias) \\
abril & inicio de la red BoquerónLibre, Santiago del Estero \\
mayo & LaBolsaLibre, Córdoba, va en linea \\
julio & inicio de la Red Fumaça Online, Río de Janeiro, Brasil con la apoyo de AlterMundi \\
deciembre & Mauricio Macri asuma la presidencia en Argentina; adopta política orientada hacia el mercado: laissez-faire capitalismo; deroga del Artículo 82 de la Ley Argentina Digital \\
 & \\
\textbf{2016} & \\
enero & inicio de la red MulukukuLibre, Nicaragua con el apoyo de AlterMundi \\
principios & Macri modifica la Ley 26.522 con un decreto: "morigerar el carácter antimonopólico de la ley, beneficiando a los principales medios de comunicación del país." (por ejemplo, los límites para lincencias introducidas por la Ley 26.522 cesan) \\
? & nace el proyecto del LibreRouter \\
 & \\
\textbf{2017} & \\
febrero & inicio de la red CaimitoLibre, Esmeraldas, Ecuador con el apoyo de AlterMundi \\
julio & el decreto de Macri llega a la Corte Suprema, que aun ha de pronunciarse sobre su legalidad \\
octubre & planificado el lanzamiento del LibreRouter
\end{longtable}


\subsection*{Mapeo de la controversia}

En la Figura~\vref{fig:mapeo} tenemos una intento de mapear los Actores-Red en la controversia de la soberania comunicativa, en acuerdo con~\autocite{Venturini2010a}.

\begin{sidewaysfigure}
  \centering
  \includegraphics[width=\columnwidth]{mapeo-controversia}
  \caption[Mapeo]{Controversia de la soberania comunicativa utilizando ANT} % The text in the square bracket is the caption for the list of figures while the text in the curly brackets is the figure caption
  \label{fig:mapeo}
\end{sidewaysfigure}




