\section{Poema XV, Tu risa und El tigre in Re-vision}

% Was ist mein Projekt/Meine These?? Ganz klar hier und in der Allgemein-Intro formulieren

% Vlt move Revision-Erklärung zu der Allgemein-Intro?
In Anlehnung an Adrienne Rich möchte ich in diesem Kapitel eine \textit{Re-vision} einiger Liebesgedichte Nerudas exemplarisch durchführen.
Rich definiert \textit{Re-vision} als ``the act of looking back, of seeing with fresh eyes, of entering an old text from a new critical direction''.
Sie unterstreicht, dass solche Revisionen von literarischen Werken essenziell sind, um, unter anderem, mögliche Lebensentwürfe für Frauen neu zu sehen, um sich der (Selbst-)zerstörung in einer Männer-dominierten Gesellschaft zu widersetzen~\cite{Rich1972}.
Dies mag auf dem ersten Blick etwas dramatisch klingen, wenn wir uns allerdings diverse Phanomäne(syn) vor Augen führen, wie Feminizide, den Kampf um das Recht am eigenen Körper (z.B. Abtreibungsrechte, reproduktive Selbstbestimmung), Essstörungen, ..., die Frauen jeden Tag unmittelbar erleben, wird es klar, dass es sich keineswegs um eine Überspitzung, sondern immer noch um die triste Realität handelt.

Was wir in Frage stellen und dekonstruieren sollen sind, wie Annette Kolodny anmerkt, gemeinsame kulturelle Annahmen, die so tief verwurzelt sind und schon so lange vorherrschen, dass Kritiker*innen und Leser*innen aufgehört haben, diese als solche wahr zu nehmen und kritisch zu hinterfragen~\cite{Kolodny1980}.
Männerdominanz und Frauenpassivität werden anstatt dessen als natürliche Gegebenheiten gesehen, die durch ihre Darstellung und gar Verherrlichung in literarischen Werken weiter verfestigt werden.
%vgl auch Duncan

In diesem Kapitel möchte ich gezielt alternative Lektüren der Gedichte ``Poema XV'' (``Me gustas cuando callas\ldots''), ``Tu risa'' und ``El tigre'' anbieten. 
% alternative - weitere Möglichkeit nach Kolodny; auch, im Sinne von "from a new critical direction" -> welche Direction ist gemeint?
%Ich möchte Erkenntnisse aus den Women und Gender Studies (auf Deutsch?: Frauen- und Geschlechterstudien?) heranziehen, um die etablierten Leseweisen dieser Werke aufzubrechen. % rausgenommen, da ich keinen neuen theoretischen Rahmen an der Stelle mit einbeziehe
%TODO wenn das so ist, bräuchte ich evtl besagte Erkenntnisse in Kapitel 3
Das Ziel ist, aus der Bequemlichkeit auszubrechen, die uns vorgefertigte (männer-zentrierte) Interpretationen für kanonisierte Werke anbieten (siehe Kapitel~\ref{chap:canon}).

Die Gedichte wurden unter anderem deswegen ausgewählt, weil sie aus zwei verschiedenen Gedichtsbändern kommen, die in ihrer Verfassung fast 30 Jahre auseinander liegen:
``Poema XV'' gehört zu \textit{Veinte poemas de amor y una canción desesperada}, die 1924 veröffentlicht wurden, und ``Tu risa'' und ``El tigre''—zu \textit{Los versos del capitán}, dessen Erstveröffentlichung im Jahr 1952 war.
Das lässt uns vermuten, dass sich der Stil, die Bilder, die Anschauung des Autors gewandelt haben können.
Im Weiteren gilt, die Frauen- (und Männer-)bilder, die durch diese Gedichte evoziiert werden, zu analysieren und zu diskutieren, in wie fern eine Entwicklung dieser stattgefunden hat.

\subsection{Poema XV}
% Veinte poemas: Allgemeiner Eindruck
Die gesamte \textit{Veinte poemas}-Sammlung wird durch melancholische Begriffe durchzogen:
Dunkelheit, Dämmerung, Nacht, Herbst, Kälte, Tod, Abwesenheit und damit assoziierte semantische Felder sind häufige Bilder im Band.
Die weibliche Geliebte ist kaum da und wenn schon ist sie ``callada'', ``muda'', ``lejana''.

Wenn sie überhaupt angeredet wird, adressiert sie das lyrische Ich nicht als gleichberechtigte, vollwertige Person mit eigenen Gefühlen, Wünschen und Gedanken.
Vielmehr werden auf sie die Wünsche und Gefühle des lyrischen Ichs projiziert:
``A nadie te pareces desde que yo te amo'', ``Ah déjame recordarte cómo eras entonces, cuando aún no existías'', ``te pareces a mi alma'';
beziehungsweise das lyrische Ich richtet sich in seinem Diskurs an einzelne ihrer Körperteile — ``tu cintura de niebla'', ``tus brazos de piedra transparente'' — oder an sie als Körper (``Amé desde hace tiempo tu cuerpo de nácar soleado''), nicht als Person. %nácar=Perlmutt
``Muñeca'' ist eine weitere wiederholte Anrede, der eine sexistische Haltung zugrunde liegt.
Eine zentrale Eigenschaft von Puppen, für die das Wort auch sehr häufig metaphrosich steht, ist hübsch auszusehen. %noch ein Kommentar dazu? So wirkts etwas knapp
Zudem sind sie bequem stumm und eben als unbelebte Gegenstände lassen sich von einem Menschen (Mann) ganz nach seinen Wünschen und Vorstellungen einsetzen. % anderen rumkommandieren lassen
% https://books.google.de/books?hl=bg&lr=&id=Y1eVeyOTuNQC&oi=fnd&pg=PT6&dq=calling+women+dolls&ots=BV9aH9j-Mw&sig=kmypmucRZ1apB1rTIqyvRjv9eWI#v=onepage&q=calling%20women%20dolls&f=false

% Poema XV
Poema XV wird oft als einen Hochpunkt der romantischen Liebe zelebriert~\cite{Lagos1975} (mind noch 1 Quelle!).
Dies scheint besonders brisant, weil wenn man die ersten zwei Zeilen liest, da wortwörtlich steht, dass das lyrische Ich es am liebsten mag, wenn die Geliebte gar nichts sagt und es so ist als ob sie gar nicht da wäre.
Das erinnert fast an den Alte-Männer-Humor, der sich immer davon beschwert, wie lästig die Ehefrau ist.
(Aber trotzdem sind die besagten alten Männer ja weiterhin seit Jahrzehnten verheiratet, weil es eben bequem ist, dass jemand da ist, die ihnen die Socken stopft, jeden Tag Essen zubereitet und über die man sich ständig auslassen kann.)
Man kann sich beim Lesen fast fragen, was am Text von Liebe spricht:
Wir interpretieren den als Liebesgedicht, da es ein Teil von einem Band ist, der \textit{Veinte poemas de \textbf{amor}} heißt.
Sonst sind die einzigen Zeichen, die uns an Liebe denken lassen, das wiederholte ``me gustas'' und die Erwähnung der eigenen Seele, die Bezeichnung der Geliebten als ``llena del alma mía''.
Beide sind allerdings mit Vorsicht zu genießen:
Das dreifache ``me gustas'' wird jedes Mal eher wieder negiert durch das, was folgt: ``cuando callas''.
Die beste Eigenschaft der Frau ist also quantitativ gesehen, dass sie ``como ausente'' ist.
Ihre Bedeutung für die Seele des lyrischen Ichs wird auch durch Wiederholungen unterstrichen: ``Como todas las cosas están llenas de mi alma / emerges de las cosas, llena del alma mía. / Mariposa de sueño, te pareces a mi alma''.
Anstatt allerdings die Liebe zu einer anderen Person zu bekräftigen, zeigen diese Zeilen eher die Liebe zu sich selbst auf.
Die begehrte Frau wird gar nicht wegen ihrer eigenen Eigenschaften geliebt, sondern weil sie sich besonders gut nach den Wüschen des lyrischen Ichs formen lässt und ihn an sich selbst erinnert.
Sie wird von ihm gefüllt, geschöpft, verwirklicht; sie existiert an sich nicht gar nicht.
Wenn die Frau stumm und abwesend ist, ist das lyrische Ich frei, die eigenen Vorstellungen und Wünsche auf sie zu projizieren, ohne dass ihre Persönlichkeit dem in die Quere kommt und dadurch fühlt sich das Ich wichtig und geliebt.

Vielleicht wäre es eben zwingend nötig, sich zu fragen, was Liebe eigentlich bedeutet.
Denn der Begriff wird so häufig und so selbstverständlich angewandt, dass er alles und nichts zu bedeuten scheint, oder noch schlimmer — in einer gewissen Mystik gehüllt wird.
Nur wenn wir eine gemeinsame Definition zugrunde legen, können wir urteilen welche Handlungen und Bilder dieser entsprechen und welche davon (und in welcher Form) abweichen.
An der Stelle würde ich mich gerne auf die feministische Wissenschaftlerin und Aktivistin bell hooks beziehen, die Liebe als eine Mischung folgender Bestandteile definiert: ``care, affection, recognition, respect, commitment, and trust, as well as honest and open commmunication''
und ferner unterstreicht, wie wichtig es ist, diese Definition früh zu lernen, um liebevolle Beziehungen aufbauen zu können~\cite{hooks2001}.
Kaum eines dieser Elemente scheint in ``Poema XV'' vorhanden zu sein.
Vlt eine twisted form von affection (für die Abwesenheit der Frau oder einzelne ihrer Körperteile).

\begin{comment}
  ** Adrienne Rich: "the word "love" in itself is in need of re-vision"

%Laurie Penny's Beziehungskonzept
"Both demand that we see another person as less than human, merely a body filling a prewritten role in our script or romantic or erotic extasy.
Both are wildly unrealistic, and both set us up to fail."

"We talk a lot about women as sex objects, but the reduction of women to love objects does just as much damage;[..]
The love object is a thing to be desired and pitied all at once.
She is the helpmeet, the saint, the fantasy.
She is never a complete person.
Whatever attributes make her interesting — maybe she can cook, or sew, or shoot a gun, or solve a crime — she exists ultimately for the hero's edification.
She is nothing without him."

"The colonisation of love by capitalist patriarchy is a deeply painful thing.
It means that structural sexism and cultural violence are played out on small, private stages, which is what makes them so very hard to recognise and resist.
    Human love is radical, and it is devastating.
And human love has been thoroughly captured by neoliberalism, by the mindset and mechanisms of profit."

"You can use it [human passion] to isolate people in antagonistic pairs and let them blame each other for the structural lack of sweetness in the world."

\end{comment}
%Es gibt in der Psychologie einen Begriff dafür und dieser heißt nicht Liebe sondern Limerence~\cite{WillBent2015}.
%Limerence zeichnet sich durch obsessive Gefühle und Gedanken für eine idealisierte Person aus, deren Erwiderung die limerente Person um jeden Preis sucht und eine Abweisung fürchtet.
%Die Person, die Limerence erfährt, sucht obsessiv nach Zeichen für die Erwiderung ihrer Gefühle und (über)interpretiert alles, was die ``geliebte'' Person tun oder sagt (oder nicht tun oder sagt).
%Limerence wird ferner mit Störungen in den Bindungseinstellungen assoziiert: % überlegen ob das reinkommt, aber kommt mir grad nach zu viel ausschweifen vor
%Also Menschen, die keine sicheren Bindungen zu Bezugspersonen in ihrer Kindheit erfahren haben, die eine Trennung von einer wichtigen Bezugsperson erfahren oder erwartet haben, tendieren mehr dazu Limerence zu anderen Personen zu erfahren, sich nach solchen Bindungen zu sehnen und die fehlende Erwiderung ihrer Gefühle sehr zu fürchten.
% "the authors definition of Limerence being “an involuntary potentially inspiring state of adoration and attachment to a LO involving intrusive and obsessive thoughts, feelings and behaviours from euphoria to despair, contingent on perceived emotional reciprocation”
% "ruminative thinking being an obsessive, repetitive, addictive, time consuming distractive state which impacted on normal daily activity."
% "These included feelings of confusion, destabilisation and being out of control, even to the point of stalking a LO" -> vgl El Tigre
% ", with the LO as an “idealised” (R6) person, as reflected in the perception of LO characteristics as “flawless and godlike” (R5)."
% "it is a creation within the Limerent person themselves"
% "The person experiencing Limerence has unmet relational needs, wants, desires, often from childhood, that somehow crystalise into a laser focused, intense desire for a single person (i.e., the LO) (R5)."
% "Limerence is an unexpected, overwhelming and debilitating experience that relates to the feeling of ‘being in love’ but in an intense form which is often, though not always, unreciprocated usually resulting in negative outcomes. The condition has been implicated as a major cause of relationship and family breakdown, as well as being related to anti-social behaviours, including stalking and self-harm (Tennov, 2005). The term was originally described by Tennov (1979), who noted that a Limerent individual becomes obsessed with securing emotional reciprocation, uses imagined reciprocation as temporary relief and has an intense fear of rejection from the focus of their attention (the Limerent Object/LO)."

%unerreichbarkeit; vorstellungen des sängers auf die Geliebte projiziert
%Diese haben wir durch den Minnesang gelernt mit (tragischer) romantischer Liebe zu assoziieren, sie haben per se allerdings kaum was mit Liebe zu tun.
%Ein Kernmerkmal dieser Liebe war immer, dass sie unerreichbar bleibt und sich eher in der Fantasie des Sängers als in der Realität abspielt. %TODO Quelle??
%vlt ganz raus, weil mir grad nicht einfällt wofür es gut ist, nur mit einem Satz zu erwähnen, dass unerreichbare Liebe aus der Minnesangtradition kommt

Alle anderen Bilder im Gedicht entstammen den semantischen Feldern \textit{silencio} -- \textit{noche-oscuridad} -- \textit{sueño}.
Das gezeichnete Bildnis der Geliebten erinnert an ein Gespenst.
Sie ist ``como ausente'', ``parece que los ojos se te hubieren volado'', ``como si hubieras muerto'', ``mi voz no te alcanza''.
``Mariposa de sueño'', ``mariposa en arrullo''.
Die Frau wird aktiv verstummt: ``parece que un beso te cerrara la boca'', wie in diesem schlechten romantischen Filmen-Cliché, in dem Frauen, die aufgebracht sind und ein ernsthaftes Gespräch führen wollen, einfach mal (zunächst gewalttätig) geküsst werden und dann ist alles gut. -> wrong on so many levels.
Der allgemein triste/negative Ton des Gedichts, die Atmosphäre von Unwirklichkeit und Tod werden einzig durch die letzten 2 Zeilen durchbrochen: ``Una palabra entonces, una sonrisa bastan'' für das lyrische Ich.
Das Wort und das Lächeln markieren den einzigen ``hellen'' bzw. ``aktiven'' Moment.
Ich finde allerdings, das Ganze ist schon seit dem 1. Satz nicht mehr zu retten.
Oder doch: Die Passivität, die Stille, die Abwesenheit der Frau können fast als Praxis der Resistenz ihrerseits betrachtet werden.
Diese Leseweise wird auch von Cynthia Duncan angeboten, die das Schweigen als eine Methode der Distanzierung vom lyrischen Ich erkennt, als Strategie der Loslösung, als Strategie, die dazu dient, ``die Andere'' zu bleiben~\cite{Duncan1992}.

\begin{comment}
  %%% form
  * 4 x 4Line verse: Quartette (lookup wie das korrekt heißt auf Deutsch!) + 2 x 2Zeiler (Couplets) --> erinnert ein bisschen an Sonnetform aber nicht ganz (Wie heißen nochma die verschiedene Sonnetformen? Nicht die von Gongora sondern die von Shakespeare meine ich grad: 3x4Zeilen und dann 2)
\end{comment}

\subsection{Tu risa}

Wie bereits erwähnt, gehört ``Tu risa'' zum Gedichtsband \textit{Los versos del capitán}.
Im Gegensatz zu den 1924 erschienenen \textit{Veinte poemas de amor} wurden die Werke in \textit{Los versos} nicht von einem jugenldichen, sondern von einem erwachsenen Neruda geschrieben.
Die Gedichte sind weniger melancholisch, es gibt sinnlichere/präsentere Bilder vom lyrischen Du und an diversen Stellen—explizite sexuelle Referenzen.
Es fällt schwer, alle Texte in \textit{Los versos del capitán} unter einem Gesamteindruck zusammen zu fassen;
dafür sind die einzelnen Teile vom Band viel zu unterschiedlich.
Diese bieten jedoch weiterhin ein fragwürdiges Weiblichkeitsmodell an:
Wir finden immer noch keine aktiven, autonom handelnden Frauenfiguren, die mit ihren Gedanken und Gefühlen berücksichtigt werden.

``Tu risa'' wurde deswegen für eine nähere Analyse ausgewählt, weil es auf dem ersten Blick als ``schön'' und ``romantisch'' (wie viele der anderen Liebesgedichte Nerudas auch) gelesen werden könnte, %TODO erklären warum schön und romantisch in Gänsefüßchen stehen, wenn ich das so behalte; das Problem ist, beide Begriffe sind so hingeklatscht etwas lapidär und ungenau
beim näheren Betrachten jedoch auch problematische Weltbilder offenbart.

Inhaltlich wird im Gedicht beteuert, dass das Lächeln der Geliebten das Allerwertvollste für das lyrische Ich ist.
Der männliche Sprecher kann ohne Nahrung (``quitame el pan'') oder Luft (``quitame el aire'') überleben, aber nicht ohne das Lächeln.
Weitschweifig wird die Schönheit dieses Lächelns beschrieben:
Es wird mit fröhlich sprudelndem Wasser (``el agua que de pronto/estalla en tu alegría'') und Blumen (``la flor azul, la rosa''), mit einer silbernen Welle (``la repentina ola/de plata que te nace'') verglichen.
%Seine Bedeutung wird auf Basis von vielen Konstrasten verdeutlicht. %noch ein beispiel für die kontraste?

Die Wahrnehmung der Geliebten als einzelne Körperteile ist auffällig abwesend im Vergleich zu anderen Gedichtent aus dem Band.
Das einzig Präsente ist die ``risa'', die metonymisch für das lyrische Du steht.
Die Frau wird entsprechend auch hier auf eine einzige Eigenschaft reduziert.
Die Poetisierung davon verfestigt die gesellschaftliche Annahme, dass schön aussehen und lächeln inherente Teile von Weiblichkeit wären.
Dies erkennen wir auch im Alltag, unter anderem im allgegenwärtigen sexistischen Spruch ``lächel doch mal''; als ob frau jedem x-beliebigen Typen auf der Straße etwas schuld wäre.
Das Glück im Privaten, das Lächeln der geliebten Frau dienen hier als Ausgleich für das Übel in der Welt, wo das lyrische Ich als aktiver Krieger unterwegs ist, der die Welt gestaltet—``la lucha es dura''.
Sie ist immer zuhause und wartet, als er mit ``ojos cansados'' zurückkehrt.
Sie existiert lediglich, um um ihn zu sorgen, ihn zu trösten, schön zu lächeln, damit alles gut wird.
Die Opposition Privat-Öffentlich beziehungsweise drinnen-draußen (die Frau ist nur im Privaten tätig (syn), der Mann dagegen bewegt sich im öffentlichen Raum), wird implizit bestätigt durch die Phrase ``al entrar''.
Die untergeordnete Position der Geliebten wird im Gedicht nicht nur durch die Bildsprache deutlich sondern auch durch die viele Imperative, mit denen sie adressiert wird.
%Fazit??

\begin{comment}
    %TODO entweder hier mit in die Analyse einbauen oder in Beziehung zu bell hooks's definition of love setzen; oder in das Fazit für chap4 für alle Gedichte auslagern

* was für romantische Beziehungskonzepte werden in unserer Gesellschaft akzeptiert (Laurie Penny) (die romantische Zweierbeziehung als einziges gesellschaftlich anerkanntes Model + deren Funktion als Ausgleich für strukturell-polit. Probleme/Missstände)
  ** hoher Stellenwert der 2er heterosexuellen Beziehung
  ** für Frau: die ultimate (einzig mögliche?) persönliche Erfüllung?
  ** Duncan: "the position of the woman in the texts is an enviable one"
  ** Duncan: "Neruda's women  possess the qualities we have been taught to covet: they are beautiful,  sensual, desirable, and eminently agreeable.[..] They want what all  women have been conditioned to want in the way of self-fulfillment:  romantic love. They may have no identity, no voice, no sense of purpose,  but, in exchange, they are promised the reward of man's eternal  devotion if they agree to play their role properly. [...] The message to  female readers is clear: woman bears the responsibility of attracting,  nurturing, and keeping man's love alive. He is not obliged to love her  and, as he often reminds her, he can fare better without her than she  can without him. Without love, he is still a man, but she, without love,  is nothing."
  ** Adrienne Rich: "the myth of the special woman" and its destructiveness: "We seem to be special women here, we have liked to think of ourselves as special, and we have known that men would tolerate, even romanticize us as special, as long as our words and actions didn't threaten their privilege of tolerating or rejecting us according to their ideas of what a special woman ought to be [..] how divisive and ultimately destructive is this myth of the special woman"
  ** Molly Haskell (in Duncan's text): "the idea of woman's inferiority, a lie so deeply ingrained in our social behavior that merely to recognize it is to risk unraveling the entire fabric of civilization"
  ** Laurie Penny: "Mit dem Posen der romantischen Liebe, besonders der heterosexuellen romantischen Liebe in der Ehe, meinen wir uns der bitteren Realität von Arbeit und Tod verweigern zu können - dabei präparieren sie uns genau dafür, verkuppeln uns und stecken uns in lauter kleine Schubladen aus Leid und Leidenschaft: Du und ich gegen den Rest der Welt, Baby. Wir verlieben uns, weil das leichter ist als wenn wir lernen müssten, uns freizuschwimmen."
  ** Laurie Penny: "Man kann Menschen in antagonistischen Paaren zusammenspannen und vor ihrer Umwelt isolieren, damit sie sich gegenseitig die strukturelle Herzlosigkeit dieser Welt in die Schuhe schieben." (S.226)
  ** Laurie Penny: "Man kann die Suche nach einer einfachen Bindung zu einem elenden, ermüdenden RItual machen, das eine strenge Geschlechtskonformität voraussetzt und den menschlichen Geist unterdrückt."
  ** Laurie Penny: "Frauen aller Gesellschaftsschichten wird beigebracht, dass sie sich zuallererst um die Liebe von Männern bemühen müssen, um darüber ihren Wert zu taxieren, Männer auf sich aufmerksam zu machen. Und in allen Gesellschaftsschichten wird die romantische Demütigung dafür genutzt, Frauen klein zu machen. Jeder heterosexuelle Mann, mit dem ich mich je über Partnersuche unterhalten habe, dass Frauen in romantischen Dingen alle Macht haben, auch die ultimative Macht, auf die sexuelle Avancen eines Mannes einzugehen oder sie zurückzuweisen und einen Mann in den Kreis ihrer "Freunde" einzureihen, was dieser als völlig schwachsinnig empfindet, denn natürlich will kein echter Mann für eine Frau nur ein Freund sein. Die Macht, nein zum Sex zu sagen, finden Männer ungeheuerlich und erachten es daher nur als gerecht, dass Frauen und Mädchen alle anderen Spielarten der Macht seit vielen trostlosen Generationen vorenthalten werden. Das hätte etwas für sich, wenn die Macht, nein zum Sex zu sagen, in der Praxis respektiert würde." (S.225)

\end{comment}

\subsection{El tigre}

Anstatt exemplarisch ein Gedicht aus dem dritten Liebeslyrikband Nerudas—\textit{Cien sonetos de amor}—zu lesen, hab ich mich dafür entschieden, noch auf ``El tigre'' aus \textit{Los versos del capitán} einzugehen.
%Ich konnte es mir nicht verkneifen, weil das Gedicht so krass ist.
Es wurde aufgrund seiner Eklatanz(syn?) ausgewählt und kann gewiss nicht als ``typischer'' Vertreter für das gesamte Band stehen, da, wie bereits erwähnt, die Gedichte dadrin so heterogen sind.
``El tigre'' stellt, meiner Meinung nach, den Hochpunkt der Bestätigung von Sexismus und \textit{rape culture} in Nerudas Liebeslyrik dar.

Überhaupt ist der gesamte Abschnitt ``El deseo'', in dem die drei Gedichte nach Tieren benannt sind, sehr problematisch.
Die bereits in \textit{Veinte poemas de amor} angehende Tendenz, das lyrische Du nicht als Person, sondern als einzelne Körperteile zu begreifen, wird hier zugespitzt und noch stärker sexualisiert.
Wie das berühmte Essay der Filmwissenschaftlerin Laura Mulvey argumentiert, werden hier ganz besonders Bilder von einer männlichen Perspektive für ein männliches Publikum (oder für das was die vorherrschende patriarchale Gesellschaft für ``männlich'' hält) erschaffen und diese durch scheinbare Objektivität und Entfremdung als universal/universell? menschlich dargestellt~\cite{Mulvey1975}.
Das lyrische Ich wendet sich/schwärmt von \textit{caderas} und \textit{pies}, von \textit{un cráter, una rosa/de fuego humedecido}—seine Metapher für die weiblichen Genitalien.

Zudem klingen die Gedichte, vor allem ``El condor'' und ``El tigre'', bedrohlich und sinister, Gewalt gegen das lyrische Du wird grafisch beschrieben und glorifiziert.
Die in ``El tigre'' evoziirten Atmospäre und Bilder spannen Urwald und Gefahr auf, wo das lyrische Ich, in der Gestalt eines Tigers, zuhause ist.
Der Tiger lauert die Geliebte auf (``Te acecho entre las hojas''), die übrigens wieder bloß als vereinzelte Körperteile auftritt: ``pecho'', ``caderas'', ``miembros''.
Im Mittelpunkt steht, dass diese Körperteile wörtlich und besonders grausam vom lyrischen Ich auseinander gestückelt werden:
``de un zarpazo derribo/tu pecho, tus caderas.''
Der letzte Teil vom Gedicht, der im Vergleich am längsten ist, beschreibt noch die einsame Wache vom Tiger und vom Regen, die die Knochen und Asche der Ermordeten behüten.
Man könnte vielleicht erstmal meinen, dass es sich dabei um eine Art Reue handelt, ein freiwilliges Exil in die Einsamkeit für die grausame Tat.
Das scheint allerdings nicht sehr wahrscheinlich.
Viel mehr ist(syn) der Wald das ``natürliche Habitat'' vom Tiger, der an der Stelle eine Metapher für einen starken männlichen Männermann ist, der in der Einsamkeit gedeiht.
Die letzten zwei Wörter des Gedichts bezeichnen die begangene Gewalttat als ``amor asesino''—
ein Oxymoron, da Mord und Gewalt genaue Gegenteile von Liebe sind, die, wie bereits angesprochen, auf Respekt, Fürsorge, Zuneigung, Vertrauen und Verantwortung beruht~\cite{hooks2001}.
%"genuine love ( a combination of care, commitment, trust, knowledge, responsibility, and respect)"

Man könnte ``El tigre'' mit drei Wörtern beschreiben, diese wären allerdings nicht ``universelle menschliche Gefühle'', sondern ``Glorifizierung von \textit{rape culture}''.
Denn genau diese Normalisierung und gar Verherrlichung von Gewalt gegen Frauen machen \textit{rape culture} aus.

In der \textit{Antología poética}, die 1983 vom spanischen Dichter (und engen Freund Nerudas) Rafael Alberti für den Madrider Verlag Espasa-Calpe kuratiert wurde, fehlt ``El tigre''.
Alberti hat, nach eigenen Worten, die schönsten Gedichte für dieses Sammelband ausgewählt, und er fand anscheinend auch (obwohl ``die Auswahl schwer war''~\cite{Alberti1983}), dass ``El tigre'' nicht unbedingt dazu gehört.


\subsection{Fazit für die Section}

\begin{comment}
% Erklärung für die positive Rezeption?? -> Not quite sure where to put it: whether here or in 4.
[Kolodny1980]
* die von der Gesellschaft angesehenen Werte werden in unsere Lektüren reinprojiziert.
--> Weiterführend (Laurie Penny) Die Gesellschaft sagt uns, dass romantische Liebe schick und erstrebenswert (vor allem für Frauen) ist, dass sie sich nur darin als Menschen verwirklichen können.
--> Also projizieren Leser*innen (und wirklich auch *innen!) diese Werte in Nerudas Liebeslyrik und feiern sie

* Und anders rum: die Kunst hilft auch diese Normen zu verfestigen/untermauern/bekräftigen
"not only the mirroring of life in art, but also the normative
impact of art on life."
\end{comment}


% Aus der Einleitung: Im Weiteren gilt, die Frauen- (und Männer-)bilder, die durch diese Gedichte evoziiert werden, zu analysieren und zu diskutieren, in wie fern eine Entwicklung dieser stattgefunden hat.
% TODO: also hat eine Entwicklung stattgefunden?

Das männliche lyrische Ich ist in Nerudas Liebeslyrik häufig nicht als solches grammatikalisch markiert.
Es ist allerdings konstruiert und definiert in binärer Opposition zum weiblichen lyrischen Du (was auch nicht unbedingt grammatikalisch markiert ist).
Durch die Beschreibungen/zugewiesenen Position/evozierten Bilder sind diese eindeutig als männlich/weiblich bestimmbar.
Die Rollenverteilung und die Klischeebilder lassen allerdings keinen Zweifel zu wer wer ist.

\begin{comment}
% Allgemeine Gedanken. Am Anfang oder ans Ende stellen; oder Zwischendurch als Zwischenschlussvolgerungen; (Aus dem Handout)
   * Duncan erklärt weiterhin, dass unter dem Vorwand der Lobpreisung des Frauseins und dessen Einzigartigkeit, diese Einschätzungen über Nerudas Liebeslyrik textuelle Widersprüche vorenthalten, welche symptomatisch für die Unterdrückung der Frau in einer patriarchalen Kultur seien. Nur dadurch, dass die Frau zu einem Liebesobjekt wird, erwacht sie zu Leben. Ohne ihren männlichen Liebhaber ist sie “vacía, sin substancia“. Diese Darstellung der Frau in den Texten kontrastiert drastisch neben der des männlichen Sprechers, wessen Existenz nicht von der physischen Präsenz oder der weiblichen Liebe abhängig ist. Ohne ihre Liebe mag er sich verloren fühlen, aber seine Identität gerät nie in eine Krise, er sagt, ohne zu zögern, “todos saben quién soy“.

    [Kolodny1980]
"Kennard appropriately cau-
tions us against drawing from her work any simplistically reductive
thesis about the mimetic relations between art and life. Yet her
approach nonetheless suggests that what is important about a fic-
tion is not whether it ends in a death or a marriage, but what the
symbolic demands of that particular conventional ending imply
about the values and beliefs of the world that engendered it"
--> can be used as an argument in the analysis itself?

"The power relations inscribed in the form of conven-
tions within our literary inheritance, these critics argue, reify the
encodings of those same power relations in the culture at large."
--> what is written about adorns the subjects with importance, not the other way round, vgl auch "Little Women"

auch
"these critics also insist upon exam-
ining not only the mirroring of life in art, but also the normative
impact of art on life."

  %Revision. Adrienne Rich
  [Rich1972]
%  Definition:
``Re-vision-the act of looking back, of seeing with fresh eyes, of entering an old
text from a new critical direction-is for us more than a chapter in cultural history:
it is an act of survival. Until we can understand the assumptions in which we are
drenched we cannot know ourselves. And this drive to self-knowledge, for woman,
is more than a search for identity: it is part of her refusal of the self-destructiveness
of male-dominated society. A radical critique of literature, feminist in its impulse,
would take the work first of all as a clue to how we live, how we have been living,
how we have been led to imagine ourselves, how our language has trapped as well
as liberated us; and how we can begin to see-and therefore live-afresh.''
-- das letzte, der Teil mit dem new beginning, Ausblick, vlt in den Fazit integrieren
zum Thema "self-destructiveness" -> schädliche Stereotypen von Passivität, von erstrebenswerter Position als "Begehrungsobjekt", von romantischer Liebe, die der Frau die Agenz absprechen

% myth making and romantic tradition; vgl Laurie Penny
``Jane Harrison, the great classical anthropologist, wrote in 1914 in a letter to her
friend Gilbert Murray:
By the by, about "Women," it has bothered me often-why do women never want
to write poetry about Man as a sex-why is Woman a dream and a terror to man and
not the other way around? \ldots Is it mere convention and propriety, or something
deeper?
I think Jane Harrison's question cuts deep into the myth-making tradition, the
romantic tradition; deep into what women and men have been to each other; and
deep into the psyche of the woman writer.''

"It strikes me that in the work of both [Sylvia Plath and Diane Wakoski] Man appears as, if not a dream,
a fascination and a terror; and that the source of the fascination and the terror is,
simply, Man's power-to dominate, tyrannize, choose, or reject the woman. The
charisma of Man seems to come purely from his power over her and his control
of the world by force"

``One answer to Jane Harrison's question has to be that historically men and
women have played very different parts in each others' lives. Where woman has
been a luxury for man, and has served as the painter's model and the poet's muse,
but also as comforter, nurse, cook, bearer of his seed, secretarial assistant and copy-
ist of manuscripts, man has played a quite different role for the female artist.''

"We seem to be special women here, we have liked
to think of ourselves as special, and we have known that men would tolerate, even
romanticize us as special, as long as our words and actions didn't threaten their
privilege of tolerating or rejecting us according to their ideas of what a special
woman ought to be.
An important insight of the radical women's movement, for
me, has been how divisive and how ultimately destructive is this myth of the
special woman, who is also the token woman."
--vgl auch Tu risa; Laurie Penny

"she meets the image of Woman in books written by men. She
finds a terror and a dream, she finds a beautiful pale face, she finds La Belle Dame
Sans Merci, she finds Juliet or Tess or Salome, but precisely what she does not
find is that absorbed, drudging, puzzled, sometimes inspired creature, herself, who
sits at a desk trying to put words together."
--> what roles are absent?

``Because I was also determined to have a "full" woman's life, I plunged in my early twenties into marriage
and had three children before I was thirty. There was nothing overt in the environ-
ment to warn me: these were the '50's, and in reaction to the earlier wave of fem-
inism, middle-class women were making careers of domestic perfection, working to
send their husbands through professional schools, then retiring to raise large
families. People were moving out to the suburbs, technology was going to be the
answer to everything, even sex; the family was in its glory. Life was extremely
private; women were isolated from each other by the loyalties of marriage. I have
a sense that women didn't talk to each other much in the fifties-not about their
secret emptinesses, their frustrations.''
--vlt benutzen als Erklärungsversuch zu warum Frauen die Gedichte toll fanden?2
--auch Laurie Penny mit was uns vorgegauckelt wird mit dem Mythos der Romantischen Liebe

``But in those earlier years I always felt the conflict as a
failure of love in myself. I had thought I was choosing a full life: the life available
to most men, in which sexuality, work, and parenthood could coexist. But I felt,
at 29, guilt toward the people closest to me, and guilty toward my own being.''
--vgl Laurie Penny at the beginning "We can have it all... The army of nannies and cleaning ladies which this life style requires, can they have it all too??"


``Yet I began to feel that my fragments
and scraps had a common consciousness and a common theme, one which I would
have been very unwilling to put on paper at an earlier time because I had been
taught that poetry should be "universal," which meant, of course, non-female.''
-- universal = non-female

"and egotism-a force directed by men into creation, achievement, ambition, often at the expense of others, but justifiably so.
For
weren't they men, and wasn't that their destiny as womanly love was ours? I know
now that the alternatives are false ones-that the word "love" is itself in need of
re-vision."
-> vgl die Liebe im Sinne von Nerudas lyrischem "ichs"
-> vgl bell hooks "all about love"

"so man will have to learn to gestate and give birth to
his own subjectivity-something he has frequently wanted woman to do for him."
"but women can
no longer be primarily mothers and muses for men: we have our own work cut
out for us."
-- vlt use in Fazit; Männer sollten auch für sich Verantwortung übernehmen und aufhören sich als Frauen beschützende Helden zu inszenieren


[Duncan1992]
"his frequent treatment of her as a fragmented body part, designed for man's pleasure, and by his insistence that his love for her is the only solid core to her being.
Neruda, consciously or not, subordinates the female to the role of man's creation in his poems.
Man is the speaker, the creator of meaning, while woman is the silent, passive listener.
Only by becoming an object of love does the woman come into being.
Without her male lover, she is ``vacía, sin substancia'' (El amor, LVDC)."

"At times, the woman's abscence is even considered preferable, since it allows the male to recreate her in the text, and thus provides him with a heightened level of inspiration."
"Without her love, he may feel lost, but his identity is never in crisis"

"Araya notes that in Veinte poemas, ``la amada es esencialmente carne'' (165), and adds, ``siendo ella objeto pasivo del amor, no manifiesta su identidad ni dentro del libro ni fuera de él''(168).
He also notes, ``Ella no tiene voz propia en este poemario.[...]
Más que un sujeto que actúa y ama, la amada es un objeto al que se dirige la capacidad amatoria del yo'' (168).
Despite these astute observations, however, Araya continues to view the image of woman that emerges from the text as a positive one, for he believes that she is ``sometida a la ley natural'' (182) which conditions her to accept the male speaker as a superior being (159)."

Citing Araya:
``No es difícil entonces que la muchacha y la mujer que leen este libro se identifiquen con la imagen femenina que brota de él''(187)
-> "culturally accepted norm of romantic love"

"As E. Diedre Pribram has observed with reference to film, ``The function of a text is to position the spectator to receive cetain favoured—and restricted—meaning which the text `manages' for the viewing subject in keeping with dominant ideology. In this model the spectator is not an active part of the production of textual meaning but passive side of a unidirectional relationship in which the text disperses meanings while the speactator-subject receives them'' (5).
A similar situation occurs in Neruda's poems, for the reader's gaze is controlled by the ``yo'' who generates the discourse, much as the movie spectator's view is controlled by the seemingly invisible camera eye. Both situations place the reader/spectator in a limiting framework, without appearing to do so, by entrusting a fictional character with the omnipotent and coercive gaze which, in actuality, belongs to the machinery of neunciation.
In Neruda's poems, the male speaker is seen as the voice of authority; his view is rarely, if ever, questioned, since it is presented as the only view possible.
Because he duplicates dominate cultural values in his text, his voice has the ringof truth to it.
But as E. Ann Kaplan has noted, this ``dominating male gaze, carrying with it social, political, and economic as well as sexual power, relegates women to absence, silence, and marginality...'' (5). 
In this light, Neruda's text can be seen as both container and disseminater of ideoloy, for it pormotes the long-standing notion of male superiority and female passivity as natural facts when, in reality, they are merely culturally determined conditions."

"The first poem of the \textit{Veinte poemas} collection, which sets the tone for the whole work, is addressed not to the woman but to her body: ``Cuerpo de mujer''."
"the woman is his possession: ``cuerpo de mujer mía''."
"addresses her as a silent, passive recepient of his male desire"
"The speaker occasionally declares his love for the ``tú'' of the poems but, more often, he speaks of his love for ``tu cuerpo''."
Love for body parts:
``amo tu cuerpo alegre'' (19)
``senos perfumados'' (14)
``grandes ojos fijos'' (20)
``tu color y forma son como yo los quiero''(16)
``mujer de labios dulces''(16)
``carne, carne mía'' (Canción desesperada)

"En \textit{Los versos del capitán} published 28 years after the \textit{Veinte poemas}, we find entire poems dedicated to fragmented parts of the beloved's body: ``Tus pies'', ``Tus manos'', ``Tu risa''.
In each of these poems, the body part acquires importance because it evokes pleasure in the male"

"her purpose in life is to give him pleasure and comfort; she is ``hecha para mis brazos/hecha para mis besos/hecha para mi alma'' (El inconstante)"

"In ``Bella'', the speaker enumerates the parts of the woman's body which are most beautiful to him: ``La sonrisa en tu rostro,'', `finas manos y delgados pies'', ``tu cabeza'', ``tus ojos'', ``tus senos'', ``tu cintura'', ``tus caderas'', ``tu cuerpo'', ``tu voz, tu piel, tus uñas''.
These parts make up a whole which belongs to him: ``todo eso es mío''"

"Although the speaker is generally fascinated by the body of his beloved, at times it holds no special interest for him. On these occasions, her body becomes an empty symbol, signifying nothing"

"in the \textit{Cien sonetos de amor}, the speaker once again takes up the female body as fetish. He calls the woman's body ``mi territorio de besos y volcanes'' and speaks of ``tus pies creados para mí''(V)."

"defines his beloved in the same terms used by the adolescent lover in \textit{Veinte poemas}: ``eres cuerpo''."
-> Kein Fortschritt 30 Jahre später. Gleiche Bilder und Symbole in allen 3 Bändern

From \textit{Cien sonetos}
``quiero comer tu piel como una intacta almendra'' (XI)
``manzana carnal'' (XII)
``pan que devoro'' (XIII)

"Mulvey has observed that ``Woman... stands in patriarchal culture as signifier for the male other, bound by a symbolic order in which man can live out his phantasies and obsessions through linguistic command by imposing them on the silent iage of woman still tied to her place as bearer of meaning, not maker of meaning''."

"What allows her to occupy all of these positions with equal ease is her silence, her lack of a voice. She becomes whatever her lover imagines her to be at the moment"

"He refers to the time before she met him as the time ``cuando aún no existías'' (``La noche en la isla'', LVDC)"

-> Vlt move to Reception
"Emir Rodríguez Monegal has praised this section, specifically the poem ``El tigre'', as being representative of ``el deseo de todos los hombres''.
If this is indeed so, it is a frightening thought, for the poems deal overtly with rape and contain numerous images of violence directed agains the female figure."
-> Rape culture for analysis. 
Rape ist nicht Liebe oder Desire; Es geht um Macht;
Und genau durch solche Aussagen, die solche Vorstellungen normalisieren, anstatt sie als Gewalt zu diffamieren, wird diese weiter aufrechterhalten.

``El tigre''
- male speaker "portray[ed] as a wild animal, lying in the wait for his female victim"
- "spies on the female"
- "Sated after his feast, the tiger/male speaker settles down to guard what is left of the woman (``tus huesos, tu ceniza'') and gloats over what his ``amor asesino'' has accomplished"
-> news flash, das ist definitiv keine Liebe. (vgl vlt Definition von Liebe von bell hooks "all about love")

Vlt check aus "Der Ursprung der Liebe" von Liv Stromquist als potentielle Analyse-Inspiration.

"Las vidas" in \textit{Los versos del capitán}:
"more socially and politically committed male speaker"
"his attitude toward the female beloved has changed only in superficial ways"
"She is not his equal in the political struggle, but merely a campfollower who looks after the needs of er man.
He is still the one in charge, the one who gives orders"
``Limpia ese fusil, camarada'' (``El amor del soldado'')
``Te veo/lavando mis pañuelos,/colgando en la ventana/mis calcetines rotos'' (``No sólo el fuego'')
"basic premise that woman's only contribution to social and political change is to stay at home and wait for her man to return"

"Without him, she is nothing: ``Si me apartas tu vida / morirás / aunque vivas. / Seguirás muerta o sombra / andando sin mí por la tierra''(``El desvío'')."


\end{comment}

\begin{comment}
   %evtl nach 3 Schieben
* Male gaze
  ** das ist auch was die Menschen von der Kunst und Gesellschaft im Allgemeinen kennen:
     dass Kunst aus männlicher Perspektive dargestellt und rezipiert wird; dass Frauen erotisiert werden
  ** In Nerudas Gedichten hat die Stimme des männlichen Sprechers Autorität, da dessen Standpunkt/Betrachtung selten oder gar nicht in Frage gestellt wird, weil seine Stimme als die einzig mögliche Perspektive dargestellt wird.
  ** Filmtheorie: Übereinstimmung von 3 Blicken: Kamera, (männl.) Protagonist und Zuschauer (vgl Laura Mulvey)
     -> Objektifizierung der Frau, Frau als passive Rezipientin sexueller Aufmerksamkeit
     -> Verfestigung patriarchalischen Normen

* "als nah, natürlich wahrgenommen", die Leser*innen können sich damit identifizieren
  ** Menschen können sich direkt mit den Figuren identifizieren, dadurch dass es nicht extradiagetisch ist
  ** Duncan spricht von "real", "natural/beautiful expression of male/female relationships", die Menschen können sich damit identifizieren, sie sehen die Gedichte als Reflexion der gesellschaftlichen Idealen, die sie gelernt haben anzustreben
  ** Duncan:  "a sort of manual on the workings of modern romantic love and have come  to be regarded by many, as a standard agains which real life  relationships can be judged"
  ** Duncan: ".. in fact it seems extraordinarily "real" and "natural", as if it were not a literary construct at all, but, rather a reflection of real-life experiences. Readers who turn to these poems to learn "what love is supposed to be like", "what men are like", and "what women are like", ultimately receive a skewed message told from the traditional dominant male position"

* Letztendlich wechselt die Rolle der Frau in Nerudas Liebeslyrik, wobei all diese Rollen männliche
Fantasien involvieren: Manchmal ist sie die umsorgende, geerdete Mutter zu seiner jungenhaft
verlorenen Person, andere Male ist sie die gefährliche Verführerin, welche ihn seiner Seele zu berauben
droht. Andere Male ist sie die unbewegliche/unantastbare Ikone, ein erstarrtes Bild von Schönheit oder
Sinnlichkeit, und andere Male wiederum ist sie die beschäftigte, kleine Hausfrau. Durch die
Verstummung der weiblichen Stimme kann der männliche Liebhaber die Frau zu allem werden lassen,
wohin ihn seine Vorstellungskraft in diesem Moment treibt. “If she is unique in any way, it is only
because she is his special creation“.
Auch wird Sie als Spiegelung seiner Seele gesehen und ebenso als seine Gefangene, welche sich zu
fügen hat und sich glücklich schätzen darf, unter all den anderen auserwählt worden zu sein. (Bsp.
hierfür p.434 rechts mittig bis unten, diese Denkweise ist bis heute u.a. in lateinamerikanischen
Ländern nicht ganz unüblich.)

\end{comment}
