\section{El proyecto AlterMundi}

\subsection{Desarollo histórico}
% oder \subsection{Altermundi: fundación, objetivo, estructura, desarrollo histórico}

La asociación civil AlterMundi comienza a formalizarse en el año 2012 con el objetivo de facilitar talleres de redes libres dentro del proyecto Arraigo Digital, iniciado por el Ministerio de Educación en Argentina con el fin de ``capacitar a los jóvenes en conocimiento, funcionamiento y armado de una red digital comunitaria, además de la introducción y práctica del software libre''~\autocite{Picolli2015}.
AlterMundi surge desde la cooperativa CodigoSur\footnote{\url{https://www.codigosur.org/quienes-somos/codigo-sur}}, un grupo de activistas de diferentes países de América Latina que facilita el desarrollo y uso de tecnologías libres\footnote{Cuando
 hablamos de tecnologías libres, no nos referimos a productos gratuitos, sino a tales que conceden a lxs usuarixs ciertas libertades, como la posibilidad de utilizar el software para lo que quieran, estudiar el funcionamiento del software, modificar el software según sus necesidades y compartir las versiones modificadas.
Para más información, veáse también la definición propuesta por la Free Software Foundation: \url{https://www.fsf.org/about/what-is-free-software}.}
de información y comunicación para diferentes movimientos sociales.
La forma legal de asociación civil les permite hacer alianzas con otras organizaciones formales, como por ejemplo la Universidad Nacional de Córdoba que provee un enlace al Internet o diferentes fundaciones que les han prestado suporte financial.
Lo interesante es que AlterMundi en si no es una red comunitaria, sino una organización de activistas que suporte el despliegue de tales redes por toda Argentina.

% add cronología --> in appendix

Al final, aquell proyecto no se realizó en la forma prevista~\autocite{Vaseva2016a},~\autocite{iv-nico-pillku}, pero la primer red local, QuintanaLibre, ha ido en linea igualmente.
Desde entonces el grupo básico de activistas, que es Altermundi, ha ayudado a diversas comunidades a construir redes comunitarias locales:
hasta hoy existen ....

\subsection{La red como infraestructura técnica}
Para las metas del trabajo actual, cuando hablamos de redes comunitarias vamos a entedender redes comunitarias inalámbricas (computer networks).
-- bzw las redes de AlterMundi son redes inalámbricas; <-- importante para el hardware que se usa, flexibilidad, etc.


% oder La red como infraestructura física

Bastante intrigante, aunque en un primer momento quizás un poco contralógica, resulta la observación de Latour que ``the expansion of digitality has enormously increased the material dimension of networks''~\autocite{Latour2010}.
Sin embargo, si reflejionamos un poco sobre el asunto, nos damos cuenta que sí, la tecnología digital deja muchas huellas y artefactos en el mundo físico: si nos fijamos, podemos descubrir las antenas, los routers y los cables per tot arreu(esp??).
Gulli deckel marcan los lugares de los cables transatlánticos~\autocite{video-internet-landscapes}.
Si seguimos las antenas y los routers, podemos reconstruir hasta un cierto grado las redes físicas de proyectos comunitarios.

% Commons; moechte ich aus der Ecke argumentieren?
En cierto sentido, las redes comunitarias se pueden ubicar en la lógica de los bienes comunes~\autocite{FiTre2015}.
Ya la denominación ``redes \textit{comunitarias}'' insinua esta idea.
Interesante sería fijarse en el valor ideológico que lleva esta noción.
Según Rieder, el término ``comunidad'' es exclusivamente positivo en la historia cultural anglo-sajona~\autocite{Rieder2012}, mientras en otros contexto el significado sea más ambivalente.
Por ejemplo en alemán, la noción se relaciona también con la ``Volksgemeinschaft'' (``comunidad popular'') de lxs nacis. % so what
No obstante, en el sentido/caso de ``commons'' y ``redes comunitarias'' la conotación es más la de la tradición anglo-sajona.

Aunqué probablemente partes de la técnica partenecen tecnicamente(syn!) a personas privadas, otras partes son más una propiedad comunal.
Aunqué la tecnología que utilizan las redes comunitarias es bastante robusta (vease Cap. X), está en el interés común de arreglar problemas que usuarixs particulares tienen, ya que si un aparato malfunciona toda la comunidad sufre.
También, como señalan De Fillippi y Tréguer, esto es particularmente el caso para las redes comunitarias inalámbricas, ya que estas dependen exclusivamente a free-to-use airwaves (``spectrum commons'')~\autocite{FiTre2015}.
Ellxs argumentan que en contraste con la wide-spread noción que la gestión de bienes comunes no funciona muy bien en práctica, en el caso de open spectrum, el approach tiene éxito:
el spectro está usado más thorough/racionalmente/..
% of spectrum and exclusive licensing still have the upper hand, they have often come short of fostering public interest goals, for instance by causing a very significant underutilization of this public resource"
Además, infraestructura pública, creada por el estado con el dinero de impuestos ha de ser bien común y no regalada a empresas privadas, argumentan lxs científicxs~\autocite{FiTre2015}.

\begin{comment}

[nico-iv-pillku]
"antenas que fabricamos localmente."

"El firmware que instalamos: AlterMesh, permite la auto-configuración de los equipos de manera que al arrancar el nodo ya “sabe” como participar de la mesh. "

"todo el pueblo funcione como una gran red local, lo que facilita la implementación de servicios y compartir contenidos."

% Sicherheit
"Los riesgos de sufrir ataques informáticos en una red libre no difieren particularmente de los riesgos que afrontan otros tipos de redes"
"El hecho de que la red permita con más facilidad a las personas ofrecer servicios y compartir contenidos, genera también una conciencia diferente sobre el uso de la red y de sus dispositivos. Creemos que esta toma de conciencia es positiva también en relación a la seguridad."

[FiTre2015]
% la infraestructura técnica
autonomous local networks + uplink to the internet (which is a potential bottleneck)
gibts in Altermundis redes local services? (vgl nicos iv)
ja --> "Contamos con un portal del pueblo donde todos pueden publicar información e inclusive clasificados. También tenemos un servicio interno de chat, independiente de redes externas, que al igual que el servicio de llamadas de Voz sobre IP (en experimentación) nos permite conectarnos entre vecinos de manera fluída. La Radio FM del pueblo aprovecha la red para hacer transmisión en vivo de su programación, tanto a la red local como a Internet."


%TODO
[Latour2010]
"the technology.. /digital technology .. makes networks material and explicit" (infrastruktur ist präsent und sichtbar im öffentlichen Raum) --> wem gehört die Infrastruktur? wer kontrolliert die? Wer hat (physischen) Zugang dazu?
  geografía: terreno difícil; quien controla la infraestructura?

"subversion it introduces in the notion of distance (the adjectives “close” and “far” are
made dependant on the presence of conduits, bridges, and hubs)," <-- physical network
\end{comment}

% Bild: artefactos físicos en las redes facilitadas por Altermundi
Reference the figure composed of multiple subfigures as Figure~\vref{fig:esempio}. Reference one of the subfigures as Figure~\vref{fig:ipsum}. % The \vref command specifies the location of the reference

\begin{figure}[tb]
\centering
\subfloat[Outdoor router en la red DeltaLibre.]{\includegraphics[width=.45\columnwidth]{deltalibre8}} \quad
\subfloat[Antenas en DeltaLibre.]{\includegraphics[width=.45\columnwidth]{deltalibre10}\label{fig:ipsum}} \\
\subfloat[Instalación de antenas en NonoLibre.]{\includegraphics[width=.45\columnwidth]{nonolibre7}} \quad
\subfloat[Autoproducción de antenas.]{\includegraphics[width=.45\columnwidth]{nonolibre8}} \\
\subfloat[Jesi en QuintanaLibre]{\includegraphics[width=.45\columnwidth]{2013_quintanalibre2}} \quad
\subfloat[Antena en las altes cumbres con autonomía energética solar]{\includegraphics[width=.45\columnwidth]{16-todos_trabajando}}
\caption[A number of pictures.]{AlterMundi: artefactos físicos. (Fotos: AlterMundi)} % The text in the square bracket is the caption for the list of figures while the text in the curly brackets is the figure caption
\label{fig:esempio}
\end{figure}

% eher zum kapitel gemeinschaftsnetzwerke; vlt den kapitel mit dem hier verschmelzen
Proyectos de redes comunitarias como Altermundi se ubican claramente en el contexto de la segunda onda del ciberoptimismo, que Rieder esboza~\autocite{Rieder2012}.
Sus valores claves (core values) son la decentralización, la distribución del control, la autogovernanza y la organización sin jerarquías~\autocite{FiTre2015}.
Estos proyectos de base (grassroots projects) intentan a revolucionar la comunicación desde abajo, a empoderar a lxs usuarixs a tomar su comunicación en sus propias manos.

\begin{comment}
* community network projects: are to be found in the context of (2nd wave) cyber optimism: "decentralization", "distributed control", "self-governance", "non-hierarchical organization"
an attempt at a capillary revolution -> ist es erfolgreich?
\end{comment}


\subsection{La red social de Altermundi}
% \subsection{La red como tejido social}

Como ya hemos señalado, Altermundi es una asociación social que ayuda a comunidades locales en la construcción de redes inalámbricas.
En cierto sentido, podemos entender a Altermundi y las redes juntos como un único actor en la ANT.
No obstante, si hacemos un zoom in, este se descompone en el grupo básico de activistas y redes autónomas locales.
Y un zoom más allás nos deja aun con personas con necesidades y opiniones individuales.

En esta constelación el papel de Altermundi es sobre todo educar y concienciar.
A través de talleres de formación el grupo intenta a empoderar a las comunidades locales de manejar sus propias necesidades comunicativas.
La comunidad toma las decisiones relevantes de forma autónoma y Altermundi tiene un papel(syn) consultativo.


Bastante interesante es también la recepción del proyecto por parte de las redes.
Algun de los problemas mencionados más frecuentemente es la falta de conocimientos técnicos en la comunidad local.
Entonces, observamos que aunque el proyecto intenta a facilitar la construcción de redes para personas non-técnicas, esto funciona solamente hasta un cierto grado.

\begin{comment}

[nico-iv-pillku]
%Problemas
"Más allá de los desafíos típicos relacionados a la dinámica social, las estrategias de transferencia de conocimiento y las problemáticas económicas típicas de las poblaciones pequeñas donde se despliegan estas redes comunitarias, la región sufre otros problemas estructurales en cuanto al acceso a la red global."
"En nuestro país, como en el resto de Latinoamérica y en general en los países que no son históricamente centrales en el desarrollo de Internet, la principal dificultad para cualquier intento de democratización del acceso a la vida digital es el “impuesto al más débil” que pagamos cuando queremos comprar ancho de banda dedicado, para conectar estas redes al resto de Internet. Para dar una idea de la dimensión de este problema basta un ejemplo: en el Punto Neutro (Internet Exchange) de Cataluña, se puede comprar 1Gbps, es decir “mil megas por segundo” en vocabulario coloquial, por un valor mensual de 780 us dolares. En la Argentina, aún con los esfuerzos colectivos realizados por los pequeños y medianos operadores nucleados en la Cámara Argentina de Internet, el costo por 1Mbps en un Punto Neutro local ronda los u$s 40. Es decir que 1Gbps costaría unos u$s 40.000 por mes: un 5.100 por ciento (51 veces) más caro que en Europa.
Para entender cómo impacta esto en la democratización del acceso a la red de redes, pensemos que con lo que cuesta comprar 1Gbps durante un mes en Argentina, se podría contratar durante más de 4 años el mismo servicio en Cataluña, que tampoco es la plaza más económica del mundo.
Hoy en día, que en la Argentina los abusos en las diferentes cadenas de valor se han puesto en la mira con una política activa, sería muy positivo que hubiera una evaluación seria sobre los costos reales que afrontan los proveedores mayoristas en esta actividad y sobre todo tomar dimensión de los valores que se manejan en el mundo para desentramar este abuso que sufrimos a manos de un núcleo muy concentrado de empresas multinacionales.
"

[FiTre2015]

% Social
"Rather than being driven by profits, they focus on the actual needs of the needs of its participants. They
also experiment with novel models of distributed governance"
"flat organizations and a peer-to-peer approach to decision-making, based on deliberation and consensus"
-- stimmt fuer die redes

%% polit. engagement der Projekte
eigentlich sehen sie sich nicht zwangslaeufig als politisch y operan más fuera de la lógica de la política convencional de instituciones o del activismo político disidente (protestas, etc.)
"One way is to address the issue from within the political system, as ‘insiders’, formally interacting with the power holders"
"Another solution is to fight the problem as ‘outsiders’, pressuring both regulators and
incumbents from outside the political system, by means of protests, demonstrations and other campaigning tactics"
"Yet, most of the community networks we surveyed do not properly qualify as what social movement scholars define as ‘insiders’ (although they
sometimes do interact with policy-makers), and much less as ‘outsiders’. Mostly, they fall within the third category—what Milan identifies as
‘beyonders’."
"remaining beyond the political system"
--> vgl auch "route around laws" from Rieders text (zitat eff mensch)

allerdings sind sie schon in essenz politisch:
"hese networks are ‘prefigurative realities’ that
challenge the status quo and ultimately contribute to a new political order (Milan, 2013, pp.126-38): these networks – built ‘for the people, by the
people’ – fundamentally embody a form of political action."

"distributed counter-power to traditional telecoms operators"
"adhere to specific ethical commitments and governance structures."
"From locally-grown food to locally-grown networks, WCNs form part of a wider movement
focused on empowering local communities to directly produce and manage the resources that matters the most to them."
-- paralelo soberania de comunicación - soberanía alimentaria (vgl Interview Isa)

und fazit: es lohnt sich politische Loesungen fuer polit. Probleme zu suchen
"these examples show that ‘insider’ strategies, i.e. direct engagement with policy-makers, are worth pursuing."
-- davor kommentar: es ist fuer solche Iniziativen oft nicht feasible sich mit Politik zu beschaeftigen, v.a. da sie dafuer keine Ressourcen haben;
es ist eine volunteer community, die netzwerke bauen moechte und nicht sich mit gesetzestexten beschaeftigen.
weiss nicht in wie fern das alles auf altermundi zutrifft, aber die sind im grunde auch alles techies;
und auch wenn deren motivation weniger faszination mit technik und mehr hippie-soberania comunicativa-derecho a comunicación ist, haben die vermutlich genau so wenig menschen, die von legal issues plan haben und nicht unbedingt riesenbock sich da reinzusteigern
\end{comment}

\begin{comment}
  Recepción desde las Redes

% licencia de todos los videos: Creative Commons Attribution license (reuse allowed) /Creative Commons Atribucion-Compartir Obras Derivadas Igual 2.5./CC BY-SA 2.5 AR

%  LaBolsaLibre https://www.youtube.com/watch?v=x6nONJXcUQ8
* somos grupo de vecinos/amigos, que ha surgido desde el despliegue de la red, después de la semana de hackathon de construcción de la red
* nos faltan conocimientos técnicos, somos dependientes de otras redes, si un nodo se quema o la red se cae

% QuintanaLibre https://www.youtube.com/watch?YMKlvUS7B-A&v=hsPjT2R-ToQ
* proyecto grande: diferentes grados de involucración
* mucho trabajo
* cómo atraer miembros nuevos?: reuniones con gente ya involucrada, que expliquen que es lo que les gusta en la red, como es su experiencia, "como cambió su vida por la parte de la red"

% AnisacateLibre https://www.youtube.com/watch?YMKlvUS7B-A&v=iasMfOAaDpk
* 10 familias conectadas
* 1 persona responsable, empezó sola, el único nodo en el barrio; se reunió con gente de Altermundi después, intercambiando experiencia, etc.
* desde 2012-2013, después de José de la Quintana
* comparten datos, tráfico, etc.. --> para el responsable el aspecto social es clave: estar conectado con más personas, que comparten
* red comunitaria != internet gratis, sino se hace entre todosa -> importante que lxs participantes lo entiendan

% LaSerranitaLibre https://www.youtube.com/watch?YMKlvUS7B-A&v=Kv2246EDPPg
* desde fines de 2013-2014
* 2016: 15 familias conectadas?
* tenían la necesidad, los proveedores no querían prestarle servicio y conocieron a QuintanaLibre
* ir de casa a casa, algo comunitario, sin un fin económico que estaba en vista
* la mayor dificultad entre vecinos: entender que es una red libre, sacarte un poco de la lógica del mercado, algo que se hace en la comunidad
* desafío: unir La Serranita con Córdoba, para tener conección; el tereno es duro, habían de subir mucho más alto;

% NonoLibre https://www.youtube.com/watch?YMKlvUS7B-A&v=RaXZqlALRII
* 25 nodos operativos: 18 familias y resto cabañeros?
* comenzó sep? 2014
* Nono: localidad túristica; por lo tanto la Camara de Comercio fue un de los impulsores del proyecto, cuando se enteró de ello
* conectividad importante para el desarollo personal pero también para el desarollo laboral del pueblo
* proyecto previo: Nono digital (tener wifi libre en Nono), se juntó con Altermundi -> NonoLibre;
* tenían 40 nodos, pero solamente 25 operativos, ya que el terreno es difícil
* desafío más grande: montar una antena en la Pampa de Achala (Altas Cumbres) --> conectar las redes del un lado de la monte con las del otro
* experiencia muy linda, gratificante: trabajar en comunidad
---
Reunión en LaQuintana 19 Marzo 2017
"Nono Libre: Tiene la dificultad que en momentos de mucha afluencia turística se satura la salida a internet. "

\end{comment}
\begin{comment}

[Freeman1970]
% wichtig fuer die analyse, weiss noch nicht genau wo
Klammer: In wie fern ist das bei mir eigentlich anwendbar? Mein Usecase hat doch eine gewisse Struktur: Kerngruppe + Lokalgruppen, in den Lokalgruppen gibts ne VV, die Entscheidungen trifft.
Vlt kann man argumentieren, dass konkrete Zuständigkeiten fehlen? Dafür kenne ich die Organisation aber viel zu schlecht. Sollte man nochmal mit denen reden?

"Elites are nothing more, and nothing less, than groups of friends
who also happen to participate in the same political activities."
Vmtl kann man ziemlich zutreffend argumentieren, dass die Kerngruppe ne Art Elite ist.
vgl auch "Superhero-Status" (im Gespräch erwähnt)

[Castells2015]
"There are usually a handful
of persons, sometimes just one, at the start of a movement."
--> vgl auch Super hero..


gatekeepers to infrastructure/indispensability (vgl auch 4 Charakteristiken einer erfolgreichen Kleingruppe): ist die Kerngruppe Infrastructure Gatekeeper? Man kann bestimmt von Knowledge als Ressource ausgehen und eine Wissenshierarchie feststellen. (Auch wenn große Bestrebungen gibt, der entgegen zu wirken)

7 Principles of democratic structuring:
1. delegation: Ich glaube das gibts; So was wie Kassenwart? Und eine lokale Ansprech-/zuständige Person für die Maintainence? (TODO: vgl mit Interview transcript)
2. responsibility: Es ist glaub ich schwierig, Personen zu Rechenschaft zu ziehen und Konsequenzen durchzuziehen, da ganz oft nicht gut möglich ist die Person an einer Position auszutauschen, weil sonst niemand den Posten haben will.
3. distribution: kp ob das zutrifft
4. rotation: dito
5. allocation of tascs along rational criteria (skills, interest, responsibility,..); apprenticeship program -> ich denk, das trifft auf jeden Fall zu (zumindest den Teil mit dem "apprenticeship program", es finden mehrere talleres,.. statt; Bildungsauftrag, das wird ernst genommen)
6. diffusion of information to everyone as frequently as possible - kp.
7. equal access to resources: das trifft vermutlich nicht zu;

\end{comment}

\subsection{Zusammenfuegen}

Al final, la separación aguda entre capa técnica y capa social es difícil y no necesariamente útil.
% mach das sinn?
% vlt ciberoptimismo hier?

\begin{comment}

[FiTre2015]
% Zum Zusammenfuegen chapter?
"we show how the current revival of grassroots community networks can counterbalance the erosion of autonomy of Internet users that results from current
telecom policies."

---
[Rieder2012]

> framing the Internet alternately as lawless, anarchic,
> free, “a world where anyone, anywhere may express his or her beliefs, no matter how singular,
> without fear of being coerced into silence or conformity” (Barlow 1996) (p.1)

* la infraestructura está prerequisito para participación

## Conclusión del texto [Rieder2012]

> If technology won’t deliver us from the conundrums of
> governance, negotiation, and struggle, we
> may as well reengage politics proper[ly].

* sobre todo problemas sociales/políticos: organización de grupos (no tanto técnicos): ¿cómo decidimos como grupo? ¿quién hace qué? ¿quién está responsable?
* soluciones políticas para problemas políticos
\end{comment}

\subsection{Mapeo: área conflictiva}

\begin{comment}
[Castells2015]
Programmers and Switchers
"Thus, who holds power in the network society? The *pro-
grammers* with the capacity to program each one of the main
networks on which people’s lives depend (government, par-
liament, the military and security establishment, finance,
media, science and technology institutions, etc.). And the
*switchers* who operate the connections between different
networks (media moguls introduced in the political class,
financial elites bankrolling political elites, political elites bail-
ing out financial institutions, media corporations intertwined
with financial corporations, academic institutions financed
by big business, etc.)."

ANT: podemos hacer aquí zooms in y out!

                          * da talleres
                          * apoyo técnico
                          * Bildungsauftrag

            Altermundi <-----------------------> redes
         /   ^                                     ^
individuos   |                                     |
             |                                     |
             |                                     |
             |                                     |
             v                                     v
            estado<-------------------------> infraestructura

Altermundi:
* individuos, gente que se comunica entre si, con todos los problemas y desafíos que surgen de esto
* tiene una forma organizativa legal: asociación civil (siehe IV) para poder entrar en alianzas, cooperaciones (con el estado y otras organizaciones formales, p.e. la Universidad de Córdoba)
* función auxiliar (las redes son autónomas)

Estado:
* está responsable para regulaciones que ayudan o impiden el trabajo de tales proyectos (ley telemediatica)
* la última milla
* solucionar el problema de conexión de manera completa/de gran escala (siehe papel del estado)
* apoyo financial (eher nicht, vgl Entstehungsgeschichte)
[FiTre2015]
% Mapamiento/papel del estado
--Law is more often than not, a hurdle (siehe auch firmware lockdown)
--Allerdings gabs in ARG ein progressives Telemediengesetz 2009, was Macri grad versucht zu kassieren.
"law should respond by ‘implementing policies that predictably diversify the set of options that all individuals are able
to see as open to them’ (Benkler 2006, p. 152)."
-- gibt eine Liste von was der Staat alles machen koennte, um die Arbeit der Community Networks zu erleichtern; man koennte vergleichen was davon alles auf ARG zutrifft;
auf jeden Fall, varying Erfolg, es gibt Versprechen, die nicht eingehalten werden; relativ progressive Gesetzgebung, dann wiederrum Versuche die rueckgaengig zu machen (hier sieht man wie auch der Actor "staat" in einzelne Regierungen bzw Personen zerfaellt, die alle verschiedene Ziele verfolgen)


Infraestructura:
* ¿donde está?
* quien la mantiene/gestiona/instala
* quien tiene acceso?

Redes
* autónomas
* tensiones: problemas con la comunicación dentro de la comunidad
* falta de conocimientos técnicos

Proveedores de servicios tradicionales
* intereses económicos comerciales
* interesados en el uso exclusivo de recursos del espectro
* no interesados en proveer acceso a internet en regiones pobres y con poca población
% Mapamiento -- commercial ISPs como actor --contrapoder!! [FiTre2015]
"a policy overhaul focused on community networks can indeed help create effective counter-powers to the
dominance of commercial operators in the communications infrastructure."
[Castells2015]
Contrapoder def:
"deliberate attempt to change
power relationships, is enacted by reprogramming networks
around alternative interests and values, and/or disrupting the
dominant switches while switching networks of resistance
and social change."


Foreign Regulatory Bodies: EC und FCC
* intento a regular el espectro
* sobre todo mesuras contra la interferencia con los rádares (weather radar) <-- no está relevante para regiones rurales
* verabschieden vage Richtlinien, interpretadas por los productores de hardware como aufruf, prohibir software libre
* igualmente que se trata de instituciones del EU/EEUU, esto va a influenciar todo el mundo, ya que sonst cada productor tiene que producir productos distintos para los diferentes mercados: con y sin implementar las regulaciones

\end{comment}

\begin{comment}
[Rieder2012]

%Analyse
"the Internet allows capillary configurations
of power – local initiatives, ad-hoc pressure groups, fan cultures, “issue publics” – to challenge
the statutory powers that be. "
Allerdings:
Wird argumentiert, dass alle mitmachen koennen, das stimmt aber nicth so direkt.
Koloniale Zusammenhaenge bestehen;
Geographische Schwierigkeiten;
Oekonom. Probleme (oft in Kombi mit schwierigem Terrain); --> Kommerzielle Provider haben kein Interesse
Staatliche Repression --> dem Staat passt nicht dass X oder Y kommunizieren kann und eine oeffentliche Plattform hat

% Analyse: parte social, papel del estado
"the convergence of this mutating counterculture with laissez-faire capitalism"
--> vlt kritische Perspektive: community networks as counterculture, aber andererseits spielen sie dem Staat teilweise in die Haende (laissez-faire capitalism, vgl Interview Gui)

% Analyse: redes, parte social
"Democracy rests on the idea that, except for technical details for which experts
may be useful, the important decisions of society are within the capability of ordinary citizens.
Not only can ordinary people make decisions about these issues, but they ought to, (Zinn 2003)"
--> vgl Altermundi: Experts: tech. Details; las redes: toman decisiones de manera autónoma

% Análisis
> If technology won’t deliver us from the conundrums of
> governance, negotiation, and struggle, we
> may as well reengage politics proper[ly].
--> Altermundi es explicitamente un proyecto político?
Ich glaub, it always boils down to thinking critically about what one's doing and assuming responsibility for it.

% analyse: estructura social
"This mediation has to rely on at least some communality – Rawls
speaks of “reasonable” plurality – but the challenge is indeed to manage difference." (p.16)
// Also Altermundis Probleme sind nicht technischer sondern sozialer Natur --> passt zur Schlussfolgerung des Texts

% analyse: estructura social --> superhero
"In many of the “self-organized” sys-
tems that make up Web 2.0, we find that a small group dominates structures of visibility." (p.15)

\end{comment}

\subsection{Ciberoptimismo}
\begin{comment}
% Analyse
se ubican en el contexto del ciberoptimismo (DEF)

reflexión crítica: está la solución fiable en gran escala: vgl papel del estado (IV Gui); última milla

  community network projects: are to be found in the context of (2nd wave) cyber optimism: "decentralization", "distributed control", "self-governance", "non-hierarchical organization"
  "the second wave of Inter-
  net enthusiasm was able to transpose key terms such as “decentralization”, “distributed control”,
  “self-governance”, or “non-hierarchical organization”, from the language of countercultural
  community-building into the realm of entrepreneurial cyber-capitalism where scarcity doesn’t
  exist, without them losing their anti-establishment ring and affective value."

## técnica y política

> framing the Internet alternately as lawless, anarchic,
> free, “a world where anyone, anywhere may express his or her beliefs, no matter how singular,
> without fear of being coerced into silence or conformity” (Barlow 1996) (p.1)

"framing the Internet alternately as lawless, anarchic,
free, “a world where anyone, anywhere may express his or her beliefs, no matter how singular,
without fear of being coerced into silence or conformity” (Barlow 1996) or, more recently, as a
space of surveillance, commercial manipulation, and sweeping monopoly."
* la infraestructura está prerequisito para participación
* Community networks operieren in diesem Spannungsfeld

"My first line of critique will therefore try to show that digital networks may very well
produce effects of centralization as much as decentralization, and give rise to new mechanisms of
power that imply new vectors of domination and abuse." (anderer Pol: google & co)
decentralización, pluralidad, democracia --> diese ganzen ciberoptimismo Zeugs in Basics:
Community Networks: die werden idealerweise angestrebt; aber gelingt das?

%ANT:
[Rieder2012]
"Software now habitually provides specific answers to questions that do not seem technical at all: What is communication? What is cooperation? Which information is valuable? What is decision-making?"
"The idea that technology implements values and should be seen as an “actor” in the shaping of
social processes is of course not new; from actor-network theory"
"In many of the “self-organized” sys-
tems that make up Web 2.0, we find that a small group dominates structures of visibility."
Zoom out: big internet providers + social media corporations (Google, Facebook, Cisco, AT&T)
Zoom in to community networks: you'd see above all Altermundi, but not the redes

"The hope for magical technological
solutions to the messy realities is counterproductive if it leads to an attitude that disengages tradi-
tional political process to simply “route around it” ." --> LibreRouter routs around the firmware lockdown quite litereally

[Castells2015]

"Mass self-­
communication is based on horizontal networks of interactive
communication that, by and large, are difficult to control by
governments or corporations."
// die sind nicht horizontal und grad solche netzwerke werden sehr wohl von coroprations und governments kontrolliert. Projekte wie Altermundi wirken dem entgegen.

"communication networks are decisive sources of power-making." // auf jeden Fall! Siehe auch Gene Sharp

\end{comment}

\begin{comment}
  https://globalvoices.org/2017/01/02/librerouter-why-buy-a-router-when-you-can-build-your-own/
  Librerouter
  GV: Why does a community network such as AlterMundi need a LibreRouter?
  GI: Las redes que fomentamos desde AlterMundi están construidas y mantenidas por gente relativamente no técnica. Con lo cual, desde el principio nos concentramos en que tanto la puesta en marcha como el mantenimiento de los nodos sean lo más simple posible. Sin embargo, con el escalamiento de las redes fuimos encontrando complejidades (como por ejemplo la necesidad de montar dos o más routers en ciertas ubicaciones) que complican el entendimiento por parte de la población en general, y por eso veníamos dándole vueltas a la idea desde 2013. El punto de inflexión ocurrió con las mencionadas restricciones de fábrica, que directamente hacen inviable la posibilidad de que gente no técnica transforme un router hogareño (económicamente accesible) en un nodo comunitario, poniendo en peligro la continuidad de las redes en todo el mundo.

%Poder
[Sharp2010]

"The dictators’ access to material resources also directly affects
their power. With control of financial resources, the economic
system, property, natural resources, transportation, and means of
communication in the hands of actual or potential opponents of
the regime, another major source of their power is vulnerable or re-
moved. Strikes, boycotts, and increasing autonomy in the economy,
communications, and transportation will weaken the regime." (s.68)
As previously discussed, the dictators’ ability to threaten

%IV Gui
[Vaseva2016]

Def redes comunitarias (gibts auch wo anders):
"infraestructura ... internet al principio, que la monta y la sostiene y la gestiona la propia gente que la usa."
"normalmente no es? lo que pasa: viene la empreza que pone algo que es de la empreza en tu casa y..ellos tienen toda la decisión sobre eso], solamente le puedes contratar o salir del servicio y nada más.[...]"

Entstehung/Aktivitäten:
"primero en el pueblito de Córdoba, cerca de Buenos Aires y luego se fueron formando otros pueblos vecinos en Córdoba, pués nada, como fuimos tejiendo redes sociales y hicimos un taller en el pasado en Santiago de Fe? en Argentina y, como en otra provincia.. y después en otros países: como en Brasil (estabamos? en Rio de Janeiro) y en Nicarágua hace poco y -"
"darles capacitación. No de instalar nada de los equipos, sino de ayudarle para que aprendan a manejar.. un poco esta tecnología.." --> Bildungsauftrag

Vinculos con el estado:
"un proyecto armado con una  una persona contratada/conectada? por el estado,del gobierno que quería hacer despliege loco? de talleres por todo el país y no sé qué, como talleres de software libre, de comunicación y entonces Nico que en este momento se juntaba con esta persona? propuso? también hacer redes comunitarias, redes libres y "
"esta persona le cambiaron, bueno dejó de estar en este lugar y entonces se cortó todo este proyecto, "
"como tuvimos la perspectiva que nos daba la confianca de tener el apoyo del gobierno. Finalmente no lo tuvimos, que lo hicimos igual, pero tuvimos confianca en un principio de tener eso."

" La tarea de asegurar internet para la gente.. no te parece que es una tasca del gobierno? y no lo vez problemático, en el sentido: si el gobierno ve que hay activistas que ya se ocupan de esto se dice: "vale, alguíen ya se encarga, entonces nosotros no tenemos que hacer nada". "
"no lo veo así.
Pero por la experiencia que tuvimos"
"on cualquier organización del gobierno [con la que hablabamos?] era como: "che, que bueno que esten haciendo, en lo que podemos ayudar, les queremos ayudar, que puede ser poco o mucho..." algunas casos simplemente.. como bueno ..a veces puede ser cosas que no servían,  "
"fue divertido el intercambio.. pero no de todo útil"
"La Entidad Reguladora de comunicaciondes, de frenquencias etc...y entonces, y.. un plan de fibra, ..  fibra por Argentina y estabamos charlando siempre .. usando ese plan de tomar las redes comunitarias como modelo para .. la última milla de esa fibre y la resonancia? era positiva"
"el gobierno actual sí hubo, sacar esa ley, o la cancelaron y toda la gente que conocíamos que estaba trabajando que ... que desponía de recursos, de plata, de herramientas, de tiempo, de.. conecciones sociales o lo que sea, ya no está.."
"como que si el estado quiera hacer algo, quiera resolver un problema
L: lo haría"
"o sea normalmente tiene mucho más recursos que cualquier iniciativa privada. Entonces, si no está resolviendo el problema es porqué o no quiere.. o no puede, no puede en terminos de, no puede resolverlo en una forma apropiada por el estado."
"o sea "lo puede hacer vos como individuo o como grupo de vecinos, pero yo no puedo, como municipio poner internet en un barrio, nada más. porqué es como yo ya sé que van a saltar los del barrio del lado y decir "hey, por què pusiste, el estado puso internet en este barrio y no acá. Entonces como estado estoy un poco con manos atadas hasta.. o lo resuelvo todo con plan de trabajo.. o no hago nada." Y es como "vos como individuo tienes la libertad de hacerlo en tu barrio y que nadie te diga nada, porqué no lo hace de resolver al barrio del lado, es tu barrio. Que yo como estado no puedo elegir ningún barrio.." Puede ser un poco mentira, lo que me estaba diciendo, puede ser un poco verdad, a mi me resultaba rasonable la explicación y..bueno.. También, o sea, también hay un poco de falta de voluntad,"
" no solo que no quieren o no pueden, sino que no se enteran. "

Entstehung weiter:

"ya habíamos montón de gente de barrio que se había comprado un router y estabamos haciendo experimentación y todo.. y decidimos de segir adelante, sin financimiento"

"y enseñarnos a hacer redes que podemos a enseñar a hacer a otra gente que no es técnica, como eso era la meta: simplificar toda la capa de software lo máximo posible y todo el despliege fuera accesible para una persona no técnica y luego con el tiempo fuimos dando talleres como los quizás hubieramos dado en el proyecto original pero de forma autónoma, "

"o sea, ya a cabo de 1 año teníamos un firmware, una solución, que podíamos enseñarle en un par de días a otras personas.. no todas las capas, pero si en nivel de, puede instalar un nodo o replazar un nodo o sí.."

"Pero, la necesidad original era la misma. Nico y Jessi estaban viviendo en su pueblito sin internet,"
"había comprado un router para mi casa también y había replicado la experiencia en mi barrio y allí tenía la misma repercusión."

"esde codigosur, empezó altermundi.."



Problemas sociales, no tecnicos
"después la capa complicada y que nos requirió mucho más aprendizaje fue la capa social o es todavía la capa social, o sea, es decir, los routers son super fáciles de flashar e instalar y mantener, es como bastante trivial, se puede aprender por una persona en 1 día, o en un taller de un día, pero el desafío, tanto para la gente de los pueblos, como para nosotros también, es gestionar la comunidad."

Estructura: Altermundi vs Redes
"Aproximadamente 10 personas, sí, que siempre como.. activas.. pero hace 1 año logramos finalmente terminar de tener una forma legal que fue difícil"
-> auch estrctura/forma legal para poder formar alianzas
"para gestionar recursos, ante..
L: Donaciones por ejemplo?
G: Sí, tambíen, y alianzas, o sea, firmamos un convenio con una universidad, con una universidad nacional.."
En las redes:
"sí 100 personas conectando se con la red, pero como usuarias, no como que venian a ninguna reunión y siempre como ni se enteraban que era una red libre porqué era como una red abierta"
"por el tema incluso de gestión de plata, para decir cuanta plata estamos marcando/bancando ahora? y allí en la lista era de unas 50 personas
Pero, sí, como para repartir hacia? barrios y no sé qué y las reuniones organisativas nunca fueron de más de 10 personas."
"13) Y estas 10 personas son las personas quienes al final ponen los routers no? y los gestionan?
22:15
G: no.. más o menos.. no necesariamente"
"En otros pueblos no, o sea, hubo talleres que hicimos y hubo más o menos 15 personas en el taller y.. unos 15 nodos y se lo [instalara?] cada persona
[y hicimos todo juntos]
como hicieron una compra colectiva y después ..en tu casa
..y alguna persona no podía que era muy viejita y le ayudó otra pero bueno.."
" en otro pueblo en Cordoba, que se llama Nono, también pasó que en los primeros meses era como todo el mundo estaba basicamente haciendo todo, pero después cuando ya tenía su nodo y tenía su internet y todo y se.. en su casa y se limitaron en quedarse en su casa? y poner la plata cada mes o lo que sea y ir a alguna reunión pero que hubo un pibe más técnico haciendo se carga de todos los problemas, "

Concepto:
"que la gente del pueblo que tenga interes, que no tenga Internet al final, se reuna y entonces o junten plata (como en ese caso último que comentaste) o colectan plata después para comprar la técnica, después la instalan o vosotros la instalaís..
ca.24:14

G: no, no
L: ellos la instalan"
"L: y entonces después colectan plata cada més para pagar los gastos del Internet,no
G: sí, basicamente sí"
"16) y después hay una o en el caso ideal más que una persona que se encargue del mantenimiento técnico?
24:55

G: sí, tal y cual, sí, depende de la escala."

Motivacion personal:
"había por primera vez, creo, había trabajado con grupos de gente ..y como me entusiasmó un montón esta idea de.. de dar talleres.. como que me pareció genial.."
"e alejé de toda la informática y volví con esto pero no por las computadoras sino por-
L: la parte social"
"aprendiendo durante este proceso, como ayudar mejor o como manejar un poco grupos y poder resolver situaciones que no tenía ninguna experiencia de eso como freaky"
" no puedo negar que me divierta? resolver problemas técnicos "
"lo que seguro me mantuvo todos estos años es esta perspectiva y la experiencia también de haber dado talleres, etc."

Problemas:
"Tenemos esta separación un poco, la idea es que somos esta asociación que fomentamos el despliege de redes y ayudamos a que se despliguen redes, que no son.. o sea, que son redes independientes.Como entonces la red que empezamos en el Delta se llama delta libre y tiene su autonomía. Y nosotros le ofrecemos ayuda, capacitación de software, etc., como le damos soporte y apoyo y.. bueno, pero son como organizaciones/redes sociales indepentientes que.. y en las redes, bueno, siempre hay lios sociales porqué son mucha gente. "
"En altermundi no tanto, porqué somos amigos desde hace muchísimos años o que sea (también poca gente?) y bueno allí? tenemos más problemas organizativos, de cordinarnos bien internamente, de comunicación y todo.
yo estoy bastante malo de comunicarme a distancia,"
"el mayor problema para mi es en las redes, son muchas veces gente que no tiene experiencia de participar en ningún grupo, ni asamblea, ni nada, ni ningún proyecto colaborativo
la parte técnica es bastante trivial, y la parte social - no"
"tuvo que invertir muchísimo tiempo en el delta para gestionar todos los problemitas sociales, no sé qué..
basicamente por ocupar este lugar de haber iniciado la red  ----------> Superhero!!
entonces la mayor parte de la gente me conocía y me respetaba"

Finaciamiento
"pero nos propusimos desde un par de años, hacer un escquema que cada red aportaba a la asociación una plata como decir de 1 euro al mes por nodo a cambio del suporte "
" no, no lo logramos. "
"  nunca llegó una escala para poderamos sostenernos económicamente "
"recibimos varías subvenciones o sea un premio y un par de subvenciones "
"de la OTI, de LACNIC? y de Shuttleworth,"
"y bueno, eso nos permitió comprar equipos, también pagarnos un sueldo para algunos meses para desarrollo de software que haya obtenido? .."


Comparacion Rhizomatica
"la misma lógica, y encima con un número muy parecido, como que era ca 1 euro-1.20 euro por persona que usaba el celular, o sea las cuentas? eran muy parecidas
L: y para ellos funcionaba..
G: Sí, porqué estaban mucho más, era más fácil de.. como todo era un poco más estricto."
"La comunidades, sí, o sea cada pueblo, había comprado equipo carísimo, había invertido x mil dolares,"
"a estructura comunal 40 pesos (como 3 euros) por mes para tener servicio de celular, que era un número que habían decidido en asamblea y de esos 40 pesos, se pagaban 15 para llegar a Rhizomatica. Entonces y Rhizomatica también le daba suporte."
"puede ser que también era, no sé, que para ellos/ellas era parte del concepto desde el principio..
G: Sí
L: y como que vosotras habeís intentado introducirlo después y .. no?
G: Sí, puede ser
L: y puede ser que ..
G: sí, eso seguro pasó en las 2 redes originales en Delta Libre y Quintana Libre era "imposible",
como grave, tenía que pasar por asambleas o algo así, mientras que el Nono, el pueblito este, el único pueblito que por toda la .. meses de quota? fue el día de taller, el día 0, .. "bueno, proponemos esto" y toda la gente dijo "buenísimo" 
L: claro, porqué no había otra propuesta, no.. 

-----------------------
https://pillku.org/article/altermundi-y-las-redes-comunitarias-digitales/


Otra experiencia que resulta digna de mención es la interacción que se ha conseguido entre las redes del Valle de Paravachasca con la Universidad Nacional de Córdoba, a través del Laboratorio de Redes y Comunicación y de la Prosecretaría Informática. Desde este año, gracias a un enlace propio de 50Km, las redes comunitarias han establecido interconexión con la red de la Universidad. La UNC, aparte de comenzar a fomentar la investigación y el estudio académico de este tipo de redes, también se ha ido estableciendo como un importante centro de convergencia de las redes regionales, alojando tanto al NAP de CABASE en Córdoba como al futuro nodo de Arsat.
Para nuestras redes comunitarias, lograr este tipo de interconexiones representa un grado de madurez que no se había conseguido previamente en el país y me animaría a decir en el continente.


José de la Quintana es un pueblito serrano, con muy baja densidad de población, que tampoco es destino turístico y por lo tanto para las empresas tradicionales es muy poco atractivo como mercado. Los celulares sólo funcionan en las lomas altas y hay muy pocas líneas de telefonía fija, que están más tiempo fuera de servicio que funcionando. Los proveedores de Internet (inalámbrica) existentes antes de la llegada de QuintanaLibre ofrecían conexiones de 512Kbps (“medio mega”) a quienes tenían la suerte de tener visibilidad con sus torres, por valores inaccesibles para mucha gente.

Hoy en día contamos, gracias a la UNC y a la empresa Silica Networks, con un enlace propio dedicado de 20Mbps (veinte “megas”) simétricos, que se distribuye entre más de cuarenta familias, a las que se suma también la escuela secundaria, la radio comunitaria y los espacios públicos del pueblo, donde es muy común ver chicos usando la red con sus netbooks de Conectar Igualdad. La situación de conectividad hoy es comparable a la de ciudades medianas, por unos costos ínfimos que socializamos para mantener en funcionamiento la red.
"
----------------
"This virtuous relation between hardware vendors and the community has been threatened by new regulation from the Federal Communitations Commision (FCC) – U.S.A. – which has led vendors to globally close up their routers to third party modifications, hindering open innovation and effectively closing the door to Community Networks in terms of access to the hardware they depend on. \url{https://librerouter.org/what-and-why}"

\end{comment}
