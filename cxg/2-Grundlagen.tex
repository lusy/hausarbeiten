\section{Grundlagen}

\subsection{Kognitive Linguistik}
Was ist Kognitive Linguistik?

* Der Kognitionswissenschaften unterzuordnen: es geht um Infromationswahrnehmung und -verarbeitung.
* kein abgegrenztes Sprachmodul im Gehirn. (Ziem08?)

In \cite{Ellis08}: ``The meaning of the words of a given language, and how they can be used in combination, depends on the perception and categorization of the real world around us.''

``Figure/ground segregation and perspective taking, processes of vision and attention, are mirrored in language and have systematic relations with syntactic structure.''

``In laguage production, what we express reflects which parts of an event attract our attention;
depending on how we direct our attention, we can select and highlight different aspects of the frame, thus arriving at different linguistic expressions.''

``.. language cognition cannot be separated from semantics and the rest of cognition.''

``.. denies clear boundaries between the traditional linguistic separations of syntax, lexicon, phonology, and pragmatics.''
\begin{itemize}
    \item pattern matching
    \item usage based perspective

        In \cite{Ellis08} usage based theories ``hold that structural regularities of language emerge from learners' lifetime analysis of the distributional characteristics of the language input and, thus, that the knowledge of a speaker/hearer cannot be understood as an innate grammar, but rather as a statistical ensemble of language experiences tat changes slightly every time a new utterance is produced.''

        ``.. the acquisition of grammar is the piecemeal learning of many thousands of constructions and the frequency-biased abstraction of regularities within them.''

        ``frequency of patterns in the input affects acquisition''

        Look at Bybee text from seminar.

\end{itemize}


\subsection{Konstruktionsgrammatik}
Alle Konstruktionsgrammatiken gehören zum Bereich der Kognitiven Linguistik (vgl Seminarhandouts)

\begin{itemize}
\item Alle sprachliche Phänomene lassen sich innerhalb derselben Theorie erklären (es gibt keine Kernsprache und Randerscheinungen!)
\item Konstruktion - Form/Bedeutungs-Paar (Goldbergs Definition?)

    In \cite{Ellis08}: ``Any construction with unique, idiosyncratic formal or functional properties must be represented independently in order to capture speakers' knowledge of their language.''

    ``Frequency of occurance may lead to independent representation of even ``regular'' constructional patterns.''
\end{itemize}

\subsection{Spracherwerb}
\begin{itemize}
    \item L1: Tomasello - Abstraktion aus konkretem Input -> Chunks -> item-gestützte Konstruktionen -> abstrakte Kategorien

        In \cite{Ellis08} ``children are picking up frequent patterns from what they hear around them, and only slowly making more abstract generalizations as the database of related utterances grows.''
    \item Unterschiede/Gemeinsamkeiten L1/L2

        In \cite{Ellis08}: ``in child language acquisition knowledge of the world and knowledge of language are developing simultaneously whereas adult SLA builds upon pre-existing conceptual knowledge;''

        ``adult learners have sophisticated formal operational means of thinking and can trea language as an object of explicit learning''

        ``typical L1 pattern of acquisition results from naturalistic exposure .... whereas classroom environments for second or foreign language teaching can distort patterns of exposure''

        ``adult SLA builds on pre-existing L1 knowledge... adults have already acquired knowledge of these cathegories and their lexical membership for L1, and this knowledge may guide creative combinations in their L2 interlanguage to variously good and bad effects.''

        ``naturalistic SLA develops in broadly the same fashion as does L1 - from formulae, through low-scope patterns, to constructions''
    \item Gebrauchbasierte/Konstruktionsgrammatische Perspektive auf dem Zweitspracherwerb
\end{itemize}

\subsection{Die Zweitspracherwerbsforschung in historischer Perspektive}
\cite{Weinreich79} ``... discussed how two language systems relate to each other in the mind of the same individual.
The key concept was interference, defined as ``those instances of deviation from the norms of either language which occur in the speech of bilinguals as a result of ther familiarity with more than one language'' \cite{Weinreich79} '' \cite{Cook93}

can happen on all levels of a language:
\begin{itemize}
    \item phonology: ``Speakers may carry over the L1 phonological system by ignoring distinctions made in the L2 but not in the L1;'' \cite{Cook93}
        ``some French learners fail to distinguish between the two English phonemes /i:/ and /i/ as in ``keen'' /ki:n/ and ``kin'' /kin/ because they are not distinct in their L1.''

        \cite{Weinreich79} `` For example, neither French nor Russian have /\textipa{D}, $\theta$/ phonemes, but in contact
        with English, French speakers tend to render /\textipa{D}, $\theta$/ as [z, s], while Russian
        speakers generally pronounce [d, t].
        In other words, the French perceive the continuance of /\textipa{D}, $\theta$/ which distinguishes them from /d, t/ as most relevant, while
        the Russians consider the mellowness of /\textipa{D}, $\theta$/ which distinguishes them from /dz, ts/ as decisive.''


    \item word order: ``German learners of English produce ``Yesterday came he'' modelled on the equivalent German sentence ``Gestern kam er''.''

        \cite{Weinreich79} ``(1) The replica of the relation of another language explicitly conveys an unintended meaning.
        Example: A German speaker says in English `this woman loves the man' on the model of German
        `diese Frau liebt der Mann', intending to communicate the message `the man
        loves this woman', but producing the opposite effect.
        (2) The replica of the relation of another language violates an existing relation pattern, producing
        nonsense or a statement which is understandable by implication.
        Example: A German speaker says in English `yesterday came he' on the model of German
        `gestern kam er', meaning `he came yesterday'.''


\end{itemize}

In \cite{Lado71} The ``fundamental assumption'' is transfer;``individuals tend to transfer the forms and meanings, and the distribution of forms and meanings of their native language and culture to the foreign language and culture'' (in der 1. Auflage, s.2)

According to \cite{Cook93}: ``In this view L2 learning consists largely of the projection of the system of the L1 on to the L2.
This will be successful when the two languages are the same -- called ``positive'' transfer by some;
it will be unsuccessful whenever the L2 fails to correspond to the L1 -- ``negative'' transfer.''

``Spanish learners add an ``e'' before English consonant clusters starting with /s/ so that ``school'' /sku:l/ becomes /esku:l/ in order to conform to the syllable structure of Spanish;''

grammatical structure as ``system of habits''(s. 57) - speakers can produce speech automatically and without thinking -> die Konstruktionen haben einen Eintrag im mentalen Lexikon (check Einführung \cite{Ziem13} Kapitel 8 ``Lexikon-Grammatik Kontinuum'')

----------------

Effects of L1 on L2-Learners in \cite{Braidi99}


Adoption of a L1 rule in the IL grammar:
``I go not to school'' German-speaking L2 learners of English ``reflecting the post-verbal negation structure
of the German L1 (``Ich gehe nicht in die Schule'')''

``... English-speaking L2 learners of Spanish may overgeneralize the usage of present progressive to indicate a
future event, an inappropriate context for this structure in Spanish (*``Estoy saliendo mañana''/``I am leaving tomorrow''
for ``Voy a salir mañana.'')


Lado according to Braidi:
Similar structures are easily transfered from L1 in L2;
``structures could be similar in 3 ways: they could be signalled by the same formal device, have the same meaning,
or have a similar distribution in the language system'' (Lado 1957: 66)

``The student tends to transfer the sentence forms, modification devices, the number, gender and case patterns of
his native language.'' (Lado 1957: 58)

``Transfer is the influence resulting from similarities and differences between the target language and any other
language that has been previously (and perhaps imperfectly) acquired'' (Odlin 1989: 27 Language Transfer: Cross-linguistic Influence in Language Learning
JFK F 107.4/ O 24)

* use of L1 typological organization and overproduction of structures (Schachter and Rutherford 1979 Discourse function and language transfer; Working Papers on Bilingualism 19 3-12)


Check Grundlagen in \cite{Ziem13}
