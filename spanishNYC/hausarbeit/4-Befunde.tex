\section{Befunde}

% vlt anderer Titel, wenn man genau weiß, was die These ist
%\subsection{Hauptthese}
%Bzw vlt neue Struktur zum gesamten Kapitel überlegen

Die automatisch erkannten Gebrauche von Englischen Floskeln wurden per Hand angeguckt und in Kategorien unterteilt.
Insgesamt wurden in den 1419 Artikeln 3271 unique? Wörter identifiziert.
Das heißt wenn das selbe Wort in einem Artikel öfter vorkam, wurde es hierfür nur einmal gezählt.

% vlt kommt die Beschreibung der Ergebnisse auch hier rein

Die vorgefundenen Code-Switches lassen die Vermutung aufstellen, dass es sich bei den Artikeln um eine Art Performance handelt, die auf ein bestimmtes Zielpublikum (welches?) zugeschnitten ist.
%Welche Wirkung versuchen diese zu erzielen
Da es dabei um Kommunikation in geschriebener Sprache geht, sind die Code-Switches ganz bewusst ausgewählt und benutzt worden und sind nicht das Produkt einer spontanen ``mir-fällt-grad-das-Wort-nicht-ein'' Aktion.

% Evtl hier Medientheorie und wie Zeitschriften die Gesellschaft (mit)gestalten heranziehen

%------------------------------------------
\subsection{Ergebnisse beschreiben}

Folgende Kategorien wurden bei der Sichtung der Daten identifiziert: / Im Folgenden wird ein Überblick über die identifizierten Kategorien von Code-Switches und dazugehörige Beispiele gegeben:

%TODO:
% Bsp für jede Kategorie aufzählen
% auf Deutsch übersetzen

\begin{enumerate}
  \item Named Entities, dadrunter products (heap of commercials); journals, magazines, newspapers; organizations; book and movie titles; TV-Sendungen; Events; Musik; Orte; Läden/Firmen; Youtube Videos; US-amerikanische Feiertage;
  \item Direct quotes:
  \item English vocab which has more or less entered Spanish (presumably): (unisex, piercing, party)
  \item English discourse markers as part of the Spanish text:
  \item English interjections as part of the Spanish text:
  \item Half translated English idiomatic expressions
  \item English Idiomatic expressions / collocations
    * "Mix and match , la pareja ideal"
  \item Whole phrases in English, not necessarily idiomatic % todo: check: what's idiomatic anyway?
  \item Misc: one English word without category at the moment
  \item "One night stand"/sex stuff (72933)
  \item Dating
  \item Cool/slang/hip/life-style
  \item Further TV/Media/Movies stuff
  \item Music stuff
  \item Food/cooking stuff
    \begin{itemize}
      \item Organic/healthy/lifestyle bla
      \item und dann meat; fish; fruit; vegetables; candy; junkfood; drinks; herbs and spices; misc
    \end{itemize}
  \item Diet
    \begin{itemize}
      \item Products
      \item Diet names
    \end{itemize}
  \item Fashion stuff
    \begin{itemize}
      \item Clothing
      \item Hair
      \item Colors
      \item Makeup
      \item misc: trendy; look; vintage
    \end{itemize}
  \item Cosmetic stuff:
  \item Personal descriptions
  \item Tech/Internet stuff (strange, would've expected more of this to have appeared by now; possibly tokens cathegorized as UNK-> have a look)
    \begin{itemize}
      \item general
      \item Social media:
      \item Internet
      \item Smartphone
    \end{itemize}
  \item Fitness stuff:
  \item Berufe/Stellenbezeichnungen: ``nanny'', ``babysitter'', ``dog walker'', ``coach de lifestyle''
  \item Pets
  \item Electrodomestics
\end{enumerate}

Es können für die vorgefundenen Kategorien von Code-Switches verschiedene Erklärungen geliefert werden.

% Named Entities
Die erste, und zugleich umfangreichste Kategorie, hat man vielleicht am Anfang nicht mitgedacht/geahnt..., sie ist jedoch kaum überraschend.
Es handelt sich dabei um die Gruppe der Named Entities (Eigennamen).
Wenn man überlegt in welchen Kontexten eventuell der Code gewechselt hat, sind Eigennamen vielleicht nicht das Erste was einer in den Sinn kommt.
Jedoch überrascht (syn!) es nicht, dass Produkte, Filme oder Organisationen bei ihrer Originalnamen auf Englisch genannt werden.
Man kann gewiss argumentieren, dass für einige davon bestimmt auch spanische Übersetzungen existieren (die Film- und Buchtiteln, einigen Produkten, ...), die Publikation richtet sich allerdings an ein Publikum, das diese Produkte auf dem US-amerikanischen Markt erlangen sollte und deshalb macht es total (syn!) Sinn, dass sie auf Englisch benannt werden.

% Entlehnungen
Bei manchen handelt es sich um Englische Vokabeln, die bereits größtenteils ins Spanische eingedrungen sind und mehr als Entlehnungen funktionieren.
Beispiele hierfür wären die Vokabeln ``unisex'', ``party'', ``piercing'', ``fitness'', ``blazer'', ``fan'', ``club'', ``shock''\footnote{All dieser Wörter werden beim Online-Wörterbuch \url{https://pons.eu} als Spanisch aufgelistet. \url{https://leo.org} führt alle bis auf ``party'' und ``fitness'' als Spanisch. Und die Online-Version des Wörterbuchs der konservativen Real Academia Española \url{https://dle.rae.es} erkennt alle davon bis auf ``piercing'' und ``fitness'' als Spanisch an.}.
Manche davon, wie ``piercing'' oder ``blazer'', beschreiben Gegenstände, für die es kein einzelnes Wort auf Spanisch existiert.
Bei anderen, wie ``party'', ``fan'' oder ``shock'',  ist die Entlehnung vielleicht weniger einfach zu erklären, da es entsprechende Äquivalenten auf Spanisch gibt (nämlich ``fiesta'', ``aficionadx'' und ``choque''). % also warum werden sie dann benutzt??
Die Benutzung von Vokabeln wie ``party'' oder ``fitness'' kann vielleicht auch mit der Inszenierung eines gewissen Lebensstils/Life Styles erklärt werden (siehe unten).

% Prestige
Ohne Zweifel ist in den USA Englisch die prestigevollere/angesehenere Sprache.
Spanischsprachige Bürgerinnen und solche, die als Latinas identifiziert werden, werden öfter auf dem Arbeitsmarkt aber auch im alltäglichen Leben systematisch diskriminiert.
Wer gut Englisch beherrscht hat bessere Chancen einen Job zu finden und sozial aufzusteigen.
Wiederum wird Spanisch mit Marginalisierung, ... asoziiert. % so für das alles brauchen wir eine Quelle.
Diese Tatsache lässt uns die Vermutung aufstellen, dass öftere Benutzung Englischer Vokabeln eine gewisse Überlegenheit in der sozialen Hierarchie signalisieren/andeuten sollte, bzw. dass Menschen, die auch so reden, signalisieren(syn!)/darauf hinweisen/zu verstehen geben, dass sie Englisch beherrschen und somit denen ein sozialer Aufstieg offen steht.

% English discourse markers
Genau damit würde ich die Kategorie der englischen Discourse Markers (DE?) erklären.
Sonst ist es kaum nachvollziehbar, warum im Text ``anyway'', ``must'' oder ``pros and contras'' erscheinen sollten, anstatt auf diese komplett zu verzichten, bzw. die spanischen Übersetzungen zu benutzen.
Bei gesprochener Sprache kann man sich diese Gebrauche noch mit spontanem Ausdruck (besser erklären) erklären, bei dem Medium Zeitschrift jedoch, wo die Autorinnen der Artikel Zeit zum Nachdenken hatten und der Text vermutlich einen (mehrstuffigen) Redigierprozess durchlaufen hat, kann von Spontanietät kaum die Rede sein.

% Sex
Eine weitere spannende Kategorie habe ich ``one night stand'' bzw Sachen, die mit Sex zu tun haben benannt.
Wir kennen den platten Ausdruck ``sex sells'' und die kapitalistische Gesellschaft tut Frauen gnadenlos als sexualisierte Objekte ausbeuten (besser ausführen, nach Kapitalismuskritik).
Warum aber werden Wörter und Ausdrücke aus diesem semantischen Feld auf Englisch gebraucht?
Ich würde hier wieder das Argument mit der Prestige, bzw coolness heranziehen. (vgl auch Laurie Penny!)

% Essen
Eine andere umfangreichere Kategorie, die vielleicht nicht vom Anfang an erwartet wurde, ist ``Essen''.
Diese habe ich wiederum in mehrere Unterkategorien aufgeteilt.
Zunächst haben wir es mit Lebensmitteln in Rezepten oder Dietprogrammen zu tun, die oft auf Englisch, oder auf Spanisch mit der Englischen Übersetzung in Klammern erscheinen.
Es handelt sich bei manchen dabei vielleicht um Nahrungsmitteln, die in den Heimaten (countries of origin) nicht besonders populär sind und demzufolge die Menschen kaum die spanischen Namen kennen würden.
Oft sind es unkonventionellere Lebensmittel, die in diese Kategorie auftauchen: ``kale'' (Grünkohl), ... (aber auch nicht ausschließlich, wir haben hier auch ``banana'', ``steak'' oder ``baking soda''). % zu prüfen: eigentlich sind gar nicht so viele weirde sachen am start!
Es kann also auch eine einfachere Erklärung für deren Vorkommen auf Englisch gefunden werden:
nämlich die Leserinnen, die ein Rezept oder Diät ausprobieren möchten, leben schlussendlich in den USA und sollten in der Lage sein im Geschäft die entsprechenden Produkte auf Englisch einzukaufen.
Dann ist es durchaus sinnvoll zu wissen, dass ``col risada'' auf Englisch ``kale'' heißt.

%% Lifestyle Essen
Eine weitere Unterkategorie, die ich hier identifiziert habe, hat auch mit ``life style'' zu tun.
Es handelt sich dabei um Sachen wie ``baby greens'', ``pavo wild'', ``gluten free'' etc.
Diese würde ich, wie bereits angedeutet, mit dieser Prestigeinszenierung und das Schaffen einer gewissen Atmosphäre von Coolness, Hippness und dadurch von Erfolg und sozialen Aufstieg. % na bzw grad hier wird ein gesunder Lebensstil inszeniert
% cool, flexibel, schwer beschäftigt wird eher durch ``coffeehouse'', ``dip'', ``snack'' evoziert.

%% US Zeug
Nicht zuletzt gibt es unter den Nahrungsmitteln auch welche, die vielleicht als spezifisch US-amerikanisch identifiziert werden können (oder zumindest sehr start mit der US-amerikanischen Kultur in Verbindung gebracht werden) und für die es dementsprechend vielleicht auch gar keine spanischen Begriffe existieren.
Beispiele hierfür sind ``brownies'', ``cheesecake'' oder ``peanut butter''.

% Technik
Eine der gefundenen Kategorien, die auch vom Anfang an vermutet wurde, ist Technik (bzw. Computer/Internet/Smartphonekram).
Hier ist ein erster Erklärungsversuch:
einerseits entstehen ganz viele der Geräte/Phänomenen in einem englischsprachigen Kontext (zb Silicon Valley);
und werden von da aus in anderen Teilen der Welt übernommen, und da die Konzepte/Gegenstände vorher nicht existiert haben, werden auch gleich ihre Bezeichnungen mitentlehnt.
Manche davon werden schlussendlich auf der lokalen Sprache übersetzt/(``angedeutscht'' für die entsprechende Sprache) (zb ``computadora'' oder ``en nube'' auf Spanisch), bei anderen wird die Benutzung der ursprünglichen Bezeichnung unverändert verstättigt (zb ``internet'', ``software'', ``wifi'').
Dieses Phänomen (oder diese Kategorie) können wir vermutlich bei jeder beliebigen Sprache finden und nicht nur im Spanisch-Englisch Kontext der Zeitschrift \textit{Siempre mujer}.

% Look
Auffällig oft kommt in den Artikeln das Wort ``look'' vor.
Zusammen mit anderen Vokabeln (Bsp!) habe ich dieses in die Kategorie ``Cool/hip/Life-style'' einsortiert.
(Bzw auch life-style bei der Kategorie ``Essen'') würde dazugehören.
Ich würde argumentieren, dass es sich dabei um den Versuch handelt, das Publikum/die Leserschaft der \textit{Siempre Mujer} als coole, hippe, gesund lebende, moderne, schöne, erfolgreiche ... Superfrauen zu inszenieren (syn!)/darzustellen.
``Look'', also das äußere Aussehen, wird besonders bei Frauen oft mit Erfolg (oder sein Fehlen) gleichgestellt, wie die britische Periodistin und Autorin Laurie Penny argumentiert:

\begin{quote}
For modern women in this axious age, the makeover is a ritual of health and devotion and social conformity.
[\ldots]
Cosmetic surgery companies plaster public transport with promises to deliver not just physical changes, but emotional ones like `confidence'.
Fashion editorials advise us to spend money we don't have on skirt suits and handbags as `investment pieces'; you're not supposed to dress and style your body simply to please yourself but with one eye on your financial future.
That skirt suit really is an `investment' in a one-woman business whose product is you, only glossier.
This is what power, health and success means to the modern, emancipated woman: terminal exhaustion and a wardrobe full of expensive disuises.~\cite[p.41-42]{Penny14}
\end{quote}

%Kapitalismuskritik
Man kann die oben anlautende Kritik erweitern/weiter denken/verfolgen/... und ... daran anknüpfen:
Schlussendlich sind die hier erweckten/angedeuteten/.. Werte wie Flexibilität, (hip, jung), Mobilität, Kreativität, .. vom Kapitalismus aufgegriffen und angeeignet worden.
Wie Luc Boltanski und Ève Chiapello schreiben: .. ~\cite[]{BolChi07}.

    \begin{comment}
      Kapitalismuskritik
      [BolChi07]
      "Finally, capitalist restructuring over the last two decades -which, as we have seen, occurred around financial markets and merger-acquisition activities in a context of favourable government policies as regards taxation, social security and wages - was also accopmanied by significant inventives to greater labour flexibility. Opportunities for hiring on a temporary basis, using a temporary workforce, flexible hours, and a reduction in the costs of layoffs, have developed considerably in all the OECD countries, gradually whittling down the social security systems established during a century of social struggles." (p.xxxviii, Prologue) <-- damit sind die 70er und 80er gemeint
      "This process was widely encouraged by a significant number of the protesters of the era, who were especially sensitive to the themes of the artistic critique - that is to say the everyday oppression and sterilization of each person's creative, unique powers produced by industrial, bourgeois society. The tranformation in working methods was thus effected in large part to respond to their aspirations, and they themselves contributed to it, especially after the left's accession to government in the 1980s. Once again, one cannot fail to stress the fact that critique was effective." (p.199, 1968: The Crisis and Revival of Capitalism)
      "Correlatively, however, at the level of security and wages various gains of the previous period were clawed back - not directly, but via new mechanisms that were much less supervised and protective than the old full-time permanent contact which was the standard norm in the 1960s. Autonomy was exchanged for security, opening the way for a new spirit of capitalism extolling the virtues of mobility and adaptability." (p.199)
      "The displacements operated by capitalism allowed it to escape the constraints that had gradually been constructed in response to the social critique, and were possible without provoking large-scale resistance because they seemed to satisfy the demands issuing from a different critical current." (p.200)
      "What we have observed of the role of critique in the .. also the displacements and transformations, of capitalism .. always conductive to greater social well-being - leads us to underscore the inadequacies of critical activity, as well as the incredible flexibility of the capitalist process. This process is capable of conforming to societies with aspirations that vary greatly over time (but also in space, though that is not our subject), and of recuperatig the ideas of those who were its enemies in a previous phase." (p.200-201)
      "By contrast, it was by opposing a social capitalism planned and supervised by the state - treated as obsolete, cramped and constraining - and leaning on the artistic critique (autonomy and creativity) that the new spirit of capitalism gradually took shape at the end of the crisis of the 1960s and 1970s, and undertook to restore the prestige of capitalism. Turning its back on the social demands that had dominated the first half of the 1970s, the new spirit was receptive to the critiques of the period that denounced the mechanization of the world (post-industrial society against industrial society) - the destruction of forms of life conductive to the fulfilment of specifically human potential and, in particular, creativity - and stressed the intolerable character of the .. of oppression which, without necessarily deriving directly from historical capitalism, had been exploited by capitalis mechanisms for organizing .." (p.201)
      "By adapting these sets of demands to the description of a new, liberated and even libertarian way of making profit - which was also said to allow for realization of the self and its most personal aspirations.." (p.201)
      "By helping to overthrow the conventions bound up with the old domestic world, and also to overcome the inflexibilities of the industrial order - bureaucratic hierarchies and standardized production - the artistic critique opened up an opportunity for capitalism to base itself on new forms of control and commodify new, more individualized and 'authentic' goods." (p.467, The Test of the Artistic Critique)
\end{comment}


\begin{comment}
  1. Versuch der Analyse:
\begin{itemize}
  \item English vocab which has more or less entered Spanish (presumably): (unisex, party, piercing, fitness, blazer, shock, fans, club, picnic)
  pons.eu findet folgende davon bei Spanish: unisex, party, piercing, fitness, blazer, shock, fan, club, pícnic
  leo.org : unisex, piercing, blazer, shock, fan, club, picnic
  dle.rae.es: unisex, party, blazer, shock, fan, club, pícnic
  \item ``look'' % vlt haben die letzten 3 Kategorien was gemeinsames: hippe/internet/jugend sprache? die für flexibilität/verfügbarkeit/bla steht? (kapitalismuskritik?)
    \begin{comment}
      auch zum Thema "look": [Penny14]
      "For modern women in this axious age, the makeover is a ritual of health and devotion and social conformity. It's he central transfigurative myth of modern femininity, and it's lucrative. Playing the woman game, the game of artifice and self-annihilation, is serious business. A recent survey by shpping channel QVC claimed that the average British woman spends 2,055 pounds per year, or 11 per cent of the median full-time female salary, on maintaining and updating the way she looks. Men, by contrast, spend just 4 per cent of their salary on their appearance, most of which goes on shaving and the gym.
      [...]
      Cosmetic surgery companies plaster public transport with promises to deliver not just physical changes, but emotional ones like 'confidence'. Fashion editorials advise us to spend money we don't have on skirt suits and handbags as 'investment pieces'; you're not supposed to dress and style your body simply to please yourself but with one eye on your financial future. That skirt suit really is an 'investment' in a one-woman business whose product is you, only glossier. This is what power, health and success means to the modern, emancipated woman: terminal exhaustion and a wardrobe full of expensive disuises." (p.41-42, Fucked-up Girls)
    \end{comment}
    \begin{comment}
      Kapitalismuskritik
      [BolChi07]
      "Finally, capitalist restructuring over the last two decades -which, as we have seen, occurred around financial markets and merger-acquisition activities in a context of favourable government policies as regards taxation, social security and wages - was also accopmanied by significant inventives to greater labour flexibility. Opportunities for hiring on a temporary basis, using a temporary workforce, flexible hours, and a reduction in the costs of layoffs, have developed considerably in all the OECD countries, gradually whittling down the social security systems established during a century of social struggles." (p.xxxviii, Prologue) <-- damit sind die 70er und 80er gemeint
      "This process was widely encouraged by a significant number of the protesters of the era, who were especially sensitive to the themes of the artistic critique - that is to say the everyday oppression and sterilization of each person's creative, unique powers produced by industrial, bourgeois society. The tranformation in working methods was thus effected in large part to respond to their aspirations, and they themselves contributed to it, especially after the left's accession to government in the 1980s. Once again, one cannot fail to stress the fact that critique was effective." (p.199, 1968: The Crisis and Revival of Capitalism)
      "Correlatively, however, at the level of security and wages various gains of the previous period were clawed back - not directly, but via new mechanisms that were much less supervised and protective than the old full-time permanent contact which was the standard norm in the 1960s. Autonomy was exchanged for security, opening the way for a new spirit of capitalism extolling the virtues of mobility and adaptability." (p.199)
      "The displacements operated by capitalism allowed it to escape the constraints that had gradually been constructed in response to the social critique, and were possible without provoking large-scale resistance becausethey seemed to satisfy the demands issuing from a different critical current." (p.200)
      "What we have observed of the role of critique in the .. also the displacements and transformations, of capitalism .. always conductive to greater social well-being - leads us to underscore the inadequacies of critical activity, as well as the incredible flexibility of the capitalist process. This process is capable of conforming to societies with aspirations that vary greatly over time (but also in space, though that is not our subject), and of recuperatig the ideas of those who were its enemies in a previous phase." (p.200-201)
      "By contrast, it was by opposing a social capitalism planned and supervised by the state - treated as obsolete, cramped and constraining - and leaning on the artistic critique (autonomy and creativity) that the new spirit of capitalism gradually took shape at the end of the crisis of the 1960s and 1970s, and undertook to restore the prestige of capitalism. Turning its back on the social demands that had dominated the first half of the 1970s, the new spirit was receptive to the critiques of the period that denounced the mechanization of the world (post-industrial society against industrial society) - the destruction of forms of life conductive to the fulfilment of specifically human potential and, in particular, creativity - and stressed the intolerable character of the .. of oppression which, without necessarily deriving directly from historical capitalism, had been exploited by capitalis mechanisms for organizing .." (p.201)
      "By adapting these sets of demands to the description of a new, liberated and even libertarian way of making profit - which was also said to allow for realization of the self and its most personal aspirations.." (p.201)
      "By helping to overthrow the conventions bound up with the old domestic world, and also to overcome the inflexibilities of the industrial order - bureaucratic hierarchies and standardized production - the artistic critique opened up an opportunity for capitalism to base itself on new forms of control and commodify new, more individualized and 'authentic' goods." (p.467, The Test of the Artistic Critique)
    \end{comment}
  \item English discourse markers as part of the Spanish text  (``anyway'', ``pros and contras'', ``must'', etc.)% (gehört vlt auch dazu?) oder ist dieses "Oh, guckt ma, ich kann voll Englisch und bin deshalb voll dabei." Wie heißt auf Deutsch "upward mobility"? -- "Sozialer Aufstieg" oder "soziale Aufstiegsfähigkeit"
  \item Music stuff (v.a. Genres)
  \item "One night stand"/sex stuff (72933) (und Dating) --> sex sells! (aber warum auf EN?)
  \item Essen:
    \begin{itemize}
      \item Kochrezepte bzw. Werbung/Diättips: auf Spanisch, die Englische Übersetzung in Klammern (damit die Menschen es im Supermarkt finden und kaufen können)
presumption: people living in the US would know (specific) foods better by there English name;
people living in the US should be able to buy stuff under their English name, since it's unlikely that the supermarket has Spanish labels
e.g. if the food is not very popular in their country of origin; (couldn't say that for baking soda probably though^^)
      \item lifestyle shit (``baby greens'', ``pavo wild'', ``gluten free'', ``nutritional yeast''..)
Organic/healthy/lifestyle bla
      \item Zeug, das spezifisch US-amerikanisch ist (oder relativ krass mit der Kultur in Verbindung gebracht wird): ``brownies'', ``cheesecake''; (was ist mit ``black licorice''??)
      \item auch lifestyle aber mehr das "immer unterwegs, schwerbeschäftigt, cool, hipster bla": ``coffeehouse'', ``dip'', ``snack'', ``brunch''
      \item Diätkram: v.a. ``light'', aber auch ``thigh gap''
    \end{itemize}
  \item cosmetic (nicht zwangsläufig um modeerscheinungen zu beschreiben)
\end{itemize}

\end{itemize}
\end{comment}


%--------------------------------------
\subsection{Interpretation}

\begin{itemize}
  \item English is the more prestigious language;
  \item the dominant (both in numbers and power) group of people in the USA speaks English
  \item serves to avoid misunderstandings/clarify (zb cooking recipes)
  \item es wird ein bestimmter Lebensstil kreiert/bedient
  \item jung, flexibel, kreativ?, gut ankommend aufm Arbeitsmarkt
\end{itemize}

\begin{comment}
Earlier Intents on Explanation
------------------------------
* Movies, Book names are not translated;
  audience living in the USA much more likely to know the stuff under their English names
* the rest of the named entities: even less expected to be translated?
* multiple clinics, hospitals and universities among the organizations (but that doesn't look that exciting)

  %Language Shift
Language Shift (Spanish -> English) : Speech community of a language shifts to speaking another language (source? Thomason?)
Can we talk about language shift in this context?
Who is shifting? The editors of the magazine? Or do they induce language shift?
Should that be a focus at all? Or focus more on communities of practice/style projection through language/sozioindexikalität, etc.

Ein Shift Spanisch->Englisch ist nicht nur mit Prestige zu erklären, sondern auch mit allgemeinen Anpassungs/Integrationsprozessen:
Wer in den USA Englisch kann, hat bessere Chancen einen Job zu finden, etc. na, aber das hat wohl auch mit Prestige zu tun gewissermaßen

%Language Prestige
English is the more prestigious language;
the dominant(both in numbers and power) group of people in the USA speaks English
serves to avoid misunderstandings/clarify (zb cooking recipes)
--> das kommt alles zur Interpretation!


  <aside>
    The first one is probably the least surprising of all.
    It's what we call "Named Entities" in Natural Language Processing.
    So, names of things: products, media, companies, organizations, events, you name it^^.
    And this definitely makes sense: mostly, these are references to US media (or products, companies, organizations, etc.), so their names are originally English.
    And even if an established Spanish translation exists (which, I imagine is often not the case), people living in US (regardless of their first language) would be far more likely to recognize the entity in question under its English name.
  </aside>
  <aside>
    The second category are things that have to do with technology, computers, or the Internet.
    Again, it's one that you've probably expected from the beginning.
    After all, most of this technology stuff has happened in English the first time around, so often only the English term exists and gets adopted (borrowed) in other languages as well.
    Some of them get translated after a while, others don't, but mostly even if translations exist, the English word is still recognized and used.
  </aside>
  <aside>
    The third category is something I hadn't really thought of at the beginning, but you'd found it quite logical once you'd reflected on it for a second.
    It's about food.
    There are bunch of cases where you've got the name of a specific food in Spanish, followed by the English translation in parenthesis.
    Mostly, these are articles containing cooking recipies or diet advice.
    And it makes total sense: after all, people living in predominantly English context should be able to seek out and buy the foods in questions in supermarkets where goods are most probably labeled in English only.
    Moreover, sometimes we've got some quite specific foods people may not even know the names of in Spanish, even if Spanish is their dominant language.
  </aside>
  <aside>
    And then, we have this other stuff related to food appearing in English which I labeled "life style".
    It's all about projecting some healthy, organic, hip, successful way of being.
    And it's happening in English.
  </aside>
  <aside>
    There is actually more I labeled "life style", not necessarily food related stuff.
    We've also got "trendy", and "look", and "vintage" more than a couple of times.
    My guess is that by using these terms in English a particular air of hipness, maybe sometimes of upward mobility is projected.
  </aside>
  <aside>
    The final category I'm showing you before trying to wrap things up may be somewhat puzzling.
    It consists of (half-translated) English idioms/collocations.
    (For those of you who are not quite sure, an idiom is something like "it's raining cats and dogs" and a collocation is a fix phrase, stuff that often goes together, such as "machine learning" or "decision makers".)
    Here you can certainly argue that some of them may not have an exact translation in Spanish and that the English phrase is much more accurate in the particular case.
    But that's not always the case.
    And it doesn't explain why the phrases are sometimes half in Spanish and half in English.
    If it was spoken language, you could argument with spur of the moment: you couldn't think of the exact phrase in Spanish right then, the other person understood English anyway, so you just used what was on your mind.
    Well, this explaination certainly doesn't hold when we're dealing with written language, stuff published in a magazine, where more than one person has looked at it and consciously decided to use the particular wording.
    And there's indeed another explaination theory, that the publishers of the magazine try to establish proximity, intimacy with their readers.
    By purposefully using a combination of English and Spanish in their texts, much in the same fashion a bilingual person would do, more or less consciously, in their oral discourse, the publishers and authors are signalling "we're part of the same club".[Mahootian05]
    Für solche Art Switches wäre es im Grunde egal, was genau geswitcht wird, Hauptsache, die kommen vor, und unterstreichen dabei die Gemeinsamkeit mit den Leserinnen.
    Eine andere Erklärung, zB auch für Englische Interjektionen u.ä. ist "draws your attention by the fact that it's not Spanish" [Mahootian05].
    "to identify themselves as a group separate from their predecessors’ generation"[Mahootian05] --> grad bei den Englisch Switches in dem vorwiegend Spanischen Text: sich als modern, hip, .. projizieren
  </aside>
\end{comment}

