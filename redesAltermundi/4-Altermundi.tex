\section{El proyecto AlterMundi}

\subsection{Desarollo histórico}
% oder \subsection{Altermundi: fundación, objetivo, estructura, desarrollo histórico}

La asociación civil AlterMundi comienza a formalizarse en el año 2012 con el objetivo de facilitar talleres de redes libres dentro del proyecto Arraigo Digital, iniciado por el Ministerio de Educación en Argentina con el fin de ``capacitar a los jóvenes en conocimiento, funcionamiento y armado de una red digital comunitaria, además de la introducción y práctica del software libre''~\autocite{Picolli2015}.
AlterMundi surge desde la cooperativa CodigoSur\footnote{\url{https://www.codigosur.org/quienes-somos/codigo-sur}}, un grupo de activistas de diferentes países de América Latina que facilita el desarrollo y uso de tecnologías libres
\footnote{Cuando hablamos de tecnologías libres, no nos referimos a productos gratuitos, sino a tales que conceden a lxs usuarixs ciertas libertades, como la posibilidad de utilizar el software para lo que quieran, estudiar el funcionamiento del software, modificar el software según sus necesidades y compartir las versiones modificadas.
Para más información, veáse también la definición propuesta por la Free Software Foundation: \url{https://www.fsf.org/about/what-is-free-software}.}
de información y comunicación para diferentes movimientos sociales.
La forma legal de asociación civil les permite hacer alianzas con otras organizaciones formales, como por ejemplo la Universidad Nacional de Córdoba que provee un enlace al Internet o diferentes fundaciones que les han prestado suporte financial.
Lo interesante es que AlterMundi en si no es una red comunitaria, sino una organización de activistas que suporte el despliegue de tales redes por toda Argentina.

% add cronología
\begin{comment}
2007 reforma constitucional/de la ley telemediatica: "derecho a comunicación"

2009 ley... ??  Ley de Servicios 
de Comunicación Audiovisual No. 26.522, sancionada el 10 de octubre del 2009.  : deconcentración de la comunicación: empresas comerciales no pueden obtener más freqüencias/licencias: hay un límite de freqüencias para cada sector;
http://servicios.infoleg.gob.ar/infolegInternet/anexos/155000-159999/158649/norma.htm
Grupo Clarín se nieda a implementar la ley

y también:
"ARTICULO 97. — Destino de los fondos recaudados. La Administración Federal de Ingresos Públicos destinará los fondos recaudados de la siguiente forma:
...
f) El diez por ciento (10%) para proyectos especiales de comunicación audiovisual y apoyo a servicios de comunicación audiovisual, comunitarios, de frontera, y de los Pueblos Originarios, con especial atención a la colaboración en los proyectos de digitalización107."

2012
* El gobierno argentino conversa con codigosur con la idea de ofrecer talleres de software libre/.. por Argentina
* nace también la idea de ofrecer talleres de redes
* Altermundi se forma como proyecto desde codigosur
* el gobierno quita el apoyo financial prometido (el contacto ya no está)
* 1. hackathon (la gente ya se había comprado la técnica): QuintanaLibre en José de la Quintana en Córdoba: compartir Internet entre 2-3 familias (llega hasta 60 familias hoy)

...
* Delta Libre (1. Hackathon?)
* Quintana Libre
* Nono libre
* ..

2013
* la Corte Suprema de Justicia confirma la consitucionalidad general de la ley

2014
* Dec: Ley Argentina Digital (Ley 27.078) (https://www.enacom.gob.ar/ley-27-078_p2707):
"Artículo 82:

“fomento y resguardo de las denominadas redes comunitarias, garantizando que las condiciones de su explotación respondan a las necesidades técnicas, económicas y sociales de la comunidad en particular”" http://blog.altermundi.net/article/argentina-digital-fomento-a-las-redes-comunitarias/
"el reconocimiento, por primera vez en la legislación nacional, de las Redes Comunitarias como un actor en el escenario de las comunicaciones, afirmando su valor al proponer su fomento y resguardo, es un avance muy grande respecto de la situación legal confusa en la que estas redes deben operar hasta hoy en nuestro país."


2015
* talleres educativos con otros proyectos (cooperativa de mujeres en NIC?)
* Abr: LaBolsaLibre va en linea
* Dec: Macri llega al poder en Argentina --> política orientada hacia al mercado: laissez-faire capitalismo, los límites para lincencias cesan
  ** deroga el Artículo 82 de la Argentina Digital

reticulación (vernetzung)/intercambio de experiencias con otros proyectos internacionales
* Rhizomatica
* BattleMesh



2016
* principios: Macri modifica la Ley 26.522 con un decreto: " morigerar el carácter antimonopólico de la ley, beneficiando a los principales medios de comunicación del país. "
* FCC + EU: Firmware lockdown
* Nace el proyecto del Libre router
"This virtuous relation between hardware vendors and the community has been threatened by new regulation from the Federal Communitations Commision (FCC) – U.S.A. – which has led vendors to globally close up their routers to third party modifications, hindering open innovation and effectively closing the door to Community Networks in terms of access to the hardware they depend on. https://librerouter.org/what-and-why"

2017
* Julio: el decreto de Macri llega al Corte Suprema, que aun ha de pronunciarse hacia su legalidad
* Octubre: planificado el lanzamiento del libre router
pliegues de redes comunitarias en Argentina

% Desde la página web
Las redes aquí enumeradas utilizan tecnología desarrollada por AlterMundi y reciben soporte y asistencia de parte de la Asociación.

QuintanaLibre, Córdoba, iniciada en Marzo de 2012 
DeltaLibre, Buenos Aires, iniciada en Abril de 2012 
AnisacateLibre, Córdoba, iniciada en Septiembre de 2012 
LaSerranitaLibre, Córdoba, iniciada en Septiembre de 2013 
NonoLibre, Córdoba, Iniciada en Marzo de 2014 
BoquerónLibre, Santiago del Estero, Abril de 2015 
LaBolsaLibre, Córdoba, iniciada en Mayo de 2015

Despliegues de redes comunitarias en el exterior

Fumaça Online, Río de Janeiro, Brasil, Julio de 2015
MulukukuLibre, Nicaragua, Enero de 2016
Caimito Libre, Esmeraldas, Febrero de 2017
\end{comment}

Al final, aquell proyecto no se realizó en la forma prevista~\autocite{Vaseva2016a},~\autocite{iv-nico-pillku}, pero la primer red local, QuintanaLibre, ha ido en linea igualmente.
Desde entonces el grupo básico de activistas, que es Altermundi, ha ayudado a diversas comunidades a construir redes comunitarias locales:
hasta hoy existen ....

\subsection{La red como infraestructura técnica}
% oder La red como infraestructura física

Bastante intrigante, aunque en un primer momento quizás un poco contralógica, resulta la observación de Latour que ``the expansion of digitality has enormously increased the material dimension of networks''~\autocite{Latour2010}.
Sin embargo, si reflejionamos un poco sobre el asunto, nos damos cuenta que sí, la tecnología digital deja muchas huellas y artefactos en el mundo físico: si nos fijamos, podemos descubrir las antenas, los routers y los cables per tot arreu(esp??).
Gulli deckel marcan los lugares de los cables transatlánticos~\autocite{video-internet-landscapes}.
Si seguimos las antenas y los routers, podemos reconstruir hasta un cierto grado las redes físicas de proyectos comunitarios.

% Commons; moechte ich aus der Ecke argumentieren?
En cierto sentido, las redes comunitarias se pueden ubicar en la lógica de los bienes comunes~\autocite{FiTre2015}.
Ya la denominación ``redes \textit{comunitarias}'' insinua esta idea.
Interesante sería fijarse en el valor ideológico que lleva esta noción.
Según Rieder, el término ``comunidad'' es exclusivamente positivo en la historia cultural anglo-sajona~\autocite{Rieder2012}, mientras en otros contexto el significado sea más ambivalente.
Por ejemplo en alemán, la noción se relaciona también con la ``Volksgemeinschaft'' (``comunidad popular'') de lxs nacis. % so what
No obstante, en el sentido/caso de ``commons'' y ``redes comunitarias'' la conotación es más la de la tradición anglo-sajona.

Aunqué probablemente partes de la técnica partenecen tecnicamente(syn!) a personas privadas, otras partes son más una propiedad comunal.
Aunqué la tecnología que utilizan las redes comunitarias es bastante robusta (vease Cap. X), está en el interés común de arreglar problemas que usuarixs particulares tienen, ya que si un aparato malfunciona toda la comunidad sufre.
También, como señalan De Fillippi y Tréguer, esto es particularmente el caso para las redes comunitarias inalámbricas, ya que estas dependen exclusivamente a free-to-use airwaves (``spectrum commons'')~\autocite{FiTre2015}.
Ellxs argumentan que en contraste con la wide-spread noción que la gestión de bienes comunes no funciona muy bien en práctica, en el caso de open spectrum, el approach tiene éxito:
el spectro está usado más thorough/racionalmente/..
% of spectrum and exclusive licensing still have the upper hand, they have often come short of fostering public interest goals, for instance by causing a very significant underutilization of this public resource"
Además, infraestructura pública, creada por el estado con el dinero de impuestos ha de ser bien común y no regalada a empresas privadas, argumentan lxs científicxs~\autocite{FiTre2015}.

\begin{comment}

[nico-iv-pillku]
"antenas que fabricamos localmente."

"El firmware que instalamos: AlterMesh, permite la auto-configuración de los equipos de manera que al arrancar el nodo ya “sabe” como participar de la mesh. "

"todo el pueblo funcione como una gran red local, lo que facilita la implementación de servicios y compartir contenidos."

% Sicherheit
"Los riesgos de sufrir ataques informáticos en una red libre no difieren particularmente de los riesgos que afrontan otros tipos de redes"
"El hecho de que la red permita con más facilidad a las personas ofrecer servicios y compartir contenidos, genera también una conciencia diferente sobre el uso de la red y de sus dispositivos. Creemos que esta toma de conciencia es positiva también en relación a la seguridad."

[FiTre2015]
% la infraestructura técnica
autonomous local networks + uplink to the internet (which is a potential bottleneck)
gibts in Altermundis redes local services? (vgl nicos iv)
ja --> "Contamos con un portal del pueblo donde todos pueden publicar información e inclusive clasificados. También tenemos un servicio interno de chat, independiente de redes externas, que al igual que el servicio de llamadas de Voz sobre IP (en experimentación) nos permite conectarnos entre vecinos de manera fluída. La Radio FM del pueblo aprovecha la red para hacer transmisión en vivo de su programación, tanto a la red local como a Internet."


%TODO
[Latour2010]
"the technology.. /digital technology .. makes networks material and explicit" (infrastruktur ist präsent und sichtbar im öffentlichen Raum) --> wem gehört die Infrastruktur? wer kontrolliert die? Wer hat (physischen) Zugang dazu?
  geografía: terreno difícil; quien controla la infraestructura?

"subversion it introduces in the notion of distance (the adjectives “close” and “far” are
made dependant on the presence of conduits, bridges, and hubs)," <-- physical network
\end{comment}

% Bild: artefactos físicos en las redes facilitadas por Altermundi
Reference the figure composed of multiple subfigures as Figure~\vref{fig:esempio}. Reference one of the subfigures as Figure~\vref{fig:ipsum}. % The \vref command specifies the location of the reference

\begin{figure}[tb]
\centering
\subfloat[Outdoor router en la red DeltaLibre.]{\includegraphics[width=.45\columnwidth]{deltalibre8}} \quad
\subfloat[Antenas en DeltaLibre.]{\includegraphics[width=.45\columnwidth]{deltalibre10}\label{fig:ipsum}} \\
\subfloat[Instalación de antenas en NonoLibre.]{\includegraphics[width=.45\columnwidth]{nonolibre7}} \quad
\subfloat[Autoproducción de antenas.]{\includegraphics[width=.45\columnwidth]{nonolibre8}} \\
\subfloat[Jesi en QuintanaLibre]{\includegraphics[width=.45\columnwidth]{2013_quintanalibre2}} \quad
\subfloat[Antena en las altes cumbres con autonomía energética solar]{\includegraphics[width=.45\columnwidth]{16-todos_trabajando}}
\caption[A number of pictures.]{Altermundi: artefactos físicos.} % The text in the square bracket is the caption for the list of figures while the text in the curly brackets is the figure caption
\label{fig:esempio}
\end{figure}

% eher zum kapitel gemeinschaftsnetzwerke; vlt den kapitel mit dem hier verschmelzen
Proyectos de redes comunitarias como Altermundi se ubican claramente en el contexto de la segunda onda del ciberoptimismo, que Rieder esboza~\autocite{Rieder2012}.
Sus valores claves (core values) son la decentralización, la distribución del control, la autogovernanza y la organización sin jerarquías~\autocite{FiTre2015}.
Estos proyectos de base (grassroots projects) intentan a revolucionar la comunicación desde abajo, a empoderar a lxs usuarixs a tomar su comunicación en sus propias manos.

\begin{comment}
* community network projects: are to be found in the context of (2nd wave) cyber optimism: "decentralization", "distributed control", "self-governance", "non-hierarchical organization"
an attempt at a capillary revolution -> ist es erfolgreich?
\end{comment}


\subsection{La red social de Altermundi}
% \subsection{La red como tejido social}

Como ya hemos señalado, Altermundi es una asociación social que ayuda a comunidades locales en la construcción de redes inalámbricas.
En cierto sentido, podemos entender a Altermundi y las redes juntos como un único actor en la ANT.
No obstante, si hacemos un zoom in, este se descompone en el grupo básico de activistas y redes autónomas locales.
Y un zoom más allás nos deja aun con personas con necesidades y opiniones individuales.

En esta constelación el papel de Altermundi es sobre todo educar y concienciar.
A través de talleres de formación el grupo intenta a empoderar a las comunidades locales de manejar sus propias necesidades comunicativas.
La comunidad toma las decisiones relevantes de forma autónoma y Altermundi tiene un papel(syn) consultativo.


Bastante interesante es también la recepción del proyecto por parte de las redes.
Algun de los problemas mencionados más frecuentemente es la falta de conocimientos técnicos en la comunidad local.
Entonces, observamos que aunque el proyecto intenta a facilitar la construcción de redes para personas non-técnicas, esto funciona solamente hasta un cierto grado.

\begin{comment}

[nico-iv-pillku]
%Problemas
"Más allá de los desafíos típicos relacionados a la dinámica social, las estrategias de transferencia de conocimiento y las problemáticas económicas típicas de las poblaciones pequeñas donde se despliegan estas redes comunitarias, la región sufre otros problemas estructurales en cuanto al acceso a la red global."
"En nuestro país, como en el resto de Latinoamérica y en general en los países que no son históricamente centrales en el desarrollo de Internet, la principal dificultad para cualquier intento de democratización del acceso a la vida digital es el “impuesto al más débil” que pagamos cuando queremos comprar ancho de banda dedicado, para conectar estas redes al resto de Internet. Para dar una idea de la dimensión de este problema basta un ejemplo: en el Punto Neutro (Internet Exchange) de Cataluña, se puede comprar 1Gbps, es decir “mil megas por segundo” en vocabulario coloquial, por un valor mensual de 780 us dolares. En la Argentina, aún con los esfuerzos colectivos realizados por los pequeños y medianos operadores nucleados en la Cámara Argentina de Internet, el costo por 1Mbps en un Punto Neutro local ronda los u$s 40. Es decir que 1Gbps costaría unos u$s 40.000 por mes: un 5.100 por ciento (51 veces) más caro que en Europa.
Para entender cómo impacta esto en la democratización del acceso a la red de redes, pensemos que con lo que cuesta comprar 1Gbps durante un mes en Argentina, se podría contratar durante más de 4 años el mismo servicio en Cataluña, que tampoco es la plaza más económica del mundo.
Hoy en día, que en la Argentina los abusos en las diferentes cadenas de valor se han puesto en la mira con una política activa, sería muy positivo que hubiera una evaluación seria sobre los costos reales que afrontan los proveedores mayoristas en esta actividad y sobre todo tomar dimensión de los valores que se manejan en el mundo para desentramar este abuso que sufrimos a manos de un núcleo muy concentrado de empresas multinacionales.
"

[FiTre2015]

% Social
"Rather than being driven by profits, they focus on the actual needs of the needs of its participants. They
also experiment with novel models of distributed governance"
"flat organizations and a peer-to-peer approach to decision-making, based on deliberation and consensus"
-- stimmt fuer die redes

%% polit. engagement der Projekte
eigentlich sehen sie sich nicht zwangslaeufig als politisch y operan más fuera de la lógica de la política convencional de instituciones o del activismo político disidente (protestas, etc.)
"One way is to address the issue from within the political system, as ‘insiders’, formally interacting with the power holders"
"Another solution is to fight the problem as ‘outsiders’, pressuring both regulators and
incumbents from outside the political system, by means of protests, demonstrations and other campaigning tactics"
"Yet, most of the community networks we surveyed do not properly qualify as what social movement scholars define as ‘insiders’ (although they
sometimes do interact with policy-makers), and much less as ‘outsiders’. Mostly, they fall within the third category—what Milan identifies as
‘beyonders’."
"remaining beyond the political system"
--> vgl auch "route around laws" from Rieders text (zitat eff mensch)

allerdings sind sie schon in essenz politisch:
"hese networks are ‘prefigurative realities’ that
challenge the status quo and ultimately contribute to a new political order (Milan, 2013, pp.126-38): these networks – built ‘for the people, by the
people’ – fundamentally embody a form of political action."

"distributed counter-power to traditional telecoms operators"
"adhere to specific ethical commitments and governance structures."
"From locally-grown food to locally-grown networks, WCNs form part of a wider movement
focused on empowering local communities to directly produce and manage the resources that matters the most to them."
-- paralelo soberania de comunicación - soberanía alimentaria (vgl Interview Isa)

und fazit: es lohnt sich politische Loesungen fuer polit. Probleme zu suchen
"these examples show that ‘insider’ strategies, i.e. direct engagement with policy-makers, are worth pursuing."
-- davor kommentar: es ist fuer solche Iniziativen oft nicht feasible sich mit Politik zu beschaeftigen, v.a. da sie dafuer keine Ressourcen haben;
es ist eine volunteer community, die netzwerke bauen moechte und nicht sich mit gesetzestexten beschaeftigen.
weiss nicht in wie fern das alles auf altermundi zutrifft, aber die sind im grunde auch alles techies;
und auch wenn deren motivation weniger faszination mit technik und mehr hippie-soberania comunicativa-derecho a comunicación ist, haben die vermutlich genau so wenig menschen, die von legal issues plan haben und nicht unbedingt riesenbock sich da reinzusteigern
\end{comment}

\begin{comment}
  Recepción desde las Redes

% licencia de todos los videos: Creative Commons Attribution license (reuse allowed) /Creative Commons Atribucion-Compartir Obras Derivadas Igual 2.5./CC BY-SA 2.5 AR

%  LaBolsaLibre https://www.youtube.com/watch?v=x6nONJXcUQ8
* somos grupo de vecinos/amigos, que ha surgido desde el despliegue de la red, después de la semana de hackathon de construcción de la red
* nos faltan conocimientos técnicos, somos dependientes de otras redes, si un nodo se quema o la red se cae

% QuintanaLibre https://www.youtube.com/watch?YMKlvUS7B-A&v=hsPjT2R-ToQ
* proyecto grande: diferentes grados de involucración
* mucho trabajo
* cómo atraer miembros nuevos?: reuniones con gente ya involucrada, que expliquen que es lo que les gusta en la red, como es su experiencia, "como cambió su vida por la parte de la red"

% AnisacateLibre https://www.youtube.com/watch?YMKlvUS7B-A&v=iasMfOAaDpk
* 10 familias conectadas
* 1 persona responsable, empezó sola, el único nodo en el barrio; se reunió con gente de Altermundi después, intercambiando experiencia, etc.
* desde 2012-2013, después de José de la Quintana
* comparten datos, tráfico, etc.. --> para el responsable el aspecto social es clave: estar conectado con más personas, que comparten
* red comunitaria != internet gratis, sino se hace entre todosa -> importante que lxs participantes lo entiendan

% LaSerranitaLibre https://www.youtube.com/watch?YMKlvUS7B-A&v=Kv2246EDPPg
* desde fines de 2013-2014
* 2016: 15 familias conectadas?
* tenían la necesidad, los proveedores no querían prestarle servicio y conocieron a QuintanaLibre
* ir de casa a casa, algo comunitario, sin un fin económico que estaba en vista
* la mayor dificultad entre vecinos: entender que es una red libre, sacarte un poco de la lógica del mercado, algo que se hace en la comunidad
* desafío: unir La Serranita con Córdoba, para tener conección; el tereno es duro, habían de subir mucho más alto;

% NonoLibre https://www.youtube.com/watch?YMKlvUS7B-A&v=RaXZqlALRII
* 25 nodos operativos: 18 familias y resto cabañeros?
* comenzó sep? 2014
* Nono: localidad túristica; por lo tanto la Camara de Comercio fue un de los impulsores del proyecto, cuando se enteró de ello
* conectividad importante para el desarollo personal pero también para el desarollo laboral del pueblo
* proyecto previo: Nono digital (tener wifi libre en Nono), se juntó con Altermundi -> NonoLibre;
* tenían 40 nodos, pero solamente 25 operativos, ya que el terreno es difícil
* desafío más grande: montar una antena en la Pampa de Achala (Altas Cumbres) --> conectar las redes del un lado de la monte con las del otro
* experiencia muy linda, gratificante: trabajar en comunidad
---
Reunión en LaQuintana 19 Marzo 2017
"Nono Libre: Tiene la dificultad que en momentos de mucha afluencia turística se satura la salida a internet. "

\end{comment}
\begin{comment}

[Freeman1970]
% wichtig fuer die analyse, weiss noch nicht genau wo
Klammer: In wie fern ist das bei mir eigentlich anwendbar? Mein Usecase hat doch eine gewisse Struktur: Kerngruppe + Lokalgruppen, in den Lokalgruppen gibts ne VV, die Entscheidungen trifft.
Vlt kann man argumentieren, dass konkrete Zuständigkeiten fehlen? Dafür kenne ich die Organisation aber viel zu schlecht. Sollte man nochmal mit denen reden?

"Elites are nothing more, and nothing less, than groups of friends
who also happen to participate in the same political activities."
Vmtl kann man ziemlich zutreffend argumentieren, dass die Kerngruppe ne Art Elite ist.
vgl auch "Superhero-Status" (im Gespräch erwähnt)

[Castells2015]
"There are usually a handful
of persons, sometimes just one, at the start of a movement."
--> vgl auch Super hero..


gatekeepers to infrastructure/indispensability (vgl auch 4 Charakteristiken einer erfolgreichen Kleingruppe): ist die Kerngruppe Infrastructure Gatekeeper? Man kann bestimmt von Knowledge als Ressource ausgehen und eine Wissenshierarchie feststellen. (Auch wenn große Bestrebungen gibt, der entgegen zu wirken)

7 Principles of democratic structuring:
1. delegation: Ich glaube das gibts; So was wie Kassenwart? Und eine lokale Ansprech-/zuständige Person für die Maintainence? (TODO: vgl mit Interview transcript)
2. responsibility: Es ist glaub ich schwierig, Personen zu Rechenschaft zu ziehen und Konsequenzen durchzuziehen, da ganz oft nicht gut möglich ist die Person an einer Position auszutauschen, weil sonst niemand den Posten haben will.
3. distribution: kp ob das zutrifft
4. rotation: dito
5. allocation of tascs along rational criteria (skills, interest, responsibility,..); apprenticeship program -> ich denk, das trifft auf jeden Fall zu (zumindest den Teil mit dem "apprenticeship program", es finden mehrere talleres,.. statt; Bildungsauftrag, das wird ernst genommen)
6. diffusion of information to everyone as frequently as possible - kp.
7. equal access to resources: das trifft vermutlich nicht zu;

\end{comment}

\subsection{Zusammenfuegen}

Al final, la separación aguda entre capa técnica y capa social es difícil y no necesariamente útil.
% mach das sinn?
% vlt ciberoptimismo hier?

\begin{comment}

[FiTre2015]
% Zum Zusammenfuegen chapter?
"we show how the current revival of grassroots community networks can counterbalance the erosion of autonomy of Internet users that results from current
telecom policies."

---
[Rieder2012]

> framing the Internet alternately as lawless, anarchic,
> free, “a world where anyone, anywhere may express his or her beliefs, no matter how singular,
> without fear of being coerced into silence or conformity” (Barlow 1996) (p.1)

* la infraestructura está prerequisito para participación

## Conclusión del texto [Rieder2012]

> If technology won’t deliver us from the conundrums of
> governance, negotiation, and struggle, we
> may as well reengage politics proper[ly].

* sobre todo problemas sociales/políticos: organización de grupos (no tanto técnicos): ¿cómo decidimos como grupo? ¿quién hace qué? ¿quién está responsable?
* soluciones políticas para problemas políticos
\end{comment}

\subsection{Mapeo: área conflictiva}

\begin{comment}
[Castells2015]
Programmers and Switchers
"Thus, who holds power in the network society? The *pro-
grammers* with the capacity to program each one of the main
networks on which people’s lives depend (government, par-
liament, the military and security establishment, finance,
media, science and technology institutions, etc.). And the
*switchers* who operate the connections between different
networks (media moguls introduced in the political class,
financial elites bankrolling political elites, political elites bail-
ing out financial institutions, media corporations intertwined
with financial corporations, academic institutions financed
by big business, etc.)."

ANT: podemos hacer aquí zooms in y out!

                          * da talleres
                          * apoyo técnico
                          * Bildungsauftrag

            Altermundi <-----------------------> redes
         /   ^                                     ^
individuos   |                                     |
             |                                     |
             |                                     |
             |                                     |
             v                                     v
            estado<-------------------------> infraestructura

Altermundi:
* individuos, gente que se comunica entre si, con todos los problemas y desafíos que surgen de esto
* tiene una forma organizativa legal: asociación civil (siehe IV) para poder entrar en alianzas, cooperaciones (con el estado y otras organizaciones formales, p.e. la Universidad de Córdoba)
* función auxiliar (las redes son autónomas)

Estado:
* está responsable para regulaciones que ayudan o impiden el trabajo de tales proyectos (ley telemediatica)
* la última milla
* solucionar el problema de conexión de manera completa/de gran escala (siehe papel del estado)
* apoyo financial (eher nicht, vgl Entstehungsgeschichte)
[FiTre2015]
% Mapamiento/papel del estado
--Law is more often than not, a hurdle (siehe auch firmware lockdown)
--Allerdings gabs in ARG ein progressives Telemediengesetz 2009, was Macri grad versucht zu kassieren.
"law should respond by ‘implementing policies that predictably diversify the set of options that all individuals are able
to see as open to them’ (Benkler 2006, p. 152)."
-- gibt eine Liste von was der Staat alles machen koennte, um die Arbeit der Community Networks zu erleichtern; man koennte vergleichen was davon alles auf ARG zutrifft;
auf jeden Fall, varying Erfolg, es gibt Versprechen, die nicht eingehalten werden; relativ progressive Gesetzgebung, dann wiederrum Versuche die rueckgaengig zu machen (hier sieht man wie auch der Actor "staat" in einzelne Regierungen bzw Personen zerfaellt, die alle verschiedene Ziele verfolgen)


Infraestructura:
* ¿donde está?
* quien la mantiene/gestiona/instala
* quien tiene acceso?

Redes
* autónomas
* tensiones: problemas con la comunicación dentro de la comunidad
* falta de conocimientos técnicos

Proveedores de servicios tradicionales
* intereses económicos comerciales
* interesados en el uso exclusivo de recursos del espectro
* no interesados en proveer acceso a internet en regiones pobres y con poca población
% Mapamiento -- commercial ISPs como actor --contrapoder!! [FiTre2015]
"a policy overhaul focused on community networks can indeed help create effective counter-powers to the
dominance of commercial operators in the communications infrastructure."
[Castells2015]
Contrapoder def:
"deliberate attempt to change
power relationships, is enacted by reprogramming networks
around alternative interests and values, and/or disrupting the
dominant switches while switching networks of resistance
and social change."


Foreign Regulatory Bodies: EC und FCC
* intento a regular el espectro
* sobre todo mesuras contra la interferencia con los rádares (weather radar) <-- no está relevante para regiones rurales
* verabschieden vage Richtlinien, interpretadas por los productores de hardware como aufruf, prohibir software libre
* igualmente que se trata de instituciones del EU/EEUU, esto va a influenciar todo el mundo, ya que sonst cada productor tiene que producir productos distintos para los diferentes mercados: con y sin implementar las regulaciones

\end{comment}

\begin{comment}
[Rieder2012]

%Analyse
"the Internet allows capillary configurations
of power – local initiatives, ad-hoc pressure groups, fan cultures, “issue publics” – to challenge
the statutory powers that be. "
Allerdings:
Wird argumentiert, dass alle mitmachen koennen, das stimmt aber nicth so direkt.
Koloniale Zusammenhaenge bestehen;
Geographische Schwierigkeiten;
Oekonom. Probleme (oft in Kombi mit schwierigem Terrain); --> Kommerzielle Provider haben kein Interesse
Staatliche Repression --> dem Staat passt nicht dass X oder Y kommunizieren kann und eine oeffentliche Plattform hat

% Analyse: parte social, papel del estado
"the convergence of this mutating counterculture with laissez-faire capitalism"
--> vlt kritische Perspektive: community networks as counterculture, aber andererseits spielen sie dem Staat teilweise in die Haende (laissez-faire capitalism, vgl Interview Gui)

% Analyse: redes, parte social
"Democracy rests on the idea that, except for technical details for which experts
may be useful, the important decisions of society are within the capability of ordinary citizens.
Not only can ordinary people make decisions about these issues, but they ought to, (Zinn 2003)"
--> vgl Altermundi: Experts: tech. Details; las redes: toman decisiones de manera autónoma

% Análisis
> If technology won’t deliver us from the conundrums of
> governance, negotiation, and struggle, we
> may as well reengage politics proper[ly].
--> Altermundi es explicitamente un proyecto político?
Ich glaub, it always boils down to thinking critically about what one's doing and assuming responsibility for it.

% analyse: estructura social
"This mediation has to rely on at least some communality – Rawls
speaks of “reasonable” plurality – but the challenge is indeed to manage difference." (p.16)
// Also Altermundis Probleme sind nicht technischer sondern sozialer Natur --> passt zur Schlussfolgerung des Texts

% analyse: estructura social --> superhero
"In many of the “self-organized” sys-
tems that make up Web 2.0, we find that a small group dominates structures of visibility." (p.15)

\end{comment}

\subsection{Ciberoptimismo}
\begin{comment}
% Analyse
  community network projects: are to be found in the context of (2nd wave) cyber optimism: "decentralization", "distributed control", "self-governance", "non-hierarchical organization"
  "the second wave of Inter-
  net enthusiasm was able to transpose key terms such as “decentralization”, “distributed control”,
  “self-governance”, or “non-hierarchical organization”, from the language of countercultural
  community-building into the realm of entrepreneurial cyber-capitalism where scarcity doesn’t
  exist, without them losing their anti-establishment ring and affective value."

## técnica y política

> framing the Internet alternately as lawless, anarchic,
> free, “a world where anyone, anywhere may express his or her beliefs, no matter how singular,
> without fear of being coerced into silence or conformity” (Barlow 1996) (p.1)

"framing the Internet alternately as lawless, anarchic,
free, “a world where anyone, anywhere may express his or her beliefs, no matter how singular,
without fear of being coerced into silence or conformity” (Barlow 1996) or, more recently, as a
space of surveillance, commercial manipulation, and sweeping monopoly."
* la infraestructura está prerequisito para participación
* Community networks operieren in diesem Spannungsfeld

"My first line of critique will therefore try to show that digital networks may very well
produce effects of centralization as much as decentralization, and give rise to new mechanisms of
power that imply new vectors of domination and abuse." (anderer Pol: google & co)
decentralización, pluralidad, democracia --> diese ganzen ciberoptimismo Zeugs in Basics:
Community Networks: die werden idealerweise angestrebt; aber gelingt das?

%ANT:
[Rieder2012]
"Software now habitually provides specific answers to questions that do not seem technical at all: What is communication? What is cooperation? Which information is valuable? What is decision-making?"
"The idea that technology implements values and should be seen as an “actor” in the shaping of
social processes is of course not new; from actor-network theory"
"In many of the “self-organized” sys-
tems that make up Web 2.0, we find that a small group dominates structures of visibility."
Zoom out: big internet providers + social media corporations (Google, Facebook, Cisco, AT&T)
Zoom in to community networks: you'd see above all Altermundi, but not the redes

"The hope for magical technological
solutions to the messy realities is counterproductive if it leads to an attitude that disengages tradi-
tional political process to simply “route around it” ." --> LibreRouter routs around the firmware lockdown quite litereally

[Castells2015]

"Mass self-­
communication is based on horizontal networks of interactive
communication that, by and large, are difficult to control by
governments or corporations."
// die sind nicht horizontal und grad solche netzwerke werden sehr wohl von coroprations und governments kontrolliert. Projekte wie Altermundi wirken dem entgegen.

"communication networks are decisive sources of power-making." // auf jeden Fall! Siehe auch Gene Sharp

\end{comment}

\begin{comment}
  https://globalvoices.org/2017/01/02/librerouter-why-buy-a-router-when-you-can-build-your-own/
  Librerouter
  GV: Why does a community network such as AlterMundi need a LibreRouter?
  GI: Las redes que fomentamos desde AlterMundi están construidas y mantenidas por gente relativamente no técnica. Con lo cual, desde el principio nos concentramos en que tanto la puesta en marcha como el mantenimiento de los nodos sean lo más simple posible. Sin embargo, con el escalamiento de las redes fuimos encontrando complejidades (como por ejemplo la necesidad de montar dos o más routers en ciertas ubicaciones) que complican el entendimiento por parte de la población en general, y por eso veníamos dándole vueltas a la idea desde 2013. El punto de inflexión ocurrió con las mencionadas restricciones de fábrica, que directamente hacen inviable la posibilidad de que gente no técnica transforme un router hogareño (económicamente accesible) en un nodo comunitario, poniendo en peligro la continuidad de las redes en todo el mundo.

%Poder
[Sharp2010]

"The dictators’ access to material resources also directly affects
their power. With control of financial resources, the economic
system, property, natural resources, transportation, and means of
communication in the hands of actual or potential opponents of
the regime, another major source of their power is vulnerable or re-
moved. Strikes, boycotts, and increasing autonomy in the economy,
communications, and transportation will weaken the regime." (s.68)
As previously discussed, the dictators’ ability to threaten

\end{comment}
