\section{Introducción}

\epigraph{``Every translation is a misrepresentation.''}{Ilan Stavans in~\autocite{Albin2005}}
\epigraph{``'Patriarchy' does not mean 'the rule of men'. It means 'the rule of fathers'--literally, the rule of powerful heads of household over everybody else in society. Men further down the social chain were expected to be content with having power over women in order to make up for their lack of control over the rest of their lives.''}{Laurie Penny~\autocite[69]{Penny2014}}

% Motivación
ubicación
por qué es importante/interesante?
	\item Erläutern Sie kurz, in welchem Themenbereich Ihre Arbeit angesiedelt ist. Wo werden Sie einen Beitrag leisten?
	\item Das Ziel sollte es sein, den groben Kontext Ihrer Arbeit darzustellen.

Ciencias de tecnología
Sociología
Redes, conflictios y contrapoder

Redes como infraestructura física -- prerequisito para las herramientas digitales y aplicaciones de redes sociales y las formas de organización que facilitan

Política digital <-- Gabriella Coleman heranziehen

------

``Redes, conflictos y contrapoder'': en este ámbito se ubican muchas investigaciones decisivas/interesantes/fascinantes/substanciales/valiosas

La noción ``Redes'' tiene sentidos multiples:
* redes sociales, en el sentido de grupos de gente y las relaciones entre ellxs
* redes sociales, en el sentido de herramientas y aplicaciones digitales que permiten que la gente se relacione de cierta manera en el ámbito digital
* redes en el sentido de la ANT: un modelo abstracto para observar interacción y redistribución de relaciones/recursos/poder; permiten hacer zoom in and out
* redes como infraestructura física: cables, computadoras, antenas, routers <-- un requisito para las demás capas/la mayoría de los sentidos
** también redes sociales de gente (en un sentido amplio: gente organizada/estructurada de cierta manera) que decide/construye y mantiene la infraestructura física.
--> potencial conflictivo enorme: lxs que/quienes controlan la infraestructura, controlan la comunicación y tienen un poder enorme

% Objetivo
\noindent \emph{Welche Ziele werden mit der Arbeit verfolgt? Welche zentralen Fragen lassen sich daraus ableiten?}
	Die Ziele sollten so spezifisch wie möglich sein. Das hilft Ihnen im Verlauf der Umsetzung zu prüfen, ob Sie Ihre Ziele erreichen konnten. Bitte achten Sie darauf, dass die gesetzten Ziele realistisch sind und das Sie in der Lage sind, das erfolgreiche Erreichen dieser Ziele im Bereich Evaluation zu prüfen.

% Pregunta / hipótesis

concreta para el trabajo

* análisis: ciberoptimismo al nivel físico de las redes/al nivel social de la organización de los grupos

\subsection{Trabajos previos}

Gibt es andere Texte zum Thema Community Networks, die sie aus soz. Perspektive angucken?
keine zeit bzw kenne nichts; evtl Expanding the Internet Commons

\subsection{Estructura de la monografía}

Este trabajo está organizado de la siguiente manera.
Primero, presentaremos el marco teórico que nos ayudará analizar los proyectos okupas en un entorno urbano.
Seguiremos investigando las motivaciones para okupar viviendas y centros sociales,
cuáles proyectos o ideas se realizan (o se intentan realizar) a través de estas okupaciones
y cómo se evalúan los proyectos por las propias personas que los ponen en práctica.
Después, observaremos la recepción de los proyectos por la sociedad en general y la autoevaluación por los propios miembros del movimiento, planteando como las opiniones positivas tanto las negativas.
Pondremos un acento particular en el análisis de la situación jurídica en España.
En conclusión, discutiremos si el movimiento significa un enriquecimiento para la vida urbana y si sus metas no se pueden lograr de otra manera.

