\section{Tu risa, Poema XV und ... in Re-vision}

% Was ist mein Projekt/Meine These?? Ganz klar hier und in der Allgemein-Intro formulieren

% Vlt move Revision-Erklärung zu der Allgemein-Intro?
In Anlehnung an Adrienne Rich möchte ich in diesem Kapitel eine \textit{Re-vision} einiger Liebesgedichte Nerudas exemplarisch durchführen.
Rich definiert \textit{Re-vision} als ``the act of looking back, of seeing with fresh eyes, of entering an old text from a new critical direction''.
Sie unterstreicht, dass solche Revisionen von literarischen Werken essenziell sind, um, unter anderem, mögliche Lebensentwürfe für Frauen neu zu sehen, um sich der (Selbst-)zerstörung in einer Männer-dominierten Gesellschaft zu widersetzen~\cite{Rich1972}.
Dies mag auf dem ersten Blick etwas dramatisch klingen, wenn wir uns allerdings diverse Phanomäne(syn) vor Augen führen, wie Feminizide, Kampf um Recht am eigenen Körper (z.B. Abtreibungsrechte, Reproduktive Selbstbestimmung), Essstörungen, ..., die Frauen jeden Tag unmittelbar erleben, wird es klar, dass es sich keineswegs um eine Überspitzung, sondern immer noch um die triste Realität handelt.

Was wir in Frage stellen und dekonstruieren sollen sind, wie Annette Kolodny anmerkt, gemeinsame kulturelle Annahmen, die so tief verwurzelt sind und schon so lange vorherrschen, dass Kritiker*innen und Leser*innen aufgehört haben, diese als solche wahr zu nehmen und kritisch zu hinterfragen~\cite{Kolodny1980}.
Männerdominanz und Frauenpassivität (etc..??) werden anstatt dessen als natürliche Gegebenheiten gesehen, die durch ihre Darstellung und gar Verherrlichung in literarischen Werken weiter verfestigt werden.

In diesem Kapitel möchte ich gezielt alternative Lektüren der Gedichte ``Poema XV'' (``Me gustas cuando callas\ldots'') und ``Tu risa'' anbieten. 
% alternative - weitere Möglichkeit nach Kolodny; auch, im Sinne von "from a new critical direction" -> welche Direction ist gemeint?
Ich möchte Erkenntnisse aus den Women und Gender Studies (auf Deutsch?) heranziehen, um die etablierten Leseweisen dieser Werke aufzubrechen.
%TODO wenn das so ist, bräuchte ich evtl besagte Erkenntnisse in Kapitel 3
Das Ziel ist, aus der Bequemlichkeit auszubrechen (syn!), die uns vorgefertigte (männer-zentrierte) Interpretationen für kanonisierte Werke anbieten (siehe Kapitel~\ref{chap:canon}).

%TODO nur die Gedichte? Vlt auch "El Tigre"?
Die Gedichte wurden unter anderem deswegen ausgewählt, weil sie aus zwei verschiedenen Gedichtsbändern kommen, die in ihrer Verfassung fast 30 Jahre auseinander liegen.
(``Poema XV'' gehört zu \textit{Veinte poemas de amor y una canción desesperada}, die 1924 veröffentlicht wurden, und ``Tu risa'' zu \textit{Los versos del capitán}, dessen Erstveröffentlichung im Jahr 1952 war.)
Das lässt uns vermuten, dass sich der Stil, Bilder, Anschauung des Autors gewandelt haben und wir auch recht unterschiedliche Liebesgedichte vorfinden würden.
Im Weiteren gilt, die Frauen- (und Männer-)bilder, die durch diese zwei Gedichte evoziiert werden, zu analysieren und zu diskutieren, in wie fern eine Entwicklung dieser stattgefunden hat.

\subsection{Poema XV}
% Veinte poemas: Allgemeiner Eindruck
Die gesamte \textit{Veinte poemas}-Sammlung wird durch melancholische Bilder durchzogen:
Dunkelheit, Dämmerung, Nacht, Herbst, Kälte, Tod, Abwesenheit und damit verbundene? semantische Felder sind häufige Bilder(syn) im Band.
Die weibliche Geliebte ist kaum da und wenn schon ist sie ``callada'', ``muda'', ``lejana''.

Wenn sie überhaupt angeredet wird, adressiert das lyrische Ich sie nicht als gleichberechtigte, vollwertige Person mit eigenen Gefühlen, Wünschen und Gedanken.
Vielmehr werden auf sie die Wünsche und Gefühle des lyrischen Ichs projiziert (vgl limerence?):
``A nadie te pareces desde que yo te amo'', ``Ah déjame recordarte cómo eras entonces, cuando aún no existías'', ``te pareces a mi alma''.
bzw das lyrische Ich richtet sich in seinem Diskurs an einzelne ihrer Körperteile: ``tu cintura de niebla'', ``tus brazos de piedra transparente'' oder an sie als Körper (``Amé desde hace tiempo tu cuerpo de nácar soleado''), nicht als Person. %nácar=Perlmutt
``Muñeca'' ist eine weitere wiederholte Anrede. -> den Mund halten und hübsch aussehen; sich von anderen rumkommandieren lassen
% https://books.google.de/books?hl=bg&lr=&id=Y1eVeyOTuNQC&oi=fnd&pg=PT6&dq=calling+women+dolls&ots=BV9aH9j-Mw&sig=kmypmucRZ1apB1rTIqyvRjv9eWI#v=onepage&q=calling%20women%20dolls&f=false

% Poema XV
Poema XV wird oft als einen Hochpunkt der romantischen Liebe zelebriert (Quelle!).
Dies scheint besonders markant/brisant, weil wenn man die ersten zwei Zeilen liest, da wortwörtlich steht, dass das lyrische Ich es am liebsten mag, wenn die die Geliebte gar nichts sagt und es so ist als ob sie gar nicht da wäre.
Das erinnert fast an den Alte-Männer-Humor, der sich immer von der Ehefrau beschwert.
Man kann sich beim Lesen fast fragen, was am Text von Liebe spricht:
Wir interpretieren den als Liebesgedicht, da es Teil von einem Band ist, der \textit{Veinte poemas de amor} heißt.
Sonst sind die einzigen Zeichen, die uns an Liebe denken lassen, das wiederholte ``me gustas'' und die Erwähnung der eigenen Seele, der Bezeichnung der Geliebten als ``llena del alma mía''.
Beide sind allerdings mit Vorsicht zu genießen:
Das dreifache ``me gustas'', wird jedes Mal eher wieder negiert wird durch das, was folgt: ``cuando callas''.
Die beste Eigenschaft der Frau ist also quantitativ gesehen, dass sie ``como ausente'' ist.
Ihre Bedeutung für die Seele des lyrischen Ichs wird auch durch Wiederholungen unterstrichen: ``Como todas las cosas están llenas de mi alma / emerges de las cosas, llena del alma mía. / Mariposa de sueño, te pareces a mi alma''.
Anstatt allerdings die Liebe zu einer anderen Person zu bekräftigen, zeigen diese Zeilen eher die Liebe zu sich selbst auf.
Die begehrte Frau wird gar nicht wegen ihrer eigenen Eigenschaften geliebt, sondern weil sie besonders gut sich nach den Wüschen des lyrischen Ichs formen lässt und ihn an sich selbst erinnert.
Sie wird von ihm gefüllt, geschöpft, verwirklicht; sie existiert an sich nicht gar nicht.
Wenn die Frau stumm und abwesend ist, ist das lyrische Ich frei, auf sie ihre eigene Vorstellungen und Wünschen zu projizieren, ohne dass ihre Persönlichkeit dem in die Quere kommt (vgl Limerence) und dadurch fühlt sic das Ich wichtig und geliebt. %Limerence

Alle anderen Bilder entstammen den semantischen Feldern silencio -- noche-oscuridad -- sueño.
Diese haben wir durch die Minnessinger? gelernt mit (tragischer) romantischer Liebe zu assoziieren, sie haben per se allerdings kaum was mit Liebe zu tun.
Ein Kernmerkmal dieser Liebe war allerdings immer, dass sie unerreichbar bleibt und sich eher in der Fantasie des Sängers als in der Realität abspielt. %Limerence
Das gezeichnete Bildnis der Geliebten im Gedicht erinnert an ein Gespenst.
Sie ist ``como ausente'', ``parece que los ojos se te hubieren volado'', ``como si hubieras muerto'', ``mi voz no te alcanza''.
``Mariposa de sueño'', ``mariposa en arrullo''.
Die Frau wird aktiv verstummt: ``parece que un beso te cerrara la boca'', wie in diesem schlechten romantischen Filmen-Cliché, in dem Frauen, die aufgebracht sind und ein ernsthaftes Gespräch führen wollen, einfach mal (zunächst gewalttätig) geküsst werden und dann ist alles gut. -> wrong on so many levels.
Der allgemein triste/negative Ton des Gedichts, die Atmosphäre von Unwirklichkeit und Tod werden einzig durch die letzten 2 Zeilen durchbrochen: ``Una palabra entonces, una sonrisa bastan'' für das lyrische Ich.
Das Wort und das Lächeln markieren den einzigen ``hellen'' bzw. ``aktiven'' Moment.
Ich finde allerdings, das Ganze ist schon seit dem 1. Satz nicht mehr zu retten.

Bzw doch: ich finde die Passivität, die Stille, die Abwesenheit der Frau können fast als Praxis der Resistenz ihrerseits betrachtet werden.
  * "y me oyes desde lejos, y mi voz no te toca" - im naechsten teil, eine zeile tiefer: "y me oyes desde lejos, y mi voz no te alcanza"
Diese Leseweise wird auch von Cynthia Duncan angeboten, die das Schweigen als eine Methode der Distanzierung vom lyrischen Ich erkennt, als Strategie des desengagements, als Strategie, ``die Andere'' zu bleiben~\cite{Duncan1992}.

\begin{comment}
  %%% form
  * 4 x 4Line verse: Quartette (lookup wie das korrekt heißt auf Deutsch!) + 2 x 2Zeiler (Couplets) --> erinnert ein bisschen an Sonnetform aber nicht ganz (Wie heißen nochma die verschiedene Sonnetformen? Nicht die von Gongora sondern die von Shakespeare meine ich grad: 3x4Zeilen und dann 2)
\end{comment}

\subsection{Tu risa}
% Tu risa

\begin{comment}
  % Tu risa
  %% images
  * risa
  * pan - aire - luz - primavera  --> womit das Lächeln verglichen wird; Synonyme/symbole fürs Leben, lebenswichtige Attribute
  * rosa - flor --> was lebendiges, schönes, delikates
  * ola de plata
  * lanza -espada --> kamp, krieg, konflikt
  * puertas de la vida --> die Toren des Paradies? (auch cielo) --> der Mann als Gottes gleich, Schöpfer, gibt der Frau erst ihre Identität (das letzte taucht in dem Gedicht vlt nicht direkt auf)
  * opposition: tierra - cielo (die Frau ist unten auf der Erde) --> Hierarchie
  * mi sangre mancha las piedras de la calle --> violencia
  * hora oscura
  * el mar en otoño -- was ist dafür charakteristisch?
    ** mar -- chile
    ** opposition: otoño -- primavera --> das Ende bzw Anfang des Lebens
  * Isotopie(? image field): cascada de espuma - mar
  * patria -- calles torcidas de la isla --> Chile/Isla negra? (seit wann hat er das Haus da?)
  * noche - día
  * noche - luna
  * el muchacho torpe

  %% comparisons/metaphors/... why are they effective? (clear/strong/unusual connection between the 2)
  * risa como espada fresca: strong, giving strength, armas

  %% addressed subject(s)
  * love

  %% larger context of the poem
  * ``Los versos del capitán'': fast 30 Jahre später; Neruda hat eine persönliche Entwicklung durchlaufen
  * vlt sinnlichere/fleischlichere/präsentere Bilder vom lyr. Du; aber ist es wirklich anderes Modell für Weiblichkeit, das hier den Lesenden angeboten wird?
  * immer noch keine aktive, autonom handelnde Frauenfiguren

  %% mode of the poem
  * lyric: associative, vivid language

  %% aus dem Handout/Ideen notizen
    Los versos del capitán
      * Duncan dazu: In dem Gedichtband Los versos del capitán müht sich der männliche Sprecher wieder in ähnlicher Weise mit ihrer Stille ab. Laut Duncan fühlt sich der männliche Sprecher außer Stande ihr Schweigen oder ihre Andersartigkeit zu durchdringen und sieht sich deshalb gezwungen, ihren Körper mit Gewalt zu durchdringen, um seine Kontrolle über sie geltend zu machen, und somit auch ihre Sexualität, welche für ihn eine Bedrohung birgt, abzuwehren.

    * Dies lässt sich besonders deutlich in der Sektion “Deseo“ in Los versos del capitán ablesen.
    * Othering
    * Glorifizierung der Gewalt gegen Frauen

    Tu risa
      Tu risa (Los versos del capitán) -> die Welt bleibt vor der Tür stehen; Das Private ist *nicht* politisch
      das schöne Leben zu hause als Ausgleich für den Scheiß, der auf der Welt passiert

      Laurie Penny's Beziehungskonzept

  %% weitere Gedanken
  %%% form
      * versform: 3-6-8-8-8-14
      --> kein bestimmtes Muster; Anhäufung am Ende, Gradierung zur Kulmination

      * sound repetition
      ** am Anfang *qui*: "quitame"-quitame-quieres-no me quites

      * verbs: viele Imperative (das lyr. Du wird rumkommandiert)

      * 1. Vers: Essenz des Gedichts zusammenfassen
      ** was er aufgeben kann; und was nicht <-- kontrast; überhaupt das ganze Gedicht ist auf Basis von diesen Kontrasten aufgebaut (enjambment?)

      * 2. Vers: recht abstrakt

      * letzter vers enthält ein kontrast: auf was er verzichten kann (lebensnotwendige sachen) und auf was nicht(das lächeln, noch notwendiger anscheinend)
      * condicional "moriria"
      * ist das lyr. Du weiblich? grammatikalisch nicht markiert; jedoch durch weiblich konnotierte Objekte symbolisch repräsentiert (flor, risa); indirekt charakterisiert durch "este torpe muchacho"; steht in einer Tradition (und im Kontext des Gedichtbandes)

      %%%on metaphors
https://www.poetryfoundation.org/resources/learning/articles/detail/68420
The philosopher Ted Cohen suggests that one of the main points of metaphor is “the achievement of intimacy.” Cohen argues in “Metaphor and the Cultivation of Intimacy” that the maker and the appreciator of a metaphor are brought into deeper relationship with one other.

    %%% bilder/symbole/...
      * la lanza que desgranas -- versteh ich nicht
      * Semantische Felder:
        ** agua - ola - cascada de espuma - mar
        ** Körperteile: eigentlich auffällig abwesend im Vergleich zu anderen Gedichten aus dem Band; das einzig präsente ist die "risa", die metonymisch für das lyr. Du steht; die Frau wird vereinfacht auf eine einzige Eigenschaft; sie hat immer zu lächeln und dem Typen nen Ausgleich für die Ungerechtigkeiten der Welt bieten (vgl Laure Penny) (das private/zuhause ausgleich für die soz. Ungerechtigkeit); vlt auch bisschen infantil dargestellt? für nichts anderes außer für lächeln gut
        ** tierra -- patria -- calles torcidas de la isla --> Chile/Isla negra? (seit wann hat er das Haus da?)
        ** cielo - puertas de la vida --> die Toren des Paradies? (auch cielo) --> der Mann als Gottes gleich, Schöpfer, gibt der Frau erst ihre Identität (das letzte taucht in dem Gedicht vlt nicht direkt auf)
        ** Waffen: lanza - espada (dura lucha) <-- Kampf: der Kampf mit dem bösen Leben da draußen (in dem sinne --> die opposition draußen-drinnen, die implizit aufgemacht wird: "al entrar")
        ** flor - rosa - desgrana
        ** sachen, die für leben essenziell sind: aire, pan, agua, luz, primavera, aber auch "tu risa"
      * Oppositionen:
        ** tierra-cielo (die Frau ist unten auf der Erde und guckt nach oben) --> Hierarchie
        ** opposition: otoño -- primavera --> das Ende bzw Anfang des Lebens oder vlt auch stellvertretend für das ganze Jahr stehend (obwohl Sommer und Winter fehlen, kann vlt Frühling/Herbst als einen vollen Zyklus interpretieren)

      * la lucha es dura; los ojos cansados: der harte Typ gegen die harte weite Welt
      * la tierra que no cambia: draußen macht die Welt nicht immer das, was er möchte; das macht ihn fertig
      * er ist der handelnde: kommt von außen, führt da Kriege
      * "amor mío" <-- wieder in relation zu ihm bestimmt (aber hier wohl nicht mega present, nur das eine mal)
      * direkte vergleiche: la risa es como
        ** flor
        ** espada fresca
      * la flor azul, la rosa <-- welche ist die flor azul?? oder wieder eine Verallgemeinerung, die rose ist rot, also rot, blau, alles?
      * patria sonora
        ** direkt danach kommen im nächsten Vers ganz viele "r"-Laute
      * calles torcidas de la isla: isla negra gemeint? (der hat das haus angeblich 1938 gekauft)
      * "tu risa debe alzar/su cascada de espuma" - vgl engl. "bubling laughter": fröhlich,
\end{comment}

% El tigre

\subsection{Soneto ??}
% Soneto ???

\begin{comment}
exemplarische Lektüren: Poetisierung bestimmter Heteronormativen Perspektiven
gehört historisiert; nicht als zeitlos darzustellen


    [Kolodny1980]
"Kennard appropriately cau-
tions us against drawing from her work any simplistically reductive
thesis about the mimetic relations between art and life. Yet her
approach nonetheless suggests that what is important about a fic-
tion is not whether it ends in a death or a marriage, but what the
symbolic demands of that particular conventional ending imply
about the values and beliefs of the world that engendered it"
--> can be used as an argument in the analysis itself?

"The power relations inscribed in the form of conven-
tions within our literary inheritance, these critics argue, reify the
encodings of those same power relations in the culture at large."
--> what is written about adorns the subjects with importance, not the other way round, vgl auch "Little Women"

auch
"these critics also insist upon exam-
ining not only the mirroring of life in art, but also the normative
impact of art on life."

  %Revision. Adrienne Rich
  [Rich1972]
%  Definition:
``Re-vision-the act of looking back, of seeing with fresh eyes, of entering an old
text from a new critical direction-is for us more than a chapter in cultural history:
it is an act of survival. Until we can understand the assumptions in which we are
drenched we cannot know ourselves. And this drive to self-knowledge, for woman,
is more than a search for identity: it is part of her refusal of the self-destructiveness
of male-dominated society. A radical critique of literature, feminist in its impulse,
would take the work first of all as a clue to how we live, how we have been living,
how we have been led to imagine ourselves, how our language has trapped as well
as liberated us; and how we can begin to see-and therefore live-afresh.''
-- das letzte, der Teil mit dem new beginning, Ausblick, vlt in den Fazit integrieren
zum Thema "self-destructiveness" -> schädliche Stereotypen von Passivität, von erstrebenswerter Position als "Begehrungsobjekt", von romantischer Liebe, die der Frau die Agenz absprechen

% myth making and romantic tradition; vgl Laurie Penny
``Jane Harrison, the great classical anthropologist, wrote in 1914 in a letter to her
friend Gilbert Murray:
By the by, about "Women," it has bothered me often-why do women never want
to write poetry about Man as a sex-why is Woman a dream and a terror to man and
not the other way around? \ldots Is it mere convention and propriety, or something
deeper?
I think Jane Harrison's question cuts deep into the myth-making tradition, the
romantic tradition; deep into what women and men have been to each other; and
deep into the psyche of the woman writer.''

"It strikes me that in the work of both [Sylvia Plath and Diane Wakoski] Man appears as, if not a dream,
a fascination and a terror; and that the source of the fascination and the terror is,
simply, Man's power-to dominate, tyrannize, choose, or reject the woman. The
charisma of Man seems to come purely from his power over her and his control
of the world by force"

``One answer to Jane Harrison's question has to be that historically men and
women have played very different parts in each others' lives. Where woman has
been a luxury for man, and has served as the painter's model and the poet's muse,
but also as comforter, nurse, cook, bearer of his seed, secretarial assistant and copy-
ist of manuscripts, man has played a quite different role for the female artist.''

"We seem to be special women here, we have liked
to think of ourselves as special, and we have known that men would tolerate, even
romanticize us as special, as long as our words and actions didn't threaten their
privilege of tolerating or rejecting us according to their ideas of what a special
woman ought to be.
An important insight of the radical women's movement, for
me, has been how divisive and how ultimately destructive is this myth of the
special woman, who is also the token woman."
--vgl auch Tu risa; Laurie Penny

"she meets the image of Woman in books written by men. She
finds a terror and a dream, she finds a beautiful pale face, she finds La Belle Dame
Sans Merci, she finds Juliet or Tess or Salome, but precisely what she does not
find is that absorbed, drudging, puzzled, sometimes inspired creature, herself, who
sits at a desk trying to put words together."
--> what roles are absent?

``Because I was also determined to have a "full" woman's life, I plunged in my early twenties into marriage
and had three children before I was thirty. There was nothing overt in the environ-
ment to warn me: these were the '50's, and in reaction to the earlier wave of fem-
inism, middle-class women were making careers of domestic perfection, working to
send their husbands through professional schools, then retiring to raise large
families. People were moving out to the suburbs, technology was going to be the
answer to everything, even sex; the family was in its glory. Life was extremely
private; women were isolated from each other by the loyalties of marriage. I have
a sense that women didn't talk to each other much in the fifties-not about their
secret emptinesses, their frustrations.''
--vlt benutzen als Erklärungsversuch zu warum Frauen die Gedichte toll fanden?2
--auch Laurie Penny mit was uns vorgegauckelt wird mit dem Mythos der Romantischen Liebe

``But in those earlier years I always felt the conflict as a
failure of love in myself. I had thought I was choosing a full life: the life available
to most men, in which sexuality, work, and parenthood could coexist. But I felt,
at 29, guilt toward the people closest to me, and guilty toward my own being.''
--vgl Laurie Penny at the beginning "We can have it all... The army of nannies and cleaning ladies which this life style requires, can they have it all too??"


``Yet I began to feel that my fragments
and scraps had a common consciousness and a common theme, one which I would
have been very unwilling to put on paper at an earlier time because I had been
taught that poetry should be "universal," which meant, of course, non-female.''
-- universal = non-female

"and egotism-a force directed by men into creation, achievement, ambition, often at the expense of others, but justifiably so.
For
weren't they men, and wasn't that their destiny as womanly love was ours? I know
now that the alternatives are false ones-that the word "love" is itself in need of
re-vision."
-> vgl die Liebe im Sinne von Nerudas lyrischem "ichs"
-> vgl bell hooks "all about love"

"so man will have to learn to gestate and give birth to
his own subjectivity-something he has frequently wanted woman to do for him."
"but women can
no longer be primarily mothers and muses for men: we have our own work cut
out for us."
-- vlt use in Fazit; Männer sollten auch für sich Verantwortung übernehmen und aufhören sich als Frauen beschützende Helden zu inszenieren


[Duncan1992]
"his frequent treatment of her as a fragmented body part, designed for man's pleasure, and by his insistence that his love for her is the only solid core to her being.
Neruda, consciously or not, subordinates the female to the role of man's creation in his poems.
Man is the speaker, the creator of meaning, while woman is the silent, passive listener.
Only by becoming an object of love does the woman come into being.
Without her male lover, she is ``vacía, sin substancia'' (El amor, LVDC)."

"At times, the woman's abscence is even considered preferable, since it allows the male to recreate her in the text, and thus provides him with a heightened level of inspiration."
"Without her love, he may feel lost, but his identity is never in crisis"

"Araya notes that in Veinte poemas, ``la amada es esencialmente carne'' (165), and adds, ``siendo ella objeto pasivo del amor, no manifiesta su identidad ni dentro del libro ni fuera de él''(168).
He also notes, ``Ella no tiene voz propia en este poemario.[...]
Más que un sujeto que actúa y ama, la amada es un objeto al que se dirige la capacidad amatoria del yo'' (168).
Despite these astute observations, however, Araya continues to view the image of woman that emerges from the text as a positive one, for he believes that she is ``sometida a la ley natural'' (182) which conditions her to accept the male speaker as a superior being (159)."

Citing Araya:
``No es difícil entonces que la muchacha y la mujer que leen este libro se identifiquen con la imagen femenina que brota de él''(187)
-> "culturally accepted norm of romantic love"

"As E. Diedre Pribram has observed with reference to film, ``The function of a text is to position the spectator to receive cetain favoured—and restricted—meaning which the text `manages' for the viewing subject in keeping with dominant ideology. In this model the spectator is not an active part of the production of textual meaning but passive side of a unidirectional relationship in which the text disperses meanings while the speactator-subject receives them'' (5).
A similar situation occurs in Neruda's poems, for the reader's gaze is controlled by the ``yo'' who generates the discourse, much as the movie spectator's view is controlled by the seemingly invisible camera eye. Both situations place the reader/spectator in a limiting framework, without appearing to do so, by entrusting a fictional character with the omnipotent and coercive gaze which, in actuality, belongs to the machinery of neunciation.
In Neruda's poems, the male speaker is seen as the voice of authority; his view is rarely, if ever, questioned, since it is presented as the only view possible.
Because he duplicates dominate cultural values in his text, his voice has the ringof truth to it.
But as E. Ann Kaplan has noted, this ``dominating male gaze, carrying with it social, political, and economic as well as sexual power, relegates women to absence, silence, and marginality...'' (5). 
In this light, Neruda's text can be seen as both container and disseminater of ideoloy, for it pormotes the long-standing notion of male superiority and female passivity as natural facts when, in reality, they are merely culturally determined conditions."

"The first poem of the \textit{Veinte poemas} collection, which sets the tone for the whole work, is addressed not to the woman but to her body: ``Cuerpo de mujer''."
"the woman is his possession: ``cuerpo de mujer mía''."
"addresses her as a silent, passive recepient of his male desire"
"The speaker occasionally declares his love for the ``tú'' of the poems but, more often, he speaks of his love for ``tu cuerpo''."
Love for body parts:
``amo tu cuerpo alegre'' (19)
``senos perfumados'' (14)
``grandes ojos fijos'' (20)
``tu color y forma son como yo los quiero''(16)
``mujer de labios dulces''(16)
``carne, carne mía'' (Canción desesperada)

"En \textit{Los versos del capitán} published 28 years after the \textit{Veinte poemas}, we find entire poems dedicated to fragmented parts of the beloved's body: ``Tus pies'', ``Tus manos'', ``Tu risa''.
In each of these poems, the body part acquires importance because it evokes pleasure in the male"

"her purpose in life is to give him pleasure and comfort; she is ``hecha para mis brazos/hecha para mis besos/hecha para mi alma'' (El inconstante)"

"In ``Bella'', the speaker enumerates the parts of the woman's body which are most beautiful to him: ``La sonrisa en tu rostro,'', `finas manos y delgados pies'', ``tu cabeza'', ``tus ojos'', ``tus senos'', ``tu cintura'', ``tus caderas'', ``tu cuerpo'', ``tu voz, tu piel, tus uñas''.
These parts make up a whole which belongs to him: ``todo eso es mío''"

"Although the speaker is generally fascinated by the body of his beloved, at times it holds no special interest for him. On these occasions, her body becomes an empty symbol, signifying nothing"

"in the \textit{Cien sonetos de amor}, the speaker once again takes up the female body as fetish. He calls the woman's body ``mi territorio de besos y volcanes'' and speaks of ``tus pies creados para mí''(V)."

"defines his beloved in the same terms used by the adolescent lover in \textit{Veinte poemas}: ``eres cuerpo''."
-> Kein Fortschritt 30 Jahre später. Gleiche Bilder und Symbole in allen 3 Bändern

From \textit{Cien sonetos}
``quiero comer tu piel como una intacta almendra'' (XI)
``manzana carnal'' (XII)
``pan que devoro'' (XIII)

"Mulvey has observed that ``Woman... stands in patriarchal culture as signifier for the male other, bound by a symbolic order in which man can live out his phantasies and obsessions through linguistic command by imposing them on the silent iage of woman still tied to her place as bearer of meaning, not maker of meaning''."

"What allows her to occupy all of these positions with equal ease is her silence, her lack of a voice. She becomes whatever her lover imagines her to be at the moment"

"He refers to the time before she met him as the time ``cuando aún no existías'' (``La noche en la isla'', LVDC)"

-> Vlt move to Reception
"Emir Rodríguez Monegal has praised this section, specifically the poem ``El tigre'', as being representative of ``el deseo de todos los hombres''.
If this is indeed so, it is a frightening thought, for the poems deal overtly with rape and contain numerous images of violence directed agains the female figure."
-> Rape culture for analysis. 
Rape ist nicht Liebe oder Desire; Es geht um Macht;
Und genau durch solche Aussagen, die solche Vorstellungen normalisieren, anstatt sie als Gewalt zu diffamieren, wird diese weiter aufrechterhalten.

``El tigre''
- male speaker "portray[ed] as a wild animal, lying in the wait for his female victim"
- "spies on the female"
- "Sated after his feast, the tiger/male speaker settles down to guard what is left of the woman (``tus huesos, tu ceniza'') and gloats over what his ``amor asesino'' has accomplished"
-> news flash, das ist definitiv keine Liebe. (vgl vlt Definition von Liebe von bell hooks "all about love")

Vlt check aus "Der Ursprung der Liebe" von Liv Stromquist als potentielle Analyse-Inspiration.

"Las vidas" in \textit{Los versos del capitán}:
"more socially and politically committed male speaker"
"his attitude toward the female beloved has changed only in superficial ways"
"She is not his equal in the political struggle, but merely a campfollower who looks after the needs of er man.
He is still the one in charge, the one who gives orders"
``Limpia ese fusil, camarada'' (``El amor del soldado'')
``Te veo/lavando mis pañuelos,/colgando en la ventana/mis calcetines rotos'' (``No sólo el fuego'')
"basic premise that woman's only contribution to social and political change is to stay at home and wait for her man to return"

"Without him, she is nothing: ``Si me apartas tu vida / morirás / aunque vivas. / Seguirás muerta o sombra / andando sin mí por la tierra''(``El desvío'')."


\end{comment}

\begin{comment}
Ist es möglich, die Gedichte so zu lesen, als wäre das lyr. Ich weiblich?
Male bias in der Produktion und Rezeption von Inhalten (vgl auch "male gaze")
\end{comment}


\begin{comment}

    How-to-close-reading

http://writingcenter.fas.harvard.edu/pages/how-do-close-reading
Copyright 1998, Patricia Kain, for the Writing Center at Harvard University

 When you close read, you observe facts and details about the text. You may focus on a particular passage, or on the text as a whole. Your aim may be to notice all striking features of the text, including rhetorical features, structural elements, cultural references; or, your aim may be to notice only selected features of the text—for instance, oppositions and correspondences, or particular historical references.

 The second step is interpreting your observations. What we're basically talking about here is inductive reasoning: moving from the observation of particular facts and details to a conclusion, or interpretation, based on those observations. And, as with inductive reasoning, close reading requires careful gathering of data (your observations) and careful thinking about what these data add up to.

https://edsitement.neh.gov/blog/2015/01/05/birth-close-reading
Posted January 5, 2015 - 1:43pm | By Joe Phelan

Every word, every line, must be considered and reconsidered, as well as their place in the whole structure

http://web.uvic.ca/~englblog/closereading/
The Close Reading of Poetry
A Practical Introduction and Guide to Explication
 Posted on March 2, 2012
 © G. Kim Blank & Magdalena Kay < > English Department, University of Victoria

There is no single way to do a close reading of a poem. Sometimes an impression is a way in; sometimes the “voice” in the poem stands out; sometimes it is a matter of knowing the genre of the poem; sometimes groupings of key words, phrases, or images seem to be its most striking elements; and sometimes it takes a while to get any impression whatsoever. The goal, however, is constant: you want to come to a deeper understanding of the poem.

  Zu berücksichtigen:
  * figurale, räumliche und zeitliche Deixis
  * bei Lyrik gibts immer eine Metareflexionsebene (also die Form?)
  * Kunst befindet sich auch immer mit anderer Kunst in Dialog
  * Bei Liebeskram: Petrarca! Shakespeare?? (davon hab ich wohl leider keine Ahnung)

  Make a claim about how the poem works/What the poet is doing (your thesis!):
  Overall effect of the poem's craft elements?
  Where does the poem take us (emotionally, intellectually, narratively, etc?) ---> vgl Richards': "help us organize our minds"

\end{comment}
\begin{comment}
% Allgemeine Gedanken. Am Anfang oder ans Ende stellen; oder Zwischendurch als Zwischenschlussvolgerungen; (Aus dem Handout)
4. Exemplarische Re-Vision einzelner Gedichte
   Tu risa, Poema XV
   * bestimmte Frauen- (und Männer-)bilder werden von den Gedichten konstruiert --> einseitig, problematisch, Macht
     ** abgesprochene Agenz; Frauen sind stumm, ohne eigenes Wesen; Ansammlung von (erotisierten) Körperteilen
   * Heteronormativität, Monogamie (2er-Beziehungen) als Teilungsstrategie des Kapitalismus
   * Duncan weist darauf hin, dass Kritiker Nerudas Liebeslyrik oftmals nur vom Standpunkt des männlichen Sprechers im Text aus betrachten, anstatt die Position der Geliebten in Betracht zu ziehen. Daraus resultiert eine einseitige und repetitive Verbreitung aus männlicher Perspektive, welche schon in den Gedichten selbst zum Vorschein kommt. Die weibliche Perspektive wurde bis auf wenige Ausnahmen systematisch ignoriert.
   * Der sexuellen Dialektik, welche sich unter der Oberfläche dieser Gedichte verbirgt, wurde bisher
   kaum Aufmerksamkeit gewidmet. Interessant: Sexismus kommt nicht so sehr in der Bildsprache
   Nerudas vor als in dem Lesen dieser Bilder ("Sexism does not reside so much in the actual images as in
   the reading of those images.") > Warum wird die passive und stille (oder still gehaltene) Frau in einer
   Machtposition gesehen, wenn doch ihre eigene Existenz von der ihres männlichen Liebhabers abhängt?
   Warum wird sie als dem Mann gleichgestellt betrachtet, wenn sie in den Texten als seine Untergebene
   porträtiert wird? Warum sprechen Kritiker von einem Dialog zwischen den Liebhabern, wenn der
   weibliche Teil nie versprachlicht wird (keine Stimme erhält)?"") %% tun die Kritiker*innen das? Wenn ja ist es spannend! Einbeziehen!
   * Duncan erklärt weiterhin, dass unter dem Vorwand der Lobpreisung des Frauseins und dessen Einzigartigkeit, diese Einschätzungen über Nerudas Liebeslyrik textuelle Widersprüche vorenthalten, welche symptomatisch für die Unterdrückung der Frau in einer patriarchalen Kultur seien. Nur dadurch, dass die Frau zu einem Liebesobjekt wird, erwacht sie zu Leben. Ohne ihren männlichen Liebhaber ist sie “vacía, sin substancia“. Diese Darstellung der Frau in den Texten kontrastiert drastisch neben der des männlichen Sprechers, wessen Existenz nicht von der physischen Präsenz oder der weiblichen Liebe abhängig ist. Ohne ihre Liebe mag er sich verloren fühlen, aber seine Identität gerät nie in eine Krise, er sagt, ohne zu zögern, “todos saben quién soy“.
   * Das "Maskuline" wird als universell dargestellt
   * In Nerudas Gedichten hat die Stimme des männlichen Sprechers Autorität, da dessen Standpunkt/Betrachtung selten oder gar nicht in Frage gestellt wird, weil seine Stimme als die einzig mögliche Perspektive dargestellt wird. Referenz zum male gaze (Kaplan, E., Ann). // Universalisierung der männlichen Perspektive
   * auch wenn es schwierig ist, das lyr. Ich und du genau zu erfassen, bzw. als männlich oder weiblich zu kennzeichnen, allein aufgrund der Sprache, haben die Lesenden einen bestimmten Erwartungshorizont (auch durch Einfluss eines biografischen Lesens):
    das lyr. Ich: männlich
    das lyr. Du: weiblich
    -> es werden gewisse Klischee-Bilder für Weiblichkeit und Männlichkeit evoziert; einseitig, der Komplexität der Realität nicht gerecht
   * problematische Frauenbilder: schwach, still, kein gleichwertiges Gegenüber
   * Mann: besitzergreifend, oberflächlich (nur an Körper interessiert)

   * selbst wenn man in die Gedichte den polit. Diskurs der Zeit reininterpretiert und das lyr. Du vlt als die Heimat liest, trotzdem problematisch: warum wird dieser Diskurs mit sexualisierten Bildern weiblichen Körpers ausgetragen?

   %evtl nach 3 Schieben
* Male gaze
  ** das ist auch was die Menschen von der Kunst und Gesellschaft im Allgemeinen kennen:
     dass Kunst aus männlicher Perspektive dargestellt und rezipiert wird; dass Frauen erotisiert werden
  ** In Nerudas Gedichten hat die Stimme des männlichen Sprechers Autorität, da dessen Standpunkt/Betrachtung selten oder gar nicht in Frage gestellt wird, weil seine Stimme als die einzig mögliche Perspektive dargestellt wird.
  ** Filmtheorie: Übereinstimmung von 3 Blicken: Kamera, (männl.) Protagonist und Zuschauer (vgl Laura Mulvey)
     -> Objektifizierung der Frau, Frau als passive Rezipientin sexueller Aufmerksamkeit
     -> Verfestigung patriarchalischen Normen

* "als nah, natürlich wahrgenommen", die Leser*innen können sich damit identifizieren
  ** Menschen können sich direkt mit den Figuren identifizieren, dadurch dass es nicht extradiagetisch ist
  ** Duncan spricht von "real", "natural/beautiful expression of male/female relationships", die Menschen können sich damit identifizieren, sie sehen die Gedichte als Reflexion der gesellschaftlichen Idealen, die sie gelernt haben anzustreben
  ** Duncan:  "a sort of manual on the workings of modern romantic love and have come  to be regarded by many, as a standard agains which real life  relationships can be judged"
  ** Duncan: ".. in fact it seems extraordinarily "real" and "natural", as if it were not a literary construct at all, but, rather a reflection of real-life experiences. Readers who turn to these poems to learn "what love is supposed to be like", "what men are like", and "what women are like", ultimately receive a skewed message told from the traditional dominant male position"

* Letztendlich wechselt die Rolle der Frau in Nerudas Liebeslyrik, wobei all diese Rollen männliche
Fantasien involvieren: Manchmal ist sie die umsorgende, geerdete Mutter zu seiner jungenhaft
verlorenen Person, andere Male ist sie die gefährliche Verführerin, welche ihn seiner Seele zu berauben
droht. Andere Male ist sie die unbewegliche/unantastbare Ikone, ein erstarrtes Bild von Schönheit oder
Sinnlichkeit, und andere Male wiederum ist sie die beschäftigte, kleine Hausfrau. Durch die
Verstummung der weiblichen Stimme kann der männliche Liebhaber die Frau zu allem werden lassen,
wohin ihn seine Vorstellungskraft in diesem Moment treibt. “If she is unique in any way, it is only
because she is his special creation“.
Auch wird Sie als Spiegelung seiner Seele gesehen und ebenso als seine Gefangene, welche sich zu
fügen hat und sich glücklich schätzen darf, unter all den anderen auserwählt worden zu sein. (Bsp.
hierfür p.434 rechts mittig bis unten, diese Denkweise ist bis heute u.a. in lateinamerikanischen
Ländern nicht ganz unüblich.)

\end{comment}

\begin{comment}
  % allgemein
  %% Who is the speaker
  das männliche lyr. Ich; nicht zwangsläufig grammatikalisch markiert
  aber in binärer Opposition zum weibl. lyr Du konstruiert; Klischeebilder;
  durch die Beschreibungen/zugewiesenen Position/evozierten Bilder eindeutig als männlich/weiblich bestimmbar
  unterbewusste Parallele zw. lyr. Ich und Autor gezogen
  in "Tu risa" : "este torpe muchacho que te quiere" = speaker?
  (In "El tigre"/"El condor": der tiger bzw kondor)

  %% mode of the poem:
  * Bella: lyric (associative, vivid language)
  * El tigre/El condor: dramatic lyric (associative, vivid language + tells a story)

  %% poet's diction? (colloquial/formal/elaborate/...)?
  vgl Handout Cortázar: die alltägliche Sprache der einfachen Menschen; Kontrastierend zu der vergeistigten abstrakten europ. Liebeslyrik
  sein Freund Julio Cortázar dazu: “Muy pocos conocían a
  Pablo Neruda, a ese poeta que bruscamente nos devolvía lo nuestro, nos arrancaba a la vaga teoría de
  las amadas y las musas europeas para echarnos en los brazos una mujer inmediata y tangible, para
  enseñarnos que un amor de poeta latinoamericano podía darse y escribirse hic et nunc, con las simples
  palabras del día, con los olores nuestras calles, con la simplicidad del que descubre la belleza sin el
  asentimiento de los grandes helitropos y la divina proporción.“
  http://www.neruda.uchile.cl/critica/jcortazar.html ("Neruda entre nosotros")

  %% aus Duncan1992
     ``Sexism does not reside so much in the actual images as in the \textit{reading} of those imagees:
     why is the passive, silenced woman seen in a position of power when her very existence depends o her male lover?
     why is she viewed as man's equal when she is portrayed in the texts as his subordinate?
     Why do critics speak about a dialogue between the lovers when the woman's part is never voiced?'' <-- generell, nicht nur für Poema XV, aber passt hier; vgl again mit Laurie Penny: wäre eine Stärke, wenn sich jemand dran halten würde

%%%%%%%
  % Bella

  %% addressed subject(s)
  * love
  * weibliche Schönheit
\end{comment}

\begin{comment}
  % Kommentar zu "Tu risa"
* was für romantische Beziehungskonzepte werden in unserer Gesellschaft akzeptiert (Laurie Penny) (die romantische Zweierbeziehung als einziges gesellschaftlich anerkanntes Model + deren Funktion als Ausgleich für strukturell-polit. Probleme/Missstände)
  ** hoher Stellenwert der 2er heterosexuellen Beziehung
  ** für Frau: die ultimate (einzig mögliche?) persönliche Erfüllung?
  ** Duncan: "the position of the woman in the texts is an enviable one"
  ** Duncan: "Neruda's women  possess the qualities we have been taught to covet: they are beautiful,  sensual, desirable, and eminently agreeable.[..] They want what all  women have been conditioned to want in the way of self-fulfillment:  romantic love. They may have no identity, no voice, no sense of purpose,  but, in exchange, they are promised the reward of man's eternal  devotion if they agree to play their role properly. [...] The message to  female readers is clear: woman bears the responsibility of attracting,  nurturing, and keeping man's love alive. He is not obliged to love her  and, as he often reminds her, he can fare better without her than she  can without him. Without love, he is still a man, but she, without love,  is nothing."
  ** Adrienne Rich: "the word "love" in itself is in need of re-vision"
  ** Adrienne Rich: "the myth of the special woman" and its destructiveness: "We seem to be special women here, we have liked to think of ourselves as special, and we have known that men would tolerate, even romanticize us as special, as long as our words and actions didn't threaten their privilege of tolerating or rejecting us according to their ideas of what a special woman ought to be [..] how divisive and ultimately destructive is this myth of the special woman"
  ** Molly Haskell (in Duncan's text): "the idea of woman's inferiority, a lie so deeply ingrained in our social behavior that merely to recognize it is to risk unraveling the entire fabric of civilization"
  ** Laurie Penny: "Mit dem Posen der romantischen Liebe, besonders der heterosexuellen romantischen Liebe in der Ehe, meinen wir uns der bitteren Realität von Arbeit und Tod verweigern zu können - dabei präparieren sie uns genau dafür, verkuppeln uns und stecken uns in lauter kleine Schubladen aus Leid und Leidenschaft: Du und ich gegen den Rest der Welt, Baby. Wir verlieben uns, weil das leichter ist als wenn wir lernen müssten, uns freizuschwimmen."
  ** Laurie Penny: "Man kann Menschen in antagonistischen Paaren zusammenspannen und vor ihrer Umwelt isolieren, damit sie sich gegenseitig die strukturelle Herzlosigkeit dieser Welt in die Schuhe schieben." (S.226)
  ** Laurie Penny: "Man kann die Suche nach einer einfachen Bindung zu einem elenden, ermüdenden RItual machen, das eine strenge Geschlechtskonformität voraussetzt und den menschlichen Geist unterdrückt."
  ** Laurie Penny: "Frauen aller Gesellschaftsschichten wird beigebracht, dass sie sich zuallererst um die Liebe von Männern bemühen müssen, um darüber ihren Wert zu taxieren, Männer auf sich aufmerksam zu machen. Und in allen Gesellschaftsschichten wird die romantische Demütigung dafür genutzt, Frauen klein zu machen. Jeder heterosexuelle Mann, mit dem ich mich je über Partnersuche unterhalten habe, dass Frauen in romantischen Dingen alle Macht haben, auch die ultimative Macht, auf die sexuelle Avancen eines Mannes einzugehen oder sie zurückzuweisen und einen Mann in den Kreis ihrer "Freunde" einzureihen, was dieser als völlig schwachsinnig empfindet, denn natürlich will kein echter Mann für eine Frau nur ein Freund sein. Die Macht, nein zum Sex zu sagen, finden Männer ungeheuerlich und erachten es daher nur als gerecht, dass Frauen und Mädchen alle anderen Spielarten der Macht seit vielen trostlosen Generationen vorenthalten werden. Das hätte etwas für sich, wenn die Macht, nein zum Sex zu sagen, in der Praxis respektiert würde." (S.225)

\end{comment}
\begin{comment}
  % Poema XV  
  %% images
  * ausente - voz de lejos - silencio - callas
  3 semantische Felder: silencio -- noche-oscuridad -- mariposa -- sueno/arrullo
  * un beso te cerrara la boca
  * llena del alma mía
  * mariposa de sueño
  * mariposa en arrullo
  * lámpara
  * anillo
  * la noche callada, constelada
  * estrella

  %% addressed subject(s)
  * love
  * vlt einsamkeit?

  %% larger context of the poem
  * Teil von "Veinte poemas de amor.." (also auch in Zusammenhang mit dem Rest hier zu betrachten; wohl die Passivität und so des lyr. Du ist recht charakteristisch für alle Gedichte)
  * early poetry; young Neruda

  %% mode of the poem
  * lyric: associative, vivid language
  * aber vlt auch elegy? (laments/remembers)

  %% dominant feeling
  * passivität, minorität <-- weiblich konnotierte Eigenschaften

  %% weitere Gedanken
  %%% form
  * 4 x 4Line verse: Quartette (lookup wie das korrekt heißt auf Deutsch!) + 2 x 2Zeiler (Couplets) --> erinnert ein bisschen an Sonnetform aber nicht ganz (Wie heißen nochma die verschiedene Sonnetformen? Nicht die von Gongora sondern die von Shakespeare meine ich grad: 3x4Zeilen und dann 2)
  * 3 sections beginnen mit "Me gustas cuando callas"
  * letzte 2 Zeilen sind Wendepunkt: der einzige "helle"/aktive Moment bisher (palabra, sonrisa)
  * lyr Du ist grammatikalisch als weiblich markiert: "llena", "dolorosa"; "callada", "constelada" (unklar ob das eher die nacht meint, aber ich meine, das komma ist so gesetzt, dass es sich auf das lyr. Du bezieht)
  * wiederholungen, mit leichten variationen "Me gustas cuando callas, porque estás como ausente" - "Me gustas cuando callas, y estás como distante" - "Me gustas cuando callas, porque estás como ausente"
  * "y me oyes desde lejos, y mi voz no te toca" - im naechsten teil, eine zeile tiefer: "y me oyes desde lejos, y mi voz no te alcanza"
  * "como todas las cosas están llenas de mi alma" - "emerges de las cosas, llena del alama mía" - "te pareces a mi alma"
  * "déjame que me calle con el silencio tuyo" - "Déjame que te hable también con tu silencio"

  %%% kontext
  * soll im Kontext des Gedichtsbands betrachtet werden (gab überlegungen zum Roten Faden: Passivität und Abwesenhein des weibl. lyr. Du)

  %%% symbole/bilder again
  * ausente: ihre beste Eigenschaft
  * passiv, hört nur zu; sagt nichts (vgl Resistenzstrategie!)
  * llena de mi alma/te pareces a mi alma: also wird von ihm geschöpft, geformt, verwirklicht; Projektion; existiert an sich nicht
  * sie ist bloß ein Gefäß, das von ihm gefüll wird; dient dazu, dass er sich geliebt, groß, wichtig, .. fühlt
  * distanz; schlafzustand; vlt gar nicht reel alles; und er findet es gut, weil so kann er sich alles mögliche vorstellen, worauf er grad bock hat
  * la noche callada y constelada: weibliche symbole (noche)
  * claro como una lampara                | was können wir damit anfangen?
  * simple como un anillo (el silencio)   |

  %% aus dem Handout:
    Poema 15
      lyr. Du:
      * weiblich ("llena", "dolorosa")
      * stumm, Agens abgesprochen
      * abwesend
      * leblos
      * es werden vom lyr. Ich Eigenschaften reinprojiziert

      lyr. Ich:
      * träumerisch?
      * aber auch über das Du bestimmend

      * Frau in die Opferrolle gesteckt, passiv, Gefäß, welches man füllen kann (=Projektionsfläche)

      * Ihre Stille ist laut Duncan ein zweischneidiges Schwert, da es ihm zum einen erlaubt, seine Fantasie
        auf sie zu projizieren, ihr hingegen ermöglicht es sich von ihm zu distanzieren und “the other“ zu
        bleiben, trotz seinen Versuchen diese Differenz oder Barriere zwischen ihnen zu löschen. Unzählige
        Bsp. Auf S. 435 links mittig mit verschiedenen Ansätzen des lyrischen Ich, sich mit diesem “silencio“
        auseinander zu setzen.
\end{comment}
