\section{Fazit}

\begin{comment}
    * Einleitung und Fazit müssen zusammenpassen.
    * sind die Erkenntnisse im Fazit aus der Arbeit ableitbar?
    * Vorgehensweise zusammenfassen

\end{comment}

Vorgehensweise zusammenfassen

Kritische Zusammenfassung:
\begin{itemize}
    \item was ist mir gelungen und was nicht so?
    \item welche Fragen sind offen geblieben?
    \item in welche Richtung kann noch weiter geforscht werden?
    \item was sind andere Meinungen zum Thema?
\end{itemize}

Konkret:
\begin{itemize}
   \item Wäre cool, um Allgemeinheit bestimmter Tendenzen festzustellen, wenn man verschiedene Zeitschriften analysieren und miteinander vergleichen würde.
   \item Wäre auch cool, die umgekehrte Konstellation anzugucken: Zeitschriften, die auf Englisch erscheinen, benutzen sie und in welchen Situationen Code-Switching zum Spanischen? (ja, in den verwandten Arbeiten haben wir das schon gesehen; jedoch machen sie keine explizit sozioindexikalische Analyse)
\end{itemize}

%Language Shift
Can we talk about language shift in this context?
Who is shifting? The editors of the magazine? Or do they induce language shift?


%Minimal bilingualism
{Andr07}
"Minimal bilingualism
in media discourse often responds to (factual or assumed) limited language
competence on the part of the audience, and exploits the symbolic, rather
than the referential, function of language (cf. sections 3.2, 3.6). This is
sometimes achieved by its use as a framing device (Coupland et al. 2003:
167): tiny amounts of a second language are positioned at the margins of
text and talk units, and thereby evoke social identities and relationships
associated with the minimally used language."


%Kapitalismuskritik
Outro: Kapitalismuskritik?
    \begin{comment}
auch ``The Rise of the Creative Class'' Richard Florida
``They do not consciously think of themselces as a class. Yet they share a common ethos that values creativity, individuality, difference and merit.''
``Everywhere we look, creativity is increasingly valued. Firms and organizations value it for the results that it can produce and individuals value it as a route to self-expression and job satisfaction.''

[Peck05]
he dawn of a ‘new
kind of capitalism based on human creativity’ calls for funky forms of supply-side
intervention, since cities now find themselves in a high-stakes ‘war for talent’, one that
can only be won by developing the kind of ‘people climates’ valued by creatives —
      urban environments that are open, diverse, dynamic and cool (Florida, 2003c: 27). --> (2003c) The new American dream. Washington Monthly March, 26–33.

In the field of urban policy, which has
hardly been cluttered with new and innovative ideas lately, creativity strategies have
quickly become the policies of choice, since they license both a discursively distinctive
and an ostensibly deliverable development agenda. No less significantly, though, they
also work quietly with the grain of extant ‘neoliberal’ development agendas, framed
around interurban competition, gentrification, middle-class consumption and place-marketing — quietly, in the sense that the banal nature of urban creativity strategies in
practice is drowned out by the hyperbolic and overstated character of Florida’s sales
pitch, in which the arrival of the Creative Age takes the form of an unstoppable social
revolution.

      Kapitalismuskritik
      [BolChi07]
      "Finally, capitalist restructuring over the last two decades -which, as we have seen, occurred around financial markets and merger-acquisition activities in a context of favourable government policies as regards taxation, social security and wages - was also accopmanied by significant inventives to greater labour flexibility. Opportunities for hiring on a temporary basis, using a temporary workforce, flexible hours, and a reduction in the costs of layoffs, have developed considerably in all the OECD countries, gradually whittling down the social security systems established during a century of social struggles." (p.xxxviii, Prologue) <-- damit sind die 70er und 80er gemeint
      "This process was widely encouraged by a significant number of the protesters of the era, who were especially sensitive to the themes of the artistic critique - that is to say the everyday oppression and sterilization of each person's creative, unique powers produced by industrial, bourgeois society. The tranformation in working methods was thus effected in large part to respond to their aspirations, and they themselves contributed to it, especially after the left's accession to government in the 1980s. Once again, one cannot fail to stress the fact that critique was effective." (p.199, 1968: The Crisis and Revival of Capitalism)
      "Correlatively, however, at the level of security and wages various gains of the previous period were clawed back - not directly, but via new mechanisms that were much less supervised and protective than the old full-time permanent contact which was the standard norm in the 1960s. Autonomy was exchanged for security, opening the way for a new spirit of capitalism extolling the virtues of mobility and adaptability." (p.199)
      "The displacements operated by capitalism allowed it to escape the constraints that had gradually been constructed in response to the social critique, and were possible without provoking large-scale resistance because they seemed to satisfy the demands issuing from a different critical current." (p.200)
      "What we have observed of the role of critique in the .. also the displacements and transformations, of capitalism .. always conductive to greater social well-being - leads us to underscore the inadequacies of critical activity, as well as the incredible flexibility of the capitalist process. This process is capable of conforming to societies with aspirations that vary greatly over time (but also in space, though that is not our subject), and of recuperatig the ideas of those who were its enemies in a previous phase." (p.200-201)
      "By contrast, it was by opposing a social capitalism planned and supervised by the state - treated as obsolete, cramped and constraining - and leaning on the artistic critique (autonomy and creativity) that the new spirit of capitalism gradually took shape at the end of the crisis of the 1960s and 1970s, and undertook to restore the prestige of capitalism. Turning its back on the social demands that had dominated the first half of the 1970s, the new spirit was receptive to the critiques of the period that denounced the mechanization of the world (post-industrial society against industrial society) - the destruction of forms of life conductive to the fulfilment of specifically human potential and, in particular, creativity - and stressed the intolerable character of the .. of oppression which, without necessarily deriving directly from historical capitalism, had been exploited by capitalis mechanisms for organizing .." (p.201)
      "By adapting these sets of demands to the description of a new, liberated and even libertarian way of making profit - which was also said to allow for realization of the self and its most personal aspirations.." (p.201)
      "By helping to overthrow the conventions bound up with the old domestic world, and also to overcome the inflexibilities of the industrial order - bureaucratic hierarchies and standardized production - the artistic critique opened up an opportunity for capitalism to base itself on new forms of control and commodify new, more individualized and 'authentic' goods." (p.467, The Test of the Artistic Critique)
\end{comment}


Oder auch ganz konkret zum Thema Frauenzeitschriften als ``synthetic sisterhood''

\begin{comment}
\cite{Talbot95}
"how this imaginary community is established: the simulation of a friendly relationship. I conclude with some discussion of how "unsisterly" this feature really is"

"This discursively organized social space called femininity is articulated in commercial and mass-media discourses — especiall in the magazine, clothng, and cosmetics industries."

"Readers are drawn into a kind of complicity with the texts they read."

"In the late 1930s, magazines began to carry consumer features, in which advertising is presented as part of the editorial content. As this brief history suggests, the women's magazine has developed in the context of patriarchal and capitalist social relations."

"Of particular relevance here is women's work on their own bodies as objects to be looked at and their use of commodities, tha is, women's activities as consumers who feminize themselves. Magazines for women contain informative and facilitative elements on fahion and beauty products and their use, which appear both in advertisements and in sonsumer features produced by the editorial board."

"An important element of feminizing practices is the concept of a woman as a visible object requiring work."

"This itemization has been intensified by an endless proliferation of products by manufacturers and accompanying distinctions among colors, sking types, hair types, and so on."

"Magazines are therefore constructed within the relationship among staff, publisher, and manufacturers;"

"According to Angela McRobbie (1978:3; original emphasis), \textit{Jackie} presented ots teenage readers with a "\textit{false} sisterhood" and imposed an ideology of femininity that isolated women from one another; "(1) The girls are being invited to join a close, intimate sorority where secrets can be exchanged and advice given; and (2) they are also being presnted with an ideological bloc of mammoth proportions, one which \textit{imprisons} them in a claustrophobic world of jealousy and competitiveness, the most unsisterly of emotions, to say the least.""

    "McRobbie is deeply critical of these pages for offering consumption as the only way of compensating for inevitable failure in the natural beauty stakes, and their lack of discussion of "\textit{why} women feel ashamed or embarassed by this 'failing'" (40; original emphasis). This kind of neglect is certainly unsisterly, and is probably more harmful than the enticement to competitiveness and mistrust that McRobbie asserts she observes in \textit{Jackie}."

\end{comment}
