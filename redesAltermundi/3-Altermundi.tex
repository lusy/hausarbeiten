\section{El proyecto Altermundi}

\subsection{redes comunitarias}

\begin{comment}
redes convencionales: organizadas de manera jerárquica (Telematik heranziehen?):
topología centralizada: el nodo en el centro tiene control/papel clave: si falla, toda la red se desmantela.

redes mesh: cada nodo está conectado con más de 1 otro: topología más perdurable/estable; todos los nodos son iguales;

redes comunitarias: son en general redes mesh con una pretención política
características:
* infraestructura de comunicación decentralizada, creada y mantenida por la comunidad de lxs usuarixs
* garantizar acceso libre a información
* garantizar la libertad de expresión
* ..
* el firmware: software libre

      \item infraestructura de comunicación abierta, accesible para tod@s
      \item infraestructura creada y mantenida por la comunidad de l@s usuari@s
      \item ejemplos:
        \begin{itemize}
          \item Freifunk (Alemania)
          \item guifi.net (España/Cataluña)
          \item ninux (Italia)
          \item Funkfeuer (Austria)
        \end{itemize}

  \begin{itemize}
    \item conectar a comunidades excluidas por los proveedores convencionales de servicios Internet
    \item garantizar acceso libre a información
    \item garantizar la libertad de expreción
    \item profundizar los propios conocimientos técnicos, experimentar
    \item educar y concienciar a más gente
  \end{itemize}

surgen de necesidades locales:
por ejemplo en el caso de Altermundi: pequeños pueblos, poblaciones en las altas cumbres; localidades donde los proveedores convencionales no ven oportunidad de lucro y por eso no prestan servicio

se ubican en el contexto del ciberoptimismo (DEF)

reflexión crítica: está la solución fiable en gran escala: vgl papel del estado (IV Gui); última milla
\end{comment}

\subsection{Altermundi: fundación, objetivo, estructura, desarrollo histórico}
\begin{comment}
* existe en su forma actual desde 2012
* objetivo: facilitar la construcción y el mantenimiento de redes comunitarias inalámbricas
* grupo básico de activistas + redes locales autónomas
* educar y concienciar: talleres, empoderamiento, el grupo central en papel consultativo
* la comunidad decide
\end{comment}


\subsection{Análisis}
* community network projects: are to be found in the context of (2nd wave) cyber optimism: "decetralization", "distributed control", "self-governance", "non-hierarchical organization"
an attempt at a capillary revolution -> ist es erfolgreich?
\begin{comment}
> framing the Internet alternately as lawless, anarchic,
> free, “a world where anyone, anywhere may express his or her beliefs, no matter how singular,
> without fear of being coerced into silence or conformity” (Barlow 1996) (p.1)

* la infraestructura está prerequisito para participación

## Conclusión del texto [Rieder2012]

> If technology won’t deliver us from the conundrums of
> governance, negotiation, and struggle, we
> may as well reengage politics proper[ly].

* sobre todo problemas sociales/políticos: organización de grupos (no tanto técnicos): ¿cómo decidimos como grupo? ¿quién hace qué? ¿quién está responsable?
* soluciones políticas para problemas políticos
\end{comment}

\subsection{Mapeo: área conflictiva}

\begin{comment}
ANT: podemos hacer aquí zooms in y out!

                          * da talleres
                          * apoyo técnico
                          * Bildungsauftrag

            Altermundi <-----------------------> redes
         /   ^                                     ^
individuos   |                                     |
             |                                     |
             |                                     |
             |                                     |
             v                                     v
            estado<-------------------------> infraestructura

Altermundi:
* individuos, gente que se comunica entre si, con todos los problemas y desafíos que surgen de esto
* tiene una forma organizativa legal: asociación? fundación? (siehe IV) para poder entrar en alianzas, cooperaciones (con el estado y otras organizaciones formales, p.e. la Universidad de Córdoba)
* función auxiliar (las redes son autónomas)

Estado:
* está responsable para regulaciones que ayudan o impiden el trabajo de tales proyectos (ley telemediatica)
* la última milla
* solucionar el problema de conexión de manera completa/de gran escala (siehe papel del estado)
* apoyo financial (eher nicht, vgl Entstehungsgeschichte)

Infraestructura:
* ¿donde está?
* quien la mantiene/gestiona/instala
* quien tiene acceso?

Redes
* autónomas
* tensiones: problemas con la comunicación dentro de la comunidad
* falta de conocimientos técnicos

\end{comment}

\subsection{Cronología}

2012
* El gobierno argentino conversa con codigosur con la idea de ofrecer talleres de software libre/.. por Argentina
* nace también la idea de ofrecer talleres de redes
* Altermundi se forma como proyecto desde codigosur
* el gobierno quita el apoyo financial prometido (el contacto ya no está)
* 1. hackathon (la gente ya se había comprado la técnica)

...
* Delta Libre (1. Hackathon?)
* Quintana Libre
* Nono libre
* ..

2015
* talleres educativos con otros proyectos (cooperativa de mujeres en NIC?)
* Dec: Macri llega al poder en Argentina; cambios en la ley telemedia??


reticulación (vernetzung)/intercambio de experiencias con otros proyectos internacionales
* Rhizomatica
* BattleMesh



2016
* FCC + EU: Firmware lockdown
* Nace el proyecto del Libre router

2017
* Octubre: planificado el lanzamiento del libre router

