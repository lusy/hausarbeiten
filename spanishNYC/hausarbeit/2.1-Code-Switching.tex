%and borrowing

% wann passieren Borrowings
``Powerful socioeconomic and cultural forces stimulate borrowing whenever two cultures are in contact;
the borrowing by the subordinate group's language from that of the dominant group is always
significantly greater'' [Zentella90]
Factors, which facilitate borrowings:[Zentella90]
 1) Most items reflect a cultural reality that is new or different, e.g., \[kei/keike/keiki\]
``cake'', \[bobipín\] ``bobby pin'', \[ŷins/bluŷines\] ``(blue)jeans.''
 ``la boila'' (boiler) [Zentella97]
 2) Others may be facilitated or even triggered by a similarity in the phonological and/or morphological
 structure of a word in the dominant language that makes it sound like a possible subordinate word,
 matre(s) ``mattress'' is similar to that of madre(s) ``mother(s)''
 ZB ``libreria'' (Buchladen) für Bibliothek (library)
 3) reduction in the number of syllables, e.g., \[beis\] replaces florero for ``vase''
 4) prestige
 5) Anglicisms can play the role of neutralizer between competing dialectal variants because the
 prestigious outside language acts as the lingua franca that resolves the conflict without favoring one
 group at the expense of the other.

-> Zentellas Konzept ``linguistic insecurity'': ``consider their dialect inferior to others''[Zentella07]

``but it seems that the longer a group has been in the U.S., the less they use the homeland's term''
``second generation Cubans incorporated more loans in their speech than members of the first generation.''
``Language proficiency appears to be a more relevant factor than age'' [Zentella90]

%Borrowings
lexical borrowing - both form and content are new to the borrowing language [Thomason03]

%%%%%%%%%%%%%%%%%%%%%%%%%%%%%%%%%%%%%%%%%%%%%%%%%%%%%%%%%%%%%%%
[Thomason03]

% was wird entlehnt
When fluent speakers of language A incorporate features into A
from another language, B, the first and most common interference features
will be non-basic lexical items, followed (if contact is sufficiently intense) by
structural features and perhaps also basic vocabulary.

%borrowing vs shift-induced interference
These two types of interference were characterized in Thomason and
Kaufman (1988) as borrowing, in which features are incorporated into A by
native (L1) speakers of A, versus shift-induced interference, in which a group of
L2 learners of A carry over features from B (their L1) into A during a process
of shift from B to A.

--> i would say, we have to do with borrowings

Linguistic factors in linguistic interference:
* universal markedness
* typological distance between source and recipient language

Social factors:
* imperfect learning
* intensity of contact
* speaker attitudes (prestige, ..)

--> weiß nicht in wie fern all das brauchbar ist, weil ich eigentlich der Meinung bin, dass in diesen Artikeln die Sprache bewusst in der Form eingesetzt wird, also dass es sich um ``speakers' attitudes'' und Sozioindexikalität handelt.
Vlt auch ein bisschen intensity of contact..

Features that
are deeply embedded in elaborate interlocking structures are in general less
likely to be borrowed, because they are less likely to fit into the recipient
language’s structures; that is why the lexicon, which for all its structure is less
highly organized than other grammatical subsystems, is borrowed first, and it
is why inflectional morphology tends to be borrowed last. But highly integrated
features may be borrowed readily between systems that are typologically very
similar;

And when contact is intense enough, there appear to be no absolute linguistic
barriers at all to borrowing

%code switching --> USE!
Code-switching, as used in this section, includes both intrasentential switching
(sometimes called code mixing) and intersentential switching


Code-switching is a (perhaps the) major route by which loanwords enter a  %loanwords
language. It surely plays a role in at least some kinds of structural borrowing
as well, although the more dramatic kinds of structural interference are prob-
ably likely to result from code alternation instead (see section 2.2 below).

I believe that is impossible in principle and in practice to draw an
absolute boundary between code-switching and borrowing. % das ist mir auch zu feine Unterscheidung, ich würde damit nichts anfangen

A code-switched word or other morpheme becomes a borrowing if
it is used more and more frequently – with or without phonological adaptation
– until it is a regular part of the recipient language, learned as such by new
learners.

% 2. Mechanisms of Intereference

Mechanisms of contact-induced change fall into four categories.
* one set of mechanisms comes into play when the implementers of a
change are bilingual in both source and recipient language
* while the other set comprises second language
acquisition strategies
* A third category, “negotiation,” seems to
overlap with both of these types
* and the fourth category has to
do with more or less conscious and deliberate decisions by speakers to imple-
ment language change

%borrowing and new word inventions
The addition by borrowing of a new word for a new
concept, like bok choy in English, must begin with a single use and continue
with increasing usage by the innovating speaker(s) and by other speakers, and
the addition by invention of a new word, like photocopy, must follow the same
path.

These parallels are hardly surprising, given that all speakers draw on a
variety of repertoires, typically characterized as styles, registers, and dialects,
in using a single language. If there is no evidence to the contrary – and certainly
no convincing evidence has been presented – then it is surely most reasonable
to assume that speakers whose repertoires include more than one language
will employ the same strategies in deploying their linguistic resources. Change
resulting from code-switching between different languages does, of course,
differ from change via diffusion from another register and dialect borrowing;
but, as noted at the start of this chapter, the differences are a matter of degree,
not of kind.

%Borrowing vs code-switching: matter of frequency
morphemes, both lexical and grammatical, would be introduced directly via
code-switching, changing from code-switches to borrowings through increas-
ingly frequent usage by code-switching speakers and then (if not all members
of the speech community engage in code-switching) by adoption by other
speakers;

%Deliberate change
types of deliberate decisions:
* two-language mixtures created by bilinguals,
apparently to serve as a symbol of a new ethnic identity; (Chicana Spanish, "Kanakendeutsch"? meine Bsp)
* of speaker groups deliberately withholding their
“real” language from outsiders, using instead a distorted and simplified for-
eigner talk version that, in some cases, forms the basis for a trade pidgin
* creation of a secret language, either by phonological distortion [...] or lexical replacement.
* the motive for making a particular change has to do with emphasizing
in-group status, or differentness from other groups;

%code-switching vs shift-induced interference
lexical items predominate in code-switching, while
phonological and syntactic features predominate in shift-induced interference.

%Deliberate change
Theories of language change rarely or never allow for the possibility of deliber-
ate change, except in such trivial cases as the conscious adoption of loanwords
and even new sounds in words of foreign origin
--> vlt spannende Beobachtung, weil evtl die Beispiele aus den Zeitschriften auch genau so was sind;
--> die Frage ist, wo diese am besten reinpasst

%%%%%%%%%%%%%%%%%%%%%%%%%%%%%%%%%%%%%%%%%%%%%%%
