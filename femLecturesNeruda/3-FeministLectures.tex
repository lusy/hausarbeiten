\section{Feministische Lektüren}

% TODO vlt umdrehen und erst das und dann die rezeption???

In diesem Kapitel werden kurz die Werkzeuge erläutert, die den Rahmen bilden, in dem die anschließende Diskussion einer Auswahl Nerudas Liebesgedichte durchgeführt/entfachtet wird.

\subsection{Die Rolle des Kanons}

Einer dieser zentralen Begriffe ist der literarische Kanon.
Der Kanon ist der Konstrukt, mit dem wir uns am ehesten das Phänomen der nicht hinterfragt positiven Rezeption Nerudas Liebeslyrik erklären könnten.
Der Kanon hat eine ganz besondere Rolle:
Er hilft uns, die Kontinuitäten und Brüche, die Einflüsse und Beziehungen zwischen Werken, Autor*innen und Gattungen zu explizieren und zu ordnen~\cite{Kolodny1980}.
Und wir als Menschen versuchen immer Sachen zu ordnen, um damit kognitiv umgehen zu können~\cite{JorRus1999}.
Allerdings sollen wir dieses Modell nicht als auf ``inherenten ästhetischen Eigenschaften'' der Werke basierend wahrnehmen, sondern müssen uns dabei bewusst sein, welche Relationen durch so eine Konstruktion hervorgehoben werden und welche eben verdeckt bleiben, wie kritische feministische Wissenschaftler*innen warnen/hiweisen~\cite{Kolodny1980}:
Wie wird der Kanon überhaupt gebildet?
Welche Werke und Autor(*inn)en (in deren überwiegenden Mehrheit männlich) werden überliefert?
Wer entscheidet das?
Und nach welchen Kriterien?

Die Antworten auf diese Fragen sind intuitiv nahe liegend.
An der Entscheidung sind maßgeblich diejenigen beteiligt, die gesamt gesellschaftlich über Macht verfügen.
Diese Machtbeziehungen spiegeln sich dann auch in Wissensproduktion und Literatur(kritik) wider.
Eine Elite aus westlichen, weißen, oft konservative ältere Männer versucht ihre priviligierte Stellung dadurch zu sichern, indem sie hier eine Hierarchie herstellt, einen Mechanismus, der aufzeigen sollte, was ``gute'' und was ``schlechte'' Literatur ist~\cite{North2013}.
Und dieser funktioniert natürlich aufgrund eines komplett ``objektiven'' Kriteriums: des ``inherenten'' ästhetischen Werts der Werke~\cite{Kolodny1980}.
% Objektivität, die uns immer vorgegauckelt wird bei Wissensproduktion! (Quote??)
% ``Instead, we find ourselves endlessly responding to the riposte that the overwhelmingly male presence among canonical authors was only an accident of history and never intentionally sexist-coupled with claims to the "obvious" aesthetic merit of those canonized texts.''~\cite{Kolodny1980}
Dabei bleibt verdeckt, dass die westliche (und nicht nur) Literatur überwiegend eine androzentrische Perspektive vertritt:
Im Kanon sind vorwiegend Werke von Männern zu finden, in denen Männer als handelnde Figuren im Mittelpunkt stehen, und die wiederum an Männern als Rezipienten gerichtet sind.

% elaborieren, Zusammenhang fehlt grad ein bisschen
Wir lesen gerne und gut das, was wir gelernt haben zu lesen.
Das Vergnügen ist oft ein Resultat der erfolgreichen Anwendung bestimmten interpretativen Schemata und nicht Spaß an der Lektüre per se, vor allem wenn es sich um Texte handelt, die wir sonst schwer erfassen können~\cite{Kolodny1980} (weil sie zu weit von uns liegen, sei es zeitlich, räumlich, ideologisch, etc.).
Wie das Annette Kolodny formuliert:
``Radical breaks are tiring, demanding, uncomfortable, and sometimes wholly beyond our comprehension.''~\cite{Kolodny1980}.
Das ist einer der Gründe warum die hervorgehobene(syn) Stellung von Kanon-Werken sich immer fester etabliert:
Sie werden immer häufiger gelesen und nach den bekannten Mustern interpretiert und diese Popularität/große Anzahl Lektüre wird dann für direkten Indikator für einen hohen literarischen Wert gehalten.
Wenn ein Werk einmal im Kanon angelangt ist, ist es halt im Kanon, und dessen Wert sowie die Art und Weise seiner Rezeption werden kaum mehr hinterfragt~\cite{Kolodny1980}.

% Call to action!
Es gilt also, diesen zeitlosen Kanon, der aufgrund ``intrinsischen ästhetischen Werten'' aufgebaut wurde, zu durchbrechen, etablierte Kriterien und Rezeptionen zu hinterfragen, %<-- vlt auch erst am Ende (zu der Pluralität nehmen)??
sich, in Anlehnung an I.A. Richards und der Cambridge Liberal School, mit Texten zu beschäftigen, um diese in Frage zu stellen, und nicht, um ihre Autorität nochmal zu beteuern~\cite{North2013}.

\begin{comment}
%todo put this somewhere in this chapter
Worüber geschrieben wird und welche Schrifte überliefert werden, prägt auch welche Themen als wichtig angesehen werden.
Was wiederum maßgeblich durch vorherrschende Machtverhältnisse geprägt wird.
\end{comment}


\subsection{Close Reading}

Eins der Werkzeuge, die ich bei der Lektüre von Nerudas Liebeslyrik benutzen werde, ist \textit{close reading}.
Obwohl close reading heutzutage vor allem mit der Schule des New Criticism assoziiert wird, wo es dafür verwendet wurde, die inherenten ästhetischen Werte und ``richtige'' Botschaft von Texten aufzuzeigen und deren Kanonzugehörigkeit zu bestätigen, sind sich auch heute viele Literaturwissenschaftler*innen einig, dass die Methode wertvoll ist, weil sie dafür benutzt werden kann, um ein Werk durch genaues Lesen kritisch zu hinterfragen und verschiedene Lektüren desselben Textes zu generieren~\cite{Gallop2007},~\cite{Beehler1988}.

% alternative Formulierung:
%
%Eins der zentralen Werkzeuge, die ich zum Erschließen von Nerudas Liebeslyrik benutzen werde, ist das Close Reading.
%Bevor wir fortfahren, müssen wir uns kurz mit der historischen Entwicklung dieses Begriffs beschäftigen, denn es ist alles andere als klar und eindeutig, was die Literaturwissenschaftler*innen über die Jahre darunter verstanden haben (und immer noch heute verstehen).
In Anlehnung an der Analyse von Joseph North möchte ich die Close Reading Methode in ihrem ursprünglichen Kontext instrumenalisieren, wie sie vom liberalen britischen Literaturkritiker I.A. Richards konzipiert wurde, eben um ``...'' %TODO Zitat vervollständigen
und nicht ind dem Sinn der (konservativen) US-amerikanischen Strömung des New Criticism~\cite{North2013}.

Es geht dabei nicht darum, nur ein Werk für sich zu betrachten und jeglichen historischen und politischen Kontext außer Acht zu lassen.
% "For him, this means shifting the emphasis away from the supposedly “objective” aesthetic or formal qualities of the work of art considered in isolation, and onto the nature of the relationship between the artwork and its most important context—its audience."(I.A. Richards)\cite{North2013}
% "gives us the very misleading impression that it is somehow, at root, a practice of autonomous or idealist aesthetics, and as such originally or even necessarily dehistoricizing or depoliticizing."
Richards setzt sich (laut North) nicht für ``l'art pour l'art'' ein, sondern möchte mit seinem Close Reading viel mehr evaluieren (syn!), was der Text sagt (syn!), aber in Bezug auf die (Wechsel-)Wirkung(-en) mit den rezipierenden Leser*innen.
Er geht von einem instrumentellen Ästhetik-Begriff aus, von der Annahme, dass Ästhetik ein Mittel und nicht der Zweck ist.
Der Zweck ist, uns zu helfen/den Rezipirenden zu helfen, unsere/deren Weltanschauung zu ordnen.
Die lesende Person (mit ihren Erfahrungen, Wissen, Einstellungen, Weltanschuung, Gemütszustand) ist dabei der Kontext, in dem das Werk analysiert wird.
Und Close Reading ist die Methode, die den Lesenden zu einer Ästhetikbildung und der Verbesserung ihrer Leben verhelfen sollte.
%Wenn wir diese Annahme treffen, ist auch der Gedanke nahe liegend, dass auch multiple Lektüren pro Person möglich sind.
% vlt passt das letzte besser wo anders; sagt auch Kolodgny direkt

Und in dem Sinne, hier vorweggenommen: meine persönlichen Erfahrungen, Wissen, Werte, etc. werden der Kontext für die angebotenen Lektüren von Nerudas Liebeslyrik sein/darstellen.
% I am the context within which the close reading of Neruda's poems will occur (meine Erfahrungen, Wissensbasis, Werte, Gemütszustand..)
% vlt auch besser wo anders; bzw noch genauer elaborieren

%Weitere Merkmale von Close Reading, die uns helfen werden
Vielleicht ist die Definition Nancy Boyles' greifbarer: 
Diese unterstreicht, dass es beim Close Reading darum geht, Bedeutungsschichten aufzudecken, die ein tieferes Verständnis des Textes ermöglichen werden~\cite[90]{Boyles2016}.

Auch die Professorin für Literaturwissenschaft Jane Gallop hebt das Emanzipatorische an dem Close Reading hervor:
Es ermöglicht Studierenden (und nicht nur) nicht nur bereits vorhandenes Wissen anzuwenden, das ihnen von Lehrbüchern und Dozierenden angeboten wurde, sondern viel mehr, in einem direkten Kontakt mit Texten, ihr eigenes Wissen zu produzieren~\cite{Gallop2007}.
Dieser Gedanke ähnelt sehr an dem von North festgehaltenen: ``The `practical' in Practical Criticism had, in Richards’s usage, meant something like `directed towards the practical end of training readers';''~\cite{North2013}.

%Instrumentalisierung für mich muss konkreter werden; oder?

% keine Kanonisierung, keine "good" oder "bad" poetry, sondern in welcher Form tut ein bestimmtes Werk unsere Weltanschauung ordnen (auf Neruda bezogen: androzenztische Vorstellungen von Liebe/der Rolle der Frau in der Geselleschaft/... verfestigen)
% "Practical Criticism in his line: “It is less important to like ‘good’ poetry and dislike ‘bad,’ than to be able to use them both as a means of ordering our minds” (327)."\cite{North2013}


\begin{comment}

  % Close reading
   [North2013]
* New Criticism unterstützt und versucht einen zeitlosen Kanon, aufgrund deren "intrinischen ästhethischen Werten" aufzubauen
* Bsp für Elitismus, eine Elite, die versucht ihre eigene priviligierte Stellung dadurch zu sichern
* Wer gehört dazu?: white, male, christian, property owners
* "It is to be noted that in both these cases the concern is ultimately to en-
sure that valid considerations about texts in themselves are distinguished
from invalid considerations about their effects, psychological or moral,
on their readers." --> die Objektivität, die uns die westliche patriarchale Tradizion der Wissensproduktion uns vorgauckelt
% "For him, this means shifting the emphasis away from the supposedly “objective” aesthetic or formal qualities of the work of art considered in isolation, and onto the nature of the relationship between the artwork and its most important context—its audience."(I.A. Richards)\cite{North2013}
--> ToDo: Need more sources on this one
* quote Brooks: "I believed, to set up a kind of scale: at the bottom, poems that
relied heavily on the principle of exclusion, left out too much of human
experience, and so were thin and over simple. They tended accordingly
toward sentimentality and general vapidity. Toward the top of the scale
were poems that used successfully a high degree of inclusion”" (Cleanth Brooks: “I. A. Richards and Practical Criticism”)
--> also sehr wohl gute von schlechten Werken trennen wollen; nen Kanon errichten
* "prop up a hiearchy of aethetic values"
* "enthusiastic embrace of the idea of a hierarchy"
* "aesthetic value residing solely in the text itself"
* "we have arrived at the sterile concern with hierarchy and canonicity that
will occupy much of Anglophone literary studies throughout the Cold
War period: a series of ultimately unresolvable debates about the exact
constitution of a universal canon, as if one could determine what was
“good” art and what was “bad” art without any reference to what the art
might be good or bad for."
% keine Kanonisierung, keine "good" oder "bad" poetry, sondern in welcher Form tut ein bestimmtes Werk unsere Weltanschauung ordnen (auf Neruda bezogen: androzenztische Vorstellungen von Liebe/der Rolle der Frau in der Geselleschaft/... verfestigen)
% "Practical Criticism in his line: “It is less important to like ‘good’ poetry and dislike ‘bad,’ than to be able to use them both as a means of ordering our minds” (327)."\cite{North2013}
% engage with a text in order to critique it, not in order to assert its authority (vgl auch Kanon)


  [North2013]
* "much of the rest of Principles of Literary Criticism is devoted to trying to show how much of life that relationship involves. Morals and capacities for morals; pleasures and capacities for pleasure; opportunities and capacities for cognition and analysis—the aesthetic, considered in this contextual and instrumental sense, comes to overrun all the borders that Kant erects to divide the faculties."
* "far from trying, in proto-New Critical fashion, to strip works of their contexts in order to encourage a close attention to literary language “for its own sake,” Richards is in fact trying to find the most rigorous and precise way he can to put works of literature into a productive relation with their contexts of reception."
*  "For Richards “close reading” was a way to intervene in the context of reception, which is to say, the minds of actual, living readers."

  https://books.google.de/books?hl=de&lr=&id=Cbz7CwAAQBAJ&oi=fnd&pg=PA89&dq=close+reading&ots=8-Z-fTKNzq&sig=tkwM8Hs0aQs8blONWas-cwGtHpY#v=onepage&q=close%20reading&f=false
  On Developing Readers: Readings from Educational Leadership (EL Essentials) - ed. Marge Scherer; Verlag: ASCD, Alexandira, VA, USA
  9. Closing in On Close Reading - Nancy Boyles
  "Essentially, close reading means reading to uncover layers of meaning that lead to deep comprehension." (p.90)

  [Gallop2007]
"When the New Critics introduced the methodology called close reading
in the years just before and after World War II, what it replaced was liter
ary history (the old historicism, we might call it)."

"I would argue that the most valuable thing English ever had to offer was
the very thing that made us a discipline, that transformed us from cultured
gentlemen into a profession: close reading. Not because it is necessarily the
best way to read literature but because it, learned through practice with lit
erary texts, learned in literature classes, is a widely applicable skill, of value
not just to scholars in other disciplines but to a wide range of students
with many different futures. Students trained in close reading have been
known to apply it to diverse sorts of texts--newspaper articles, textbooks
in other disciplines, political speeches--and thus to discover things they
would not otherwise have noticed. This enhanced, intensified reading can
prove invaluable for many kinds of jobs as well as in their lives."

"For more than three decades, antielitist pedagogy has crystallized
around Paulo Freire's criticism of the banking model, in which the teacher
deposited knowledge in the student. This model remains dominant in
most academic disciplines where there is a huge gap between scholars
producing knowledge and classrooms where students receive, repeat, and
apply that knowledge. The literature classroom has represented a real
alternative to the banking model: students had to encounter the text di
rectly and produce their own knowledge; close reading meant they could
not just apply knowledge produced elsewhere, not just parrot back what
the teacher or textbook had told them"

"I would argue that close reading poses an ongoing
threat to easy, reductive generalization, that it is a method for resisting
and calling into question our inevitable tendency to bring things together
in smug, overarching conclusions. I would argue that close reading may in
fact be the best antidote we have to the timeless and the universal"

[Kolodny1980]
``feminist literary critics
are essentially seeking to discover how aesthetic value is assigned
in the first place, where it resides (in the text or in the reader), and,
most importantly, what validity may really be claimed by our aes-
thetic "judgments." What ends do those judgments serve, the fem-
inist asks; and what conceptions of the world or ideological stances
do they (even if unwittingly) help to perpetuate''
\end{comment}


\subsection{Pluralität der Lektüren}

Mit diesem Text möchte ich Leseweisen von Nerudas Liebeslyrik erkunden, die von der etablierten Rezeption abweichen.
Eine Pluralität der Lektüren einzuräumen ist bereits an sich eine feministische Handlung,
da eine inhärente Charakteristik des vorherschenden patriarchalen Literaturdiskurses ist, die ``richtige'' Interpretation, die ``Essenz'' des Textes zu suchen~\cite{Kolodny1980}.
%  ``we entertain the possibility
%  that different readings, even of the same text, may be differently
%  useful, even illuminating, within different contexts of inquiry.''
%  ``what we give up is simply the arrogance of
%  claiming that our work is either exhaustive or definitive.''
%  ``If feminist criticism calls anything into
%  question, it must be that dog-eared myth of intellectual neutrality.''

Kritische feministische Wissenschaftler*innen hingegen gehen davon aus, dass ein Text keine zugrunde liegende Wahrheit offenbart, sondern vielmehr einen Ausgangsfläche anbietet, aufgrunddessen mit Hilfe verschiedener Methoden und interpretative Strategien eine Mehrzahl an Bedeutungen erzeugt werden können~\cite{Beehler1988}.
Verschiedene Lektüren kommen nicht nur aufgrund unterschiedlicher interpretativen Schemata zustande, sondern auch aufgrund unterschiedlicher Erfahrungen, Gemütszustände und aktuelle Bedürfnisse der Lesenden.
Also ist nicht jede Lektüre von einem Werk, die durch verschiedenen Menschen getätigt wird, anders (ein ``Neu-Schreiben'' wie es Terry Eagleton nennt), sondern auch jedes Mal wenn dieselbe Person einen Text erneut liest, entdeckt sie dadrin potenziell andere Bedeutungen~\cite{Eagleton1997},\cite{Kolodny1980}.

Annette Kolodny warnt jedoch, dass oftmals diese verschiedenen Lektüren gar nicht stattfinden, da sich unsere Lesegewohnheiten verfestigen~\cite{Kolodny1980} und wir bequemerweise ``anerkannte'' Werke lesen und deren Standardinterpretationen übernehmen (was deren ``ästhetischen Wert'' nur noch bestätigt und steigert).
% TODO warum ist das doof?
Das würde zum Beispiel auch erklären warum junge Frauen in Seminaren über Neruda ausschweifig seine Liebeslyrik als ``universell menschlich'' bezeichnen.
Ich frage mich was die selbe Person heute, sechs Jahre später, in dieselben Gedichten lesen würde.
Und wenn sich ihre Leseweise nicht geändert hat, ist das ein Zeichen dafür, dass wir mehr kritische feministische Bildung brauchen.

In Anlehnung an dieser kritischen feministischen Tradition, möchte ich bei den Lektüren folgende Fragen stellen: Wie wird der ästhetische Wert der Werke vergeben? Kommt dieser vom Text oder von den Leser*innen? 
Welchem Zweck dienen diese Urteile?
Und welche Wertesysteme helfen sie aufrechzuerhalten?

Nicht zuletzt, immer wieder die selbe Sorte von Werken auf der selben Art und Weise zu lesen wird auf Dauer recht langweilig.
Rein deswegen ist eine Pluralität der Lektüren erstrebenswert.

\begin{comment}
%TODO überlegen ob ich das als Aufgabe irgendwo explizit aufnehme
feminist criticism very quickly moved
beyond merely "expos[ing] sexism in one work of literature after
another"\cite{Kolodny1980}
--> Tu ich das? Merely exposing sexism in one work of literature? Ist es ok? mach ich vlt sonst was anderes?
\end{comment}
