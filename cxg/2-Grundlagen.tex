\section{Grundlagen}

\subsection{Kognitive Linguistik}

Die kognitive Linguistik ist in die Domäne der Kognitionswissenschaften einzuordnen -- die Disziplinen, die die Wahrnehmung und Verarbeitung von Information untersuchen.
Der Forschungsgegenstand der kognitiven Linguistik ist also Wahrnehmung und Verarbeitung sprachlicher Informationen.

Mehrere Theorien sind unter dem Begriff zusammenzufassen.
Nach \cite{Ziem13} sind deren wichtigsten gemeinsamen Grundannahmen, dass Sprache kein abgekapseltes mentales Modul darstellt,
Sprachwissen gebrauchsgestützt erworben wird (also verfügen Menschen über kein angeborenes abstraktes sprachliches Wissen)
und grammatische Strukturen ``Ausdruck menschlicher Konzeptualisierungsleistungen sind''.
Ferner erklärt \cite{Ellis08}, dass die sprachliche Kognition nicht von der sprachlichen Bedeutung und dem Rest der Kognition getrennt werden kann und soll.
Die kognitive Linguistik lehnt also die Unterscheidung von sprachlichem und Weltwissen ab.
\cite{Ziem08} hält fest, dass spezifisch sprachliche Repräsentationen genau dann entstehen, wenn ``der perzeptuelle Input sprachlicher Art ist''.
Diese sind sowohl individuell als auch interindividuell -- es handelt sich einerseits um individuelle mentale Abbilde, andererseits um das Ergebnis der Kommunikation und Konventionen innerhalb einer Sprachgemeinschaft.
Es werden Abstraktionen aufgrund wiederkehrender Erfahrungen und sprachlicher Muster gebildet.
In \cite{Ellis08}: ``The meaning of the words of a given language, and how they can be used in combination, depends on the perception and categorization of the real world around us.''

\cite{Ellis08} erläutert, dass, im Gegensatz zu der traditionellen Sprachwissenschaft, die kognitive Linguistik keine klare Trennung zwischen den Bereichen der Syntax, Lexikon, Phonologie und Pragmatik unternimmt.

\subsubsection{Die gebrauchsgestützte Perspektive}
Sowohl die generativen Sprachtheorien als auch der Strukturalismus legen ihren Fokus auf das sprachliche System und räumen dem Sprachgebrauch eine nebensächliche, untergeordnete Rolle ein.
Die kognitive Linguistik dagegen geht davon aus, dass der Sprachgebrauch die Struktur von Sprache bestimmt und deswegen genau der Sprachgebrauch untersucht werden sollte, um Erkenntnisse über abstrakte sprachliche Repräsentationen zu gewinnen.
Laut \cite{Bybee06} ``a usage-based theorist would make the more specific proposal that grammar is the cognitive organization of one’s experience with language''.

Die Häufigkeit sprachlicher Muster im Input steuert den Spracherwerb -- der Erwerb einer Grammatik ist also das stückweise Erlernen einzelner Konstruktionen und die Abstraktion von Regelmäßigkeiten mittels distributioneller Analyse (vgl. \cite{Ellis08}, \cite{Bybee06}).
\cite{Bybee06} unterstreicht noch, dass sehr häufige Strukturen einen Prozess der Einschleifung (entrenchment) erfahren, welcher mögliche alternative Formulierungen einschränkt (vgl. dazu auch \cite{Tomasello06}).

%distributionelle Analyse

%\begin{itemize}
%    \item usage based perspective

%        In \cite{Ellis08} usage based theories ``hold that structural regularities of language emerge from learners' lifetime analysis of the distributional characteristics of the language input and, thus, that the knowledge of a speaker/hearer cannot be understood as an innate grammar, but rather as a statistical ensemble of language experiences that changes slightly every time a new utterance is produced.''

 %       ``.. the acquisition of grammar is the piecemeal learning of many thousands of constructions and the frequency-biased abstraction of regularities within them.''

%        ``frequency of patterns in the input affects acquisition''

%        In \cite{Bybee06}: ``high-frequency instances of constructions undergo grammaticization processes''

%        ``the study of grammaticization has played a central role in emphasizing
%        the point that both grammatical meaning and grammatical form come into being through
%        repeated instances of language use''

%        ``use of language is lexically particular;
%        certain words tend to be used in certain collocations or constructions''

%        ``A second effect of token frequency (the
%        CONSERVING EFFECT
%        ) relates to the morpho-
%        syntactic structure of a string. High-frequency sequences become more entrenched in
%        their morphosyntactic structure and resist restructuring on the basis of productive pat-
%        terns that might otherwise occur.  [...]
%        This effect applies to syntactic sequences as well, allowing
%        higher-frequency exemplars to maintain a more conservative structure''

%        ``A theory based on usage, by contrast, which takes grammar
%        to be the cognitive organization of language experience, can refer to general cognitive
%        abilities: the importance of repetition in the entrenchment of neuromotor patterns, the
%        use of similarity in categorization, and the construction of generalizations across similar
%        patterns.''

%\end{itemize}


\subsection{Konstruktionsgrammatik}
Alle konstruktionsgrammatischen Theorien gehören zum Bereich der kognitiven Linguistik.
Der zentrale Begriff der Konstruktionsgrammatik ist die Konstruktion.
\cite{Lakoff87} beschreibt eine Konstruktion als ein Form-Bedeutungs-Paar:
``Each construction will be a form-meaning pair (F,M), where F is a set of conditions on syntactic and phonological form and M is a set of conditions on meaning and use.''
Dabei sollten Form und Bedeutung nicht getrennt betrachtet werden.
Eine Veränderung der Form bewirkt immer eine Bedeutungsänderung.
Sowohl \cite{Lakoff87} als auch \cite{Goldberg95} sehen eine Konstruktion als eine holistische Einheit, deren Bedeutung durch ihre Einzelteile bestimmt wird und deren Form, Bedeutung oder Gebrauch sich aber nicht aus diesen Einzelteilen oder anderen Konstruktionen erschließen lassen.
\cite{Goldberg06} fügt noch hinzu, dass nicht nur unvorhersagbare, sondern auch häufig verwendete ``reguläre'' zusammengesetzte Strukturen (Form-Bedeutungs-Paare) im mentalen Lexikon als Konstruktionen abgespeichert werden.
Durch die gebrauchsbasierte Perspektive und die Annahme, dass alle sprachlichen Einheiten Konstruktionen sind, ermöglicht die Konstruktionsgrammatik die Erklärung aller sprachlichen Phänomene innerhalb des selben theoretischen Rahmens,
anstatt sich nur auf eine ``Kerngrammatik'' zu konzentrieren (vgl. \cite{Tomasello06}).

%``Im Gegensatz zu strukturalistischen Theorien nimmt sie jedoch nicht an, dass semantisch leere Regeln bedeutungstragende Elemente kombinieren und dass die Ableitbarkeit sich aus der Rekonstruktion der Bedeutung der Lexeme und ihrer Art der Verknüpfung sich ableiten lässt.
 %   Sie postuliert, dass sowohl die Bedeutung der Wörter als auch die Bedeutung größerer und abstrakterer Konstruktionseinheiten zur Bedeutung eines sprachlichen Ausdrucks beitragen.''

%    Syntaktische Konstruktionen bestimmen die allgemeine, grundlegende Bedeutung einer Äußerung und nicht die Verben.

%    In \cite{Goldberg95}: `` A construction is posited in the grammar if and only if something about its form, meaning or use is not strictly predictable from other aspects of the grammar, including previously established constructions.''

%    In \cite{Goldberg06}: ``[A]t the same time, unpredictability is not a necessary condition for positing a stored construction.
%    There is evidence from psycholinguistic processing that patterns are also stored if they are sufficiently frequent, even if they are fully regular instances of other constructions and thus predictable.
%    We must recognize that patterns are stored as constructions even when they are fully predictable.''

\subsection{Spracherwerb}
Zum aktuellen Zeitpunkt existieren bereits mehrere Studien, die den frühen Erstspracherwerb aus einer gebrauchsgestützten konstruktionsgrammatischen Perspektive untersuchen.
Der Ausgangspunkt für die Mehrheit davon stellen die Arbeiten des Entwicklungspsychologen Michael Tomasello dar.
Tomasello (vgl. \cite{Tomasello06}) geht davon aus, dass Sprache das Ergebnis der Umsetzung menschlicher kommunikativer Ziele ist
und dementsprechend auch ``frühe kindsprachliche Äußerungen und Konstruktionen [...] eng an unmittelbare kommunikative Ziele des Kindes gebunden [sind].''
Kinder erlernen Sprache inputbasiert in kleinen Schritten.
Menschen sagen ``wiederholt `ähnliche' Dinge in `ähnlichen' Situationen'' \cite{Tomasello06} und die daraus entstehenden sprachlichen Verwendungsmuster werden langsam in Konstruktionen abstrahiert (vgl. auch \cite{Ellis08}).
Während dieses Erwerbsprozesses machen Kinder öfters ``syntaktische Übergeneralisierungsfehler''.%geschickter einbauen
\cite{Tomasello06} nimmt vier Phasen im kindlichen Erstspracherwerb an --
(1) die Holophrasen, die ein ``zusammenhängendes sprachliches Symbol'' darstellen,
(2) Pivot-Schemata -- fixierte Äußerungen mit einer Platzhalterposition, ohne syntaktische Markierungen,
(3) item-gestützte Konstruktionen, die ``an einen spezifisch lexikalischen Inhalt gebunden'' sind und bei denen einige syntaktische Markierungen bereits vorhanden sind
und (4) abstrakte Kategorien.


%\cite{Tomasello06}
%allgemeine kognitieven Fähigkeiten, die den Erwerb sprachlicher Konstruktionen ermöglichen: Intentionszuschreibung (intention reading)/kulturelles Lernen, Schematisierung/Analogie, Beschränkung, distributionelle Analyse

%        Intentionszuschreibung: Menschen haben ein kommunkatives Ziel; Kinder ahmen nach, um ihre Ziele zu erreichen; -- imitierendes Lernen; entspricht den Holophrasen-Phase

%        Schematisierung/Analogie: werden sprachliche Schemata mit einem leeren Slot gebildet;
%        ``eine Analogie nur dann gebildet werden kann, wenn ein gewisses Verständnis vorliegt, welche funktionale Beziehung zwischen den einzelnen Bestandteilen der Strukturen besteht''
%        -- item-gestützte Konstruktionen

%        Generalisierungseinschränkungen (Beschränkung):
%        Enschleifung(entrenchment): ``wenn ein Organismus eine Handlung auf eine bestimmte Art und Weise häufig genug mit Erfolg vollzieht, diese Handlungsweise routinisiert wird, wodurch es für alternative Vorgehensweisen, dasselbe Ziel zu erreichen, sehr schwierig wird, sich durchzusetzen.''


%        Distributionelle Analyse: abstrakte Kategorien bilden --> neue Einheiten, die einer Kategorie zugeordnet werden, werden dann auf derselben Weise benutzt, wie andere Elemente dieser Kategorie, ohne dass konkrete sprachliche Erfahrung mit diesen Einheiten vorliegt

 %       In \cite{Ellis08} ``children are picking up frequent patterns from what they hear around them, and only slowly making more abstract generalizations as the database of related utterances grows.''

 %       ``form-meaning relations are probabilistic; ... the language processing mechanism is unconsciously weighing the likelihoods of all candidate interpretations and choosing among them''

 %       ``Language learning is the associative learning of representations that reflect the probabilities of occurance of form-meaning mappings.''


%        \cite{Eskildsen08} zitiert Tomasello03: ``token frequency is frequency of a concrete
%        expression which ‘in the language learner’s experience tends to entrench
%        that expression in terms of the concrete words and morphemes involved’,
%        whereas type frequency ‘of a class of expressions determines the abstractness
%        or schematicity of the resulting construction’.''

%        \cite{Ellis06}: ``Perception and
%        learning are ‘probabilistic achievements’, but our perception of the world is
%        shaped through the lenses of our prior analyses, beliefs, and preconceptions.''

\subsection{Zweitspracherwerb}
Der Bereich der Zweitspracherwerbsforschung im Rahmen der kognitiven Linguistik ist weniger ausgebaut.
Einiges finden wir in den Arbeiten von Nick Ellis -- \cite{Ellis04}, \cite{Ellis06} und \cite{Ellis08}.

Ellis hebt einige wichtige Unterschiede zwischen dem Erst- und dem Zweitspracherwerb hervor.
Der Erste davon ist, dass, während beim kindlichen Erstspracherwerb sprachliches und Weltwissen gleichzeitig angeeignet werden, Lernende einer Fremdsprache im erwachsenen Alter bereits vorhandenes konzeptuelles Wissen besitzen \cite{Ellis08}.
Zudem verfügen Erwachsene noch über ihre Erstsprache und dieses Wissen steuert zu einem gewissen Grad auch ihren Zweitspracherwerb \cite{Ellis08}.
Erwachsene L2-Lernende erwerben Sprache nicht nur unbewusst (``implizit'', vgl. \cite{Ellis04}), sondern sind auch in der Lage, mittels komplexerer Denkprozesse, Sprache als Objekt expliziten Lernens zu behandeln \cite{Ellis08}, \cite{Ellis04}.

Trotzdem vertritt \cite{Ellis08} die Meinung, dass naturalistischer Zweitspracherwerb ähnlich wie der Erstspracherwerb verläuft --
er fängt mit fixen Äußerungen an und entwickelt sich graduell über Platzhalter beinhaltende Strukturen zu einem Zustand, in dem die Sprecherin über abstrakte Konstruktionen verfügt.
%        ``naturalistic SLA develops in broadly the same fashion as does L1 - from formulae, through low-scope patterns, to constructions''
Die These, dass es auch im L2-Erwerb eine Phase der semi-fixen sprachlichen Strukturen gibt, wird auch durch \cite{Eskildsen08} unterstützt.

Jedoch warnt \cite{Eskildsen08} davor, dass direkte Rückschlüsse vom L2-Sprachgebrauch auf die abstrakten sprachlichen Repräsentationen problematisch sind.
Er unterstreicht, dass detailliertere Forschung zu ``what constitutes a schematic endpoint of L2 learning'' notwendig ist, um diese Repräsentationen besser nachvollziehen zu können.


Wie schon erwähnt, betont \cite{Ellis08}, dass erwachsene Zweitsprachlernende bereits über ihre Erstsprache verfügen.
Der L2-Erwerb ist dann gewissermaßen durch das vorhandene L1-Wissen gesteuert.
Dieser Einfluss kann verschiedene, sowohl positive, als auch negative Auswirkungen haben \cite{Ellis08}.
%Übergang!
\cite{Ellis04} zitiert Schmidts ``noticing hypothesis'' -- um bestimmte sprachliche Strukturen zu erwerben, ``[i]t is necessary that the learner notice the relevant linguistic cues'' \cite{Schmidt90}.
%Zitat Schmidt 1990 - The Role of consciousness in second language learning. Applied Linguistics 11, 129-158
Also muss ein sprachlicher Aspekt erst durch die Lernende wahrgenommen werden, bevor sich bei ihr eine mentale Repräsentation davon bilden kann.
\cite{Ellis04} unterstreicht, dass diese Wahrnehmung auf verschiedenen Arten erschwert oder verhindert werden könnte --
manche Merkmale sind nicht auffällig genug,
manche sind redundant,
manchmal besteht die Schwierigkeit darin, dass diese auf eine andere Weise als in der Erstsprache verarbeitet werden sollten.
In solchen Fällen ist ``explicit instruction'' beim L2-Erwerb hilfreich und notwendig \cite{Ellis04}.

In Fällen, wo es der Lernenden mißlingt, ein sprachliches Merkmal auf die L2-relevante Art und Weise zu verarbeiten, beobachten wir sprachliche Transferphänomene.



%\cite{Ellis06}
%``this effect of prior learning inhibiting
%new learning is called proactive inhibition (PI)''

%Probleme beim Zweitspracherwerb
%        ``fragile features of L2 acquisition'': ``available as a
%        result of frequency, recency, or context, fall short of intake because of one of the
%        factors of contingency, cue competition, salience, interference, overshadowing,
%        blocking, or perceptual learning, which are all shaped by the L1.''

% first language acquisition produces end-state model of language
        %In \cite{Ellis06}: ``If first language is rational in the sense that acquisition produces an
        %end-state model of language that is a proper reflection of input and that
        %optimally prepares speakers for comprehension and production, second
        %language is usually not.''


% die Motivation und Verlauf vom Zweitspracherwerb sind auch ähnlich zu L1
%In \cite{Ellis04}: ``The task of the language learner is to make sense of language.''

%der Sprachgebrauch von der L2 ist nicht mit Wissen gleichzustellen --> komplexe sprachliche Repräsentationen, nicht gut genug erforscht
%``UBL, then, with its proposed item-based path of language learning seems
%promising, in N. Ellis’s words, as a default in guiding investigations into
%longitudinal L2 learning. However, its operationalization of performance as a
%subset of competence (Barlow and Kemmer 2000) is a problematic modus
%operandi for SLA. Use in SLA cannot be thought of as a subset of knowledge
%because we do not know exactly what might constitute an exhaustive
%account, at the most abstract level of schematicity, of the linguistic resources
%of a given L2 learner.''

%usage based: aber eigene Grammatik für L2
%``this paper advocates the future study of L2 linguistic
%development in terms of an empirically grounded locally contextualized
%grammar, consisting of flexibly abstract units of actual language use in
%interaction.''

%``The proposed item-based path for learning an L2 seems to be valid, albeit
%with one major adjustment. Formulas, or MWEs, should be seen as
%interactionally and locally contextualized.''



