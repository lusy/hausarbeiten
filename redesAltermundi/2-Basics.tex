\section{Marco teórico}

\subsection{Commons/bienes comunes y redes comunitarias}

## "community"

> The term “community” has played a central role in American affective politics for a long time,
> and as Cherry Schrecker (2006) argues, this thoroughly positive connotation carries, with ex-
> tremely few exceptions, through most of Anglo-Saxon sociology.

* en alemán: más ambiguo (los nacis y la "Volksgemeinschaft")
* español? ("comunidad", "comunitari@");

\subsection{Actor Network Theory}
\subsection{Ciberoptimismo}

community network projects: are to be found in the context of (2nd wave) cyber optimism: "decentralization", "distributed control", "self-governance", "non-hierarchical organization"

## técnica y política

> framing the Internet alternately as lawless, anarchic,
> free, “a world where anyone, anywhere may express his or her beliefs, no matter how singular,
> without fear of being coerced into silence or conformity” (Barlow 1996) (p.1)

* la infraestructura está prerequisito para participación

\subsection{Instituciones y jerarquías}

## statutory vs capillary power

formal establishments / social mechanisms

> Even if there are no institutions (as formal establishments) regulating behavior,
> there are always institutions in the sense of mechanisms,
> rules, and established dynamics.

