\section{Nerudas Liebeslyrik: etablierte Rezeption}

Die Liebeslyrik von Pablo Neruda wird durch eine (fast absolute) Mehrheit von Lesenden und Literaturkritiker*innen ausschließlich positiv rezipiert.
Prominente Akademiker*innen wie Ilan Stavans sprechen von einem ``poeta bisexual, hermafrodita, poesía erótica, accpetada y leída tanto por hombres como por mujeres'' (quelle??).

\begin{comment}
Stavans zitieren und polemisch einsteigen: "poeta bisexual, hermafrodita, poesía erótica, accpetada y leída tanto por hombres como por mujeres"
hat er das im Seminar bei uns gesagt?
http://america.aljazeera.com/articles/2014/6/20/the-unknown-neruda.html
"Likewise, the inspiration of some of the “Twenty Love Poems and a Song of Despair” (1924), including poems XV (“Me gustas cuando callas porque estás como ausente”) and XX (“Puedo escribir los versos más tristes esta noche”), seem — there is no better way to put it! — eternal."
Kann man machen
etablierte Rezepzionsperspektive
ist weit nicht der Einzige

% Aus Duncan's text? check again!
Belén Lagos schreibt dem weiblichen Teil Macht zu und dem Mann eine Position von “cierta sutil
impotencia“. Der Mann wird laut ihr zu “una figura frágil, anorante, acosada por la presencia infinita
de la mujer“, sobald dieser mit der idealisierten Frau seiner Träume konfrontiert wird.

% Erklärung für die positive Rezeption??
[Kolodny1980]
* die von der Gesellschaft angesehenen Werte werden in unsere Lektüren reinprojiziert.
--> Weiterführend (Laurie Penny) Die Gesellschaft sagt uns, dass romantische Liebe schick und erstrebenswert (vor allem für Frauen) ist, dass sie sich nur darin als Menschen verwirklichen können.
--> Also projizieren Leser*innen (und wirklich auch *innen!) diese Werte in Nerudas Liebeslyrik und feiern sie

* Und anders rum: die Kunst hilft auch diese Normen zu verfestigen/untermauern/bekräftigen
"not only the mirroring of life in art, but also the normative
impact of art on life."
\end{comment}
