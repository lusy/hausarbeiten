\section{Marco teórico}

* nación

* modernidad

* cultura/campo cultural y su rol en la formación de la nación
  * símbolos del Peronismo? (vgl auch marco histórico)
  * cultura de masas; imprenta; todo el mundo puede leer


# cultura

* idea occidental
* cultura vs naturaleza <-- parejas binarias del pensamiento metafísico occidental ~\cite{SzIr2009}{p.71}
* (hombre vs mujer)
* quien define?
* quien es la referencia: el hombre blanco, europeo, heterosexual,... <- privilegiado
* noción evolucionista
* salvajismo, barbaridad
* cultura como instrumento para preservar la hegemonía (vgl la necesidad de aculturar l@s salvajes)
* también como instrumento para romper/desestabiliza la hegemonía (contracultura, cultura de subversión, subcultura)
* dimensión descriptiva/normativa
* set of norms and behaviors which a given society/collective has (implicitely) agreed on

* cultura material/immaterial: "la cultura material ejerce una acción sobre la cultura espiritual"~\cite{SzIr}{p.73} (la predefine?)

* high vs low culture
* cultura de masas vs prácticas de las élites
"El siglo XX le otorga una gran significación a la cultura popular y la cultura de masas, considerándose a ambas como espacios de acción y transformación humanas."~\cite{SzIr2009}{p.72}

"la idea de cultura como un conglomerado de prácticas qie norman y rigen las relaciones entrre los diferentes actores sociales entre sí y con las instituciones" (p.48) ~\cite{SzIr2009}

* viene de "culto": cultivar una práctica; "tanto a una deidad religiosa coo al cuerpo o al espíritu"~\cite{SzIr2009}{p.71}
* dedicación
* "puede ser el resultado o el efecto de cultivar los conocimientos humanos y, también, el conjunto de modos de vida"~\cite{SzIr2009}{p.71}
* "Para la semiótica, la cultura es una red de signos; es un acto comunicativo, un intercambio"~\cite{SzIr2009}{p.71}
* "uno de los problemas fundamentales de la cultura es la denominación"~\cite{SzIr2009}{p.72}

* "el conjunto de creencias y prácticas que constituyen una cultura determinada son susceptibles de ser utilizadas como una tecnología de control, [...] como un conjunto de límites dentro de los cuales la conducta social debe ser contenida"~\cite{IzIr2009}{p.72}

* "Aun cuando la cultura no es reducible a los procesos sociales, no es distinta a ellos. De ahí la circulación en los estudios culturales de términos como identidad, representación, ideología y hegemonía, así como la ideade que la cultura puede asumir una función política específica tanto en la construcción de hegemonías como en su desestabilización."~\cite{SzIr2009}{p.73}

* "diálogo en tensión entre lo local y lo global[..], entre lo rural y lo urbano, entre lo oral y lo letrado, lo nacional y lo regional"~\cite{SzIr2009}{p.73}

# campo cultural

* Bourdieu!
* segun Bourdieu: "un campo es un sistema de posiciones individuales (caracterizadas por el habitus de sus miembros) que se definen por la estructura y la cantidad del capital que se posee" ~\cite{SzIr2009}{p.48}
* un conjunto de límites, creencias y prácticas
* negocia las relaciones entre grupos: "La idea de campo permite estudiar ya no sólo las dinámicas de la cultura como bien de la élite, sino las dinámicas de las diferentes culturas que disputan la hegemonía"~\cite{SzIr2009}{p.48}
* lugar de conflicto y mecanismo de poder
* capital
  * económico - trabajo, tierras, patrimonios
  * cultural - conocimientos, calificaciones <- define las relaciones en el campo cultural
  * social - amigos, redes,
  * simbólico - reputación, prestigio

* cómo se tracen los límites del campo? quién decide?
* actores del campo cultural: "el intelectual" (nótase el artículo masculino), "el artista": quines son? cuál es su papel en la construcción de la nación?
* el poder/privilegio se han de ver en relación, no en vacuum (algunos pueden ser privilegiados en respeto a otr@s y menos privilegiad@s en respeto a un tercer grupo)
* "Las instituciones son la configuración de relaciones entre actores individuales y colectivos."~\cite{SzIr2009}{p.49}

# modernidad

* estrechamente vinculada a la idea del progreso (una idea europea); tiene sus raices en:
  * revolución industrial
  * revolución francesa (ideales democráticos y al final no tan democráticos: "todos son iguales pero algunos son más iguales que otros")
  * romanticismo alemán & humanismo
    * (auch interessant dass der Humanismus mit Sklaverei vereinbar war)
  * positivismo científico
* el proyecto/ideal europeo se sobrepone sobre todos los modelos y todos se miden a través de ello
* pluralismo, contradicciones, utopías
* a partir de la ilustración: una interacción de proces históricos, etc en los últimos 5 siglos
* Eurocentrismo: no debido a procesos "naturales", sino a la administración de colonias: recursos naturales, mano de obra
* ruptura con lo que existió antes
* dualidad: coextistencia de lo tradicional y lo moderno \cite{SzIr2009}{p.177}
* otras oposiciones perdurables: "ritual y racionalidad, mito e historia, comunidad y estado, magia y lo moderno, emoción y razón"~\cite{SzIr2009}{p.177} --> Perón versucht an sie alle zu appellieren?
* modernidades

# sujetos modernos

* sujetos de la modernidad vs sujetos modernos ~\cite{SzIr2009}{p.180}
  * sujetos de la modernidad: actores históric@s, participantes activxs en los procesos de la modernidad; sujetxs *a* estos procesos; sujetxs que moldean de manera activa los procesos;  "comprendido no sólo a las clases medias progresivas occidentalizadas, sino a campesinos, indios y trabajadores que de modo diverso hab articulado los procesos de colonialismo y poscolonialismo"
  * sujeto moderno (masculino, europeo/euroamericano, heterosexual, blanco, ...): imaginado por las discuciones y concepciones cotidianas de la modernidad
  * "Los múltiples sujetos modernos [..] son también sujetos de la modernidad, pero no todos los sujetos de la modernidad son sujetos modernos."
* quiénes identidades están privilegiadas? hombre, blanco, europeo, heterosexual..



# nación

* imaginación de América Latina por parte de l@s europe@s:
  * continente jóven, fuerza, potencia, eine unbeschriebene Seite; recursos naturales...
  * pero no ha cumplido con éxito la modernidad (europea)

* nación: de "nascere"=nacer
* la idea principal de nación: grupos de personas que provienen del mismo lugar; funciona de forma metafórica como una gran familia
* más tarde: un grupo que comparte la misma lengua y cultura~\cite{SzIr2009}{p.189}
  * "miembros compartan una historia, costumbres y tradiciones(comida, música, vestimenta, calendarios de fiestas, etc.), práctias culturales, imaginarios, cánones de literatura, de cine y de artes plásticas, así como valores éticos y morales."~\cite{SzIr2009}{p.190}
   ---> \texit{Evita}: calendario de fiestas al final!
* estado-nación (que no es realidad en ningún lugar)
  * definirse en homogenidad interna
  * y en contraste con lo que está fuera
  * sobre todo una identidad administrativa y política (no solamente cuestión de identidad, cultura...)
  * quién forma parte?: definido en primera linea por las fronteras administrativas

* la nación como una unidad/comunidad:
  * imaginada: vgl Benedict Anderson "comunidades imaginadas"
  * limitada: entidades discretas
  * soberanea
  * oft: "Con el intentode divulgar valores éticos y morales comunes [...] preservar en el Estado-nación una unidad religiosa, lo que ha llevado a procesos de exclusión para los que no son del mismo credo."~\cite{SzIr2009}{p.190}

* el concepto de ciudadanía (sólo para el sujeto masculino heterosexual por supuesto) ~\cite{SzIr2009}{p.189}

* memoria histórica colectiva ~\cite{SzIr2009}{p.191}
* "monumentos, recintos conmemorativos, manuales escolares para la enseñanza de la historia, calendarios de fiesta, ..."~\cite{SzIr2009}{p.191}

* sentido de pertenencia, más allá de los límites geográficos -> movid@s por emociones!!
* confianza en el aparato administrativo/político del estado que garantiza la preservación del estado-nación

"[...] la conformación de los Estados-nación se ha fundamentado en la aparición de un grupo social - por lo general la burguesía- capaz de establecer su hegemonía y de definir un proyecto político de autodeterminación que aglutina a todos los demás secores de la población"~\cite{SzIr2009}(p.190)
-- la burguesía derrocada durante el Peronismo: la base era la clase trabajadora

"Sobre todo las mujeres en tanto madres, primero, y en su papel de maestras, posteriormente, han contribuido a la divulgación de los discursos hegemónicos nacionales y pedagógicos pese a que no fueron los sujetos que los formularan."~\cite{SzIr2009}{p.190-191}
-- Evita hier einordnen "madre espiritual de la nación"

"Por otra parte, al relatar historias de amor y presentar parejas heterosexuales ideales en las novelas fundacionales del romanticismo, se evocó la unidad nacional más allá de diferencias raciales y sociales."~\cite{SzIr2009}{p.192-193}

"Esta falta de identificación y el sentimiento de no ser representados por los políticos electos debilita y retrasa, hasta la fecha, los procesos de democratización en la mayoría de los estados latinoamericanos y tiene como consecuencia sistemas populistas y autoritarios."~\cite{SzIr2009}{p.194}

# los estados americanos

* no hay diferencia idiomática distintiva (hay en las lenguas nativas??)
* zonas económicamente y geográficamente separadas -> falta de infraestructura
* unidades administrativas arbitrarias (origen: conquistas militares)
* relativamente independentes
* caudillismo/localismo
* ambivalencia constitutiva:
  * independizarse delcentro que nos ha socializado
  * pero sin recuperar las historias y desarrollos locales (con excepción de algunas simbiosis)
* "ficciones orientadoras"

