\section{Nerudas Liebeslyrik: etablierte Rezeption}

% TODO Vlt parallel mit den alternativen Lektüren besprechen
%hier nur zussammenfassen, was andere gesagt haben! Damit diskutieren in 4?

Meine erste umfangreichere Begegnung mit Nerudas Werk fand im Hauptseminar zu Pablo Neruda im Sommersemester 2016 statt, im Rahmen dessen die vorliegende Arbeit entsteht.
In der Sitzung, die sich der Liebeslyrik widmete, sprach eine Komilitonin von ``universellem menschlichem fleischlichem Begehren''.
Unter den mehr als 30 Seminarteilnehmenden scheinte niemand mit dieser Aussage ein Problem zu haben/ und alle scheinten mit diesem Statement einverstanden, bis auf mich, die Profesorin und 2-3 Komilitoninnen, die etwas älter als das Durchschnittsalter von 20 waren.

Die Liebeslyrik von Pablo Neruda wird durch eine (fast absolute) Mehrheit von Lesenden und Literaturkritiker*innen ausschließlich positiv rezipiert.
Prominente Akademiker*innen wie Ilan Stavans sprechen von einem ``poeta bisexual, hermafrodita, poesía erótica, acceptada y leída tanto por hombres como por mujeres'' (quelle??)
und nennt die Poesie ``eternal''\cite{Stavans2014}.

\begin{comment}
    [Kolodny1980]
"For what we are
asking be scrutinized are nothing less than shared cultural assump-
tions so deeply rooted and so long ingrained that, for the most
part, our critical colleagues have ceased to recognize them as such."
-> weswegen Nerudas Liebesgedichte als "universell" gelesen werden

"What makes it so exciting, of course, is
that it can be constantly relearned and refined, so as to provide
either an individual or an entire reading community, over time,
with infinite variations of the same text. It can provide that, but,
I must add, too often it does not"
Quote Komilitonin mit: "universell menschliches fleischliches Begehren" -> und argue that we need critical feminist thinking education in order to shift perspectives in such readings.
Fast der ganze Kurs mit Ausnahme der Professorin und ein paar Studentinnen, die über das Durschnittsalter von 22 waren, scheinten mit der Leseweise kein Problem zu haben.
Ich frage mich was die selbe Person heute darüber sagen würde, wenn sie die Gedichte nochmal liest.

---

Stavans zitieren und polemisch einsteigen: "poeta bisexual, hermafrodita, poesía erótica, accpetada y leída tanto por hombres como por mujeres"
hat er das im Seminar bei uns gesagt?
http://america.aljazeera.com/articles/2014/6/20/the-unknown-neruda.html
"Likewise, the inspiration of some of the “Twenty Love Poems and a Song of Despair” (1924), including poems XV (“Me gustas cuando callas porque estás como ausente”) and XX (“Puedo escribir los versos más tristes esta noche”), seem — there is no better way to put it! — eternal."
Kann man machen
etablierte Rezepzionsperspektive
ist weit nicht der Einzige

%Reception
viele reden von "shared, symmetrical experience", wenn es in Wirklichkeit alles andere ist als das
``Antonio González Montes, for example, observes that in Neruda's work, "el amor es una experiencia fundamental, profunda y total, compromete y envuelve de modo absoluto y absorbente a los seres que lo comparten" (25).
He further notes that "el poeta se dirige en tono dialógico y apasionado a ella para enaltecer las diversas facetas y virtudes de su condición femenina excepcional y construye así una imagen casi sublime y notablemente idealizada de la mujer que comparte su amor"(25).''
// teilt sie das wirklich? In der Realität ist ihre Perspektive völlig unsichtbar und außer Acht gelassen
// interessant wo das "dialogische" herkommt, wenn sie nie was sagt

%TODO: check these people!
``Others who have treated Neruda's love poetry from the male perspective include: Aguirre, Alazraki, Alonso, Durán, Pérez, and Pimentel. While I do not disparage the work of these or other critics of Neruda's poetry, I feel that they have contributed to the notion that the texts are "natural" reflections of male/female relationships and that the position of the woman in the texts is an enviable one. To my knowledge, the only critics to date who have acknowledged that the asence of a feminine perspective in Neruda's poetry is problematic are John Felstiner and Marjorie Agosin.''

``Guillermo Araya speaks of "la mujer homenajeada" (179), and points out that she occupies a priviledged space in the text because she is the object of the male speaker's unrelenting, adoring gaze (151).
//vgl Laurie Penny --> das wäre Macht, wenn sich jemand dran halten würde

% Aus Duncan's text? check again! -- jo!
Belén Lagos schreibt dem weiblichen Teil Macht zu und dem Mann eine Position von “cierta sutil
impotencia“. Der Mann wird laut ihr zu “una figura frágil, anorante, acosada por la presencia infinita
de la mujer“, sobald dieser mit der idealisierten Frau seiner Träume konfrontiert wird.

% Erklärung für die positive Rezeption??
[Kolodny1980]
* die von der Gesellschaft angesehenen Werte werden in unsere Lektüren reinprojiziert.
--> Weiterführend (Laurie Penny) Die Gesellschaft sagt uns, dass romantische Liebe schick und erstrebenswert (vor allem für Frauen) ist, dass sie sich nur darin als Menschen verwirklichen können.
--> Also projizieren Leser*innen (und wirklich auch *innen!) diese Werte in Nerudas Liebeslyrik und feiern sie

* Und anders rum: die Kunst hilft auch diese Normen zu verfestigen/untermauern/bekräftigen
"not only the mirroring of life in art, but also the normative
impact of art on life."

[Duncan1992]
``Keith Ellis calls \textit{Veinte poemas}, with its more than two million copies in print, `tal vez el libro de poesía más popular del mundo' (11), and René de Costa states that the influence of these poems `on subsequent love poetry in Spanish has been extraordinary' (35).''

//vlt auch in 4 in combi mit Laurie Penny
``Critics, like the millions of readers who have been drawn to these poems, gnerally tend to read them in a positive light, viewing them as a natural and beautiful expression of male/female love relationships.
In a sense, the three thematically-linked works have become, over the years, a sort of manual on the workings of modern romantic love and have come to be regarded, by many, as a standard agains which real-life relationships can be judged.
Specifically, Neruda is lauded for the adept way in which he captures the intensity of emotion between the lovers: critics unfailingly speak of love in these poems as a shared experience, one which impacts on the male and female with equal force, and constitutes in equal measure the essence of their being.''

``Neruda's idealized treatment of woman as earth goddess or icon of desire has also received considerable praise, and has won him a reputation as a poet who is sensitive to the workings of the female mind, heart, and body.''
// aber keine Behandlung der Frau als Mensch mit Bedürfnissen, Ängsten, Fehlern, ...

``the critics have tended to view love from the position of the male speaker in the texts rather than from the position of the female beloved''
``the female perspective has, with rare exception, been systematically ignored''

%Duncan's project, ist im Grunde auch was ich mache:
``There is no doubt that Neruda is a gifted poet; whether he is guilty of sexism as a writer is not a point I wish to debate here.
But, I do take issue with the critical readings of Neruda which have drawn attention to the repression, subjugation, and silencing of women in his poetry only to dismiss these factors as natural conditions which are in and of themselves praiseworthy.''

\end{comment}
