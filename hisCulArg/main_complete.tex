%%%%%%%%%%%%%%%%%%%%%%%%%%%%%%%%%%%%%%%%%
% Arsclassica Article
% LaTeX Template
% Version 1.1 (10/6/14)
%
% This template has been downloaded from:
% http://www.LaTeXTemplates.com
%
% Original author:
% Lorenzo Pantieri (http://www.lorenzopantieri.net) with extensive modifications by:
% Vel (vel@latextemplates.com)
%
% License:
% CC BY-NC-SA 3.0 (http://creativecommons.org/licenses/by-nc-sa/3.0/)
%
%%%%%%%%%%%%%%%%%%%%%%%%%%%%%%%%%%%%%%%%%

%----------------------------------------------------------------------------------------
%	PACKAGES AND OTHER DOCUMENT CONFIGURATIONS
%----------------------------------------------------------------------------------------

\documentclass[
10pt, % Main document font size
a4paper, % Paper type, use 'letterpaper' for US Letter paper
oneside, % One page layout (no page indentation)
%twoside, % Two page layout (page indentation for binding and different headers)
headinclude,footinclude, % Extra spacing for the header and footer
%BCOR5mm, % Binding correction
]{scrartcl}

\input{structure_alt.tex} % Include the structure.tex file which specified the document structure and layout

\hyphenation{Fortran hy-phen-ation} % Specify custom hyphenation points in words with dashes where you would like hyphenation to occur, or alternatively, don't put any dashes in a word to stop hyphenation altogether

%----------------------------------------------------------------------------------------
%	TITLE AND AUTHOR(S)
%----------------------------------------------------------------------------------------

\title{\textit{Evita. Libro de lectura para primer grado inferior}: un análisis desde las ciencias culturales}
%\title{\normalfont\spacedallcaps{Article Title}} % The article title

\author{%
\spacedlowsmallcaps{Lusy}\\
\normalsize{PS: Historia Cultural de Argentina}\\
\normalsize{Wintersemester 2016/2017}\\
\normalsize{Seminarleiterin: Vanessa Gómez Pereira}
}

%\date{} % An optional date to appear under the author(s)
\date{\today}

%----------------------------------------------------------------------------------------

\begin{document}

%----------------------------------------------------------------------------------------
%	HEADERS
%----------------------------------------------------------------------------------------

\renewcommand{\sectionmark}[1]{\markright{\spacedlowsmallcaps{#1}}} % The header for all pages (oneside) or for even pages (twoside)
%\renewcommand{\subsectionmark}[1]{\markright{\thesubsection~#1}} % Uncomment when using the twoside option - this modifies the header on odd pages
\lehead{\mbox{\llap{\small\thepage\kern1em\color{halfgray} \vline}\color{halfgray}\hspace{0.5em}\rightmark\hfil}} % The header style

\pagestyle{scrheadings} % Enable the headers specified in this block

%----------------------------------------------------------------------------------------
%	TABLE OF CONTENTS & LISTS OF FIGURES AND TABLES
%----------------------------------------------------------------------------------------

\maketitle % Print the title/author/date block

\setcounter{tocdepth}{2} % Set the depth of the table of contents to show sections and subsections only

\tableofcontents % Print the table of contents

%\listoffigures % Print the list of figures

%\listoftables % Print the list of tables


%----------------------------------------------------------------------------------------
%	AUTHOR AFFILIATIONS
%----------------------------------------------------------------------------------------

%{\let\thefootnote\relax\footnotetext{* \textit{Department of Biology, University of Examples, London, United Kingdom}}}

%{\let\thefootnote\relax\footnotetext{\textsuperscript{1} \textit{Department of Chemistry, University of Examples, London, United Kingdom}}}

%----------------------------------------------------------------------------------------

\newpage % Start the article content on the second page, remove this if you have a longer abstract that goes onto the second page

%----------------------------------------------------------------------------------------
%	MAIN PART
%----------------------------------------------------------------------------------------

%1-Intro--------------------------------------------------------------------------------
\section{Introducción}
El peronismo es uno de los movimientos políticos más discutidos de Argentina, Latinoamérica y el mundo entero.
Se trata de un fenómeno complejo al análisis del cual se han dedicado muchas disciplinas: desde la historia y las ciencias políticas hasta la antropología y las ciencias culturales.
Abordar todas las discuciones sobre este asunto no es posible ni es el objetivo de este texto.

% Was ist mein Ziel: Nación o Evita?
La intención del trabajo actual es más bien analizar el tropo de la nación imaginado por el peronismo reflejado en el libro de lectura para primer grado interior \textit{Evita} del editorial Luis Lassere, editado por Graciela Albornoz de Videla en año 1952\footnote{Una versión escaneada del libro se puede encontrar en~\url{http://www.taringa.net/posts/imagenes/15396831/Evita-el-libro-de-primer-grado-completo.html}}.

Se trata de un libro que facilita la enseñanza y aprendizaje de lectura y escritura dirigido a niñxs de la escuela primaria del periodo.
Como ya el título señala, el libro está vinculado muy estrechamente con los ideales y las imágenes del regimen político en vigor.
Aunqué se dirije a lxs más jóvenes, \textit{Evita} no falla de abordar los temas y símbolos claves del peronismo, apoyando la idea de un aparato de propaganda estatal extenso.
\textit{Evita} forma parte del proyecto de crear un mito homogeneo de la nación argentina
y de consolidar la autoridad de lxs líderes peronistas: Juan y Eva Perón.

Vamos a discutir este libro a través de algunas nociones claves de las ciencias culturales como ``cultura'', ``modernidad'' y ``nación''.

A continuación, este trabajo está organizado de la manera siguiente.
Primero, investigaremos el periodo histórico en el cual se ubica dicha publicación.
Después, presentaremos el marco teórico que nos ayudará analizar los rasgos claves de \textit{Evita}.
Seguiremos con el análisis del libro infantil.
En conclusión, resumiremos otra vez los resultados y plantaremos algunas ideas para una discusión más profunda.

%2-Background----------------------------------------------------------------------------
\section{Marco histórico}

En este capítulo trataremos a dar una descripción del periodo histórico que nos ayudará a discutir el libro \textit{Evita}.
Como hemos señalado en la introucción, el libro fue publicado el año 1952.

Al nivel mundial opera en este tiempo la lógica de la Guerra Fría.
La mayoría de los países del mundo pertenecen a uno de los dos campos - el bloque capitalista de los Estados Unidos o el comunista de la Unión Soviética.
Es difícil resumir un fenómeno tan complejo como la Guerra Fría en dos líneas, pero podemos afirmas que el mundo está completamente dividido y en plan mudnial predomina una atmosfera política de miedo y rechazo total de la ideología adversa.

En este contexto llama la atención que el gorbierno argentino se niega a adoptar una de estas doctrinas.
El peronismo se (auto?)denomina ``la tercera posición''.

En Argentina tenemos en este tiempo la segunda presidencia de Perón.
Aquí destacaremos los elementos claves del peronismo.
El movimiento, en su intensión de reunir más seguidorxs, se basa en un discurso de clase.

%3-Basics--------------------------------------------------------------------------------
\section{Marco teórico}

En esta parte resumiremos muy brevemente algunas de las nociones teóricas que utilizaremos para la discución de \textit{Evita}.

\subsection{Cultura}

El concepto ``cultura'' y lo que relacionamos con ello es históricamente (sobre todo) una idea evolucionista occidental.
Se trata de la noción de superar el ``salvajismo'', la naturaleza, de la superioridad imaginada de los seres humanos y lo creado por ellxs.
``Cultura'' viene de ``culto'': se trata(syn!) de cultivar una práctica, de la dedicación ``tanto a una deidad religiosa como al cuerpo o al espíritu''~\autocite[71]{SzIr2009}.

Hoy nos referimos (sobre todo) a un conjunto de normas y comportamientos acordado (más a menudo de manera implicita) por un colectivo concreto~\autocite[52]{GKS2016} que ``rigen las relaciones entre los diferentes actores sociales entre sí''~\autocite[48]{SzIr2009}.
Estas prácticas y normas tienen poder transformatorio pero también sirven como un mecanismo de control, un instrumento para preservar la hegemonía.
%* cultura/campo cultural y su rol en la formación de la nación

``Para la semiótica, la cultura es una red de signos; es un acto comunicativo, un intercambio''~\autocite[71]{SzIr2009}:
entonces, valdrá destacar los símbolos, los signos del peronismo.

\subsection{Modernidad}

La modernidad es una época histórica y un concepto cultural contradictorio con facetas múltiples.
Tiene sus raices y está estrechamente vinculada con la idea europea del progreso, con la Revolución Industrial y la Revolución Francesa y se caracteriza con la ruptura con lo que existió antes.

Por un lado, la modernidad se refiere a ``lo moderno'': el sujeto (masculino, blanco, heterosexual) que goza de la industrialización y los derechos civiles.
Por otro lado, la modernidad se distingue con el pluralismo de identidades y la coexistencia de realidades culturales, modernas y tradicionales, y los conflictos que surgen entre ellas~\autocite[177]{SzIr2009}.

\subsection{Nación}

Ya la etimología de la palabra ``nación'' viene del verbo latino ``nascere'' (nacer) nos da pistas acerca de su sigificado y conotaciones~\autocite[189]{SzIr2009}.
El término ``nación'' se refería en un principio a grupos de personas que provienen del mismo lugar, funcionaba, de forma metafórica, como una gran familia.
Más adelante el significado se ha transformado hacia grupos cuyos miembros
``compartan una historia, costumbres y tradiciones(comida, música, vestimenta, calendarios de fiestas, etc.), práctias culturales, imaginarios, cánones de literatura, de cine y de artes plásticas, así como valores éticos y morales''~\autocite[190]{SzIr2009}.

El Estado-nación es un concepto que se establece encima de la idea de la nación.
Es la tentativa de establecer una entidad administrativa y política apoyada en la homogenidad interna y en contraste con (y que excluye) lo que está fuera.

La formación de los Estado-naciones en America Latina tiene sus particularidades concretas.
Por un lado, no tenemos la diferencia idiomática distintiva que en otros lugares ha servido como el factor decisivo de diferenciación de lxs otrxs.
O más bin, sí, tenemos lenguas diferentes, pero se trata de las lenguas autóctonas indígenas o de las lenguas de lxs afrodescendientes cuya mera existencia suele ser completamente ignorada en asuntos como formación oficial de colectivos, cultura, nación./la formación de discursos oficiales

Tenemos aquí en principio un vasto territorio que está escasamente poblado, entonces faltan también los vínculos y el sentido de pertenencia, unidad colectiva entre lxs inmigrantes.

%4-Analysis------------------------------------------------------------------------------
\section{Evita: \emph{un instrumento de la construcción de la nación durante el peronismo}}

Un ejemplo particular del proyecto de consolidación de la imagen nacional argentina creada por el peronismo es el libro de primer grado infantil \textit{Evita}~\autocite{Albornoz1952}.

Nos dedicaremos en este análisis breve a las preguntas siguientes:
¿quiénes actorxs/protagonistas están dibujadxs?
¿está reflejada la realidad social en su complejidad?
¿la heterogeneidad cultural?
¿a quién se dirige?
¿cuál es la imagen de Argentina que pinta el libro?
¿cómo estan construidos las imágenes centrales de Eva y Juán Perón?

\subsection{Protagonistas}

Tenemos en el libro dos tipos de protagonistas: los personajes anónimos y las personas nombradas.

\subsubsection{Lxs trabajadorxs anónimxs}
Lxs protagonistas anónimxs respresentan el pueblo: la base popular que apoya el gobierno peronista y que está apoyada por ello.
Se trata aquí de un grupo sin nada de particular, caracterizado solamente por su amor de la patria, de los valores peronistas y de lxs líderes del regimen.
El rasgo común de todxs es la risa permamente: que nos señala que lxs protagonistas están contentxs, no puedan pedir nada más, viven en un paradiso en la tierra.
Aunqué el peronismo afirma lxs criollxs del interior como su base, como ya hemos resumido en capítulo..., su discurso se construye sobre todo en términos de clase.
La compleja realidad étnica del país no se incluye de manera directa en las políticas o el discurso.
También el libro está vacío de características étnicas (con una excepción: personas en vestidos étnicos en página 52).

Por otro lado, las referencias a la clase trabajadora son varias y muy claras:
la figura del \textit{descamisado} en página 59, los capítulos ``Día de Trabajo'' (p.35), ``Derechos del Trabajador Hoy'' (p.44-45), ``El buey'' (p.49), ``Hombre de campo'' (p.66), ``Todos deben producir'' (p.39),
con su homilía:
``La Patria pide a los niños argentinos que produzcan lo más posible. Así colaboran al bienestar común. Perón dijo:`Hay que producir, producir y producir'.''(p.39).

Lo que también llama la atención es la falta de agencia por parte de estxs protagonistas.
Son siempre lxs beneficiarixs pasivxs de los regalos, afectos y la bondad, sobre todo de la figura de Eva, y en menor grado de Juan Perón, presentadas en el libro, a lxs que miran con rostros vueltos hacia arriba y miradas de adoración (p.65).

\subsubsection{Lxs niñxs}
Otro grupo dibujado con freqüencia son lxs niñxs, así que se trata, después de todo, de un libro infantil.
Interesante aquí es notar que mucha vez tienen nombres: el libro habla de Yolanda, Inés, Zoilo y Ubaldo (p.16), Horacio (p.17), Cholita (p.21),
probablemente con el objetivo de facilitar la identificación entre lxs lectorxs jóvenes y lxs protagonistas y sus valores.

%Se transmite la impresión que lxs niñxs y el pueblo se pueden comprar
%* wiederkehrende Motive
%  * "juguetes para mí" (p.76)

\subsubsection{Eva y Juan Perón, otrxs actorxs nombradxs}
El resto de los personajes mencionados por su nombre que aparecen en el libro son exclusivamente personajes claves para la formación del Estado-nación argentino.
Domingo Faustino Sarmiento, Manuel Belgrano, José de San Martín todos tienen una página dedicada.
Y, supuestamente aparecen también Eva y Juan Perón.

No podemos hojear tres o cuatro páginas sin que aparezca al menos una vez unx de ellxs o ambxs.
Y cada vez están destacadas sus logros, sus virtudes y el amor incondicional que merecen y que el pueblo le da.
Tanto a Eva como a Juan Perón se refiere con un conjunto/campo/red de epítetos repetidos.
Eva es ``la mamá'' (p.18), ``nuestra Madre Espiritual'' (p.10), la ``madre espiritual de los niños'' (p.75), ``jefa espiritual de la nación'' (p.71).
Su imagen esta estilizada en términos de maternidad y espiritualidad religiosa.
Las características de maternidad no son arbitrarias: la madre es la persona que da nacimiento (fíjase en la relación con ``nación''), una de las personas más importantes para la formación de lxs niñxs.
De hecho, Eva y Juan Perón están varias veces representadxs como padres figurativxs de cada uno y cada una, de toda Argentina:
en página 3--``Mi mamá, mi papá, Perón, Evita'', ``Mamá y papá me aman. Perón y Evita, nos aman.'' (p.7), ``Mi hermanita y yo, amamos a mamá, papá, Perón y Evita.'' (p.8).
El culto a la personalidad y las semejanzas con las representaciones de la Virgen María son tan fuertes, que es fácil de comprender las inquietudes de la iglesia católica al respeto~\autocite{Chamosa2010}.
% obviamente después del fallacimiento de Eva Perón, se le dedica una página; la muerte ha convertido el culto de persona más fuerte? quizá no, se nota lo mismo también para Juán Perón
Además, se le atribuyen cualidades con conotación univocamente femenina.
Ella es ``buena'' (p.15), ``cariñosa'' (p.15), ``querida'' (p.67), ``amó a todos'' (p.13).

Juan Perón, por otra parte, es el líder, presidente, general, buen gobernante.
Tiene un papel más ``fuerte'', más activo, típico ``masculino''.
No podemos dejar de notar los papeles estereotípos de género que operan aquí.
Mientras Eva es la compasiva, la madre, Juan Perón es el líder activo que se ocupa de actividades ``masculinas'': el ejercito, las deudas, etc.
El deminutivo ``Evita'' con el que está denominada con freqüencia y era conocida en Argentina es aún otro rasgo en apoyo de este argumento.
En la vida pública, a mujeres a menudo se refiere por su nombre o un nombre diminutivo, mientras que los hombres están llamado más bien por su apellido o combinación de nombre y apellido (pensamos en nombres que aparecen juntos como ``Dilma y Temer'' o ``Hilary and Trump'').
Aún unx puede argumentar que a través del deminutivo se crea una intimidad entre Eva Perón y el pueblo (que ciertamente no es falso),
esta denominación también invoca asociasiones con una niña, alguien infantil, con una persona que no puede ser tomada en serio.

Sin embargo, lo que es nuevo es que Eva Perón como mujer no esta relegada al privado, sino tiene un papel activo político/público;
La vemos en representaciones de manifestaciones y discursos públicos (p.60,71), muy típicos por el movimiento, se destaca también su trabajo para el voto femenino.

En las representaciones de ambxs es interesante también la referencia repetida al trabajo:
Evita es ``Mártir del trabajo'' (p.22),
Juan Perón es ``Primer Trabajador Argentino'' (p.35).
Esto quiere destacar su vinculación con la clase trabajadora, se representan como una parte de ella.


\subsection{Imagen de la nación}

La imagen peronista de la nación en \textit{Evita} se transmite no sólo por lxs protagonistas del libro, sino a través de relatos inmediatos de Argentina.

%La nación argentina tiene un papel activo
Comenzando con el mapa en la página 29, podemos descubrir varias representaciones directas del Estado.
Los símbolos estatales de ``la Nueva Argentina'': la bandera, el escudo, la escarapela y el himno, también están dibujados explicitamente (p.18-19).

Como aborda en su análisis de Patagonia Gabriela Nouzeilles, el poder estatal se manifiesta, entre otras cosas, en su capacidad para someter la naturaleza y configurar el espacio~\autocite{Nou1999}.
La fetichisación de la naturaleza, su sumisión al servicio de la gente y su valor símbolico para la imagen del estado encontramos en \textit{Evita} en los capítulos dedicados a las ``bellezas de la patria'' (p.56-57) y en las referencias a ``tierras ajenas'' como las Islas Malvinas o la Antártida, que forman parte del imaginario de la nación (p.24-25).

Demás, el libro forja la imagen de una Argentina moderna, industrializada.
Proyectos y artefactos de la infraestructura están representados varias veces:
tenemos el capítulo ``Caminos'' (p.50), así como dibujos múltiples de ferrocarriles(p.48,72), carros, coches, del teléfono (p.58), avión (p.72), barcos (p.61,72).

Otro elemento fundamental de la figuración de la nación es el calendario de los días festivos~\autocite{SzIr2009}.
También estos encontramos en \textit{Evita}:
páginas se dedican al
``Día de la Libertad'' (p.22),
a la ``Independencia Política e Independencia Económica'' (p.30-31),
al ``Día del Trabajo'' (p.35),
y al ``Día de la Recuperación Nacional'' (p.48).
Además, hay al final del libro una lista de las ``Fechas que conviene recordar'' (p.79-80).


\subsection{Destinatarixs}

En primer lugar, lxs destinatarixs de este libro son lxs niñxs de escuela primaria, lxs niñxs que aprenden a leer y escribir.
El lenguaje es simple, repetitivo, con accento en lo más importante (las figuras centrales de Perón y Evita, la imagen de Argentina).
Las temas son una mezcla entre lo que suposadamente ha de interesar niñxs: juquetes, amor hacia la familia, .. y propaganda política.
El mensaje para ellxs es claro: Argentina es parte de su familia, Juan y Eva Perón son prácticamente una segunda pareja de padres.

Aunqué el libro está claramente dirigido a lxs seguidorxs del peronismo,
hecho señalado por el lenguaje y los temas abordados,
es interesante observar que esto no se nota necesariamente en los dibujos de las figuras.
Con pocas excepciones (las figuras en vestidos étnicos en página 52 y el \textit{descamisado} dibujado en p.59),
las figuras (syn!) suelen mostrar un aspecto más burgués que de clase trabajadora.

%X-Conclusion---------------------------------------------------------------------------
\section{Conclusión}

A forma de conclusión, volvemos a las preguntas:

En general, después de todas las reflexiones hasta este momento,
%----------------------------------------------------------------------------------------
%	BIBLIOGRAPHY
%----------------------------------------------------------------------------------------

\renewcommand{\refname}{\spacedlowsmallcaps{Referencias}} % For modifying the bibliography heading

\nocite{*}
\addcontentsline{toc}{section}{Referencias}
%\bibliographystyle{unsrt}
%\bibliographystyle{alpha}
%\bibliographystyle{natdin}
%\bibliography{literature} % The file containing the bibliography

\printbibliography
%----------------------------------------------------------------------------------------
\end{document}
