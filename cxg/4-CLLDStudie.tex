\section{CLLD und CLD Konstruktionen im Spanischen und Englischen}

\subsection{Hauptthese}
%\begin{enumerate}
%    \item \sout{Lernerinnen einer Fremdsprache greifen oft auf schematisierte Äußerungen zurück.
%        Schlagwort: formulaic phrases}
%    \item Lernerinnen einer Fremdsprache projizieren Strukturen/Konstruktionen aus ihrer Muttersprache in die L2.
%\end{enumerate}

Lernerinnen einer Fremdsprache projizieren Strukturen/Konstruktionen aus ihrer Muttersprache in die L2.
Oder tendieren zu Übergeneralisierungen einer gelernten Konstruktion aus der L2.
Beispiele wären die CLD-Konstruktionen im Englischen und CLLD-Konstruktionen im Spanischen.
Lernerinnen mit L1 Englisch tendieren dazu die CLLD Topikkonstruktionen im Spanischen zu benutzen, unabhängig von der Spezifität (also empfinden den Bedeutungsunterschied nicht).
Lernerinnen mit L1 Spanisch tendieren dazu, bei spezifischen Topiks eine CLLD Konstruktion einzusetzen?
\cite{Valenzuela05}


Behauptungsgrundlage

%soll das hier nicht zu 2-Grundlagen über? (ergo Zweitspracherwerb Forschung in historischer Perspektive oder so)

\subsection{Vorgeplänkel}

\cite{Valenzuela05}
``When the L2 input data are insufficient and the L2 speaker cannot restructure his/her grammar, the learner is unable to `let go' of
one of the two forms of the construction.
The presence of two forms for one interpretation results in a permanent optionality between both variations.
This permanent optionality is not found in the native grammar of the target language and is therefore a form of incomplete L2 acquisition.''

Was sind Topik-Konstruktionen?

<TOPIK, Rest>
``in Germanic and Romance languages, topicalization is typically expressed by setting a phrase apart
from a clause in order to reintroduce it into the discourse.''

``Topic constructions are expressed using contrastive left dislocation (CLD) in Germanic languages'':
\begin{exe}
    \ex \uline{The hat}, I bought in Toronto.
    \ex \label{shoes} \uline{These shoes}, I bought \textbf{Op} in Madrid.
\end{exe}

``clitic left dislocation (CLLD) is a typical form of topicalization in Romance languages'':
\begin{exe}
    \ex \label{zapatos} \uline{Estos zapatos}, \textbf{los} compré en Madrid.
\end{exe}

``The parametric difference between the two language groups is that in English,
the preposed topic connects to the comment via a null anaphoric operator (Op),
(as in (\ref{shoes}), whereas in Spanish, the connection between the topic and the comment is made using a clitic (as
in (\ref{zapatos})).
% wie wissen wir dass es da ein ``null anaphoric operator'' gibt? ich meine, da gibts ja nichts..

``In Spanish, the notion of specificity is crucial for topicalization. When a topicalized element is
specific, it takes a clitic (CLLD structure) however, when a topicalized element is non-specific or generic
(shown in (\ref{libro}) – (\ref{revistas}) respectively), it does not take a clitic''

``with the CLLD construction the topic is
interpreted as specific whereas with the CLD construction the topic in interpreted as non-specific.''

\begin{exe}
    \ex \label{libro} Este libro, *(\textbf{lo}) he leído muchas veces.
    \ex \label{revistas} Revistas, (*\textbf{las}) leo a menudo.
\end{exe}

``Such a contrast is not available in English topicalization. English is thus restricted to the CLD construction
regardless of the interpretation of the topic''

\begin{exe}
    \ex \uline{This} book, I have (*it) read many times.
    \ex Magazines, I often read (*them).
\end{exe}

``In CLLD the left-dislocated phrase can occur in either root or embedded clauses, as in (9). The left-dislocated phrase in CLD, on the other hand, can only occur at the left periphery of root clauses, as in (10):''
--> no non-specific topics allowed in embedded clauses (auf Spanisch)!
\begin{exe}
    \ex \begin{xlist}
        \ex Un libro, \textbf{lo} leí anoche
        \ex Me pregunto que, a María, el libro, quién \textbf{se} lo dio
        \end{xlist}
    \ex \begin{xlist}
        \ex Un libro, leí anoche
        \ex[*] {Me pregunto que, tarjetas, a amigos, quién mandará}
        \end{xlist}
\end{exe}

\subsubsection{Research Question}
\begin{itemize}
    \item Will adult L2 Spanish learners be able to acquire the syntactic properties associated with CLLD?
    \item Will adult L2 English learners be able to let go of the syntactic properties associated with CLLD?
    \item will it be easier to acquire or to ‘let go’ of a property?
\end{itemize}

\subsection{Methoden}

\begin{itemize}
    \item L1 Spanisch -> L2 Englisch lalal
    \item L1 Englisch -> L2 Spanisch: 15 English speakers of L2 Spanish; had had their first exposure to Spanish after puberty;
        from England, Canada and US; living in Spain; work in Spanish, often work + home life in Spanish;
        25 monolingual Spanish speakers tested as control group;
    \item Sentence Selection(tests acception) and Sentence Completion(tests production) tasks;
    \item specific vs non-specific topics in root clauses
    \item specific vs non-specific topics in embedded clauses
\end{itemize}

\subsection{Funde}

Englisch -> Spanisch

Sentence Selection: Root Topik-Konstruktionen
\begin{itemize}
    \item both participant groups (natives and near-natives) correctly selected the specific topic with a clitic
    \item bei root non-specific Topiks gab es bei near-natives eine höhere Tendenz zur Wahl eines Klitikons (falsch) - etwa 37\% vs 14\% der Kontrolgruppe
\end{itemize}

Sentence Selection: Embedded Topik-Konstruktionen
\begin{itemize}
    \item ``both groups correctly preferred the CLLD construction with tokens eliciting specific topics in embedded contexts''
    \item non-specific Topik: gibt es eigentlich nicht in Embedded Konstruktionen: 77\% der Kontrolgruppe bestätigt das; 67\% der near-natives wählen eine klitische Konstruktion (CLLD) und nur 31\% bezeichnen sowohl CLLD als CLD als inkorrekt
\end{itemize}

Sentence Completion: Root Topik-Konstruktionen
\begin{itemize}
    \item ``Both groups correctly provided clitics with specific left-dislocated topics in root environments''
    \item ``In contexts forcing non-specific interpretation, however, near-natives completed them with a clitic over 50\% of the time.'' (falsch)
       -> Sie tun übergeneralisieren!

\end{itemize}

Sentence Completion: Embedded Topik-Konstruktionen
\begin{itemize}
    \item ``both groups correctly produced sentences with a clitic in contexts where a specific clitic was provided.''
    \item Non-specific Topik ist nicht so wirklich erlaubt in Embedded Konstruktionen laut der Einführung, deshalb verstehe ich diesen Teil der Studie nicht;
        es gab wohl keine Option anzugeben, dass der Satz ohnehin komisch ist und den einfach nicht zu vervollständigen;
        Ergebnisse zeigen: natives entscheiden sich gegen einen Klitikon, near-natives dafür;
\end{itemize}

So
\begin{itemize}
    \item near-native haben die CLLD Konstruktion gelernt;
    \item sie scheinen sich aber manchmal, die Bedeutungsunterschiede zwischen Klitikon/kein Klitikon nicht bewusst zu sein
    \item Also es gibt schon eine gewisse Tendenz dazu, dass near-natives viel zu oft klitische Konstruktionen benutzen. Auch da wo es nicht geht.-->übergeneralisierung!
\end{itemize}

Spanisch -> Englisch

Sentence Selection: Root Topik-Konstruktionen
\begin{itemize}
    \item specific context: beide Gruppen entscheiden sich richtig für ``no pronoun'' zu 48\%; bei den L1 Spanisch Leuten gibt es immer noch eine gewisse Tendenz dazu Klitika zu benutzen (30\%)
    \item non-specific context: die Mehrheit von beiden Gruppen entscheidet sich richtig für ``no pronoun''; allerdings sind die near-natives, die diese Option wählen mehr als die natives (63\% vs 55\%)
\end{itemize}

Sentence Selection: Embedded Topik-Konstruktionen --> gibts nicht; keine CLDs in embedded Konstruktionen
\begin{itemize}
    \item specific: beide Gruppen tendieren dazu, sich richtig zu entscheiden (weder Klitikon noch keins ist eine Option); allerdings, mehr als 50\% (bei beiden!) wählt die eine oder die andere oder beide Optionen
    \item non-specific: same here
\end{itemize}

Sentence Completion: Root Topik-Konstruktionen
\begin{itemize}
    \item die Mehrheit der beiden Gruppen entscheidet sich für kein Pronomen sowohl bei specific als auch bei non-specific contexts
    \item allerdings beobachtet man bei den Spanisch L1 Sprecherinnen eine gewisse Tendenz zur Wahl eines Klitikons (54\% bei den specific und 36\% bei den non-specifics) --> also Einfluss der L1
\end{itemize}

Sentence Completion: Embedded Topik-Konstruktionen
--> gibts vernünftigerweise nicht, es hieß ja vom Anfang an, es gäbe keine CLD-Konstruktionen in Embedded Clauses..

So
\begin{itemize}
    \item L1 Spanisch Speaker: haben die Unterschiede zwischen CLLD und CLD im großem und ganzem gelernt: ``appear to know the constraints on the contexts in which CLD can appear (only in root contexts).''
    \item ``However, the near-native group is both accepting and producing pronouns (overt resumptive elements like clitics) with the specific topics there by exhibiting L1 influence.''--> `letting go' die L1-Konstruktion ist schwer
\end{itemize}

% sollen mehr konkrete Beispiele rein?

\subsection{Kritik}
bisschen zu großzügige Interpretation der gemessenen Daten zugunsten von der Arbeitshypothese

\subsection{Aus konstruktionsgrammatischer Perspektive}

\subsection{Daten}
\begin{itemize}
    \item \cite{Valenzuela05} CLD und CLLD: Spanisch -> Englisch, Englisch -> Spanisch
%    \item Kindlichem L2 Erwerb Spanisch -> Englisch: \cite{Wong-Fillmore76} in \cite{Haberzettl06}
%    \item Kindlichem L2 Erwerb Türkisch -> Deutsch: Wegener\cite{} -- leider unveröfftentlichte Schrift
%und \cite{Haberzettl05} in \cite{Haberzettl06}
%
%SOV-Konstruktionen aus dem Türkischen ins Deutsche projiziert
%
%    \item Evidenzialität Quechua -> Spanisch??
\end{itemize}
