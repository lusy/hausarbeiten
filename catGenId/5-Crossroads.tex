\section{Una noia a l'encreuament. Lectura interseccional}

\subsection{Interseccionalitat}

En l'any 1989 la professora de dret i advogada de drets civils Kimberle Crenshaw fa servir el terme ``interseccionalitat'' per primera vegada~\autocite{Crenshaw1989}.
Amb aquesta noció, que prové de la metàfora de l'encreuament dels camins, l'acadèmica vol descriure l'opressió multiple que sofreixen persones que pertanyen a la vegada a diversos grups marginalitzats.
Ella critica que fins aquest moment els discursos acadèmics, legals i activistes analitzen la discriminació al llarg d'un únic eix,
tractant les opressions de gènere i racistes com dos fenòmens mútuament excloents,
concentrant-se en els membres privilegiats de cada grup i d'aquesta manera amaguen la multidimensionalitat d'experiències de les persones més vulnerables, com per exemple les dones negres.
Segons Crenshaw, la interseccionalitat denomina un tipus de discriminació que consta de més que la suma de les seves parts.
Això vol dir que les dones negres per exemple experimenten una discriminació molt particular que és diferent de la combinació d'experiències de les persones negres i les dones en general (que no significa que les dones negres no siguin discriminades com dones o com negres, però que també experimenten una opressió particular com a ``dones negres'').

Des de llavors, aquest concepte s'ha fet servir en contextos diferents per analitzar situacions i col·lectius que pateixen d'una opressió multiple.
Nosaltres l'aplicarem a la situació de la filla en ``L'últim patriarca''.

\subsection{Entre dues cultures}

Com adverteix el fundador de les ciències postcolonials Edward Said en el seu llibre ``Culture and Imperialism'', avui dia cap persona pot ser definida amb una sola etiqueta
i identitats (atribuïdes o seleccionades per a una mateixa) com ``dona'', o ``musulmà'', o ``americà'' serveixen només per a una primera orientació~\autocite{Vidal2012}.
A més, aquestes categories són subjectives i existeixen sempre dins un narratiu particular que les ha creat, afirma la professora de ciències de traducció Mona Baker, citada en~\autocite{Vidal2012}.
Said planteja també la idea que tothom construeix la seva pròpia identitat cultural i ètnica.
Ja no tenim cultures i identitats ``pures'' i ``estables''~\autocite{Vidal2012}.
Però encara, amb la mateixa força, l'altre, el diferent, l'estrany està percebut com una amenaça per l'ordre establert i per això es rep amb por i prejuicis.

Entre una cultura d'origen, representada pel pare, per la mare, pel reste dels familiars a Marroc i, en part, pels moros a Catalunya, i una cultura d'``acollida'', la filla en ``L'últim patriarca'' ha de trobar les parts per construir la seva identitat.

%la mare
La mare està descrita com a una dona que compleix tots els estereotips de la societat occidental per a una dona musulmana:
és una bona esposa, segueix quasi literalment tots els ordres absurds i humiliants del seu marit, pateix les seves escapades en silenci, s'ocupa dels fills, fa excepcionalment totes les feines de casa.
Ella representa un model a seguir per la filla segons les normes marroquines, una possibilitat, que a la noia no li agrada gaire:
``La mare deia surts massa i començava aquells discursos de jo a la teva edat ja feia\ldots Jo a la seva edat no hauria sabut què fer perquè ella només netejava i netejava i no sabia com anar al metge sense el pare, com havia de comprar sense el pare, com havia de viure sense el pare.''~\autocite[188]{ElHachmi2008}.
Trobem aquí el que és no només un xoc de cultures sinó també un conflicte generacional, i, fins i tot, una atribució masclista per com ha de ser una nena.
La mare està convençuda que a l'edat de la filla era molt més capaç i la filla està decepcionada que encara a aquesta edat la mare no és capaç de funcionar a la societat per si mateixa.
Hi ha també rares ocasions d'amiració per la mare per part de la filla--quan ella demostra la seva força silenciosa:
``Va dir o la deixes, o et deixo. Jo no em vaig creure el que sentia, però era la meva mare que parlava, era la Mila que s'havia afartat de netejar capelles i relíquies, la Colometa que fugia de tot per trobar-se.''~\autocite[222]{ElHachmi2008}.
Però, podem resumir, que a la majoria de les vegades, la filla es definiria més en contrast amb la seva mare.

%el pare
El personatge de Mimoun està caracteritzat, com ja hem assenyalat, amb una crueltat exagerada i excessiva.
El també compleix els estereotips occidentals per l'home arab qui oprimeix la seva esposa, té una sexualitat no limitable i no respecta cap dona com a esser humà.
A més, fa servir els prejuicis dels altres pels seus propis objectius:
fixem-nos a l'episodi quan prohibeix a la filla anar a les colònies, perquè sí, per demostrar que ell està en control, i argumenta davant les mestres de l'escola amb la mare, qui suposadament hagi emès la probihició, presentant-se així com a un home progressiu en contrast amb la mare--una dona musulmana conservativa.

% la cultura d'acollida
La cultura catalana que sovint es presenta com alliberada tampoc és gaire acollidora per a una noia percebuda com ``altra''.
Els veïns i les veïnes no saluden la filla quan es posa el mocador i observen en silenci els assalts del seu pare sense obtenir ni per un sol moment la idea d'ajudar-li.
Ella rep els prejuicis i la hipocrecia de la societat catalana:
``jo ja feia temps que no podia anar a casa dels seus pares\ldots a l'amiga número dos la seva mare la portava a vendre espelmes d'aquelles amb les quals s'ajuden els nens pobres del món, d'Àfrica i tot, però es veu que només valia per als nens negres de debò i no pas per a mig brunes com jo''~\autocite[285]{ElHachmi2008}.
Tothom ja té les seves idees com ha de ser suposadament una nena marroquina:
``\ldots i jo que havia trencat lleis no escrites i havia decidit que no volia ser ni auxiliar d'infermeria ni administrativa de grau u ni mecànic ni electricista.
  Pesaven força espases de Dàmocles damunt meu: que si jo a la teva edat ja estava casada, que si en la teva cultura ja se sap que no val la pena, que us acaben casant tard o d'hora, la d'aquest és l'últim curs i alguna altra que tenia el pare al cap, com allò de les dones que no traeixen mai els pares però que sí que acaben traint els homes.
  Tot això duia jo a la motxilla, però ningú se'n va adonar.''~\autocite[273]{ElHachmi2008}.

% autoidentificació
Entre aquests exemples, una noia adolescent ha de trobar un model per a si mateixa, que no és una tasca fàcil.
Aparentament ninguna de les dues cultures no pot ser triada sense objeccions.
El món occidental amb la seva autopercepció de superioritat, d'haver establert l'única manera possible de fer i viure les coses, i les seves opressions sobre tot el que és diferent, no pot ser acceptat incondicionalment.
Però, el mateix està en vigor per la cultura marroquina.
El seu únic projecte de feminitat conservadora, relegada a les feines de casa i les necessitats dels homes, és encara menys satisfactori.

\begin{comment}
  1) la cultura d'origen representada per
     * el pare
     * la mare
     * la família amb els avis i les tietes
     * els moros de Catalunya
  2) la cultura d'acollida, representada per
     * l'escola
     * les amigues
     * les amants del pare

+ stereotips mútuus de cada grup per l'altre
\end{comment}

\begin{comment}
La mare
-------
"La mare era massa tossuda per ser la dona de Mimoun. Ell neceissitava una dona que es deixés domesticar per a tot i ella en les coses que i eren importants [versteht sich putzen und kochen], no en sabia, de cedir." (p.108)
-- am Ende wird sie auch gebrochen..

"Mimoun pensava que aquella separació posaria a prova els lligams que havia creat amb la seva dona i es veuria si ja l'havia domesticada prou." (p.124)

"Jo deia a la mare au, va, anem a mercat, que ell no hi és, o anem a aquella botiga de teles i te les tries tu mateixa per fer-te els vestits, anem a fer el volt o a veure alguna amiga teva. Ella que no, que no, que ell no hi és però ho sap gairebé tot. Allà vaig començar a entendre fins a quin punt estava domesticada i que potser aquell lligam ha era per a tota la vida." (p.237)

"L'avi va dir vesteix-te, dona, que véns amb mi, i ella que no, que no, que Mimoun em mataria si ho sabés, que no. Mimoun ja t'està matant, li havia respost l'avi i a mi em va quedar la frase per sempre." (p.164)
--representasions del pare i de la mare

"No hi teníem res que no fossin les veïnes de davant, que deien denuncieu-lo, que tothom ja ho veu el que està fent i que si voleu us acompanyem a serveis socials. La mare deia no, jo mai he demanat caritat i aqeusta no serà la primera vegada i estirava els estalvis tant com podia." (p.211)
-- aquest concepte d'estar pobre és vergonyós; isolación, patir en silenci
   falscher Stolz
-- interseccionalitat amb classisme?

"Li ho havia explicat tot a la mare. Em separo. Que t'ha pegat? No. Que t'ha insultat. No. Que no et dóna diners per pagar el menjar? No. Doncs no entenc per què et vols divorciar, no ho saps que una noia divorciada ja és de segona categoria? Què penses fer?, el teu pare t'ho farà pagar car, si tornes a casa. No tornaré a casa, i ella ja no entenia res." (p.326)

"I no sabia quin era el però perquè de fet tota la situació era un però enorme a tota la tradició, a tot l'ordre establert que a ella li havien ensenyat. Un ordre que ja s'acabava, almenys el la nostra família." (p.326)
-- normes d'allà
-- l'enteniment de la mare què és un matrimoni?
-- seguretat social, ja que no es suposa que la dona pot proveir per a ella mateixa


La família
----------
"[...] desprès del seieu, fills meus, seieu i què és això que portes a les dents, aquest anys véns tota plena de plata, quina gràcia, i dels que no us donen de menjar, allà a l'estranger? Després del va, agafa la cuixa del pollastre que ja sé que és el que més t'agrada i us he guardat figues tan bones que us en recordareu la resta de l'any. Després de tot, venia aquella mena de nus a la gola just abans d'anar a dormir, amb el vaivé del vaixell encara bressolant-te i una lleugera certesa que aquell no era el teu destí però que tampoc no sabies quin havia de ser i eres com Zaida de <textit>Nule Parte</>." (p.275-276)

"L'àvia havia pres sang d'eriçó, s'havia banyat amb aigua on havia diluït l'esperma del seu home; s'havia fet fumejar l'entrecuix amb la barreja que cremava al foc, feta de sofre, gallarets esmicolats i excrements de colom secs.
  Tots els remeis que les àvies de l'època li havien recomanat." (p.12)
-- exotisme? aber so bisschen ins laecherliche gezogen; vlt sich lustig machen ueber rueckstaendigkeit? vlt ist es auch gar nicht so gemeint, kann man aber so lesen; però per altra banda no parla "d'un altre lloc", sino de "l'època".. doncs la història sóna molt propera...

"A la mare no li deixaven fer res i ella es vestia amb caftans bonics, que així era com havien de vestir les dones dels homes importants i rics. Perquè es veu que érem rics, nosaltres. A mi tot allò em fer gràcia al principi, però aviat ja vaig tornar a tenir ganes d'anar a casa." (p.249)

Autoidentificació
----------------
"Fins que els vaig sentir en una d'aquelles converses de germans i vaig començar a pensar que aquell no era el meu món ni ho seria mai. [...] Mira, tu ja tens edat de casar-te. [...] havíem parlat força vegades i havíem dit que com que ens hem estimat tant, què millor que el seu fill per fer-te d'espòs. [...] Que et casis amb ell és millor que no pas anar a parar a una família estranya que no coneguis de res. Jo no em vull casar. Les tietes van riure totes, que tothom es casa tard o d'hora, hona, no pots quedar-te per vestir sants. No em penso casar ni ara ni mai." (p.250--251)

Els moros a Catalunya
---------------------
"però per tu ho faré, treballaré deu hores al dia i tindré prou diners per al dot. Jo és que estic estudiant i m'agradaria anar a la universitat. Cap problema. Jo és que no seré una esposa de les que es queden a casa a netejar i cuinar, vull treballar, vull sortir, les feines de casa han de ser compartides. Cap problema. Devia ser la pressió del seu entrecuix que li feia dir que sí a tot o potser s'ho creia de debò. Jo em vaig sentir commoguda que un home que havia nascut al mateix lloc que tots nosaltres pogués ser tan diferent del pare." (p.294)

Encreuament: pare vs escola (vlt auch lectura autobiogràfica)
-------------------------------------------------------------
Expectacions del pare
"Va començar a fer-me entrar a les cambres quan venia algun dels cosins [...] El pare va voler anar més enllà, no vull que hi parlis, amb aquests voltors [...]" (p.250)

"Va arribar un dia i var dir t'he comprat un regal. Unes faldilles i una camisa, que jo vaig pesar que eren per a la mare. Se'm va escapar el riure, on vols que vagi així? Coi, vull que vesteixis decentement i no pas amb aquests pantalons tan ajustats." (p.253)


\end{comment}

\begin{comment}
--hier oder Unterkapitel "intertextualitaet" oder aehnlich in language

Qui es català?

* estereotips: die Mutter kann und tut nur putzen und kochen und ihren Mann gehorchen obwohl er sich ungeheuerlich verhaelt; der Vater ist richtig ueberzogen als Arsch dargestellt, der seine Familie schlaegt, rumvoegelt und absurde Vorstellungen von Ehre, Ehe und Familie hat;
--> vlt sind die stereotypen so überzogen um den inneren (und äußeren) konflikt noch drastischer hervorzuheben?
    vlt auch um ihre Absurdität aufzuzeigen?

"Jo hi volia anar, a veure Isabel [...] i així podria saber quina cara feia una dona com aquella. Lletja, segur. Havia de ser lletja i pudent, com havia dit la mare tantes vegades que eren les dones que mengen porc." (p.185)
--estereotips (dones cristianes, dones musulmanes, gegen einander ausgespielt)

"La mare sempre deia que hauries d'estar fent això o hauries d'estar fent allò altre i jo ja havia vist que les nenes de la meva edat no sabien ni agafar bé una escombra i que no tenien cap interès a aprendre'n." (p.239)
-- entre cultures: com són les amigues

\end{comment}

\subsection{Entre el desig d'autoidentificació i les atribucions del masclisme}

\begin{comment}
  1) autoidentificació
  buscant la seva identitat
  * religió
  * literatura

  2) les atribucions
  * com han de ser les nenes/les dones?
    ** alleine fuer die ganze care arbeit zustaendig, wird als natuerliche neigung inszeniert
  * les dones i les "altres" són culpables per tot (vgl Laurie Penny)
    --> keine Solidaritaet zwischen den Frauen (zb la mare i les amants) moeglich, obwohl sie alle Opfer sind; sie werden gegen einander ausgespielt
  * frauen werden auf ihr aeusseres reduziert
  * die frauen wuerden nie genuegen;
\end{comment}

Una altra noció teòrica central per a l'obra és el patriarcat.
Com assenyala la periodista i activista Laurie Penny, aquest terme és refereix al comandament/regiment d'una elit d'homes sobre la resta de la societat.
I historicament, aquesta elit és bastant petita.
Llavors, els homes que no disposen de poder polític, tenen al menys la satisfacció de manar els membres de la seva família com a compensació per la falta de control sobre la resta de les seves vides, destaca l'autora~\autocite[69-70]{Penny2014}.
Aquestes són les estructures del patriarcat, i exactament aquesta descripció trobarem molt aplicable.

Exemple!

Una idea interessant aquí és que les dones tenen un paper bastant important per la reproducció de l'ordre patriarcal, ja que elles són en la majoria dels casos encara les persones que primerament s'ocupen de l'educació dels fills i de les filles i llavors aqulles que perpetuen les normes opressives.
``Some of the things that happen around him justify what he does. I was very surprised when I was writing the book that women in the family were so important in making him become the monster he becomes'' assenyala Najat El Hachmi en una entrevista, referint-se a Mimoun~\autocite{HaAM2011}.
Aquest concepte d'excuses i justificacions es recolza també en el text de la novel·la: ``però ja ho saps, filla, que en el fons el teu pare és de bona fusta i no faria mai mal. És només això, que els espants no li han acabat de marxar mai del cos i això l'ha fet algú diferent''~\autocite[18]{ElHachmi2008}.
Doncs, Mimoun creix a creure que per a ell tot està permès.
``Era així com Mimoun aconseguia sempre que les dones de la seva vida l'anessin convertint en patriarca.''~\autocite[99]{ElHachmi2008}.

Les eines del patriarcat i del masclisme per manar les dones i la sexualitat femenina són multiples.

%reducció de les dones en la seva aparença física
Una de les més classiques és la reducció de les dones en la seva aparença física, que es converteix en l'única característica relevant per valuar una dona.
El que és encara més alarmant i trist aquí és que aquesta postura es manté també per dones en les seves relacions entre elles i fins i tot en l'evaluació de si mateixes.
Observem els pensaments de filla sobre una de les amants del seu pare:
``Sí que era lletja, feia cara de dolenta de pel·lícula, de les dolentes que no sedueixen, que només maquinen [\ldots] Bruixa!, vaig pensar. Que la mare és més guapa, que ho sàpigues, que és més bonica''~\autocite[186]{ElHachmi2008}
o sobre una putativa amant de la seva parella:
``Un cul més gros que el meu no faria malbé la meva llibertat sostinguda.''~\autocite[308]{ElHachmi2008}.
Una dona està aquí reduïda a una part del cos i el verdader culpable per la situació s'ofusca.
%"La roba sempre havia estat un problema. Des que havies fet el canvi [...] Que mira si els malucs et creixen més del compte què voleu que hi faci i no teniu més talles que quaranta. Una quaranta-dos en una botiga de roba moderna i per a joves no s'estilava. Si ets jove es dóne per fet que estàs prima" (p.284)
%--körperimage, die einer aufgedrungen wird
%"Encara ara tinc pànic als emprovadors de les botigues i em fa mandra anar a comprar roba per no haver-m'hi de ficar." (p.285)
%-- keine solidaritaet zwischen den "opfern", die werden gegeneinander ausgespielt;
Les possibilitats per solidaritat i suport mutu entre dones es distrueixen.

% atribucions: com són les nenes
Una altra eina són les atribucions assídues de qualitats i capacitats que tingui una nena o una dona.
Perquè si tothom et diu continuament que siguis així o d'aquesta altra manera, al final també acabes creient-lo.
I les atribucions són, naturalment, no de qualitats vinculades amb control i poder, sinó tot el contrari: de qualitats submisses, suaus, agradables, no poderoses i quals no canvien el món.
Sobre les dones cau la responsabilitat per totes les tasques domèstiques, tota la feina de cura i afecte i això, per descomptat, sense cap agraïment, sense res a canvi.
La filla està sotmesa de prescripcions vàries:
``[i] t'ho demostren, les nenes, et demostren que t'estimen facis el que facis i el seu amor és sempre incondicional. Jo ja vaig néixer amb aquest deure afectiu''~\autocite[147]{ElHachmi2008},
``[l]a mare sempre deia que jo era més responsable que els meus germans, més treballadora, més estudiosa, més de tot, però em penso que l'única cosa que jo era més que ells era nena''~\autocite[191]{ElHachmi2008},
``[l]a mare, estesa al llit, em va dir no podies haver donat roba neta als teus germans, que la porten tota tacada de tomàquet d'ahir?''~\autocite[220]{ElHachmi2008}.
% la dona qui satisfà les necessitats dels homes

% destruir l'autoestima: una dona sense un home...; nie genuegen

% la dona qui satisfà les necessitats dels homes

\begin{comment}
Autoidentificació - religió
---------------------------
Kapitel Estima Déu i ell t'estimarà
- Identitaetssuche
- Ohne vernuenftige Vorbilder + Erklaerung, ohne Sinn und Verstand
- Escaipismus
- zu Extremhaltungen neigend
- sich aufgehoben fuehlen
"Jo em vaig proposar de ser una bona musulmana, la millor." (p.226)

"Déu meu, fes que el pare torni al bon camí, però ho deia en la llengua de la capital de comarca perquè en la llegua dels musulmans jo no hauria sabut com dir-ho. S'hi valia: en l'última part de l'oració, on demanes alguna cosa directament a Déu, podies fer servir la llenuga que et fos més còmoda." (p.226)
-- llengua-idenitat

Hijab
-----
quote Interview El Hachmi:
"Los pornógrafos de la etnicidad acentúan rasgos de ti que en tu país encontrarías ridículos."
"El inmigrante no quiere pertenecer a una asociación de inmigrantes, sino a una de vecinos..."
"Cada mujer lleva el velo por motivos diferentes. Y no se
puede pretender salvar, de manera paternalista, a las pobres moritas del yugo de
sus maridos.
-El yugo existe.
—Como en otras culturas."
"El problema es que la mayoría tiene permiso de residencia sin permiso de trabajo. La ley de extranjería las condena a la clandestinidad laboral y eso hace que dependan del marido. Les corta la vía de emancipación."

"Treu-te això del cap, que em fas passar vergonya. I ella que no, que em sentiré despullada, que no. Mira que aquí les coses són diferents i a mi em coneix molta gent i tinc una empresa i no hi ha cap nexessitat de portar aquests draps." (p. 183)
-- der stellt sich sehr befreit und so dar; und spaeter findet er gut, dass seine tochter nen hijab traegt..

"Me'l posava per resar, primer. Després per estar per casa. FIns que vaig sentir que era imprescindible, que no podria viure mai més passant amb el cap descobert davant de ningú. Me'l vaig posar per anar a comprar i vaig sentir les mirades estranyades de les botigueres que em coneixien [...] Vaig sortir així un parell de vegades i un dia el pare em va veure. [...] Au, no surtis més amb aquest drap al cap." (p.228)
-- warum gucken sie die Verkaeuferinnen komisch an? ist sie etwa nicht dieselbe person? was sehen sie im hijab? ein symbol der anderen? ein symbol der opression?

"La mare em va fer anar a ca la Soumisha a buscar alguna cosa i jo vaig posar-me el mocador, [...] Sembles un àngel, m'havia dit ella, segur que entraràs al cel directament, per la porta gran. [...] Els nostres ulls es van trobar i allà mateix ho vaig saber, que no hauria hagut de posar-me el mocador. [..] no sé com no vaig caure. Ell no deia res, però jo ja el sentia derrere meu i quan va dir para, para o encara serà pitjor, jo ja no sé si vaig córrer o em vaig aturar, però em recordo a terra, amorrada a la clavaguera i ell vinga donar-me puntades de peu. No recordo els cops, no recordo si em va picar a la cara, a l'estòmac. [...] I llavors vaig mirar tot al meu voltant i vaig veure els clients del bar de davant de casa amb la beguda a la mà que no deien res i els que passaven pel costat que no deien res i els que ens coneixien i no deien res i allò era estar sola." (p.228-229)
--estar sola
-- la indiferència de la gent davant "els\_les altres"

Hijab (Kübra bei der Friedrich Ebert Stiftung)
---------------------------------------------
frauen mit kopftuch wird die legitimation abgesprochen
- kann sie überhaupt sprechen? (nicht nur deutsch, sondern überhaupt)
- darf sie überhaupt etwas sagen?
- kann man ihr vertrauen?

Andere sprechen für sie? Was waren wohl ihre Beweggründe ein Kopftuch zu tragen?

Persönliche Erfahrungen ins Licht gestellt, hinterfragt, das Mensch-Sein dieser Person wird in Frage gestellt

https://www.youtube.com/watch?v=tcGaDPUSJL0


Autoidentificació?
-------------------
la mestra amiga
"Jo li havia parlat de crisis, de crisis que encara era incapaç de reconèixer com a indentitàries, de pits que creixen massa, de la mare que no volia que em depilés i de com m'havia llençat els tampons per por que jo perdés la virginitat, així, sense ni parlar-ne ni res, havia vist el dibuix de les instruccions i els havia llençat a les escombraries." (p.268)

Autoidentificació - recerca de la identitat
"El metge va dir això són atacs d'ansietat i sonava tan greu que encara em vaig espantar més. Tens motivus per estar així, algun problema personal? No, doctor, no, la meva vida és perfecta, volia dir-li, com al de qualsevol adolescent que ha de fer-se gran i no sap com fer-ho. COm tots, suposo, li vaig dir, i em va donar aquells tranquil·litzants que m'havia de posar sota la llengua si em venia allò una altra vegada." (p.302)

Autopercepció/autoestima
-----------------------
(ell a ella crec)
"I tu suposo que deus ser d'aquests immigrants que viuen sols i tal. No t'ho creus ni tu, devia pensar, i vam caminar fins al Club que hi havia més amunt, on mig a les fosques ja em va dir que estava boig per mi. Què? Si no em coneixes de res. Et conec i t'he seguit durant els últims sis mesos, t'estimo. T'estimo, t'estimo, sonava dins del meu cap i jo només vaig poder riure. No em pots estimar si no saps com sóc. I és clar que puc. Només digues-me que em donaràs una oportunitat. En sentir allò qualsevol altra amb l'autoestima equilibrada hauria fugit corrents i hauria notat que anava més cremat del que solen anar els homes a la seva edat, que aquell no era el meu tipus ni de molt lluny. Però encara tenia assumit que si un home em mirava era perquè tenia alguna cosa a la cara[...]"(p.288-289)
--stalker
--nein sagen lernen
--autoestima

"No vaig trigar gaire a veure-m'hi estirada, per què havia d'anar tot tan de pressa? [...] que ja estava només en calces i sostenidor davant d'ell [...] Per què no vaig dir no, encara no, jo no ho vull, això. Volia demostrar que tenia tanta pressa com ell, que a pesar de la diferència d'edat, jo sabia molt bé el que em feia. I no en tenia ni idea." (p.293)
-- immer noch nicht in der lage, nein zu sagen; schaemt sich weil sie keine erfahrung hat


Putes
-----
"Jo ja no recordava que la la mare no em deixava ni dormir de bocaterrosa, que és de putes, deia, gira't de costat, és la millor postura per dormir de manera decent. Feia tant temps que m'explicava allò que jo no sabia ni què volia dir devent. De tant en tant em posava de bocaterrosa, quan tots dormien, i tenia un orgasme recordant el pes de la Laia damunt meu i els seus pits tan erectes tocant els meus, rodons." (p.241)

"Ell havia dit no et vull veure mai més parlant amb un home al mig del carrer, que no sigui dit que la filla del Driouch és una puta qualsevol." (p.269)
--ser puta és una ofença greu i es fa servir molt sovint

"[...] oficialment no podia entrar a cap local on seure i prendre alguna cosa, que una norma no escrita es veu que deia que allò era de putes. Jo començava d'estar-ne farta, de la paraula, i que totes les dones del món fossin la mateixa cosa" (p.285)
--doncs cal reinvidicació!

"Tot eren pregutnes retòriques, valia més no contestar. I et vas quedar allà, al seient del darrere del contxe metre el cor et tornava a bategar ben de pressa, la mà et bategava de dolor, el pare que cridava i tu que no sabies contenir les llàgrimes, escoltant com unsultaven la teva millor amiga, amb ganes de cridar: tu bé que te les has follat totes, les pudents porques cristianes! Bé n'has tingudes tantes com n'has volgut menjant-te-les. Però això no li ho podies dir. Allà vas entendre que el problema real no eres tu ni l'amiga número dos ni la seva vestimenta ni res de res. El problema era que a ell se li havien posat aquells ullets que posava quan li agradava una dona, de tant que l'havies vist així, ja li'n coneixies l'expressió. Ella li agradava i el territori li era tan desconegut que se sabia sense cap mena de possibilitat. Per això era una puta, per provocar el seu desig i que ell no pogués evitar-ho." (p.282-283)

Fuehlt sich nicht zugehoerig; die Frauen, die nie genuegen
----------------------------------------------------------
"La Laia havia dit: em penso que li agrades, a l'Arumí. Et mira d'una manera, cada cop que passem davant seu. Jo tenia clar que no podia agradar a ningú i encara menys a un d'aquí de tota la vida. Hi havia una premissa que m'explicava el món a pesar de les seves incongruències: als moros els agraden totes les dones, però especialment les mores. Als d'aquí, en canvi, no els havien d'agradar mai les mores. Era contranatural. Si no com s'explica que el pare amagués la seva dona de totes les mirades que no fossin cristianes?" (p.243)
-- fuehlt sich nicht zugehoerig

"Hi havia altres motius per creure que jo no podia agrada a ningú: 1) No havia tingut mai cap noviet a classe, cosa que havia passat amb la majoria de companyes a aquelles edats. 2) [...] ningú em feia mai cap petó [...] 3) La mare sempre em feia portar una trena llarga que ja semblava part del meu cos, els cabells ben enrere, duia ulleres i havia fet el canvi tan de pressa que semblava una geganta al costat dels meus companys d'escola, la mare de tots." (p.243-244)

"[...] havia dit a la Laia no em presentes la teva amiga, tan guapa que és? Jo vaig dir te'n fots o què? No, no, ho dic de debò, però sempre em va quedar el dubte perquè ho deia amb aquell mig somriure que fa ell." (p.244)
--fuehlt sich verarscht wenn sie andere menschen huebsch nennen; die frauen sind nie genug: genug jung, schoen, wuerdig fuer liebe

"Escolta, va dir un dia, vols sortir amb mi? Mira que n'ets, de cruel, li vaig dir i ell que no devia entendre res i jo encara vaig entendre-ho menys." (p.245)

Control sobre la sexualitat femenina
------------------------------------
 http://www.elcritic.cat/entrevistes/najat-el-hachmi-tothom-esta-sorpres-que-jo-escrigui-sobre-sexe-per-que-perque-soc-marroquina-3627
 P: A ‘L’últim patriarca’ sembla que la responsabilitat de l’honor i de la vergonya recaigui sobre la dona.
A: Això és un dels pilars bàsics del patriarcat. Crec que la majoria de religions monoteistes ho són; el control de la sexualitat femenina és bàsic.

--vgl Laurie Penny

el moral doble és increïble:
"Si ell es divorciés, la seva dona li hauria de continuar sent fidel fins a la mort; per alguna cosa havia estat ell el primer de tenir-la." (p.151)
--> quina intenció té aquest relat? solidifica estereotips..

A banda d’això, sí, la veritat és que al món musulmà segueix molt vigent la idea que l’home té uns instints naturals que no pot controlar i que és la dona qui s’ha d’encarregar de protegir-se.

altra violació:
"Però el Manel tenia aquell mena d'instint de caçador que han de tenir per força els que estan destinats  a ser gran patriarques i no entenia que era no." (.91)
-- no means no

Atribucions del masclisme: la dona serveix per satisfer les necessitats dels homes
-----------------------------------------------------------------------------------
"Quan ens tornem a veure? [...] Mira que si em dius que no em vols veure més, jo em mato, eh?, i allà també hauria d'haver fugit, però vaig dir demà, a la biblioteca, a quarts de cinc." (p.289)
--sehr verlanged; der anderen seite pflicht und verantwortung aufhalsend

"No em puc creure que jo no m'estimis i jo ja n'estic fart, que tot sigui tan difícil. Jo només el vaig abraçar i vaig repetir que és clar que t'estimo i no vaig gosar preguntar allò que havia estat un camell, que deia el pare." (p.302)
--es faengt an, mit den Anschuldigungen; die Frauen, die die ganze Care Arbeit machen sollen

"Una hora i vint minuts donava per dutxar-se, per a un clau ben ràpid més de compromís que de plaer. Jo no tenia prou temps per excitar-me i aquell fica treu només li anava bé a ell, però jo feia ah, uh, mm, ui, ai. Perquè el que volia més eren abraçades, petons tendres, mirades als ulls d'aquelles que et fan sentir acompanyada al món, única al món. Ell les sabia fer, aquella mena de coses, fins que ja estava tan cansat que després del clau ràpid, ah, ah, aaaah, uau, agafava el comandament de la tele i l'abraçada era a mitges i les mirades escadusseres.
  Després va venir el dia que vaig entrar i hi havia una noia asseguda al llit que fiea de sofà" (p.306)
-- kein spass fuer sie
-- sie will zuneigung
-- am ende werden aber nur die beduerfnisse vom typen befriedigt

Món mascliste: les dones percebudes com a propietat dels homes
--------------------------------------------------------------
"[...] ell que va començar a dir-me que millor que no fos de cara al públic, no? És qu les feines de bars i restaurants, creu-me, són molt pesades, la gent no és respectuosa i a mi no m'agradaria que et vinguessin a dir res, que ets la meva dona, ara." (p.319)

"Jo encara rumiava en aquell parell d'opinions, encara dubtava si no era que volia començar aquella mena de control subtil que solen fer els amrits, quan ja va trucar el pare, quan ja va trucar el pare. La teva mare vol parlar amb tu. Que diu que us ha vist en un bar i que tu anaves destapada i sense mocador i ensenyant el cul a tothom. Filla, pots verstir com una dona casada, si et plau?" (p.320)
-- control de la família segueix

"Vaig penjar i li ho vaig explicar a ell. Va callar, massa que va callar. Potser t'ho hauries de pensar. Què, portar mocador? Vaig riure només d'imaginar-me a mi mateixa d'aquella manera, no hi sabria ni caminar, vestida així. No, el mocador no, però una mica més tapadeta, que la gent ho ha de saber, que estàs casada." (p.320)

"Casada, casada, les dones casades van així, no estudien, no treballen, cuinen molt bé i tenen les coses totes ben endreçades." (p.320)
-- vlt interseccionalitat amb cultura?

"No va venir a picar a la meva porta, no va demanar-me de genolls que tornés, diuen que va desaparèixer. Sï que havia trucat i havia dit allò de mira que si et veig amb un altre jo no ho podré suportar, mira que to ets meva i de ningú més. Jo li vaig dir que mirava massa aquella mena de programes de televisió i que pobre d'ell que se m'acostés, que ja no hi havia res entre ell i jo. No et donaré el divorci, et quedaràs sempre més penjada i no podràs tornar-te a casar. Ni em vull casar ni estaré gaire més temps casada amb tu, que jo ja he damanat el divorci als jutjats d'aquí" (p.326)
-- moral doble

"He portat la teva mare perquè he pensat que si t'havia trobar amb un home aquí potser et mataria i ella sempre impedeix que faci aquesta mena de coses. Un home? He vist que eren les quatre de la matinada i encara tenies els llums oberts, segur que hi ha algun home amb tu, què ha de fer, si no, una dona que viu sola?" (p.328)
-- jaja genau, wenn da kein penis ist, sollte sie sich hinlegen und sterben, ist klar dass es so nicht weiter gehen kann -\_-
-- una dona sense un home no és una persona integra i saludable (vgl "Una dona sense un home és com un peix sense bicicleta")
-- moral doble, per a un home (sol) està acceptat tenir relacions amoroses, per a una dona - no (putes)

Unsortiert
----------
les atribucions del masclisme
sehr interessanter Ehren-Begriff vgl "Feminism is the radical notion that women are people." also bei ihm trifft das definitiv nicht zu.
(Isabel):
"Se li oferia sempre que ell ho necessitava i allò era còmode, però ella ja havia estat d'altres homes abans i Mimoun no havia de preservar cap seu honor perquè considerava que havia nascut sense." (p.156)

"Ella no li havia estat mai fidel, ja havia estat amb altres homes abans que amb ell, perquè li havia de ser fidel ell, doncs?" (p.156)
-- wieder sehr selektive Interpretation von Sachen

el pare té problemes amb pel·lúcules amb gent despullada o contingut sexual:
"El reglament començava a ser confús si tenim en compte que a les pel·lícules que més li agradaven a ell no hi deixaven de sortir homes mig despullats. Bruce Lee, Jean-Claude Van Damme, els mateixos Terence Hill i Bud Spencer." (p.268)

"Sí, ho podem provar, tens els preservatius? És que a mi no em solen anar bé, ja ho saps, em costa molt que m'entrin i se'm fa incòmode. La primera vegada no passa res i et prometo que pararé a temps." (p.299)
--oha. jetzt auch noch penetration ohne kondome.. -\_-
-- necessitats dels homes
-- cap respecte per la dona, les seves necessitats, salut, etc

"Jo sabia que la primera vegada podien passar moltes coses, però no vaig objectar gaire." (p.299)
-- und sie ist immer noch nicht in der lage "nein" zu sagen

"Si no et pots dignar a fer-lo, ni a rentar els plats, ni a fer cap feina de casa, com a mínim ajuda'm a saber què vols per sopar. Que no et faig prous feines a casa? Que no t'he rentat mai els plats? Doncs mira, si això és el que penses de mi, ja t'ho faràs perquè no penso tocar res mai més, faré com fan la resta d'homes i a veure si així estàs contenta." (p.324)
--soso, hat einmal den Abwasch gemacht und verdient jetzt Medaille oder was

"Si anava o no a l'institut depenia de tants factors i cap d'ells tenia a veure amb si jo em portava bé o no, o amb si jo treia bones notes o no, o amb si jo feia cas o no. Estranyes desaparicions havien tingut lloc en els últims dos anys a l'escola i encara gràcies que no m'havia tocat a mi. Desaparicions de noies com jo que venien d'un lloc semblant al lloc on jo vaig néixer però que potser eren molt diferents de mi o devien tenir molta menys sort que jo. Noies que ara tenen tres o quatre fills" (p.271)
-- entre culture i masclisme

"A mi em tocava desaparèixer de l'escenari escolar i encara no sé com no va passar. Un factor era l'avi, que era l'únic que em preguntava què, com han anat els exàmens, ho has aprovat tot?" (p.271)
"La mare ja m'ho havia dit, el teu pare diu que aquest és l'últim any que vas a escola, i era una cantarella que es repetia cada final de curs. Aquest ñes l'últim, i jo deia val, però sabia que no seria així. Potser l'altre factor va ser la professora massa amiga del pare, que en alguna cosa devia influir-lo, que deia la teva filla ha d'estudiar una carrera" (p.271--272)
-- entre culture i masclisme

\end{comment}

\begin{comment}
\subsubsection{Quotes}
%------
"Aquesta és la història de Mimoun, fill de Driouch, fill d'Allal, fill de Mohamed, fill de Mohand, fill de Bouziane, i que nosaltres anomenarem, simplement, Mimoun." (p.7)
-- die Wichtigkeit, das Mystische, sein Stammbaum, seine Geschichte wird ihm entzogen?
-- die Auflistung ist Symbol des Patriarchats

"És la seva història i la història de l'últim dels grans patriarques que formen la llarga cadena dels avantpassants de Driouch. Cadascun d'ells havia viscut, actuat i influït en la vida de tots els que els envoltaven amb la fermessa de les gran figures bíbliques." (p.7)

"Aquesta és l'única veritat que us volem explicar, la d'un pare que ha d'afrontar la frustració de no veure acomplert el seu destí, la d'una filla que, sense haver-s'ho proposat, va canviar la història dels Driouch per sempre." (p.7)
-- el destí té un paper central; es menciona bastant sovint

altra violació:
"Però el Manel tenia aquell mena d'instint de caçador que han de tenir per força els que estan destinats  a ser gran patriarques i no entenia que era no." (p.91)
-- no means no
\end{comment}

\begin{comment}

\subsubsection{Notes:[Penny2014]}

"How are men supposed to cope with this loss of power in a society that still insists that the only way to be a man is to grab as much power as possible, to be rich, to be capable of extreme violence, to dominate other men physically and to dominate women sexually and emotionally? The received wisdom is that they're not supposed to cope. Without power over others, particularly over women, men are supposed to crumble, to lash out, to collapse in an extravagant welter of identity implosion that leaves a suspicious mess on the carpet." (p.64)
-- das schafft M. ziemlich erfolgreich auchwenn er weder in der Schule noch mit seinen sonstigen beschaeftigungen so mega Erfolg hat in der Regel.

"'Patriarchy' does not mean 'the rule of men'. It means 'the rule of fathers' - literally, the rule of powerful heads of household over everybody else in society. Men further down the social chain were expected to be content with having power over women in order to make up for their lack of control over the rest of their lives.
[...]
Most individual men do not rule very much, and they never have. Most individual men don't have a lot of power, and now the small amount of social and sexual superiority they held over women is being questioned." (p.69-70)

"There are two big secrets about 'traditional masculine power' that mainstream culture does not want us to discuss, and it is imperative that we discuss them honestly [...]
The first big secret is this: most men have never really been powerful. Throughout human history, the vast majority of men have had almost no structural power, except over women and children. In fact, the power over women and children - technical and physical dominance within the sphere of one's own home - has been the sop offered to men who had almost no power outside of it." (p.75)

"Thus, a poor man working a job he hated could once expect to feel, at the very least, superior to his wife and children, to be master of his home even if he was treated like a slave outside it." (p.76)
\end{comment}

\subsection{A la intersecció. Les pobres morites}

\begin{comment}
* Identitaetssuche (vgl. versuchen den eigenen Platz zu finden; die Freund*innen angucken; in die Religion; )
\end{comment}

