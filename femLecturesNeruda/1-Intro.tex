\section{Einleitung}

Pablo Neruda, der Nationalpoet Chiles, ist ein der bekanntesten und beliebtesten, wenn nicht der bekannteste überhaupt, Lyriker*innen Lateinamerikas.
Seine Werke werden heute noch rege rezipiert und sind Tausendmal(syn!) von der Kritik kommentiert worden.

%Liebeslyrik

In dieser Arbeit möchten wir Nerudas Liebeslyrik aus einer feministischen Perspektive lesen.
Das Ziel dieser Arbeit(syn!) ist es nicht, die ``richtige'' Interpretation Nerudas Liebeslyrik auszuarbeiten, sondern viel mehr eine Pluralität der Lektüren (in Anlehnung an Annette Kolodny) anzuregen, die Leseweisen ermöglicht, die im androzentrischen Literaturkritikdiskurs verdeckt bleiben.

% Weitere Werkzeuge

Durch das Close Reading exemplarisch ausgewählten Gedichte soll die Poetisierung bestimmten heteronormativen Perspektiven aufgezeigt und in Frage gestellt werden.

\begin{comment}
1. Intro
  * Ziel von Feministischen Lektüren: zugrunde liegende Machtstrukturen in Werken und deren Rezeption aufzudecken
    ** androzentrische Perspektive der Literatur:
       *** Männer in Mittelpunkt (als Figuren)
       *** von Männern gemacht
       *** an Männer gerichtet
  * Wie erreicht? Durch eine Pluralität der Lektüren und Close Reading
ohne das ouevre Nerudas nicht als ganzes in Frage stellen
exemplarische Lektüren: Poetisierung bestimmter Heteronormativen Perspektiven
gehört historisiert; nicht als zeitlos darzustellen
\end{comment}
