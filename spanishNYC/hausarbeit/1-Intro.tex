\section{Einleitung}

\begin{comment}
* ca 3/4 Seiten
* Untersuchungsgegenstand
* Erkenntnisinteresse
* Forschungsstand
* Vorgehensweise: also Section 2 macht blabla, Section 3 blablup, ....
* Ergebnisse können/sollen angedeuten werden
\end{comment}


%TODO
%Generall Intro fehlt!


Das Ziel dieser Arbeit ist eine Analyse der Code-Switches zwischen dem Spanischen und dem Englischen in Artikeln der US-amerikanischen Zeitschrift ``Siempre mujer'' aus einer soziolinguistischen Perspektive.
Es gilt herauszufinden, ob diese Code-Switches in bestimmten Kontexten passieren oder bestimmten rekurenten Mustern unterliegen.
Wenn diese Erwartung erfüllt wird, wird eine sozioindexikalische Erklärung versucht.

Die Ausgangshypothese ist, dass dies sehr wohl der Fall ist.
Im hauptsächlich spanischen Text werden in speziellen Fällen englische Ausdrücke erwartet:
z.B. wenn es um technische Begriffe oder um Idiome/Redewendungen geht, für die keine genaue spanische Übersetzung existiert.
Weitere Fälle, in denen wir Englisch erwarten können, sind region-/kulturspezifische Begriffe: Essen, Tiere und Pflanzen, Bräuche, etc.

Es wäre weiterhin interessant zu untersuchen, in wie fern bewusst ins Englische geswitcht wird, da diese Sprache als prestigeträchtiger wahrgenommen wird oder um einen bestimmten Lifestyle(?) zu inszenieren, wodurch dann auch z.B. versucht wird gewisse Produkte zu verkaufen.

%(Sozioindexikalität!!)

\begin{comment}
Pregunta
--------
Vull investigar si l'altre idioma es fa servir en contextos específics.
La suposició inicial és que sí, que si tenim un text escrit principalment en castellà, trobarem expresions en anglès en casos especials,
per exemple per designar termes tècnincs o frases fixes que no en tenen una traducció exacta en castellà.
Esperem també que observarem la tendecia inversa: que les frases fixes del castellà es mantenen en textos angleses.
Un altre cas de mantenir l'idioma "original" seria si es tracte de coses típiques d'una regió que no es poden trobar als llocs on es parla l'altre idioma (per exemple plantes, animals, menjar, etc.).
A més a més, seria interessant observar si frases de "l'altra llengua" s'utilitzan a causa de tenir un estatut més "alt" o amb l'objectiu de vender un producte.

Vinculació a la investigació existent
------------------------------------
El tema del billingüísme castellà-anglès als Estats Units no és nou i ja existeixen bastants treballs que investigan aquest fenomen.
Per exemple la investigació d'Ana Zentella i Ricardo Otheguy es va concentrar massa al billingüisme i l'ensenyament,
als diferents processos que ocorren en casos de coexistència de més d'una llengua.
Tanmateix, seria interessant examinar de quina manera els mitjans de comunicació (i en aquest cas concret, les revistes feminines) utilitzan dos codis i,
d'aixó intentar a concloure quin és l'objectiu que persegueixin.

------------------------------------

\end{comment}

\subsection{Verwandte Arbeiten}
(Forschungsstand)

\begin{comment}
Was soll dann hier rein?
Ich hab keine Ahnung ob jemand schon so was analysiert hat..
Eine oberflächliche Internetsuche sagt ja :)

Soll das in die Intro? Oder eigener Kapitel? Oder in 2.?
Ich glaub, ich machs in die Intro:
[Mahootian05]
[Ticknor12]
\end{comment}

\subsection{Aufbau der Arbeit}

Im Weiteren ist diese Arbeit wie folgt aufgebaut:
Kapitel 2 gibt einen Überblick über die theoretischen Grundlagen, auf denen die vorliegende Analyse aufbaut.
In Kaptel 3 stelle ich das untersuchte Korpus und die benutzte Methodik vor.
Die Ergebnisse der Untersuchung werden in Kapitel 4 diskutiert.
Kapitel 5 gibt einen Ausblick auf Verbesserungen, Anregungen und Fragen für zukünftige Untersuchungen und fasst nochmal abschließend die Arbeit zusammen.

%\cite[vgl.][s.20]{Tomasello06}
