\section{Marco histórico}

En este capítulo trataremos a dar una descripción del periodo histórico que nos ayudará a discutir el libro \textit{Evita}.
Como hemos señalado en la introucción, el libro fue publicado el año 1952.

Al nivel mundial opera en este tiempo la lógica de la Guerra Fría.
La mayoría de los países del mundo pertenecen a uno de los dos campos - el bloque capitalista de los Estados Unidos o el comunista de la Unión Soviética.
Es difícil resumir un fenómeno tan complejo como la Guerra Fría en dos líneas, pero podemos afirmas que el mundo está completamente dividido y en plan mudnial predomina una atmosfera política de miedo y rechazo total de la ideología adversa.

En este contexto llama la atención que el gorbierno argentino se niega a adoptar una de estas doctrinas.
El peronismo se (auto?)denomina ``la tercera posición''.

En Argentina tenemos en este tiempo la segunda presidencia de Perón.
Aquí destacaremos los elementos claves del peronismo.
El movimiento, en su intensión de reunir más seguidorxs, se basa en un discurso de clase.




\begin{comment}
situar el periodo histórico:

* en el Mundo? - finales de la Segunda Guerra Mundial; principios de la Guerra Fría; división en los bloques capitalista y comunista
  * no sé si es relevante, pero: Argentina también acojió a una multitud de los nazis alemanes después de la Segunda Guerra.

* en América Latina

* que pasa an Argentina?
  * primer Peronismo!:1946-1955 --> las dos primeras presidencias de Perón
    * la clase trabajadora como base del gobierno y poder de Perón; discurso basado en clase, no en "raza";
      * aumenta sus derechos, sindicalismo
    * intención de unir más seguidorxs;
    * "descamisados", "cabecitas negras"
    * pero todavía un discurso hegemonial; la diversidad étnica y cultural no está reflejada de manera relevante; la herencia afroargentina está completamente ignorada; si está reflejada la gente no está en posiciones activa, no está representada como actorxs;
    * signos propios; no se puede presionar/apretar en el continuo tradicional de derecha-izquierda; todavía hoy muy polarizante
      * se habla de una "tercera posición"
    * pero el aparato propagandista parece ser inspirado por el Zeitgeist (vgl propaganda comunista)
    * papel de la iglesia? --> discurso inequívocamente católico; apoyandose en valores católicos; hispanista;
    * amplió los derechos de la ciudadanía
      * se consiguió el voto femenino : 1947
    * control estatal de la economía y los sectores más importantes: ferrocarriles, energía, bancos,..
      * se promueve la industria.. (vgl página sobre deudas en el Libro)

    * creación de Fundación Eva Perón (1948): facilitó ayudas estatales sociales, sobre todo para mujeres, niñxs y ancianxs

* la figura histórica de Eva Perón:
  * migrante del interior, de la clase trabajadora (por eso tiene tanta fuerza reunidora, la gente se identifica con ella)

~\autocite{Chamosa2010}
Ambivalencia
* por el peronismo: se desafía el mito elitista urbano de la nación Argentina
* pero tampoco rescata toda la diversión étnica; se incluye un poco de la herencia indígena, pero nada de la herencia africana; (ej: "Festival de la Pacha Mama"- la parte inca, que ni siquera es parte fundamental para Argentina; imperio grande etc. "civilizado")
* no se transforman de raíz los valores elitistas
* intento a homogenizar
* a lxs pocxs actorxs tampoco se le dio agencia: receptorxs pasivxs de las afecciones y regalos de Evita

* sobre todo: movimiento de la clase trabajadora, alianzas con partes del ejército y lxs pequeñxs industriales
-> gran coalición popular, con amplio apoyo de la parte del interior, que hasta este momento fue ignorada (foco en Buenos Aires)
* polarización super aguda de la sociedad argentina: peronistas vs antiperonistas
* todavía hoy tiene valor como signo aunqué Perón y Eva Perón ya no son sujetos políticos;
* en esta época se construyó un nuevo sujeto político: el pueblo

"The hisotry of Peronism cannot be isolated from the cultural and racial divide that separared the children of nineteenth-century European immigrants from Argentine \textit{mestizos}, more commonly called \textit{criollos}." (p.113)
"As thousands of rural Argentines moved to the metropolitan area of Nuenos Aires and other coastal cities during the first government of Perón, upper- and middle-class Argentines resented the new arrivals, calling them \textit{cabecitas negras}"(p.113)

"a generation of Argentines grew up learning that the culture of the rural criollo from the interior constituted the authentic manifestation of Argentine nationality" (p.114) -- spiegelt sich das im Buch wieder? (gibt es Referenzen auf criollo Kultur?)

"Peronist folklore policy implied a challenge to the historical representtion of Argentina as a white nation." (p.114)

"vindication of criollo culture was consistent with other policies aimed at improving the lifestyle of the working class, criollo or otherwise, and at securing its loyalty to Peronism." (p.115)

"mimicked, the culture of the legendary gauchos of the pampas [...] which first raised the romantic idea that the soul of the Argentine naiton laid in the deep interior" (p.115)

"The Peronist propaganda machine skillfully used the mass media, school curricula, and street demonstrations to secure a loyal following and promote a cult of personality around the figures of the presidential couple." (p.116)

"Perón seemed to adhere to a \textit{hispanista} definition of criollo culture that overemphasized the colonial Spanish influence and understated or ignored the strong indigenous and African legacies." (p.116)

"addressed workers as a social class and emphasized the economic gains that Peronism offered them" (p.116)

"Peronist leaders also embraced the racialized epithet of \textit{cabecita negra} that the urban middle class hurled against their followers." (p.116)

".. the promotion of criollo folklore as the authentic national culture was a clear step toward dignifying dark-skinned workers from the interior" (p.117)

"In this government right-wing Catholic nationalists maintained control of key offices from which they exercised lasting influence over the education and cultural apparatus." (p.122)

"Bolstering Catholicism and Hispanismo were after all important goals among the cultural authorities in the first Peronist government. The key difference between the Peronists and the Conservatives was that the celebration of criollo folklore under Perón accompanied sweeping reforms such as unionization of sugar and wine workers, as well as the regulation of rural day labor, that empowered the rural criollo, at least in the realm of public perceptions." (p.123)

"the Peronism movement responded to this stereotyping by embracing the association betwen Peronist followers and the dark-skinned criollo." (p.123)
"clearly documented in the graphic propaganda" (p.123)
"Several photographs show Eva and to a lesser extent Juan Perón fraternizing with dark-skinned Argentines." (p.123)
--> reflected in the book? eher nicht? preemptied of skin color? abgesehen von einer seite, die offensichtlich ?? darstellen moechte

"The photographs are clearly staged for propaganda purposes, but that should not be equated with hypocrisy." (p.124)
"Eva performed with real passion her role as a 'bridge between Perón and the workers'." (p.124)
"No one from the powerful classes had ever talked to them or treated them with respect until that blond fairy appeared, promising a workers' paradise, and they responded with unconcealed loyalty." (p.124)
"Peronist discourse in relation to workers subordinated ethnis specificity to class-consciousness and loyalty to Perón and Eva. This is apparent in the profuse iconography..." (p.124)
"el \textit{descamisado}, the prototypical male Peronist worker, portray him [...] most commonly as ethnically neutral."(p.124-125)
"In fact, the descamisado identity signature was an unbuttoned white shirt with sleeves rolled up, a marker of material poverty inscribed in class rather than ethnic terms." (p.125)

"Perón vowing 'to fortify the prolific patrimony of the Greco-Latin civilization bequeathed to us, of which we are the perpetuators'."(p.125)
"the core of the traditional patrimony as 'language, religion, the cult of the family, popular poetry, folklore, popular dances, and the cult of the national holidays'" (p.126)

"Peronism was a movement of public demonstrations." (p.127) --> gibts im Buch Bilder, die das wiederspiegeln?
"For events such as the national holidays and the two Peronist \textit{jornadas} (May Day and October 17), openings, festivals, party assemblies, and electoral rallies, people packed the plazas and parks to glimpse, hear and cheer Perón and Eva." (p.127)

"recreate the fundamental basis of the regime's legitimacy: the charismatic leadership of Perón based on his contact with the people without intermediaries" (p.127)

"Although the central act was Perón's or Eva's speech, many ceremonies were more complex and included parades, live music, contests, and even beauty pageants." (p.127)

provincial festivals
agricultural products such as wine, sugar and cotton
"Those festivals predated Peronism, but the new government co-opted them much as it did the May Day celebration." (p.128)

"By tapping into the local festivals, Peronism aimed to cenet its legitimacy as a federal, popular movement by approppriating the symbolic capital generated by community-building rituals." (p.132)

Frauenbild
"queen of labor pageant": ".. the ideal of female beauty and virtue espoused in this official, labor-oriented pageant was not different from the contemporary bourgeois models as expressed in the popular feminine magazines of the time. The press and official statements described the participants as delicate, feminine, and patriotic.. show young women of European descent, some of them with blonde or dyed hair, officials and journalists called them 'criollo beauties'" (p.133)

Nueva Argentina
"The well-funded Peronist propaganda machine called the period that began with the election of Juan Domingo Perón to the presidency \textit{La Nueva Argentina}" (p.138)
"During the Peronist period Argentina experienced sweeping reforms of its political, economis, and social structures, reforms designed to improve the living conditions of the working class." (p.138)
\end{comment}

