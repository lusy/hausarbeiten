\section{Einleitung}

\begin{comment}
* ca 3/4 Seiten
* Untersuchungsgegenstand
* Erkenntnisinteresse
* Forschungsstand
* Vorgehensweise: also Section 2 macht blabla, Section 3 blablup, ....
* Ergebnisse können/sollen angedeuten werden
\end{comment}

Títol i tema
------------
Títol:
"Es hora de decir *bye-bye*" (Fussnote fuer die Quelle http://siempremujer.com/estilo/8-senales-amiga-toxica-en-redes-sociales/72941/)

"El sentimiento general que debe darte *the one* es positivo y feliz." (http://siempremujer.com/amor/senales-es-el-amor-de-tu-vida/72538/)

Utilitzant chunks en anglès a les revistes feminines(Anmerkung, ist das der korrekte Begriff?) dirigides a latines vivints als Estats Units.


Objectiu
--------
Aquest treball intenta a analitzar el context d'us de paraules i expresions en castellà en revistes feminines en anglès dirigides a Latines als Estats Units

i al revés:
el context d'us de paraules i expresions en anglés en revistes en castellà;


Pregunta
--------
Vull investigar si l'altre idioma es fa servir en contextos específics.
La suposició inicial és que sí, que si tenim un text escrit principalment en castellà, trobarem expresions en anglès en casos especials,
per exemple per designar termes tècnincs o frases fixes que no en tenen una traducció exacta en castellà.
Esperem també que observarem la tendecia inversa: que les frases fixes del castellà es mantenen en textos angleses.
Un altre cas de mantenir l'idioma "original" seria si es tracte de coses típiques d'una regió que no es poden trobar als llocs on es parla l'altre idioma (per exemple plantes, animals, menjar, etc.).
A més a més, seria interessant observar si frases de "l'altra llengua" s'utilitzan a causa de tenir un estatut més "alt" o amb l'objectiu de vender un producte.


Vinculació a la investigació existent
------------------------------------
El tema del billingüísme castellà-anglès als Estats Units no és nou i ja existeixen bastants treballs que investigan aquest fenomen.
Per exemple la investigació d'Ana Zentella i Ricardo Otheguy es va concentrar massa al billingüisme i l'ensenyament,
als diferents processos que ocorren en casos de coexistència de més d'una llengua.
Tanmateix, seria interessant examinar de quina manera els mitjans de comunicació (i en aquest cas concret, les revistes feminines) utilitzan dos codis i,
d'aixó intentar a concloure quin és l'objectiu que persegueixin.


%\cite[vgl.][s.20]{Tomasello06}
