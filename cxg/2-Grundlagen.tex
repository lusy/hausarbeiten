\section{Grundlagen}

\subsection{Kognitive Linguistik}
\begin{itemize}
    \item pattern matching
    \item usage based perspective
\end{itemize}


\subsection{Konstruktionsgrammatik}
\begin{itemize}
\item Alle sprachliche Phänomene lassen sich innerhalb derselben Theorie erklären (es gibt keine Kernsprache und Randerscheinungen!)
\item Konstruktion - Form/Bedeutungs-Paar
\end{itemize}

\subsection{Spracherwerb}
\begin{itemize}
    \item Tomasello - Abstraktion aus konkretem Input -> Chunks -> item-gestützte Konstruktionen -> abstrakte Kategorien
    \item Unterschiede/Gemeinsamkeiten L1/L2
    \item Gebrauchbasierte/Konstruktionsgrammatische Perspektive auf dem Zweitspracherwerb
\end{itemize}

Check Grundlagen in \cite{Ziem13}
