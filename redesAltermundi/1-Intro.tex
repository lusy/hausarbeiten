\section{Introducción}

\epigraph{``All bits are created equal.''}{Dicho conocido sobre el principio de la neutralidad de la red.}
\epigraph{``It is democracy that guarantees free Internet and not the other way around.''}{Jens Jessen in~\autocite{Rieder2012}}

% Introducción general
Hoy en día no podemos imaginarnos pasar ni un sólo día sin leer noticias, intercambiar mensajes de chat, escribir correos electrónicos o comunicarse de cualquier otra manera con otrxs.
Estamos siempre en linea y damos esta situación por sentada.
Pocas veces nos damos cuenta quien y como mantiene la infraestructura comunicativa de la que dependemos.
Nuestro derecho a comunicación lo amenazan no sólo regimenes antidemocráticos, sino también otrxs actorxs.
En realidad, el estado de conexión permanente no está normalidad para mucha gente al rededor del mundo.
Lxs que viven en estados autocráticos, en regiones con terreno difícil donde proveedores de servicios de comunicación no ven oportunidad de lucro, o simplemente en barrios pobres, no pueden disfrutar del acceso libre a información o libertad de expresión.

En el espíritu eufórico del ciberoptimismo, muchxs han celebrado el concepto de las redes comunitarias como un modelo alternativo de gestionar la propia comunicación;
un modelo, en/con lo/el cual, lxs usuarixs mismxs construyen y mantienen su infraestructura comunicativa y por lo tanto retienen el control sobre su comunicación.

Cuando hablamos de ``redes'' en el contexto digital cada unx de nosotrxs tiene una noción diferente.
Una gran parte se imagina seguramente las herramientas de redes sociales, las aplicaciones como Facebook, Twitter, Instagram, Google+, Snapchat, etc.
Otrxs piensan en la misma gente que se comunica a través de estas herramientas y relaciones entre estas personas.
En un contexto científico más específico habría seguramente también lxs que recurren a la Teoría de Actor-Red (Actor Network Theory o ANT) de Bruno Latour.
Y aún otro grupo de personas comprende ``redes'' sobre todo en un sentido físico: la infraestructura compuesta por cables, routers, antenas y computadoras, cual es la premisa para la existencia de las aplicaciones mencionadas arriba.

% ubicación científica
El tema de redes comunitarias se ubica entre todos estos conceptos, en el entrecruzamiento de las ciencias de tecnología, la sociología y la informática.

\begin{comment}
por qué es importante/interesante?
	\item Erläutern Sie kurz, in welchem Themenbereich Ihre Arbeit angesiedelt ist. Wo werden Sie einen Beitrag leisten?
	\item Das Ziel sollte es sein, den groben Kontext Ihrer Arbeit darzustellen.

Política digital <-- Gabriella Coleman heranziehen

\end{comment}

% Motivación
Para la investigación actual nos interesarán sobre todo las redes en el sentido de infraestructura física, pero también las redes sociales de personas cuales utilizan, construyen y mantienen esta infraestructura.
Esta imbricación entraña un potencial conflictivo inmenso, ya que, como ya hemos señalado, lxs que controlan la infraestructura, controlan al final la comunicación y tienen un poder enorme.
Un papel importante tienen aquí también las topologías de las redes, como que una arquitectura centralizada concentra poder.
Entonces, surge un antagonismo binario entre decentralización/libertad y centralización/control~\autocite{FiTre2015}.
Justo en este ámbito se ubica el proyecto argentino Altermundi que será el foco del presente análisis.

% brauchen wir hier eine kurzdefinition von community networks?

% Pregunta / hipótesis concreta para el trabajo

Con este trabajo examinaremos Altermundi y las redes cuyo despliegue el proyecto facilita desde la ANT.
Observaremos tanto la capa física de la redes como la capa social de los grupos involucrados.
Situando la idea/el concepto de redes comunitarias en el contexto del ciberoptimismo esbozado/resumido por Bernhard Rieder~\autocite{Rieder2012}, intentaremos un análisis crítico de la controversia acerca de la soberania comunicativa que Altermundi y proyectos similares persiguen.

\subsection{Trabajos previos}

En los últimos años se han realizado varias investigaciones que analizan redes comunitarias desde perspectivas diferentes: técnica, sociológica, ...
Sin embargo, hasta hoy hay pocos trabajos que se dedican específicamente a la realidad latinoamericana con sus idiosincracias y desafíos particulares.
%stimmt das??

El libro ``Wireless Networking in the Developing World'' por ejemplo incluye un estudio de caso venezolano~\autocite[437-452]{WNDW2013}, no obstante, el trabajo es más un manual para el despliegue de redes inalámbricas con los recursos disponibles en comunidades locales y un presupuesto limitado, explicando en detalle las tecnologías necesarias, entonces bastante técnico, y no una investigación desde la sociología o las ciencias de tecnología.

Vamos a mencionar aquí aquellas investigaciones que se han estudiado y que nos han de interesar como que abordan uno o varios aspectos que intentaremos a analizar.

Uno de los trabajos que analiza redes comunitarias desde una perspectiva social es el artículo ``Expanding the Internet Commons: The Subversive Potential of Wireless Community Networks'' de Primavera de Fillippi y Félix Tréguer~\autocite{FiTre2015}.
Lxs investigadorxs se centran en el contexto europeo y nos dan una perspectiva general sobre las diferentes motivaciones, formas de organización y regulaciones a las que los diferentes proyectos están sometidos.
Hacen un análisis desde la teoría de los bienes comunes (``commons'') y concluyen su investigación con una lista de propuestas para las legislaciones nacionales que resume cómo ellas podrían facilitar el trabajo de los proyectos de redes comunitarias.

\subsection{Estructura de la monografía}

Este trabajo está organizado de la siguiente manera.
Primero, presentaremos el marco teórico que nos ayudará analizar el proyecto Altermundi en el contexto de las ciencias de tecnopolítica.
Abordaremos el concepto de bienes comunes y de ciberoptimismo como la Teoría de Actor-Red (ANT) y haremos una reflexión sobre instituciones y jerarquías.
Seguiremos observando las redes comunitarias wifi y la historia/génesis de Altermundi.
Después, intentaremos un análisis del proyecto y su entorno a través de la Teoría del Actor-Red y ...
En conclusión, ...

