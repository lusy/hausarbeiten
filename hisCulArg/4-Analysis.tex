\section{\textit{Evita}: un instrumento de la construcción de la nación durante el peronismo}

Un ejemplo particular del proyecto de consolidación de la imagen nacional argentina creada por el peronismo es el libro de primer grado infantil \textit{Evita}~\autocite{Albornoz1952}.

Nos dedicaremos en este análisis breve a las preguntas siguientes:
¿quiénes actorxs/protagonistas están dibujadxs?
¿está reflejada la realidad social en su complejidad?
¿la heterogeneidad cultural?
¿a quién se dirige?
¿cuál es la imagen de Argentina que pinta el libro?
¿cómo estan construidos las imágenes centrales de Eva y Juán Perón?

% quienes protagonistas estan dibujadxs/esta reflejada la complejidad social
Tenemos en el libro dos tipos de protagonistas: los personajes anónimos y las personas nombradas.
Lxs anónimxs respresentan el pueblo: la base popular que apoya el gobierno peronista y que está apoyada por ello.
Se trata aquí de un grupo sin nada de particular, caracterizado solamente por su amor de la patria, de los valores peronistas y de lxs líderes del regimen.
El rasgo común de todxs es la risa permamente: que nos señala que lxs protagonistas están contentxs, no puedan pedir nada más, viven en un paradiso en la tierra.
Están dibujadxs muchxs niñxs, es, después de todo, un libro infantil.
Todxs....
Aunqué el peronismo afirma lxs criollxs del interior como su base, como ya hemos resumido en capítulo..., su discurso se construye sobre todo en términos de clase.
La compleja realidad étnica del país no se incluye de manera directa en las políticas o el discurso.
También el libro está vacío de características étnicas (con una excepción: personas en vestidos étnicos en página 52).
Lo que también llama la atención es la falta de agencia por parte de estxs protagonistas.
Son siempre lxs beneficiarixs pasivxs de los regalos, afectos y la bondad, sobre todo de la figura de Eva, y en menor grado de Juan Perón, presentadas en el libro, a lxs que miran con rostros vueltos hacia arriba y miradas de adoración (p.65).
%TODO: buscar los atributos del descamisado! p.59!
% clase trabajadora!
%* "todos deben producir"
%* p39: "La Patria pide a los niños argentinos que produzcan lo más posible. Así colaboran al bienestar común."

Los personajes mencionados por su nombre que aparecen en el libro son exclusivamente personajes claves para la formación del Estado-nación argentino.
Domingo Faustino Sarmiento, Manuel Belgrano, José de San Martín todos tienen una página dedicada.
Y, supuestamente aparecen también Eva y Juan Perón.
No podemos hojear tres o cuatro páginas sin que aparezca al menos una vez unx de ellxs o ambxs.
Y cada vez están destacadas sus logros, sus virtudes y el amor incondicional que merecen y que el pueblo le da.

Tanto a Eva como a Juan Perón se refiere con un conjunto/campo/red de epítetos repetidos.
Eva es ``la mamá'' (p.18), ``nuestra Madre Espiritual'' (p.10), la ``madre espiritual de los niños'' (p.75), ``jefa espiritual de la nación'' (p.71).
Su imagen esta estilizada en términos de maternidad y espiritualidad religiosa.
Las características de maternidad no son arbitrarias: la madre es la persona que da nacimiento (fíjase en la relación con ``nación''), una de las personas más importantes para la formación de lxs niñxs.
% obviamente después del fallacimiento de Eva Perón, se le dedica una página; la muerte ha convertido el culto de persona más fuerte? quizá no, se nota lo mismo también para Juán Perón
De hecho, Eva y Juan Perón están varias veces representadxs como padres figurativxs de cada uno y cada una, de toda Argentina:
en página 3--``Mi mamá, mi papá, Perón, Evita'', ``Mamá y papá me aman. Perón y Evita, nos aman.'' (p.7), ``Mi hermanita y yo, amamos a mamá, papá, Perón y Evita.'' (p.8).
El culto a la personalidad y las semejanzas con las representaciones de la Virgen María son tan fuertes, que es fácil de comprender las inquietudes de la iglesia católica al respeto~\autocite{Chamosa2010}.
Además, se le atribuyen cualidades con conotación univocamente femenina.
Ella es ``buena'' (p.15), ``cariñosa'' (p.15), ``querida'' (p.67), ``amó a todos'' (p.13).

Juan Perón, por otra parte, es el líder, presidente, general, buen gobernante.
Tiene un papel más ``fuerte'', típico ``masculino''.
No podemos dejar de notar los papeles estereotípos de género que operan aquí.
Mientras Eva es la compasiva, la madre, Juan Perón es el líder activo que se ocupa de actividades ``masculinas'': el ejercito, las deudas, etc.
% deminutivo
%* nominación: "Evita y Perón" : la mujer es la que está denombrada por un deminutivo: cute, no tomada en serio? pequeña... (podemos notar la tendencia que sigue... "Dilma vs Temer", "Hilary and Trump")
Sin embargo, lo que es nuevo es que Eva como mujer no esta relegada al privado, sino tiene un papel activo político/público;
La vemos en representaciones de manifestaciones y discursos públicos, se destaca su trabajo para el voto femenino.

En las representaciones de ambxs es interesante también la referencia repetida al trabajo:
Evita es ``Mártir del trabajo'' (p.22),
Juan Perón es ``Primer Trabajador Argentino'' (p.35).
Esto quiere destacar su vinculación con la clase trabajadora, se representan como una parte de ella.

Se transmite la impresión que lxs niñxs y el pueblo se pueden comprar
* wiederkehrende Motive
  * "juguetes para mí" (p.76)

% Imagen de la nación
* símbolos estatales "de la Nueva Argentina" dibujados explicitamente: p.18-19

Bellezas de la patria (p.56-57) und am Anfang
  * "bellezas de la patria": sehenswuerdigkeiten --> mito de la estado-nación p.24.25

* cuál es la imagen de Argentina que pinta el libro?
  * modernidad!
    * revolución industrial: caminos, ferrocarriles, ..
    * los libres como artefacto material son representaciones de la modernidad (alfabetismo, la circulación masiva de medios impresos?)
  * los espacios que sirven para crear Argentina como nación (vgl las pampas y los parques naturales): página con sehenswuedrigkeiten
    * campañas del desierto, reimaginar las pampas; el estado reconfigura ele espacio, "civilizar" las pampas
* Kapitel ueber infrastruktur: "Caminos", "Recuperación nacional", "El buey"; dibujos de ferrocarriles, carros, coches, cabales <--- modernidad inszeniert!/Industrialización: auch teléfono, avión, ferrocarril, barcos (p.72)

* días festivos (utilizados para establecer la nación) mencionado explicitamente und in relación con el Peronismo p.22 (auch am Ende Kalender), p.35, 


% a quien se dirige
Aunqué el libro está claramente dirigido a lxs seguidorxs del peronismo,
hecho señalado por el lenguaje y los temas abordados,
es interesante observar que esto no se nota necesariamente en los dibujos de las figuras.
Con pocas excepciones (las figuras en vestidos étnicos en página 52 y el \textit{descamisado} dibujado en p.59),
las figuras (syn!) suelen mostrar un aspecto más burgués que de clase trabajadora.


\begin{comment}

* "Policlínico Evita" (p.43), "Fundación Eva Perón"(p.43), "clubes infantiles Evita" (donde se practica futbol^^)(p.55)
* se rescatan los éxitos (el voto femenino, el pago de las deudas,...)

* p.10: oraciones, rezar, catolicismo; pero también aquí "nunca olvido a Eva Perón, nuestra Madre Espiritual"
  * gottesgleichheit (hat anscheinend die kirche etwas verstoert^^ -- Chamosa o FF?)
  * kann alle wuensche verwirklichen
\end{comment}
