\section{Befunde}

% vlt anderer Titel, wenn man genau weiß, was die These ist
%\subsection{Hauptthese}

Die automatisch erkannten Vorkommnisse von englischen Floskeln wurden einzeln gesichtet und in Kategorien unterteilt.
Insgesamt wurden in den 1419 Artikeln 3271 eindeutige englische Tokens identifiziert. %TODO Tokens oder Vorkommnisse?
Das heißt, wenn das selbe Token in einem Artikel öfter vorkam, wurde es hierfür nur einmal gezählt.
Im Kapitel~\ref{chap:results-desc} werden diese Kategorien beschrieben und exemplarisch dargestellt.
Anschließend werden in Kapitel~\ref{chap:results-analysis} verschiedene Erklärungsversuche dafür angeboten.


\subsection{Ergebnisse beschreiben}
\label{chap:results-desc}

Im Folgenden wird ein (quantitativer/qualitativer?) Überblick über die identifizierten Kategorien von Code-Switches und dazugehörige Beispiele gegeben.

%TODO:
% Bsp für jede Kategorie aufzählen
% auf Deutsch übersetzen

\subsubsection{Named Entities/Eigennamen}
Eigennamen bilden die erste und zugleich umfangreichste Kategorie von gefundenen Code-Switches.
Es handelt sich dabei um Namen von Organisationen, Produkten, Medien, Veranstaltungen oder US-amerikanischen Feiertagen.
Einige Beispiele hierfür sind ``Kindle'', ``Maybeline Fit Me Foundation'', ``Converse All-Star''; ``American Jourcnal of Clinical Nutrition'', ``Daily Mail'', ``The Huffington Post'', ``Fitness Magazine'';
``Sony Music'', ``Washington State University'';
``The Girl with the Dragon Tattoo'', ``Fifty Shades of Grey'', ``Letters to Juliet'', ``Frozen'', ``American Idol'', ``Animal Planet'', ``BBC News'';
``New York Fashion Week'', ``Super Bowl''
``Times Square'';
``Labour Day'';

\subsubsection{Entlehnungen aus dem Englischen}
Beispiele hierfür wären die Vokabeln ``unisex'', ``party'', ``piercing'', ``fitness'', ``blazer'', ``fan'', ``club'', ``shock''\footnote{All dieser Wörter werden beim Online-Wörterbuch \url{https://pons.eu} als Spanisch aufgelistet. \url{https://leo.org} führt alle bis auf ``party'' und ``fitness'' als Spanisch. Und die Online-Version des Wörterbuchs der konservativen Real Academia Española \url{https://dle.rae.es} erkennt alle davon bis auf ``piercing'' und ``fitness'' als Spanisch an.}.
Manche davon, wie ``piercing'' oder ``blazer'', beschreiben Gegenstände, für die es kein einzelnes Wort auf Spanisch existiert.
Bei anderen, wie ``party'', ``fan'' oder ``shock'',  ist die Entlehnung vielleicht weniger einfach zu erklären, da es entsprechende Äquivalenten auf Spanisch gibt (nämlich ``fiesta'', ``aficionadx'' und ``choque''). % also warum werden sie dann benutzt??

\subsubsection{Nahrungsmittel}
Eine andere umfangreichere Kategorie, die vielleicht nicht vom Anfang an erwartet wurde, ist ``Essen''.
Zunächst haben wir es hier mit Lebensmitteln in Rezepten oder Dietprogrammen zu tun, die oft auf Englisch, oder auf Spanisch mit der Englischen Übersetzung in Klammern erscheinen.
Oft sind es unkonventionellere/spezifischere Lebensmittel, die in diese Kategorie auftauchen: ``kale'' (Grünkohl), ... (aber auch nicht ausschließlich, wir haben hier auch ``banana'', ``steak'' oder ``baking soda''). % zu prüfen: eigentlich sind gar nicht so viele weirde sachen am start!

%% Lifestyle Essen
Manche der Switches in dieser Kategorie haben etwas mit einem gesunden Lebensstil zu tun:
Es handelt sich dabei um Sachen wie ``baby greens'', ``pavo wild'', ``gluten free'', ``smoothie'', ``olive oil'', ``light'' etc.
Andere wiederum werden eher mit einer Atmosphäre von ``Coolness'' oder ``Hippness'' assoziiert:
``brunch'', ``coffeehouse'', ``dip'', ``snack'', ``steak'', ``bacon'', ``mint julep'', ``Dirty Martini''

%% US Zeug
Nicht zuletzt gibt es unter den Nahrungsmitteln auch welche, die vielleicht als spezifisch US-amerikanisch identifiziert werden können (oder zumindest sehr stark mit der US-amerikanischen Kultur in Verbindung gebracht werden) und für die es dementsprechend vielleicht auch gar keine spanischen Begriffe existieren.
Beispiele hierfür sind ``brownies'', ``cheesecake'' oder ``peanut butter''.


\begin{enumerate}
  \item Named Entities, dadrunter products (heap of commercials); journals, magazines, newspapers; organizations; book and movie titles; TV-Sendungen; Events; Musik; Orte; Läden/Firmen; Youtube Videos; US-amerikanische Feiertage;
  \item Direct quotes:
      Foto titles ``A photo posted by..''
        tweets
  \item English vocab which has more or less entered Spanish (presumably): (unisex, piercing, party)
  \item English discourse markers as part of the Spanish text:
      ``anyway'', ``must''
  \item English interjections as part of the Spanish text:
      ``me di quenta que wow!'', ``pense wow esto es lo que quiro''
  \item Half translated English idiomatic expressions
      ``vivimos el presente a full'', ``la voz en off'', ``casarse y vivir su happily ever after'', ``que se embarcan en un road trip''
  \item English Idiomatic expressions / collocations
    * "Mix and match , la pareja ideal"
        ``son verdaderas decision makers''
        ``echale un vistazo al behind the scenes''
        ``no es a big deal''
  \item Whole phrases in English, not necessarily idiomatic % todo: check: what's idiomatic anyway?
  \item Misc: one English word without category at the moment
  \item "One night stand"/sex stuff (72933) + dating
      ``sexy'', ``sex appeal'', ``hot'', ``dating'', ``crush''
        ``ya has encontrado a the one''
  \item Cool/slang/hip/life-style
      ``cool''
  \item Further TV/Media/Movies stuff
      ``reality'', ``show'', ``TV''
  \item Music stuff
      ``country'', ``pop'', ``folk'', ``jazz'', etc.
  \item food
    \begin{itemize}
      \item Organic/healthy/lifestyle bla
      \item und dann meat; fish; fruit; vegetables; candy; junkfood; drinks; herbs and spices; misc
    \end{itemize}
  \item Diet
    \begin{itemize}
      \item Products
      \item Diet names
    \end{itemize}
  \item Fashion stuff
    \begin{itemize}
      \item Clothing
          ``outfit'', ``jeans'', ``
      \item Hair
          ``bob'', ``pixie''
      \item Colors
          ``nude'' ``coral reef'', ``snow shadow'', ``cherry red''
      \item Makeup
          ``smoky eye'', ``cat eye''
      \item misc: trendy; look; vintage
          ``fashion'', ``trendy'', ``look'', ``vintage'', ``look'', ``glamour''
    \end{itemize}
  \item Cosmetic stuff:
      ``shampoo'', ``frizz'', ``cleanser''
  \item Personal descriptions
      needy, creepy, charming
  \item Tech/Internet stuff (strange, would've expected more of this to have appeared by now; possibly tokens cathegorized as UNK-> have a look)
      ``chip'', ``software'', ``clic'' <--- schon mit spanischer Ortographie!
        ``LED'', ``upgrade'', ``laptop''
    \begin{itemize}
      \item general
      \item Social media:
          ``hashtag'', ``trending topic'', ``likes'', ``postear''
      \item Internet
          ``wifi'', ``meme'', ``online'', ``live stream'', ``hacker'', ``email'', ``chat''
      \item Smartphone
          ``app'', ``smartphone'', ``selfie''
    \end{itemize}
  \item Fitness stuff:
      ``spinning'', ``gym'', ``workout''

  \item Berufe/Stellenbezeichnungen: ``nanny'', ``babysitter'', ``dog walker'', ``coach de lifestyle''
  \item Pets
  \item Electrodomestics
      ``juicer'' (in Klammern, next to the spanisch text), ``microwave''
\end{enumerate}

\subsection{Interpretation}
\label{chap:results-analysis}

Die vorgefundenen Code-Switches lassen die Vermutung aufstellen, dass es sich bei den Artikeln um eine Art Performance handelt, die auf ein bestimmtes Zielpublikum (bilinguale Latinas, die einen sozialen Aufstieg und Anpassung an die dominante Leitkultur anstreben?) zugeschnitten ist.
%Welche Wirkung versuchen diese zu erzielen
Da es dabei um Kommunikation in geschriebener Sprache geht, sind die Code-Switches ganz bewusst ausgewählt und benutzt worden (vgl.~\cite{Mahootian05}) und sind nicht das Produkt einer spontanen Sprachproduktion, wo der Sprecherin ein Begriff nicht eingefallen ist.

Es können für die vorgefundenen Kategorien von Code-Switches verschiedene Erklärungen geliefert werden.

%TODo eher nach Erklärungen aufteilen, nicht die Kategorien von oben wiederholen:
\begin{comment}
% Leichtere Wiederekennung
Bei den Named Entities aber auch bei bestimmten andere Sachen, wie z.B. Rezeptzutaten
% Entlehnungen für Begriffe, für die es auf Spanisch keinen genauen Äquivalent existiert
% Stil, Zugehörigkeit zu einer bestimmten Gruppe/Lifestyle
Hier kann man die Sozioindexikalität und die indexikalischen Felder wieder hervorrufen;
Zugehörigkeit auf verschiedenen Ebenen:
- Evokation(ist das ein Deutsches Wort?) einer bilinguale Identität/Komplizenschaft zwischen Herausgeberinnen und Leserinnen
- Zugehörigkeit zu einem hippen erfolgreichen Lifestyle (vlt geht das bereits auch in Prestige über?)

% Prestige

% Aufmerksamkeit erregen durch den Code-Wechsel
\end{comment}

\subsubsection{Leichtere Wiedererkennung}
% Named Entities; aber auch gewisse Nahrungsmitteln, oder andere Produkte, die die Leserinnen auf einem Englisch-sprachigen Markt konsumieren sollten.
Der erste Erklärungsversuch geht zunächst von der Gruppe der Named Entities aus.
Diese hat man vielleicht am Anfang nicht mitgedacht, sie sind jedoch kaum überraschend.
Es existieren gewiss für einige davon auch spanische Übersetzungen (zum Beispiel für die Film- und Buchtiteln, einige Produkte, ...).
Die Publikation richtet sich allerdings an ein Publikum, das diese Produkte auf dem US-amerikanischen Markt erlangen sollte, bzw. in einem dominierend englischsprachigen Kontext lebt und deshalb auch mit höherer Wahrscheinlichkeit einfach die englischen Namen der Filme oder Bücher kennen würde.
Deshalb erscheint es nur logisch, dass sie auf Englisch benannt werden.

Die leichtere Wiedererkennung kann auch hinter den Code-Switches bei bestimmten (vielleicht seltener vorkommenden) Nahrungsmitteln, z.B. in Rezeptzutaten oder Diätanweisungen oder.... vermutet werden, da die Leserinnen diese auf einem Englisch-sprachigen Markt konsumieren (und finden) sollten.
Es handelt sich bei manchen dabei vielleicht um Nahrungsmitteln, die in den Ursprungsländern der Leserinnen nicht besonders populär sind und demzufolge die Menschen kaum die spanischen Namen kennen würden.
Wenn frau in der Lage sein sollte in den Geschäft zu gehen und Produkte auf Englisch zu kaufen, ist es durchaus sinnvoll zu wissen, dass ``col risada'' auf Englisch ``kale'' heißt.
Insgesamt kann man sagen, dass es sich hier um ein Versuch handelt, Zutaten wirklich klar zu stellen und Missverständnisse zu vermeiden.

\subsubsection{Entlehnungen/``Linguistic necessity''}
% Entlehnungen für Begriffe, für die es auf Spanisch keinen genauen Äquivalent existiert
% alt. Titel: Der Begriff existiert auf Spanisch nicht.
Wie wir bereits gesehen haben, handelt es sich bei manchen Code-Switches um englische Vokabeln, die bereits größtenteils ins Spanische eingedrungen sind und mehr als Entlehnungen funktionieren.
Manche davon werden wir nach Zentella (cite!!) mit dem Fehlen eines exakten Äquivalents auf Spanisch erklären können, andere vielleicht durch linguistische Sparsamkeit. %eigentlich ist linguistische Sparsamkeit bereits ein separates Argument
Einige davon kommen aus dem Feld der neuen Technologien und die entsprechenden Geräte oder Phänomenen sind ursprünglich in einer englisch-sprachigen Umgebung (z.B. in der Sillicon Valley) entstanden und demzufolge wurden sie auch zunächst (nur) auf Englisch benannt.
Die Konzepte/Gegenstände wurden von da aus in anderen Teilen der Welt übernommen, und da sie vorher nicht existiert haben, werden auch gleich ihre Bezeichnungen mitentlehnt.
Manche davon werden schlussendlich auf der lokalen Sprache übersetzt (z.B. ``computadora'' oder ``en nube'' auf Spanisch), bei anderen wird die Benutzung der ursprünglichen Bezeichnung unverändert verstetigt (z.B. ``internet'', ``software'', ``wifi'').
Dieses Phänomen (oder diese Kategorie) können wir vermutlich bei jeder beliebigen Sprache finden und nicht nur im Spanisch-Englisch Kontext der Zeitschrift \textit{Siempre mujer}.
Selbst wenn ein Wort für ein Gegenstand oder Phänomen auf der Lokalsprache (hier Spanisch) existiert und sich etabliert, werden die englischen Bezeichnungen trotzdem oft wiedererkannt und parallel weiterverwendet.

% 1) Most items reflect a cultural reality that is new or different, e.g., [kei/keike/keiki]
% «cake», [bobipín] «bobby pin», [ŷins/bluŷines] «(blue)jeans.»
Andere Felder, in denen Entlehnungen oft vorkommen sind...
Beispiele dafür sind...

\subsubsection{Zugehörigkeit zu einer bestimmten Gruppe/Lifestyle}
Eine Zugehörigkeit zu einer bestimmten Gruppe bzw. Lifestyle werden bereits auf mehreren Ebenen evoziiert/inszeniert.
Wie von~\cite{Ticknor12} gefunden, wird durch eine Benutzung des bilingualen Codes vermutlich versucht, eine Art gemeinsame Identität/Komplizenschaft zwischen den Herausgeberinnen der Zeitschrift und den Leserinnen herzustellen.
Schließlich gehört Code-Switching eben zu den Charakteristiken einer bilingualen Latina-Identität.
Für solche Art Switches wäre es im Grunde egal, was genau geswitcht wird, Hauptsache, die kommen vor, und unterstreichen dabei die Gemeinsamkeit mit den Leserinnen.
Zu einer Identitätsbildung gehört oft ``to identify themselves as a group separate from their predecessors’ generation''~\cite[]{Mahootian05}.
Und grad bei den Englisch Switches in dem vorwiegend Spanischen Text kann man wieder meinen, dass sich die Sprecherinnen (bzw. in dem Fall die Zeitschriftredakteurinnen, die diese 2.generation Latina-Sprecherinnen erreichen möchten) sich als modern, hip und erfolgreich projizieren.

Darüber hinaus wird eine nächste sozioindexikalische Ebene aufgemacht, in der gewisse (englische Begriffe) für einen (erfolgreichen), hippen, gesunden Lebensstil stehen. % (vlt geht das bereits auch in Prestige über?)
Die Benutzung von Vokabeln wie ``party'' oder ``fitness'' kann vielleicht mit der Inszenierung so eines Lebensstils/Life Styles erklärt werden.
%TODO weitere beispiele?

% Look
Auffällig oft kommt in den Artikeln das Wort ``look'' vor.
Die Benutzung von dem, zusammen mit anderen Vokabeln (Bsp!) aus der Kategorie ``Cool/hip/Life-style'', versucht, die Leserschaft der \textit{Siempre Mujer} als junge, coole, hippe, gesund lebende, moderne, schöne, erfolgreiche ... Superfrauen zu inszenieren (syn!)/darzustellen.
``Look'', also das äußere Aussehen, wird besonders bei Frauen oft mit Erfolg (oder sein Fehlen) gleichgestellt, wie die britische Periodistin und Autorin Laurie Penny argumentiert:

\begin{quote}
For modern women in this axious age, the makeover is a ritual of health and devotion and social conformity.
[\ldots]
Cosmetic surgery companies plaster public transport with promises to deliver not just physical changes, but emotional ones like `confidence'.
Fashion editorials advise us to spend money we don't have on skirt suits and handbags as `investment pieces'; you're not supposed to dress and style your body simply to please yourself but with one eye on your financial future.
That skirt suit really is an `investment' in a one-woman business whose product is you, only glossier.
This is what power, health and success means to the modern, emancipated woman: terminal exhaustion and a wardrobe full of expensive disguises.~\cite[p.41-42]{Penny14}
\end{quote}

%TODO Laurie Penny + Kapitalismuskritik ins Fazit auslagern?
%Otherwise: re-write rest of this subchapter

%Kapitalismuskritik
Man kann die oben anlautende Kritik erweitern/weiter denken/verfolgen/... und ... daran anknüpfen:
Schlussendlich sind die hier erweckten/angedeuteten/.. Werte wie Flexibilität, (hip, jung), Mobilität, Kreativität, .. vom Kapitalismus aufgegriffen und angeeignet worden.
Wie Luc Boltanski und Ève Chiapello schreiben: .. ~\cite[]{BolChi07}.

% Sex
Eine weitere spannende Kategorie habe ich ``one night stand'' bzw Sachen, die mit Sex zu tun haben benannt.
Wir kennen den platten Ausdruck ``sex sells'' und die kapitalistische Gesellschaft tut Frauen gnadenlos als sexualisierte Objekte ausbeuten (besser ausführen, nach Kapitalismuskritik).
Warum aber werden Wörter und Ausdrücke aus diesem semantischen Feld auf Englisch gebraucht?
Ich würde hier wieder das Argument mit der Prestige, bzw coolness heranziehen. (vgl auch Laurie Penny!)


\subsubsection{Prestige}
Ohne Zweifel ist in den USA Englisch die angesehenere Sprache/ist besser angesehen als Spanisch.
Sie ist die offizielle Sprache im Land und wird von der Mehrheit der US Bevölkerung gesprochen.
Spanischsprachige Bürgerinnen und solche, die als Latinas identifiziert werden, werden öfter auf dem Arbeitsmarkt aber auch im alltäglichen Leben systematisch diskriminiert.
Wer gut Englisch beherrscht (und am besten noch relativ weiß aussieht und einen angelsächsischen Namen hat), hat bessere Chancen, einen Job zu finden und sozial aufzusteigen.
Wiederum wird Spanisch mit Marginalisierung, Armut, Rückstand, fehlender Bildung und dem nicht-weißen Anderen asoziiert~\cite{Zentella07}. % so für das alles brauchen wir eine Quelle.
Zentella spricht von einem ``remapp[ing of race] from biology onto language''~\cite{Zentella07}
und macht klar, dass die generelle Erwartungshaltung vorherrscht, von Medien und Institutionen gepusht, dass sich Menschen anstrengen und gut Englisch lernen, um respektiert zu werden.
Diese Tatsache lässt uns die Vermutung aufstellen, dass öftere Benutzung Englischer Vokabeln (egal aus welchem Gebiet) eine gewisse Überlegenheit in der sozialen Hierarchie signalisieren/andeuten sollte, bzw. dass Menschen, die auch so reden, zu verstehen geben, dass sie Englisch beherrschen und somit denen ein sozialer Aufstieg offen steht.

% English discourse markers;
Genau damit würde ich die Kategorie der englischen Discourse Markers (DE?) erklären.
Sonst ist es kaum nachvollziehbar, warum im Text ``anyway'', ``must'' oder ``pros and contras'' erscheinen sollten, anstatt auf diese komplett zu verzichten, bzw. die spanischen Übersetzungen zu benutzen. % oder mit Attention devices!
Bei gesprochener Sprache kann man sich diese Gebrauche noch mit spontanem Ausdruck erklären, bei dem Medium Zeitschrift jedoch, wo die Autorinnen der Artikel Zeit zum Nachdenken hatten und der Text vermutlich einen (mehrstuffigen) Redigierprozess durchlaufen hat, kann von Spontanietät kaum die Rede sein.

\subsubsection{Aufmerksamkeit erregen durch den Code-Wechsel}
Eine andere Erklärung, zB auch für Englische Interjektionen u.ä. ist, dass sie Aufmerksamkeit erregen einfach weil sie nicht Spanisch nicht~\cite[]{Mahootian05}. % vgl auch hk magazines

% (halb)übersetzte englische Idiome/Redewendungen (iwie mit der oberen Kategorie verschmelzen)
Vielleicht können damit auch einige (zur Hälfte übersetzt oder komplett auf Englisch) englische Idiome und Redewendungen erklärt werden.
(Bsp!)
Eventuell werden manche davon benutzt, weil sie sich in dem konkreten Kontext besonders gut dafür eignen, um einen Sachverhalt zu veranschaulichen und keine genaue Übersetzung auf Spanisch existiert.
Das ist jedoch nicht immer der Fall und Redewendungen, die zur Hälfte auf Spanisch und zur Hälfte auf Englisch abgedruckt sind (``vivir su happily ever after'', ``la voz en off''), dienen höchstwahrscheinlich einfach als ein Mechasnisum zum Erregen der Aufmerksamkeit.

\begin{comment}

% Anglizismen als neutralisierende Begriffe zw. Spanischen Varianten?
       5) Anglicisms can play the role of neutralizer between competing dialectal variants because the
       prestigious outside language acts as the lingua franca that resolves the conflict without favoring one
       group at the expense of the other.
      * influence of the media: "they do contribute to popular acceptance and use of some new vocabulary" [Zentella90]
        (siehe oben bei Spanglish)
        -> Spielen auch für leveling eine Rolle: versuchen neutrale Varianten zu nutzen, um möglichst mehr Menschen zu erreichen; und die ist manchmal eben auf Englisch?

------------------
{Andr07}
"Language mixing is no doubt part of the symbolic capital that
  lifestyle magazines like Latina (the ‘Magazine for Hispanic Women’) and rap
  stars like N.O.R.E. (‘Oye Mi Canto’) sell to their audiences."

"A few media genres seem to approach this
  authenticity, e.g. live radio talk or computer-mediated interaction; but in
  advertising, television shows, song lyrics, movies or fashion magazines, the
  decision to use two or more languages is subject to careful planning, editing
  and staging. What could be less authentic than that?"

According to his classification of media, mein Fall fällt hier rein:
"Private commercial media conceive of their
  audience as consumers rather than citizens. Broadcasting a meaningful
  programme is far less important than attracting the attention of potential
  consumers of the media output and of the products and services advertised
  therein."
anders als non-commercial community media, die ein emanzipatorisches Bildungsprojekt haben

"Impersonal bilingualism proliferates in
both the commercial and the non-profi t sector, its most prominent forms being
commercially framed (e.g. popular music, advertising, lifestyle magazines)."

and according to media genres classification:
"The fourth group encompasses various non-fi ctional genres of written
discourse in e.g. ethnic, fan or fashion magazines as well as mainstream
newspapers (section 3.6);"

"Minimal bilingualism
in media discourse often responds to (factual or assumed) limited language
competence on the part of the audience, and exploits the symbolic, rather
than the referential, function of language (cf. sections 3.2, 3.6). This is
sometimes achieved by its use as a framing device (Coupland et al. 2003:
167): tiny amounts of a second language are positioned at the margins of
text and talk units, and thereby evoke social identities and relationships
associated with the minimally used language."

"Advertising discourse has a long tradition of ‘language display’, i.e. the
appropriation of out-group language ‘to attract potential customers by
appealing to their sense of what is modern, sophisticated, elegant, etc.’
(Eastman and Stein 1993: 198)"

English in ads:
"English is ‘the single most favored language selected for global mixing’
in advertising (Bhatia 1992)."
"What sets English apart
is the range of values it can be associated with, and the range of commodities
it promotes. It has been attributed symbolic values such as novelty, modernity,
internationalism, technological excellence, hedonism and fun, as opposed
to the stereotypical restriction of French to elegance and eroticism, Italian to
food, German to technology (cf. Piller 2001)."
"English, by contrast, illustrates how ethnosymbolism is left behind,
as its distribution to types of commodities is more signifi cant than the origin
of the commodities themselves."
"Cheshire and Moser (1994) suggest that the type of product is a better predictor
of language choice than the type of media that hosts the advertisement."

\cite{Beer2002}
"the magazines define Latinas primarily as consumers, they are a limited forum for women to explore noncosumerist identities, challenge hegemony, or express oppositional points of view."
"In addition, contemporarary consumer magazines offer advertisers a variety of "value-added" promotions, such as covert ads in the form of product mentions, and, although they are loathe to admit it, tolerate advertiser input into editorial content."


"It might be argued that, precisely because sexual imagery is designed to make the sale, it must, in some way, fulfill the passions it promises."
"We have noticed recently a trend toward a particular type of ad, increasingly more focused on sexuality, yet done with a tone wholly devoid of affect. Looking at these images one might correctly observe that after almost a hundred years of selling sex, the thrill seems to be gone."

"With the commodification of sex, the basic proposition is unatteinable. In terms of human passions, sex ads fail to satisfy because they confuse sexual gratification with the possession of objects. They attempt to substitute a state of *being* with the promise of *having*."

"The lack of affirmative portrayals of women's sexual passions leaves representations of female fulfillment impossible. [...] At worst, these images reinforce a patriarchal predisposition to disrespect women and do violence against them."
\end{comment}

