\section{Grundlagen}

\subsection{Kognitive Linguistik}

Die kognitive Linguistik ist in die Domäne der Kognitionswissenschaften einzuordnen -- das sind die Disziplinen, die die Wahrnehmung und Verarbeitung von Information untersuchen.
Der Forschungsgegenstand der kognitiven Linguistik ist also Wahrnehmung und Verarbeitung sprachlicher Informationen.

Mehrere Theorien sind unter dem Begriff zusammenzufassen.
Nach \cite{Ziem13} sind deren wichtigsten gemeinsamen Grundannahmen, dass Sprache kein abgekapseltes mentales Modul ist, Sprachwissen gebrauchsgestützt erworben wird und grammatische Strukturen ``Ausdruck menschlicher Konzeptualisierungsleistungen sind''.
Ferner erklärt \cite{Ellis08}, dass die sprachliche Kognition nicht von der sprachlichen Bedeutung und dem Rest der Kognition getrennt werden konnte und sollte.
Die kognitive Linguistik lehnt also die Unterscheidung vom sprachlichen und Weltwissen ab.
\cite{Ziem08} hält fest, dass spezifisch sprachliche Repräsentationen genau dann entstehen, wenn ``der perzeptuelle Input sprachlicher Art ist''.
Diese sind sowohl individuell als auch interindividuell -- es handelt sich einerseits um individuelle mentale Abbilde, andererseits um das Ergebnis der Kommunikation und Konventionen innerhalb einer Sprachgemeinschaft.
Es werden Abstraktionen aufgrund wiederkehrender Erfahrungen und sprachlicher Muster gebildet.
In \cite{Ellis08}: ``The meaning of the words of a given language, and how they can be used in combination, depends on the perception and categorization of the real world around us.''

\cite{Ellis08} erläutert noch, dass im Gegensatz zu der traditionellen Sprachwissenschaft, die kognitive Linguistik keine klare Trennung zwischen den Bereichen der Syntax, Lexikon, Phonologie und Pragmatik unternimmt.


\begin{itemize}
    \item usage based perspective

        der Sprachgebrauch bestimmt die Struktur von Sprache (und deswegen ist es komisch, dass er bei generativen Theorien und Strukturalismus eine untergeordnete Rolle bekommt)

        In \cite{Ellis08} usage based theories ``hold that structural regularities of language emerge from learners' lifetime analysis of the distributional characteristics of the language input and, thus, that the knowledge of a speaker/hearer cannot be understood as an innate grammar, but rather as a statistical ensemble of language experiences that changes slightly every time a new utterance is produced.''

        ``.. the acquisition of grammar is the piecemeal learning of many thousands of constructions and the frequency-biased abstraction of regularities within them.''

        ``frequency of patterns in the input affects acquisition''

        In \cite{Bybee06}: ``high-frequency instances of constructions undergo grammaticization processes''

        ``a usage-based theorist would make the more specific proposal that grammar
        is the cognitive organization of one’s experience with language''

        ``the study of grammaticization has played a central role in emphasizing
        the point that both grammatical meaning and grammatical form come into being through
        repeated instances of language use''

        ``use of language is lexically particular;
        certain words tend to be used in certain collocations or constructions''

        ``A second effect of token frequency (the
        CONSERVING EFFECT
        ) relates to the morpho-
        syntactic structure of a string. High-frequency sequences become more entrenched in
        their morphosyntactic structure and resist restructuring on the basis of productive pat-
        terns that might otherwise occur.  [...]
        This effect applies to syntactic sequences as well, allowing
        higher-frequency exemplars to maintain a more conservative structure''

        ``A theory based on usage, by contrast, which takes grammar
        to be the cognitive organization of language experience, can refer to general cognitive
        abilities: the importance of repetition in the entrenchment of neuromotor patterns, the
        use of similarity in categorization, and the construction of generalizations across similar
        patterns.''

        In \cite{Eskildsen08} ``A core principle uniting these theories is the rejection of the syntax–lexicon
        dichotomy''
\end{itemize}


\subsection{Konstruktionsgrammatik}
Alle Konstruktionsgrammatiken gehören zum Bereich der Kognitiven Linguistik (vgl Seminarhandouts)

\begin{itemize}
    \item Alle sprachlichen Phänomene lassen sich innerhalb derselben Theorie erklären (es gibt keine Kernsprache und Randerscheinungen!) -- sagt \cite{Tomasello06}
\item Konstruktion - Form/Bedeutungs-Paar (Goldbergs Definition?)
    In \cite{Lakoff87}: ``Each construction will be a form-meaning pair (F,M), where F is a set of conditions on syntactic
    and phonological form and M is a set of conditions on meaning and use.'' --> \cite{Ziem13}: ``dass erstere nicht nur phono-
    logische, sondern auch syntaktische Aspekte umfasst, und letztere nicht nur semantische
    Aspekte, sondern auch pragmatische Gebrauchsbedingungen einschließt.''

    \cite{Lakoff87}: ``We will argue that grammatical constructions in general are holistic, that is, that the
    meaning of the whole construction is motivated by the meaning of the parts, but is not computable
    from them.''

     --> also bringt eine ``Veränderung der syntaktischen Form stets eine Veränderung der Semantik mit sich''

    Syntaktische Konstruktionen bestimmen die allgemeine, grundlegende Bedeutung einer Äußerung und nicht die Verben.

    In \cite{Goldberg95}: `` A construction is posited in the grammar if and only if something about its form, meaning or use is not strictly predictable from other aspects of the grammar, including previously established constructions.''

    In \cite{Goldberg06}: ``[A]t the same time, unpredictability is not a necessary condition for positing a stored construction.
    There is evidence from psycholinguistic processing that patterns are also stored if they are sufficiently frequent, even if they are fully regular instances of other constructions and thus predictable.
    We must recognize that patterns are stored as constructions even when they are fully predictable.''

    ``Im Gegensatz zu strukturalistischen Theorien nimmt sie jedoch nicht an, dass semantisch leere Regeln bedeutungstragende Elemente kombinieren und dass die Ableitbarkeit sich aus der Rekonstruktion der Bedeutung der Lexeme und ihrer Art der Verknüpfung sich ableiten lässt.
    Sie postuliert, dass sowohl die Bedeutung der Wörter als auch die Bedeutung größerer und abstrakterer Konstruktionseinheiten zur Bedeutung eines sprachlichen Ausdrucks beitragen.''

    In \cite{Ellis08}: ``Any construction with unique, idiosyncratic formal or functional properties must be represented independently in order to capture speakers' knowledge of their language.''

    ``Frequency of occurance may lead to independent representation of even ``regular'' constructional patterns.''

    \cite{Eskildsen08}: ``Accordingly,
    some scholars have advocated a view of language knowledge as a formulaic-
    creative continuum (e.g. Nattinger and DeCarrico 1992; Weinert 1995). The
    maximalistic model of language knowledge envisioned by UBL entails
    precisely such a view.''
\end{itemize}

\subsection{Spracherwerb}
    L1:

    Tomasello - Abstraktion aus konkretem Input: Chunks(Holophrasen - ``zusammenhängendes sprachliches Symbol'') -> Pivot-Schemata (mit Platzhalter aber ohne syntaktische Markierungen) -> item-gestützte Konstruktionen ``an einen spezifischen lexikalischen Inhalt gebunden''(einige syntaktische Markierungen vorhanden) -> abstrakte Kategorien

        \cite{Tomasello06} Kinder erlernen Sprache auf Basis von Input in kleinen Schritten;
        Kein Prozess des Kombinierens mithilfe abstrakter Regeln, sondern Prozess des Aufspaltens;
        gebrauchsgestützt: Menschen sagen ähnliche Sachen in ähnlichen Situationen -> das enstehende sprachliche Verwendungsmuster wird in Konstruktionen abstrahiert;

        ``Frühe kindsprachliche Äußerungen und Konstruktionen sind eng an unmittelbare kommunikative Ziele des Kindes gebunden.''

        allgemeine kognitieven Fähigkeiten, die den Erwerb sprachlicher Konstruktionen ermöglichen: Intentionszuschreibung (intention reading)/kulturelles Lernen, Schematisierung/Analogie, Beschränkung, distributionelle Analyse

        Intentionszuschreibung: Menschen haben ein kommunkatives Ziel; Kinder ahmen nach, um ihre Ziele zu erreichen; -- imitierendes Lernen; entspricht den Holophrasen-Phase

        Schematisierung/Analogie: werden sprachliche Schemata mit einem leeren Slot gebildet;
        ``eine Analogie nur dann gebildet werden kann, wenn ein gewisses Verständnis vorliegt, welche funktionale Beziehung zwischen den einzelnen Bestandteilen der Strukturen besteht''
        -- item-gestützte Konstruktionen

        Generalisierungseinschränkungen (Beschränkung):
        Enschleifung(entrenchment): ``wenn ein Organismus eine Handlung auf eine bestimmte Art und Weise häufig genug mit Erfolg vollzieht, diese Handlungsweise routinisiert wird, wodurch es für alternative Vorgehensweisen, dasselbe Ziel zu erreichen, sehr schwierig wird, sich durchzusetzen.''

        spricht von ``syntaktischen Übergeneralisierungsfehlern''

        Distributionelle Analyse: abstrakte Kategorien bilden --> neue Einheiten, die einer Kategorie zugeordnet werden, werden dann auf derselben Weise benutzt, wie andere Elemente dieser Kategorie, ohne dass konkrete sprachliche Erfahrung mit diesen Einheiten vorliegt

        \cite{Tomasello06} zitiert Croft01 (Radical Construction Grammar) ``keine universelle syntaktische Kategorien... schon die bloße Idee von syntaktischen Kategorien als autonome sprachliche Einheiten in die Irre führt''

        In \cite{Ellis08} ``children are picking up frequent patterns from what they hear around them, and only slowly making more abstract generalizations as the database of related utterances grows.''

        \cite{Eskildsen08} zitiert Tomasello03: ``token frequency is frequency of a concrete
        expression which ‘in the language learner’s experience tends to entrench
        that expression in terms of the concrete words and morphemes involved’,
        whereas type frequency ‘of a class of expressions determines the abstractness
        or schematicity of the resulting construction’.''

        \cite{Ellis04}: ``the acquisition of grammar is the piecemeal learning of many thousands of constructions and the frequency-based abstraction of regularities between them''

        ``form-meaning relations are probabilistic; ... the language processing mechanism is unconsciously weighing the likelihoods of all candidate interpretations and choosing among them''

        ``Language learning is the associative learning of representations that reflect the probabilities of occurance of form-meaning mappings.''

        ``Fluent language processing is intimately tuned to input frequency and probabilities of mappings at all levels of grain: phonology and phonotactics, reading, spelling, lexis, morphosyntax, formulaic language, language comprehension, grammaticality, sentence production, and syntax.''

        ``Studies of sentence processing show that fluent adults have a vast statistical knowledge about the behavior of the lexical items of their language''

        \cite{Ellis06}: ``Perception and
        learning are ‘probabilistic achievements’, but our perception of the world is
        shaped through the lenses of our prior analyses, beliefs, and preconceptions.''

\subsection{Zweitspracherwerb}
\subsubsection{Unterschiede/Gemeinsamkeiten L1/L2}

        In \cite{Ellis08}: ``in child language acquisition knowledge of the world and knowledge of language are developing simultaneously whereas adult SLA builds upon pre-existing conceptual knowledge;''

        ``adult learners have sophisticated formal operational means of thinking and can treat language as an object of explicit learning''
        --> es findet bei Erwachsenen wohl beides statt: sowohl implizites(unbewusstes) als auch explizites (bewusstes) Lernen (vgl auch \cite{Ellis04})

        ``typical L1 pattern of acquisition results from naturalistic exposure .... whereas classroom environments for second or foreign language teaching can distort patterns of exposure''

        ``adult SLA builds on pre-existing L1 knowledge... adults have already acquired knowledge of these cathegories and their lexical membership for L1, and this knowledge may guide creative combinations in their L2 interlanguage to variously good and bad effects.''

        ``naturalistic SLA develops in broadly the same fashion as does L1 - from formulae, through low-scope patterns, to constructions''

        In \cite{Ellis04}: ``in addtion to communicative effort, cognitive effort is a necessary condition for successful adult SLA (Schmidt, 1984)''

        noticing hypothesis: ``It is necessary that the learner notice the relevant linguistic cues.''

        ``aspect must be notices before a mental representation of it can first be found''

        ``The noticing hypothesis subsumes various ways in which SLA can fail to reflect the input: failing to notice cues because they are not salient, failing to notice that a feature need to be processed in a different way from that relevant to L1, failing to acquire a mapping because it involves complex associations that cannot be acquired implicitly, or failing to build a construction as a result of not being developmentally ready in terms of having the appropriate representational precursors.''

        manchmal sind Merkmale nicht nur nicht besonders auffällig, sondern auch redundant --> LernerInnen erwerben diese nicht (zb Vergangenheitsform des Verbs --> Vergangenheit oft mit temporalen Adverben markiert; diese sind dann die salienten Vergangenheitsmarker)

        in solchen Fällen ist ``explicit instruction'' hilfreich und notwendig

        ``preservation and transfer: failing to notice that a feature needs to processed in a different new way --> distorting the preception of items in their vicinity to make them seem more similar to the prototype''

        ``The communicative functions of language motivate the learner to the task.
        Noticing lays out the problem.
        Consciousness-raising can speed its solution.
        Figuring provides the final tally of native levels of fluency and idiomacity.''

        In \cite{Ellis06}: ``If first language is rational in the sense that acquisition produces an
        end-state model of language that is a proper reflection of input and that
        optimally prepares speakers for comprehension and production, second
        language is usually not.''

        ``fragile features of L2 acquisition'': ``available as a
        result of frequency, recency, or context, fall short of intake because of one of the
        factors of contingency, cue competition, salience, interference, overshadowing,
        blocking, or perceptual learning, which are all shaped by the L1.''

\subsubsection{Gebrauchbasierte/Konstruktionsgrammatische Perspektive auf dem Zweitspracherwerb}

In \cite{Eskildsen08}: ``The data also suggest that semi-fixed linguistic
patterns, here operationalized as utterance schemas, deserve a prominent place
in L2 developmental research.''

``UBL, then, with its proposed item-based path of language learning seems
promising, in N. Ellis’s words, as a default in guiding investigations into
longitudinal L2 learning. However, its operationalization of performance as a
subset of competence (Barlow and Kemmer 2000) is a problematic modus
operandi for SLA. Use in SLA cannot be thought of as a subset of knowledge
because we do not know exactly what might constitute an exhaustive
account, at the most abstract level of schematicity, of the linguistic resources
of a given L2 learner.''

--> vielleicht keine so gute Idee L2-Lernende mit L1-Sprecherinnen der selben Sprache in terms of Kompetenz zu vergleichen; genau so wie Kinder im L1-Erwerb und deren sprachliche Repräsentationen nicht mit der Grammatik einer\_s Erwachsenen verglichen werden sollte

``his paper advocates the future study of L2 linguistic
development in terms of an empirically grounded locally contextualized
grammar, consisting of flexibly abstract units of actual language use in
interaction.''

``The proposed item-based path for learning an L2 seems to be valid, albeit
with one major adjustment. Formulas, or MWEs, should be seen as
interactionally and locally contextualized.''

``productivity enhancement is partially concrete, based on utterance schema
development, and traceable to previous experience''

``more principled discussion of what constitutes a schematic endpoint of L2 learning.''

In \cite{Ellis04}: ``The task of the language learner is to make sense of language.''

``.. the language learner's kit consists of constructions that map forms and meanings -- the recurrent patterns of linguistic elements that serve some well-defined linguistic function.''

In \cite{Ellis06}: ``when multiple
traces are associated to the same cue, they tend to compete for access to
conscious awareness'' --> reason for transfer

``this effect of prior learning inhibiting
new learning is called proactive inhibition (PI)''

``Robert Lado proposed in his
Contrastive Analysis Hypothesis (CAH) (Lado 1957): ‘We assume that the
student who comes in contact with a foreign language will find some features
of it quite easy and others extremely difficult. Those elements that are similar
to his native language will be simple for him, and those elements that
are different will be difficult’ (Lado 1957: 2)''

Another paper describing L2 from a usage-based/cxg perspective:
In \cite{}
``It seems intuitively correct to assume
that speakers choose language for a particular purpose based on numerous
factors surrounding the act of speaking, and that they use what they have to
take them as far as possible and that this involves both Lt knowledge and
L2 knowledge. An L2 speaker's choice of expression is not always conven-
tional in the sense that the utterance directly corresponds to conventional
usage, and most often this is not the case. For the purposes of this paper,
conventional usage is confined to combinations that are socially anticipated,
culturally generated, and in some sense entrenched in the community. Con-
ventional usage entails specific or concrete expressions that have attained
some degree of unit status through use within a language community.''

Check Grundlagen in \cite{Ziem13}
