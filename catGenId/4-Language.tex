\section{Llengua i traducció. Lectura llingüística}

El llibre està escrit en un llenguatge molt particular.
Tenim tota la història explicada des de la perspectiva de la filla i encara una instància narrativa auctorial.
L'estil directe al llarg del llibre no està marcat de manera gràfica en ningún lloc; el discurs s'assembla més a un corrent de consciència.

Com adverteix el fundador de les ciències postcolonials Edward Said en el seu llibre ``Culture and Imperialism'', avui dia cap persona pot ser definida amb una sola etiqueta
i identitats (atribuïdes o seleccionades per a una mateixa) com ``dona'', o ``musulmà'', o ``americà'' serveixen només per a una primera orientació~\autocite{Vidal2012}.
A més, aquestes categories són subjectives i existeixen sempre dins un narratiu particular que les ha creat, afirma la professora de ciències de traducció Mona Baker, citada en~\autocite{Vidal2012}.
Said planteja també la idea que tothom construeix la seva pròpia identitat cultural i ètnica.
Ja no tenim cultures i identitats ``pures'' i ``estables''~\autocite{Vidal2012}.
Però encara, amb la mateixa força, l'altre, el diferent, l'estrany està percebut com una amenaça per l'ordre establert i per això es rep amb por i prejuicis.

Un aspecte fundamental/fundador de la nostra identitat i de la nostra cultura (llevat de que entenem sota aquests dos termes) és/són la/les llengüa/es a través de les quals ens comuniquem.
A més a més, a través del llenguatge les cultures diferents es poden entendre, que segons Vidal Claramonte és crucial, ja que avui dia més que mai gran quantitats de persones no viuen en la seva llengua materna.~\autocite{Vidal2012}.
O, més específicament, viuen en més que una llengua i això produeix constantement conflictes i trastorns i fa necessari la reorganització de les esquemesque organitzan la seva vida~\autocite{Albin2005}.

Deleuze i Guattari, citats per Vidal Claramonte, posen una pregunta interessant:
``How many people today live in a language that is not their own? Or no longer, or not yet, even know their own and now poorly the major language that they are forced to serve?''~\autocite{Vidal2012}
i elaboren que aquest problema sigui molt propi dels i de les immigrants i els/les seves filles.

En el seu treball sobre l'obra de Najat El Hachmi María Carmen África Vidal Claramonte destaca la traducció com l'element bàsic de ``L'últim patriarca''.
Explica que la traducció té lloc a nivels multiples: podem llegir no només una traducció entre dues llengües, sinó també entre dues cultures i fins i tot entre dues generacions~\autocite{Vidal2012}.
Tenim per això exemples multiples:
la filla qui tradueix els preus dels aliments a la seva mare, no només canviant la llengüa, sinó també convertint les pesetes en duros;
la filla qui ajuda a la seva mare embarassada de parlar amb la ginecòloga:
\begin{quote}
``Hi havia coses que no sabia passar d'un idioma a l'altre, que no volia passar d'un idioma a l'altre. Contiuanva sense entendre per què tantes dones per tot el món m'explicaven coses d'aquelles a mi. Quina va ser l'última vegada que li va venir la regla a la mare? I jo ja sabia què era allò de la regla, però no n'havia parlat mai amb ella. La primera vegada que li va venir? [\ldots] La primera vegada que va tenir relacions sexuals? Déu, Déu, volia fugir corrents de tot allò, jo no les vull saber, totes aquestes coses, i encara menys traduir-les a un idioma on no hi havia cap paraula que jo conegués per a relacions sexuals que no fossin paraulotes [\ldots] au, va, pregunta-li-ho. La mare em mirava i deia què, què t'ha preguntat, i jo hauria volgut desaparèixer així, de cop, i que elles mateixes s'entenguessin. No podia dir follar, no [\ldots] Vaig intentar de buscar un eufemisme. Quants anys tenies quan vas dormir amb el pare per primera vegada? I no la vaig mirar als ulls mentre li ho deia; ella va dir, també ben de pressa, ens vam casar que jo tenia divuit anys. Això és tot.''~\autocite[217]{ElHachmi2008}
\end{quote}
En aquest fragment la traducció multiple és especialment interessant.
Tenim les dificultats de la filla de trobar paraules adequades per relacions sexuals en una llengua, en la qual encara existeix el tabu de parlar de manera lliure sobre aquestes coses,
i en la qual ningú mai li ha parlat d'això d'una manera que no sigui vulgar.
Tenim també la seva assumpció ``natural'' qua el pare sigui el primer home amb qui la seva mare n'ha tingut--una traducció de la pregunta de la metgessa.
I tenim encara la resposta de la mare, que és una nova traducció: ``ens vam casar que jo tenia divuit anys''~\autocite[217]{ElHachmi2008}--traducció generacional, perquè ``casar-se'' substitueix ``dormir amb el pare'' i traducció llingüística, ja que la filla passa la frase a català~\autocite{Vidal2012}.

Un objecte important per a la filla, al qual es fa referència al llarg del llibre, és el diccionari de la llengua catalana.
El diccionari i, en conseqüència, la llengua, són un refugi per a la noia de les contradiccions de la vida quotidiana.
Les consultacions amb el/les lectures del diccionari serveixen també com un fil que ens ajuda seguir la història.

Hem de recordar el valor simbólic dels diccionaris d'una llengua.
El diccionari és el primer llibre d'una nació, cita Verónica Albin en una entrevista amb l'intelectual Ilan Stavans~\autocite{Albin2005}.
També, és important notar que el mític objecte ``el diccionari'' no existeix.
Cada llengua disposa de diccionaris multiples, cada un d'aquests té lxs sevxs redactorxs, persones amb ideologies i circunstàncies concretes que naturalment influeixen les seves obres. [quote!]
Com ho formula Henri Meschonnic, citat per Albin, els diccionaris reflecteixen ``les conflits, les masquages des
conflits, les clichés qui font l'album de famille d'une culture''~\autocite{Albin2005}.
Stavans es posa d'acord amb aquesta noció i subratlla que encara que siguin xenòfobs, els diccionaris serveixen per definir l'univers d'una persona,
perquè ``The limits of our language are the limits of our worldview.''~\autocite{Albin2005}.

%naming things
En una entrevista Najat El Hachmi aborda un aspecte clau del llenguatge--la possibilitat d'anomenar persones, emocions, esdeveniments i relacions al voltant nostre.
Clau, perquè si no podem parlar dels fets i els problemes que ens envolten de manera adequada, tampoc podem elaborar solucions.
En particular, l'autora es refereix aquí a la violència extrema que comet el personatge de Mimoun.
El Hachmi assenyala que si no parlem de la violència, ens convertim en seus còmplices.
Per tant, destaca ella, les lectures del diccionari per part de la filla són el primer pas cap a estant en control de la situació, cap a un alliberament.
D'altra banda, la literatura és una manera d'ampliar els propis horitzons, d'experimentar vides diferents, de descobrir que altres maneres de viure són possibles~\autocite{HaAM2011}.

\begin{comment}
  Quotes (moved)
  ---------------
\end{comment}
