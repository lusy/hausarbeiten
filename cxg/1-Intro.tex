\section{Intro}

\begin{comment}
* ca 3/4 Seiten
* Untersuchungsgegenstand
* Erkenntnisinteresse
* Forschungsstand
* Vorgehensweise: also Section 2 macht blabla, Section 3 blablup, ....
* Ergebnisse können/sollen angedeuten werden
\end{comment}

%Idee:
%\begin{itemize}
%    \item Beschreiben des theoretischen Rahmens der Konstruktiongrammatik/Gebrauchsbasierten Perspektive/Kognitiven Linguistik/lala + Zweitspracherwerb;
%    \item Beschreiben der historischen Entwicklung von cross linguistic Transfertheorien;
%    \item Beschreiben der CLD/CLLD Studie Spanisch-Englisch
%    \item Putting last 2 things in perspective bzgl des Rahmens vom 1.
%\end{itemize}

Es gibt im Zweitspracherwerb Phänomene wie Transfer aus der Erstsprache (L1) und Übergeneralisierung.
Diese lassen sich in einem kognitiv linguistischen/konstruktionsgrammatischen theoretischen Rahmen erklären.
Beispielhaft wird das auf eine Studie der Contrast-Left-Dislocation (CLD) und Clitic-Left-Dislocation (CLLD) Konstruktionen im Spanischen und Englischen demonstriert.

Im Weiteren ist die vorliegende Arbeit wie folgt aufgebaut:
Kapitel 2 erläutert die theoretischen Grundlagen -- es wird eine kurze Übersicht über ausschlaggebende Annahmen der kognitiven Linguistik, der Konstruktionsgrammatik und des Erst- und Zweitspracherwerbs im Rahmen dieser Theorien gegeben.
Kapitel 3 beschreibt die historische Entwicklung der Spracherwerbsforschung mit Schwerpunkt Wechselwirkungen (Übergeneralisierungen und Transfer) zwischen L1 und L2 aus einer konstruktionsgrammatischen gebrauchsgestützten Perspektive.
In Kapitel 4 werden die Phänomene der Übergeneralisierung und Transfer auf dem Beispiel von CLD und CLLD-Konstruktionen im Spanischen und Englischen untersucht.
Die Ergebnisse werden dann in Kapitel 5 zusammengefasst. % Ausblick + Kritik!
