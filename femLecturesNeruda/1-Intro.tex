\section{Einleitung}

Pablo Neruda, der Nationalpoet Chiles, ist einer der bekanntesten und beliebtesten, wenn nicht der bekannteste überhaupt, Lyriker*innen Lateinamerikas.
Seine Werke werden heute noch rege rezipiert und sind Tausendmal(syn!) von der Literaturkritik kommentiert worden.

%Liebeslyrik

Nerudas Werk ist umfangreich und vielfältig, der Poet hat sich mit vielen Aspekten des menschlichen Lebens und dem aktuellen Tagesgeschehnis befasst.
Er bearbeitet in seinen Gedichte seine Erlebnisse während des Spanischen Bürgerkriegs, seine Rezeption anderer Poeten, seine Begeisterung für den Kommunismus. 
Unter den vielen prägenden Werken können uns heutzutage einige stutzen lassen.
Dabei beziehe ich mich nicht nur auf die Oda an Stalin oder ???, die aus heutiger Sicht zweifelslos umstritten sind, sondern auch teilweise auf Nerudas Liebeslyrik.
Von vielen sehr positiv wahrgenommen und als Wiedergabe universellen menschlichen Gefühle angepriesen, %TODO quote
können wir in dieser zum Teil auch beunruhigende Ideologien und Frauenbilder entdecken.

Ausgehend von der Annahme, dass Sprache und Literatur unsere Weltanschauung, Gedanken und Umgang mit der uns umgebenden Welt mitgestalten/formen~\cite{Kolodny1980},~\cite{North2013},
möchte ich in dieser Arbeit Nerudas Liebeslyrik aus einer feministischen Perspektive lesen.
Das Ziel dieser Arbeit(syn!) ist es nicht, die ``richtige'' Interpretation Nerudas Liebeslyrik (syn!) auszuarbeiten, sondern, in Anlehnung an Annette Kolodny, viel mehr eine Pluralität der Lektüren anzuregen~\cite{Kolodny1980}, Leseweisen ermöglichen, die im androzentrischen Literaturkritikdiskurs verdeckt bleiben.

\begin{comment}
``In my view, our purpose is not and should not be the formula-
tion of any single reading method or potentially procrustean set
of critical procedures nor, even less, the generation of prescriptive
categories for some dreamed-of nonsexist literary canon.52 Instead,
as I see it, our task is to initiate nothing less than a playful pluralism,
responsive to the possibilities of multiple critical schools and meth-
ods, but captive of none, recognizing that the many tools needed
for our analysis will necessarily be largely inherited and only partly
of our own making. Only by employing a plurality of methods
will we protect ourselves from the temptation of so oversimplifying
any text''~\cite{Kolodny1980}
\end{comment}

% Feminismus-Begriff
\begin{comment}
  [Kolodny1980]
  feministische literaturkritik wird anscheinend zur sau gemacht, weil sie kein system/program hat: "lack of systematic coherence"
  das ist auch gleichzeitig die stärke der bewegung: diversity, pluralism
  ``All the feminist is asserting, then, is her own equivalent
  right to liberate new (and perhaps different) significances from
  these same texts; and, at the same time, her right to choose which
  features of a text she takes as relevant because she is, after all, ask-
  ing new and different questions of it.''~\cite{Kolodny1980}
  ``Robert Scholes, from whom I've been quoting, goes so far as to assert
  that "there is no single 'right' reading for any complex literary
  work,"''
  ``we entertain the possibility
  that different readings, even of the same text, may be differently
  useful, even illuminating, within different contexts of inquiry.''
  ``what we give up is simply the arrogance of
  claiming that our work is either exhaustive or definitive.''
  ``If feminist criticism calls anything into
  question, it must be that dog-eared myth of intellectual neutrality.''
\end{comment}

% These??
\begin{comment}
  * literatur/sprache shape our minds
  * wachsamkeit/Sensibilität schaffen fürs Erkennen patriarchalischen Projekte, die mittels "universelle Gefühle", "schöne Liebeslyrik" vermittelt werden/dafür verkauft werden

  immer noch zu schwammig..

  [Kolodny1980]
  Feminist Readings of male authors: discovering/exposing power relations men-women, that are taken for natural and granted

  % iwie ist mein Projekt mehr oder weniger das gleiche wie das von Frau Duncan: ``There is no doubt that Neruda is a gifted poet; whether or not he is guilty of sexism as a writer is not a point I wish to debate here. But, I do take issue with the critical readings of Neruda which have drawn attention toth repression, subjugation, and silencing of women in his poetry only to dismiss these factors as natural conditions which are in and of themselves praiseworthy.''\cite{Duncan1992}

\end{comment}

% Weitere Werkzeuge

Durch das Close Reading exemplarisch ausgewählten Gedichte aus zwei verschiedenen Gedichtsbändern soll die Poetisierung und Universalisierung bestimmten heteronormativen und männerzentrierten Perspektiven aufgezeigt und in Frage gestellt werden.

% Aufbau der Arbeit!
Im Weiteren ist diese Arbeit wie folgt aufgebaut:
Im zweiten Kapitel werden zunächst etablierte Rezeptionen Nerudas Liebeslyrik dargestellt/zusammengefasst.
In Kapitel 3 gehe ich auf einige Begriffe bzw. Methoden ein, die für die weitere Analyse eine bedeutende Rolle spielen werden.
Darauf aufbauend biete ich in Kapitel 4 examplarisch (syn!) alternative Leseweisen von den Liebesgedichten ``Tu risa'' (\textit{Los versos del capitán}) und ``Poema XV'' (\textit{Veinte poemas de amor}) an.
Abschließend wird die Arbeit nochmal zusammengefasst und ein Ausblick wird vorgeschlagen/umrissen.

\begin{comment}
1. Intro
  * Ziel von Feministischen Lektüren: zugrunde liegende Machtstrukturen in Werken und deren Rezeption aufzudecken
    ** androzentrische Perspektive der Literatur:
       *** Männer in Mittelpunkt (als Figuren)
       *** von Männern gemacht
       *** an Männer gerichtet
  * Wie erreicht? Durch eine Pluralität der Lektüren und Close Reading
ohne das ouevre Nerudas nicht als ganzes in Frage stellen
exemplarische Lektüren: Poetisierung bestimmter Heteronormativen Perspektiven
gehört historisiert; nicht als zeitlos darzustellen

* durch Sprache (und Literatur dann) bestimmte identitäre Zuschreibung geprägt: Sprache formt unser Gehirn, Gedanken, Weltanschauung, Denkweise, Kathegorien --> im Kopf behalten wenn man mit "Lektüren" umgeht
\end{comment}
