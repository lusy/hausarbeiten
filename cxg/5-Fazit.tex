\section{Fazit}

\begin{comment}
    * Einleitung und Fazit müssen zusammenpassen.
    * sind die Erkenntnisse im Fazit aus der Arbeit ableitbar?
    * Vorgehensweise zusammenfassen

Kritische Zusammenfassung
\begin{itemize}
    \item was war in den Texten nicht so gelungen?
    \item welche Fragen sind offen geblieben?
    \item in welche Richtung kann noch weiter geforscht werden?
    \item was sind gegnerische Meinungen zum Thema?
\end{itemize}
\end{comment}


Das Ziel dieser Arbeit war, die Phänomene des Transfers und der Übergeneralisierung im Zweitspracherwerb aus einer kognitiv-linguistischen Perspektive zu beleuchten.
Dazu wurden diese zunächst in der historischen Entwicklung der Zweitspracherwerbsforschung verfolgt.
Darauf anschließend wurden Transfer und Übergeneralisierung am konkreten Beispiel von \textit{contrast left dislocation} und \textit{clitic left dislocation} Konstruktionen in den Zweitsprachen Englisch und Spanisch in \cite{Valenzuela05} untersucht.

\begin{comment}
Die vorliegende Arbeit konzentriert sich auf den (erwachsenen) Zweitspracherwerb (L2-Erwerb).
Es werden dabei Phänomene wie Transfer aus der Erstsprache (L1) und Übergeneralisierung bestimmter L2-Konstruktionen beobachtet.
Diese lassen sich in einem kognitiv-linguistischen/konstruktionsgrammatischen theoretischen Rahmen erklären.
Das wird am Beispiel einer Studie von Elena Valenzuela \cite{Valenzuela05} über die \textit{contrast left dislocation} (CLD) und \textit{clitic left dislocation} (CLLD) Konstruktionen im Spanischen und Englischen demonstriert.

Im Weiteren ist diese Arbeit wie folgt aufgebaut:
Kapitel 2 erläutert die theoretischen Grundlagen -- es wird eine kurze Übersicht über ausschlaggebende Annahmen der kognitiven Linguistik, der Konstruktionsgrammatik und der Erst- und Zweitspracherwerbsforschung im Rahmen dieser Theorien gegeben.
Kapitel 3 beschreibt die historische Entwicklung der Spracherwerbsforschung mit Schwerpunkt Wechselwirkungen (Übergeneralisierung und Transfer) zwischen L1 und L2 aus einer konstruktionsgrammatischen gebrauchsgestützten Perspektive.
In Kapitel 4 werden die Phänomene der Übergeneralisierung und Transfer auf dem Beispiel von CLD und CLLD-Konstruktionen im Spanischen und Englischen untersucht.
Die Ergebnisse werden dann in Kapitel 5 zusammengefasst. % Ausblick + Kritik!
\end{comment}
%Es werden 2 Ansätze vorgestellt, was (Zweit)Spracherwerb angeht: sequentieller Spracherwerb und Erwerb eines Netz von Konstruktionen

%Gegnerische Meinungen:

%\cite{Cook93} - generative Perspektive auf Zweitspracherwerb
%``A person who knows two languages has been through the acquisition process twice.
%Second Language research must explain the means by which the mind can acquire more than one grammar.''

%``One goal of second language research is to describe grammars of more than one language simultaneously existing in the same person.''

%``The importance of second language research lise not in its account of the knowledge and acquisition of the L2 in isolation, but its account of the second language present and acquired in a mind that already knows a first [...]''

%aber das sind doch voll peanuts.. ist überhaupt nicht der kernpunkt..
%\cite{Valenzuela05} Kritik:
%bisschen zu großzügige Interpretation der gemessenen Daten zugunsten von der Arbeitshypothese


