\section{Catalunya--una terra d'immigrants. Lectura autobiogràfica}

``L'últim patriarca'' és, entre altres coses, la història d'una família d'immigrants.
Per això, repassarem breument el desenvolupament de la immigració a Catalunya a l'útlim segle.
Hem de notar que Catalunya sempre ha sigut una terra d'acollida d'immigrants.
Només al segle XX hi havia multiples onades d'immigració, començant amb la primera als anys '20, passant per la segona, als anys '50-'60, la tercera a les '80 i arribant a l'actualitat~\autocite[123--126]{Candel1985},~\autocite{TarPaGa2013}.
Totes aquestes onades tenen trets diferents -- més prominent, per exemple, l'origen de la nova població o les raons per immigrar, però totes tenen també similituds.
Una d'elles és la recepció dels i de les recentment arribats.
Com assenyala Fransesc Candel al seu llibre ``Els altres catalans'', cada onada està rebuda per una mena de desconfiança, prejuicis i franca hostilitat~\autocite[17]{Candel1965}.
La població establerta té por del desconegut i de l'incert i sovint reacciona amb discursos i comportaments racistes:
``\ldots heu vingut a menjar-vos el pa dels catalans\ldots''~\autocite[17]{Candel1965},
``Aquests nou[sic!] catalans no coneixen Catalunya: la seva tradició, la seva història, el seu art, la seva cultura, la seva literatura, els seus costums, el seu folklore\ldots''~\autocite[17]{Candel1965},
``Parlen un català gruixut, groller i vulgar\ldots Desconeixen la gramàtica catalana. No saben llegir en aquest idioma. Escriure'l, encara menys.''~\autocite[18]{Candel1965}

La immigració de la família Driouch descrita a la novel·la té com a fons un periode històric:
les diverses onades migratòries del nord d'Àfrica cap Espanya que es produexien a partir de la dècada de les '80~\autocite{TarPaGa2013}.
Segons aquest mateix informe, a l'any 2013 els marroquins són amb 20.4\% el més gran grup d'immigrants a Catalunya.
Els autors confirmen:
``El col·lectiu procedent del Marroc ha estat el més important des dels inicis de la immigració estrangera recent a Catalunya, tant pel nombre com per la seva distribució en el territori.''~\autocite{TarPaGa2013}
A més, demostren, el que pel col·lectiu és important, que una part significant forma la immigració familiar--els nombres de la immigració infantil són, amb uns 14.6\%, bastant elevats.

La mateixa autora de ``L'útlim patriarca'' va néixer a Marroc, quan el seu pare ja vivia a Catalunya, i va immigrar a Vic amb la resta de la seva família quan tenia vuit anys~\autocite{Vidal2012}.
Llavors, una lectura de la novel·la com obra amb elements autobiogràfics (llevat del grad de proximitat entre ficció i realitat) ocorre de manera automàtica.
Aquesta percepció es solidifica si comparem l'obra amb algunes declaracions que l'autora fa personalment, en entrevistes, per exemple o en la seva primera obra ``Jo també sóc catalana''.
``En definitiva: has de fer els malabarismes que calguin per anar decent per a l’estètica nord-africana i alhora no semblar una pobra noia reprimida davant dels originaris de Catalunya.'' cita Vidal Claramonte la primera obra, amb caràcter autobiogràfic, de l'escriptora~\autocite{Vidal2012}.
Podem trobar paraules molt similars dins de les pàgines del llibre:
``La roba, la roba, sempre discutint amb la mare quina peça era adequada i quina no i jo ja no sabia com conciliar tantes exigències, entre les modes de l'institut, on no volia semblar rara, les del mercat, on no hi cabia, i les seves, la major part del temps senzillament absurdes.''~\autocite[285]{ElHachmi2008}.


\begin{comment}

\subsection{Entrevistes: lectura autobiogràfica?}

[Vidal2012]

"Para ella, hay muchos tipos de discriminación: la del racista que golpea de frente, pero también la del paternalista, que dice cosas como que, aceptando al inmigrante, nuestra cultura se enriquece."
-- identitat cultural?

\end{comment}
