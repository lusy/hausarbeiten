\section{Entre dues cultures. Lectura postcolonial}

Com adverteix el fundador de les ciències postcolonials Edward Said en el seu llibre ``Culture and Imperialism'', avui dia cap persona pot ser definida amb una sola etiqueta
i identitats (atribuïdes o seleccionades per a una mateixa) com ``dona'', o ``musulmà'', o ``americà'' serveixen només per a una primera orientació~\autocite{Vidal2012}.
A més, aquestes categories són subjectives i existeixen sempre dins un narratiu particular que les ha creat, afirma la professora de ciències de traducció Mona Baker, citada en~\autocite{Vidal2012}.
Said planteja també la idea que tothom construeix la seva pròpia identitat cultural i ètnica.
Ja no tenim cultures i identitats ``pures'' i ``estables''~\autocite{Vidal2012}.
Però encara, amb la mateixa força, l'altre, el diferent, l'estrany està percebut com una amenaça per l'ordre establert i per això es rep amb por i prejuicis.

Entre una cultura d'origen, representada pel pare, per la mare, pel reste dels familiars a Marroc i, en part, pels moros a Catalunya, i una cultura d'``acollida'', la filla en ``L'últim patriarca'' ha de trobar les parts per construir la seva identitat.

%la mare
La mare està descrita com a una dona que compleix tots els estereotips de la societat occidental per a una dona musulmana:
és una bona esposa, segueix quasi literalment tots els ordres absurds i humiliants del seu marit, pateix les seves escapades en silenci, s'ocupa dels fills, fa excepcionalment totes les feines de casa.
Ella representa un model a seguir per la filla segons les normes marroquines, una possibilitat, que a la noia no li agrada gaire:
``La mare deia surts massa i començava aquells discursos de jo a la teva edat ja feia\ldots Jo a la seva edat no hauria sabut què fer perquè ella només netejava i netejava i no sabia com anar al metge sense el pare, com havia de comprar sense el pare, com havia de viure sense el pare.''~\autocite[188]{ElHachmi2008}.
Trobem aquí el que és no només un xoc de cultures sinó també un conflicte generacional, i, fins i tot, una atribució masclista per com ha de ser una nena.
La mare està convençuda que a l'edat de la filla era molt més capaç i la filla està decepcionada que encara a aquesta edat la mare no és capaç de funcionar a la societat per si mateixa.
Hi ha també rares ocasions d'amiració per la mare per part de la filla--quan ella demostra la seva força silenciosa:
``Va dir o la deixes, o et deixo. Jo no em vaig creure el que sentia, però era la meva mare que parlava, era la Mila que s'havia afartat de netejar capelles i relíquies, la Colometa que fugia de tot per trobar-se.''~\autocite[222]{ElHachmi2008}.
Però, podem resumir, que a la majoria de les vegades, la filla es definiria més en contrast amb la seva mare.

%el pare
El personatge de Mimoun està caracteritzat, com ja hem assenyalat, amb una crueltat exagerada i excessiva.
El també compleix els estereotips occidentals per l'home arab qui oprimeix la seva esposa, té una sexualitat no limitable i no respecta cap dona com a esser humà.
A més, fa servir els prejuicis dels altres pels seus propis objectius:
fixem-nos a l'episodi quan prohibeix a la filla anar a les colònies, perquè sí, per demostrar que ell està en control, i argumenta davant les mestres de l'escola amb la mare, qui suposadament hagi emès la probihició, presentant-se així com a un home progressiu en contrast amb la mare--una dona musulmana conservativa.

% la cultura d'acollida
La cultura catalana que sovint es presenta com alliberada tampoc és gaire acollidora per a una noia percebuda com ``altra''.
Els veïns i les veïnes no saluden la filla quan es posa el mocador i observen en silenci els assalts del seu pare sense obtenir ni per un sol moment la idea d'ajudar-li.
Ella rep els prejuicis i la hipocrecia de la societat catalana:
``jo ja feia temps que no podia anar a casa dels seus pares\ldots a l'amiga número dos la seva mare la portava a vendre espelmes d'aquelles amb les quals s'ajuden els nens pobres del món, d'Àfrica i tot, però es veu que només valia per als nens negres de debò i no pas per a mig brunes com jo''~\autocite[285]{ElHachmi2008}.
Tothom ja té les seves idees com ha de ser suposadament una nena marroquina:
``\ldots i jo que havia trencat lleis no escrites i havia decidit que no volia ser ni auxiliar d'infermeria ni administrativa de grau u ni mecànic ni electricista.
  Pesaven força espases de Dàmocles damunt meu: que si jo a la teva edat ja estava casada, que si en la teva cultura ja se sap que no val la pena, que us acaben casant tard o d'hora, la d'aquest és l'últim curs i alguna altra que tenia el pare al cap, com allò de les dones que no traeixen mai els pares però que sí que acaben traint els homes.
  Tot això duia jo a la motxilla, però ningú se'n va adonar.''~\autocite[273]{ElHachmi2008}.

% autoidentificació
Entre aquests exemples, una noia adolescent ha de trobar un model per a si mateixa, que no és una tasca fàcil.
Aparentament ninguna de les dues cultures no pot ser triada sense objeccions.
El món occidental amb la seva autopercepció de superioritat, d'haver establert l'única manera possible de fer i viure les coses, i les seves opressions sobre tot el que és diferent, no pot ser acceptat incondicionalment.
Però, el mateix està en vigor per la cultura marroquina.
El seu únic projecte de feminitat conservadora, relegada a les feines de casa i les necessitats dels homes, és encara menys satisfactori.

