\section{Hauptteil}

\subsection{Hauptthesen}
\begin{enumerate}
    \item Lernerinnen einer Fremdsprache greifen oft auf schematisierten Äußerunge zurück.
Schlagwort: formulaic phrases
    \item Lernerinnen einer Fremdsprache projizieren Strukturen/Konstruktionen aus ihrer Muttersprache in die L2.
\end{enumerate}


zu 2)

\cite{Weinreich79} ``... discussed how two language systems relate to each other in the mind of the same individual.
The key concept was interference, defined as ``those instances of deviation from the norms of either language which occur in the speech of bilinguals as a result of ther familiarity with more than one language'' \cite{Weinreich79} '' \cite{Cook93}

can happen on all levels of a language:
\begin{itemize}
    \item phonology: ``Speakers may carry over the L1 phonological system by ignoring distinctions made in the L2 but not in the L1;'' \cite{Cook93}
        ``some French learners fail to distinguish between the two English phonemes /i:/ and /i/ as in ``keen'' /ki:n/ and ``kin'' /kin/ because they are not distinct in their L1.''
    \item word order: ``German learners of English produce ``Yesterday came he'' modelled on the equivalent German sentence ``Gestern kam er''.''
\end{itemize}

\subsection{Daten}
\begin{itemize}
    \item Kindlichem L2 Erwerb Spanisch -> Englisch: \cite{Wong-Fillmore76} in \cite{Haberzettl06}
    \item Kindlichem L2 Erwerb Türkisch -> Deutsch: Wegener\cite{} -- leider unveröfftentlichte Schrift
und \cite{Haberzettl05} in \cite{Haberzettl06}

SOV-Konstruktionen aus dem Türkischen ins Deutsche projiziert

    \item Evidenzialität Quechua -> Spanisch??
\end{itemize}
