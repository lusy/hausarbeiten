\section{Grundlagen}

Grundlagen
----------
Language Shift (Spanish -> English) : Speech community of a language shifts to speaking another language (source?)
Can we talk about language shift in this context?
Who is shifting? The editors of the magazine? Or do they induce language shift?
Should that be a focus at all? Or focus more on communities of practice/style projection through language/sozioindexikalität, etc.

%Communities of practice
can we say that the target reader group of these magazines is a community of practice?
--> bisschen weit hergeholt; passt meiner Meinung nach nicht zu der Definition von Community of Practice

English is the more prestigious language;
the dominant(both in numbers and power) group of people in the USA speaks English
serves to avoid misunderstandings/clarify (e.g. cooking recipes)

Speech patterns (Vortrag Spanish in NYC)
→ bilingualism: 63% census respondents (2007) who speak Spanish at home also report that they
speak English well or very well
→ shift from Spanish to English after 2nd generation migrants
→ Spanish speaking community survives due to new migrations

→ style/socioindexicality: adopt linguistic forms of groups one sympathises with/has intense
relationships with
→ "Although overt norms favor standard speech, powerful covert norms encourage
group members to remain faithful to group codes, linguistic and otherwise." (Zentella 2007)
** Dominicans keep their low prestige variety, because "overcorrectness" is associated with
"feminine"
→ media: promotes wide spread/conservative varieties  ""

-> Zentellas Konzept "linguistic insecurity": "consider their dialect inferior to others"[Zentella07]


* Spanglish /Code switching!
"Powerful socioeconomic and cultural forces stimulate borrowing whenever two cultures are in contact;
the borrowing by the subordinate group's language from that of the dominant group is always
significantly greater" [Zentella90]
Factors, which facilitate borrowings:[Zentella90]
 1) Most items reflect a cultural reality that is new or different, e.g., [kei/keike/keiki]
 «cake», [bobipín] «bobby pin», [ŷins/bluŷines] «(blue)jeans.»
 "la boila" (boiler) [Zentella97]
 2) Others may be facilitated or even triggered by a similarity in the phonological and/or morphological
 structure of a word in the dominant language that makes it sound like a possible subordinate word,
 matre(s) «mattress» is similar to that of madre(s) «mother(s).
 ZB "libreria" (Buchladen) für Bibliothek (library)
 3) reduction in the number of syllables, e.g., [beis] replaces florero for «vase»
 4) prestige
 5) Anglicisms can play the role of neutralizer between competing dialectal variants because the
 prestigious outside language acts as the lingua franca that resolves the conflict without favoring one
 group at the expense of the other.

"but it seems that the longer a group has been in the U.S., the less they use the homeland's term"
"second generation Cubans incorporated more loans in their speech than members of the first generation."
"Language proficiency appears to be a more relevant factor than age" [Zentella90]

* influence of the media: "they do contribute to popular acceptance and use of some new vocabulary" [Zentella90]
(siehe oben bei Spanglish)
-> Spielen auch für leveling eine Rolle: versuchen neutrale Varianten zu nutzen, um möglichst mehr Menschen zu erreichen;

%%%%%%%%%%%%%%%%%%%%%%%%%%%%%%%%%%%%%
[Zentella07]

%Intro

the defining role of language networks in identity, i.e., “identity is defined as the linguistic con-
struction of membership in one or more social groups or categories” (Kroskrity 2001: 107)
Latina/o identity in the USA is often linked to Spanish, presumed
to be the heritage language of more than 40 million people with roots in 20
Spanish-speaking Latin American nations, including Puerto Rico.
[Zentella07]

distinct ways of being Latina/o are shaped by the dominant language ideology
that equates working-class Spanish speakers with poverty and academic failure,
[Zentella07]
--> also eine Erklärung, warum sie zu Language Shift tendieren würden

The fate and form of the languages spoken by US
Latinas/os will be determined in part by the ways in which they respond to the
construction of their linguistic identities as a group and as members of distinct
speech communities, and those responses in turn can have a significant impact
on Latina/o unity.

%Latina/o Linguistic Capital

In
the USA, where race has been remapped from biology onto language because
public racist remarks are censored, comments about the inferiority and/or
unintelligibility of regional, class, and racial dialects of Spanish and English
substitute for abusive remarks about color, hair, lips, noses, and body parts, with
the same effect.

--> noch eine Erklärung für Language Shift

no one expects you to be able to change your color, but you are expected to
change the way you speak radically to earn respect (Urciuoli 1996).

These attitudes are communicated in everyday conversations and promoted by
the media and public institutions, but some groups of Latinas/os are more affected
than others.

%%%%%%%%%%%%%%%%%%%%%%%%%%%%%%%%%%%%%%%%%%%%%%%%%%%%%%%%%%%%%%%

Media
-----
This is especially true regarding language, because, as Trudgill notes, «the media... have almost no effect at all in
phonological or grammatical change» (1984:61), although they do contribute to popular acceptance and
use of some new vocabulary. [Zentella90]

%More stuff zu Media
%\cite[vgl.][s.20]{Tomasello06}
