\section{Fazit}

\begin{comment}
    * Einleitung und Fazit müssen zusammenpassen.
    * sind die Erkenntnisse im Fazit aus der Arbeit ableitbar?
    * Vorgehensweise zusammenfassen

Kritische Zusammenfassung
\begin{itemize}
    \item was war in den Texten nicht so gelungen?
    \item welche Fragen sind offen geblieben?
    \item in welche Richtung kann noch weiter geforscht werden?
    \item was sind gegnerische Meinungen zum Thema?
\end{itemize}
\end{comment}


Das Ziel dieser Arbeit war, die Phänomene des Transfers und der Übergeneralisierung im Zweitspracherwerb aus einer kognitiv-linguistischen Perspektive zu beleuchten.
Dazu wurden diese zunächst in der historischen Entwicklung der Zweitspracherwerbsforschung verfolgt.
Daran anschließend wurden Transfer und Übergeneralisierung am konkreten Beispiel von \textit{contrast left dislocation} und \textit{clitic left dislocation} Konstruktionen in den Zweitsprachen Englisch und Spanisch in \cite[][]{Valenzuela05} untersucht.
Es hat sich gezeigt, dass Merkmale, die nicht auffällig genug sind oder in der L1 auf einer anderen Weise verarbeitet werden als in der L2, schwieriger wahrgenommen werden
und es in solchen Fällen häufig zu einer Übergeneralisierung von erlernten L2-Konstruktionen kommt.
Das war in \cite[][]{Valenzuela05} mit den CLLD-Konstruktionen der L2 Spanisch Sprechenden der Fall.
Wenn wiederum ein Merkmal relevant für eine Konstruktion in der L1 war, ist es schwieriger, dieses bei der Sprachproduktion in der L2 zu ignorieren.
Wenn dieses Merkmal im Sprachgebrauch der L2, trotz seiner Irrelevanz für die L2, berücksichtigt wird, beobachten wir einen Transfer.
In \cite[][]{Valenzuela05} wurde das am Beispiel der Spezifität des syntaktischen Topiks und ihrer Relevanz für die Art der Topik-Konstruktion realisiert.
Valenzuela hat gezeigt, dass Spanisch L1 Sprecherinnen dieses Merkmal häufig in ihrem Englisch L2 Sprachgebrauch berücksichtigt haben.

Während die Argumentation zu den Voraussetzungen der Erscheinungen Übergeneralisierung und Transfer plausibel und schlüssig erscheint, sind jedoch die Ergebnisse Valenzuelas Studie \cite[][]{Valenzuela05} mit gewisser Vorsicht zu genießen.
Zum Einen fallen die Antworten der Kontrolgruppe aus Muttersprachlerinnen bei manchen Aufgaben ziemlich heterogen aus.
Zum Anderen fragt es sich, ob die Schlussfolgerungen, die bei zwei kleinen Gruppen von Probandinnen erzielt wurden, universell verallgemeinbar sind.

Für zukünftige Studien des Transfers und der Übergeneralisierung im Zweitspracherwerb wären folgende Fragen interessant:
bei welchen anderen Konstruktionen treten diese häufig auf?;
sind Übergeneralisierung und Transfer eher bei abstrakten Konstruktionen zu beobachten oder kommen sie auch in lexikalisch konkreten Konstruktionen vor?;
hängt das Vorkommen dieser Phänomene vom Niveau der L2 Sprachbeherrschung der Lernenden?

% wenig Menschen -- sind die Ergebnisse einfach so zu verallgemeinern?
% Kontrolgruppe manchmal uneindeutig/anders als für "richtig" gehalten aber in der Deutung der Ergebnisse doch zurecht gebogen


%Es werden 2 Ansätze vorgestellt, was (Zweit)Spracherwerb angeht: sequentieller Spracherwerb und Erwerb eines Netz von Konstruktionen

%Gegnerische Meinungen:

%\cite{Cook93} - generative Perspektive auf Zweitspracherwerb
%``A person who knows two languages has been through the acquisition process twice.
%Second Language research must explain the means by which the mind can acquire more than one grammar.''

%``One goal of second language research is to describe grammars of more than one language simultaneously existing in the same person.''

%``The importance of second language research lise not in its account of the knowledge and acquisition of the L2 in isolation, but its account of the second language present and acquired in a mind that already knows a first [...]''

%aber das sind doch voll peanuts.. ist überhaupt nicht der kernpunkt..
%\cite{Valenzuela05} Kritik:
%bisschen zu großzügige Interpretation der gemessenen Daten zugunsten von der Arbeitshypothese


