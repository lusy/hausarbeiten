\section{Conclusión}

A forma de conclusión, volvemos a la pregunta: ¿a quién pertenece la red?
¿Al estado? ¿a empresas multinacionales como Google o Facebook? ¿a lxs proveedorxs de servicios de comunicación? ¿o a sus usuarixs?

Ya hemos comprobado, que lxs partidarixs del ciberoptimismo dirían a lxs usuarixs.
Y aun, como hemos demostrado, eso lamentablemente casi nunca puede ser aplicado completamente, los proyectos de redes comunitarias hacen lo máximo posible para capacitar una infraestructura libre cual no se pueda usar para desventaja de la sociedad civil.

Hemos estudiado el proyecto argentino AlterMundi y un número de redes comunitarias que este ha ayudado a impulsar.
Nos hemos enfocado en la controversia de la soberania de comunicación la cual hemos analizado desde la Teoría del Actor-Red y la teoría de bienes comunes.

Y hemos llegado a la conclusión que los problemas que tales proyectos encuentran son en su esencia sobre todo problemas políticos.
Ahora bien, como señala Bernhard Rieder, la tecnología sóla no puede resolver problemas políticos, sino lo que se necesita son soluciones políticas~\autocite{Rieder2012}.

\begin{comment}
%ANT:
[Rieder2012]
"Software now habitually provides specific answers to questions that do not seem technical at all: What is communication? What is cooperation? Which information is valuable? What is decision-making?"
"The idea that technology implements values and should be seen as an “actor” in the shaping of
social processes is of course not new; from actor-network theory"
"In many of the “self-organized” sys-
tems that make up Web 2.0, we find that a small group dominates structures of visibility."
Zoom out: big internet providers + social media corporations (Google, Facebook, Cisco, AT&T)
Zoom in to community networks: you'd see above all Altermundi, but not the redes

"The hope for magical technological
solutions to the messy realities is counterproductive if it leads to an attitude that disengages tradi-
tional political process to simply “route around it” ." --> LibreRouter routs around the firmware lockdown quite litereally

## Conclusión del texto [Rieder2012]

> If technology won’t deliver us from the conundrums of
> governance, negotiation, and struggle, we
> may as well reengage politics proper[ly].
Ich glaub, it always boils down to thinking critically about what one's doing and assuming responsibility for it.

[FiTre2015]
und fazit: es lohnt sich politische Loesungen fuer polit. Probleme zu suchen
"these examples show that ‘insider’ strategies, i.e. direct engagement with policy-makers, are worth pursuing."
-- davor kommentar: es ist fuer solche Iniziativen oft nicht feasible sich mit Politik zu beschaeftigen, v.a. da sie dafuer keine Ressourcen haben;
es ist eine volunteer community, die netzwerke bauen moechte und nicht sich mit gesetzestexten beschaeftigen.
weiss nicht in wie fern das alles auf altermundi zutrifft, aber die sind im grunde auch alles techies;
und auch wenn deren motivation weniger faszination mit technik und mehr hippie-soberania comunicativa-derecho a comunicación ist, haben die vermutlich genau so wenig menschen, die von legal issues plan haben und nicht unbedingt riesenbock sich da reinzusteigern

[PHPHLT2016]
"As new technologies offer new ways of engaging with emergent research
environments, our actual practices as ethnographers also shift." --> man kann darüber meta reflektieren, welche konsequenzen das für meine arbeit hat

"how inequality is extended, reproduced or complicated by digital media technologies" --> kann leider nicht befriedigend beurteilen aus der ferne, aber ich würde eher für das gegenteil argumentieren beim projekt: es ist ein versuch, the digital devide zu schließen (bridging the digital devide);

\end{comment}
