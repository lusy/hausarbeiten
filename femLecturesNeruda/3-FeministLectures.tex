\section{Feministische Lektüren}

In diesem Kapitel werden wir kurz die Werkzeuge erläutern, die für die anschließende Diskussion einer Auswahl Nerudas Liebesgedichte benutzt werden sollten.

% vlt mit dem Kanon anfangen??? weil da gehts auch darum, die Essenz und "gute" Literatur zu bestimmen

\subsection{Pluralität der Lektüren}

% an sich feministisch, diese zu erlauben und nicht für DIE (einzig) richtige Lektüre eines Textes zu argumentieren
Eine Pluralität der Lektüren einzuräumen ist bereits an sich eine feministische Handlung,
da eine inhärente Charakteristik des vorherschenden patriarchalen Literaturdiskurses ist, die ``richtige'' Interpretation, die ``Essenz'' des Textes zu suchen. (Quote! Kolodny??)
\begin{comment}
   * Pluralität (vlt in die Intro vorziehen)
     Annette Kolodny: "In my view, our purpose is not and should not be the formulation of any single reading method or potentially procrustean set of critical procedures[...] Instead, as I se it, our task is to initiate nothing less than a playful pluralism, responsice to the possibilities of multiple critical schools and methods, but captive of none.."
\end{comment}
%Authority!

In anderen Worten, ein Text offenbart keine zugrunde liegende Wahrheit, sondern bieten vielmehr einen Ausgangsfläche/basis/Anknüpfungspunkt an, aufgrunddessen mit Hilfe verschiedener Methodologien/Strategien/... eine Mehrzahl an Bedeutungen erzeugt/erschlossen werden können~\autocite{Beehler1988}.

\begin{comment}
[Beehler1988]
"Consequently, what we teach in the English class-
room is not "literature" but ways of reading. By
helping students to identify different methods,
different positions from which to view a work, we
help them to realize that texts do not "reveal"
truth: they simply provide the field upon which
meanings can be produced."
\end{comment}

\subsection{Close Reading}

Eins der zentralen Werkzeuge, die wir benutzen werden, um uns Nerudas Liebeslyrik zu erschließen, ist das Close Reading.
Bevor wir fortfahren, müssen wir uns kurz mit der historischen Entwicklung dieses Begriffs beschäftigen, denn es ist alles andere als klar und eindeutig, was die Literaturwissenschaftler*innen über die Jahre darunter verstanden haben (und immer noch heute verstehen).
In Anlehnung an der Analyse von Joseph North wollen wir zwischen dem Close Reading vom liberalen britischen Literaturkritiker I.A. Richards und dem der (konservativen) US-amerikanischen Strömung des New Criticism unterscheiden~\autocite{North2013}.
Was wir hier operationalisieren werden ist das Konzept in seinem eher ursprünglichen, von Richards geprägten, Sinn\footnote{Ich verlasse mich hier stark auf die historische Begriffsanalyse von North, bzw. meine Interpretation davon, da eine erschöpfende Herausarbeitung des Begriffs meinerseits, auch wenn bestimmt sehr spannend, den Rahmen dieser Arbeit sprengen würde.}.
% Was sagt uns der?
Es geht dabei nicht darum, nur ein Werk für sich zu betrachten und jeglichen historischen und politischen Kontext außer Acht zu lassen.
% "gives us the very misleading impression that it is somehow, at root, a practice of autonomous or idealist aesthetics, and as such originally or even necessarily dehistoricizing or depoliticizing."
Richards setzt sich (laut North) nicht für ``l'art pour l'art'' ein, sondern möchte mit seinem Close Reading viel mehr evaluieren (syn!), was der Text sagt (syn!), aber in Bezug auf die (Wechsel-)Wirkung(-en) mit den rezipierenden Leser*innen.
Er geht von einem instrumentellen Ästhetik-Begriff aus, von der Annahme, dass Ästhetik ein Mittel und nicht der Zweck ist.
Der Zweck ist, uns zu helfen/den Rezipirenden zu helfen, unsere/deren Weltanschauung zu ordnen.
Die lesende Person (mit ihren Erfahrungen, Wissen, Einstellungen, Weltanschuung, Gemütszustand) ist dabei der Kontext, in dem das Werk analysiert wird.
Und Close Reading ist die Methode, die den Lesenden zu einer Ästhetikbildung und der Verbesserung ihrer Leben verhelfen sollte.
Wenn wir diese Annahme treffen, ist auch der Gedanke nahe liegend, dass auch multiple Lektüren pro Person möglich sind.

Und in dem Sinne, hier vorweggenommen: meine persönlichen Erfahrungen, Wissen, Werte, etc. werden der Kontext für die angebotenen Lektüren von Nerudas Liebeslyrik sein/darstellen.
% I am the context within which the close reading of Neruda's poems will occur (meine Erfahrungen, Wissensbasis, Werte, Gemütszustand..)

%Weitere Merkmale von Close Reading, die uns helfen werden
Vielleicht ist die Definition Nancy Boyles' greifbarer: diese unterstreicht, dass es beim Close Reading darum geht, Bedeutungsschichten aufzudecken, die ein tieferes Verständnis des Textes ermöglichen werden~\autocite[90]{Boyles2016}.

Auch die Professorin für Literaturwissenschaft Jane Gallop hebt das Emanzipatorische an dem Close Reading hervor: es ermöglicht (syn!) Studierenden (und nicht nur) nicht nur bereits vorhandenes Wissen anzuwenden, das ihnen von Lehrbüchern und Dozierenden angeboten wurde, sondern viel mehr, in einem direkten Kontakt mit Texten, ihr eigenes Wissen zu produzieren~\autocite{Gallop2007}.
Dieser Gedanke ähnelt sehr an dem von North festgehaltenen: ``The `practical' in Practical Criticism had, in Richards’s usage, meant something like `directed towards the practical end of training readers';''~\autocite{North2013}.

%Instrumentalisierung für mich muss konkreter werden; oder?

\begin{comment}
  https://books.google.de/books?hl=de&lr=&id=Cbz7CwAAQBAJ&oi=fnd&pg=PA89&dq=close+reading&ots=8-Z-fTKNzq&sig=tkwM8Hs0aQs8blONWas-cwGtHpY#v=onepage&q=close%20reading&f=false
  On Developing Readers: Readings from Educational Leadership (EL Essentials) - ed. Marge Scherer; Verlag: ASCD, Alexandira, VA, USA
  9. Closing in On Close Reading - Nancy Boyles
  "Essentially, close reading means reading to uncover layers of meaning that lead to deep comprehension." (p.90)

  [Gallop2007]
"When the New Critics introduced the methodology called close reading
in the years just before and after World War II, what it replaced was liter
ary history (the old historicism, we might call it)."

"I would argue that the most valuable thing English ever had to offer was
the very thing that made us a discipline, that transformed us from cultured
gentlemen into a profession: close reading. Not because it is necessarily the
best way to read literature but because it, learned through practice with lit
erary texts, learned in literature classes, is a widely applicable skill, of value
not just to scholars in other disciplines but to a wide range of students
with many different futures. Students trained in close reading have been
known to apply it to diverse sorts of texts--newspaper articles, textbooks
in other disciplines, political speeches--and thus to discover things they
would not otherwise have noticed. This enhanced, intensified reading can
prove invaluable for many kinds of jobs as well as in their lives."

"For more than three decades, antielitist pedagogy has crystallized
around Paulo Freire's criticism of the banking model, in which the teacher
deposited knowledge in the student. This model remains dominant in
most academic disciplines where there is a huge gap between scholars
producing knowledge and classrooms where students receive, repeat, and
apply that knowledge. The literature classroom has represented a real
alternative to the banking model: students had to encounter the text di
rectly and produce their own knowledge; close reading meant they could
not just apply knowledge produced elsewhere, not just parrot back what
the teacher or textbook had told them"

http://writingcenter.fas.harvard.edu/pages/how-do-close-reading
Copyright 1998, Patricia Kain, for the Writing Center at Harvard University

 When you close read, you observe facts and details about the text. You may focus on a particular passage, or on the text as a whole. Your aim may be to notice all striking features of the text, including rhetorical features, structural elements, cultural references; or, your aim may be to notice only selected features of the text—for instance, oppositions and correspondences, or particular historical references.

 The second step is interpreting your observations. What we're basically talking about here is inductive reasoning: moving from the observation of particular facts and details to a conclusion, or interpretation, based on those observations. And, as with inductive reasoning, close reading requires careful gathering of data (your observations) and careful thinking about what these data add up to.

https://edsitement.neh.gov/blog/2015/01/05/birth-close-reading
Posted January 5, 2015 - 1:43pm | By Joe Phelan

Every word, every line, must be considered and reconsidered, as well as their place in the whole structure

http://web.uvic.ca/~englblog/closereading/
The Close Reading of Poetry
A Practical Introduction and Guide to Explication
 Posted on March 2, 2012
 © G. Kim Blank & Magdalena Kay < > English Department, University of Victoria

There is no single way to do a close reading of a poem. Sometimes an impression is a way in; sometimes the “voice” in the poem stands out; sometimes it is a matter of knowing the genre of the poem; sometimes groupings of key words, phrases, or images seem to be its most striking elements; and sometimes it takes a while to get any impression whatsoever. The goal, however, is constant: you want to come to a deeper understanding of the poem. T

\end{comment}

\subsection{Die Rolle des Kanons}

\begin{comment}
   * Kanon
    ** Was einmal im Kanon ist, ist halt im Kanon, und das "ästhetische" Wert wird nicht hinterfragt (vgl Annette Kolodny;  wir sollten uns dringendst fragen sollten, wie der Kanon zustande gekommen ist, wer das bestimmt und was drin ist und warum)
    ** wir lesen gerne und gut, das was wir gelernt haben zu lesen
    ** Annette Kolodny: "we appropriate meaning from a text according to what we need (or desire) or, in other words, according to the critical assumptions or predispositions (conscious or not) that we bring to it. And we appropriate different meanings, or report different gleanings, at different times-even from the same text-according to our changed assumptions, circumstances, and requirements."
    ** Annette Kolodny: "It can provide that, but, I must add, too often it does not. Frequently our reading habits become fixed"
   * ..

   [North2013]
* New Criticism unterstützt und versucht einen zeitlosen Kanon, aufgrund deren "intrinischen ästhethischen Werten" aufzubauen
* Bsp für Elitismus, eine Elite, die versucht ihre eigene priviligierte Stellung dadurch zu sichern
* Wer gehört dazu?: white, male, christian, property owners
* "It is to be noted that in both these cases the concern is ultimately to en-
sure that valid considerations about texts in themselves are distinguished
from invalid considerations about their effects, psychological or moral,
on their readers." --> die Objektivität, die uns die westliche patriarchale Tradizion der Wissensproduktion uns vorgauckelt
--> ToDo: Need more sources on this one
* quote Brooks: "I believed, to set up a kind of scale: at the bottom, poems that
relied heavily on the principle of exclusion, left out too much of human
experience, and so were thin and over simple. They tended accordingly
toward sentimentality and general vapidity. Toward the top of the scale
were poems that used successfully a high degree of inclusion”" (Cleanth Brooks: “I. A. Richards and Practical Criticism”)
--> also sehr wohl gute von schlechten Werken trennen wollen; nen Kanon errichten
* "prop up a hiearchy of aethetic values"
* "enthusiastic embrace of the idea of a hierarchy"
* "aesthetic value residing solely in the text itself"
* "we have arrived at the sterile concern with hierarchy and canonicity that
will occupy much of Anglophone literary studies throughout the Cold
War period: a series of ultimately unresolvable debates about the exact
constitution of a universal canon, as if one could determine what was
“good” art and what was “bad” art without any reference to what the art
might be good or bad for."

\end{comment}

\subsection{Lektüre im historischen Kontext}
\begin{comment}

  % Wieder in Kontext integrieren; Close Reading ist schön, aber wir laufen Gefahr, Werke zu kanonisieren und (Bedeutungs-)Schichten zu vergessen
  % Allerdings argumentiert die Frau, dass sie genau um dieser "timeless" Kanonisierung entgegenzuwirken, sich für Close Reading einsetzt
  % Das ist auch was North argumentiert, was das Ziel von Close Reading ursprünglich war.
  % Wenn wir uns für diese Interpretation von Close Reading entscheiden (und tun wir sehr wohl) bleibt die frage, in wie fern das ein anderer Unterkapitel ist als Close Reading itself
  [Gallop2007]
  re and complained that de
  constructionist literary criticism as practiced in United States English
  departments was in fact all too much like the old New Criticism--elitist,
  canonical, and ahistorical.

  "It is precisely my opposition to timeless universals that makes
  me value close reading. I would argue that close reading poses an ongoing
  threat to easy, reductive generalization, that it is a method for resisting
  and calling into question our inevitable tendency to bring things together
  in smug, overarching conclusions. I would argue that close reading may in
  fact be the best antidote we have to the timeless and the universal."

  --> Verständnis, das sich mit North's account on Richards' idea zusammenfügt

\end{comment}

