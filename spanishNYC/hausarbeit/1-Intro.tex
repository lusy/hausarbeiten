\section{Einleitung}

%TODO
% vlt mit ner Geschichte/genauem Beispiel starten

Sprachkontakt ist eins der zentralen Phänomene, die die Soziolinguistik untersucht.
In einer Kontaktsituation kommt es besonders oft zu Code-Switchings, manche von denen dann graduell in Entlehnungen übergehen.
Diese sind manchmal das Ergebnis einer spontanen Sprachproduktion, in der der Sprecherin ein Wort oder Ausdruck nicht eingefallen ist.
Oft sind allerdings die Switches auch eine sehr bewusste Sprecherin-Entscheidung, die dazu dient, einen Sachverhalt möglichst präzise auszudrücken,
sich als Insiderin einer bestimmten Gemeinschaft zu positionieren und sich von einer anderen abzugrenzen oder noch subtilere ideologische Einstellungen auszudrücken.

%TODO
%Notiz zum generischen Femininum

Das Ziel dieser Arbeit ist eine Analyse der Code-Switches und Entlehnungen aus dem Englischen in Artikeln der primär auf Spanisch verfassten US-amerikanischen Zeitschrift \textit{Siempre Mujer} aus einer soziolinguistischen Perspektive.
Es gilt herauszufinden, ob diese Code-Switches in bestimmten Kontexten passieren oder bestimmten rekurrenten Mustern unterliegen.
Wenn diese Erwartung erfüllt wird, wird eine sozioindexikalische Erklärung versucht.

Die Ausgangshypothese ist, dass dies sehr wohl der Fall ist.
Im hauptsächlich spanischen Text werden in speziellen Fällen englische Ausdrücke erwartet:
Zum Beispiel wenn es um technische Begriffe oder um Idiome/Redewendungen geht, für die keine genaue spanische Übersetzung existiert.
Weitere Fälle, in denen wir Englisch erwarten können, sind region- beziehungsweise kulturspezifische Begriffe: Tiere und Pflanzen, Bräuche, Gerichte, etc.

Um diese Analyse zu realisieren, wurden Artikel aus der elektronischen Ausgabe der Zeitschrift heruntergeladen und Code-Switches ins Englische extrahiert.
Diese wurden anschließend in Kategorien sortiert und verschiedene Erklärungen für die einzelnen Kategorien wurden in Betracht gezogen.


\subsection{Aufbau der Arbeit}

Im Weiteren ist diese Arbeit wie folgt aufgebaut:
Kapitel~\ref{chap:basics} gibt einen Überblick über die theoretischen Grundlagen, auf denen die vorliegende Analyse aufbaut.
Verwandte Arbeiten werden in Kapitel~\ref{chap:related-works} zusammengefasst.
In Kapitel~\ref{chap:corpus} stelle ich das untersuchte Korpus und die benutzte Methodik vor.
Die Ergebnisse der Untersuchung werden in Kapitel~\ref{chap:results} diskutiert.
Kapitel~\ref{chap:conclusion} gibt einen Ausblick auf Verbesserungen, Anregungen und Fragen für zukünftige Analysen und fasst nochmal abschließend die Arbeit zusammen.

%\cite[vgl.][s.20]{Tomasello06}

\begin{comment}
%Media
% erstmal: außen vor lassen; wenn das hier aufgemacht wird, kriegt die arbeit ne völlig neue dimension
* media: promotes wide spread/conservative varieties

* influence of the media: ``they do contribute to popular acceptance and use of some new vocabulary'' [Zentella90]
(siehe oben bei Spanglish)
-> Spielen auch für leveling eine Rolle: versuchen neutrale Varianten zu nutzen, um möglichst mehr Menschen zu erreichen;


%http://www.pbs.org/speak/ahead/mediapower/media/#talk
% Jack Chambers
% “Talk the talk?” [author’s title “TV and Your Language.”]. Website “Do You Speak American?”
% McNeil - Lehrer Productions. http://www.pbs.org/speak/ahead/mediapower/media
% 2005

A final common assumption is that the media leads language changes. In fact, it belatedly reflects the changes.

The same fallacy seems to underlie the casual assumption that the mass media drives all kinds of language changes.

If the mass media can popularize words and expressions, then “presumably” it can also spread other kinds of linguistic changes. We generalize from one limited effect to a host of others.
%%%%%%%%%%%%%%%%%%%%%%%%%%%%%%%%%%%%%%%%%%%%%%%%%%%%%%%%%%%%%%%%%%%

Media
-----
This is especially true regarding language, because, as Trudgill notes, «the media... have almost no effect at all in
phonological or grammatical change» (1984:61), although they do contribute to popular acceptance and
use of some new vocabulary. [Zentella90]

Marshall McLuhan: hot and cold media
hot media: ``Hot media do not leave so much to be filled in by audience.
Hot media are, therefore, low in participation or completion by audience.'' [Willie79]

(http://www.cliffsnotes.com/sciences/sociology/contemporary-mass-media/the-role-and-influence-of-mass-media)

``Communities and individuals are bombarded constantly with messages from a multitude of sources including TV, billboards, and magazines, to name a few. These messages promote not only products, but moods, attitudes, and a sense of what is and is not important. Mass media makes possible the concept of celebrity: without the ability of movies, magazines, and news media to reach across thousands of miles, people could not become famous. In fact, only political and business leaders, as well as the few notorious outlaws, were famous in the past. Only in recent times have actors, singers, and other social elites become celebrities or “stars.” '' (http://www.cliffsnotes.com/sciences/sociology/contemporary-mass-media/the-role-and-influence-of-mass-media)

%The limited effects theory
The limited‐effects theory argues that because people generally choose what to watch or read based on what they already believe, media exerts a negligible influence.
%Criticism
First, they claim that limited‐effects theory ignores the media's role in framing and limiting the discussion and debate of issues. How media frames the debate and what questions members of the media ask change the outcome of the discussion and the possible conclusions people may draw. Second, this theory came into existence when the availability and dominance of media was far less widespread.

%The class-dominant theory
The class‐dominant theory argues that the media reflects and projects the view of a minority elite, which controls it.
For example, owners can easily avoid or silence stories that expose unethical corporate behavior or hold corporations responsible for their actions.
The issue of sponsorship adds to this problem. Advertising dollars fund most media.
elevision networks receiving millions of dollars in advertising from companies like Nike and other textile manufacturers were slow to run stories on their news shows about possible human‐rights violations by these companies in foreign countries.

%The culturalist theory
The culturalist theory, developed in the 1980s and 1990s, combines the other two theories and claims that people interact with media to create their own meanings out of the images and messages they receive.
This theory sees audiences as playing an active rather than passive role in relation to mass media.
Therefore, culturalist theorists claim that, while a few elite in large corporations may exert significant control over what information media produces and distributes, personal perspective plays a more powerful role in how the audience members interpret those messages.

%http://www.pbs.org/speak/ahead/mediapower/media/#talk
% Jack Chambers
% “Talk the talk?” [author’s title “TV and Your Language.”]. Website “Do You Speak American?”
% McNeil - Lehrer Productions. http://www.pbs.org/speak/ahead/mediapower/media
% 2005
Second, the lasting power of words that spread via the mass media has nothing to do with the various media themselves. Punk’d lasted just two seasons (2002-03). While it lasted, its name was raised into common parlance. What are the chances a word will persist for another five to 10 years? Not good. In buzzwords as in outré attire, there is a direct relationship between the height of the craze and the decline into oblivion. Fads mark their users as members of an in-group. The faster fads spread, the more pressure there is to find a new marker. Only your mother, if she was a beatnik, thinks rimless specs are groovy. Only your grandmother, if she was a gate, thinks black horn-rims are crazy.

A final common assumption is that the media leads language changes. In fact, it belatedly reflects the changes.

The same fallacy seems to underlie the casual assumption that the mass media drives all kinds of language changes.

If the mass media can popularize words and expressions, then “presumably” it can also spread other kinds of linguistic changes. We generalize from one limited effect to a host of others.

Finally, we should note that high mobility has even greater social significance than the media explosion.


\end{comment}
