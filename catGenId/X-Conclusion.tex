\section{Conclusions}

Amb aquest treball hem proposat lectures possibles de la novel·la ``L'últim patriarca'' de Najat El Hachmi.
Partint de nocions teòriques com ``cultura'', ``llenguatge'' o ``patriarcat'' hem discutit facetes múltiples d'aquesta obra essencial per a la literatura catalana.
Hem intentat a tractar les preguntes ``Qui és català?'', ``Qui decideix?'', ``Com s’ha de comportar una dona entre dues cultures i dues societats?''
i oferir direccions per reflexionar.
Com ja es va assenyalar, aquestes idees són només una proposta i es poden interpretar com a punt de partida per a les reflexions de cadascú i cadascuna amb l'objectiu d'augmentar la consciència i la sensibilitat pels problemes als quals les dones avui encara s'enfronten.

\begin{comment}
[Vidal2012]
"Y el mundo de El Hachmi no es sino un
palimpsesto, por eso también lo son sus libros, tablillas en la que se aprecian hue-
llas de escrituras/culturas anteriores y en las que leemos la última, que se nos apa-
rece más perceptible. Sin embargo, las huellas persisten."


\end{comment}
