\section{Conclusión}

A forma de conclusión, volvemos a la pregunta: ¿a quién pertenece la red?
¿Al estado? ¿a empresas multinacionales como Google o Facebook? ¿a lxs proveedorxs de servicios de comunicación? ¿o a sus usuarixs?

Ya hemos comprobado, que lxs partidarixs del ciberoptimismo dirían a lxs usuarixs.
Y aun, como hemos demostrado, lamentablemente eso casi nunca puede ser aplicado completamente, los proyectos de redes comunitarias hacen lo máximo posible para viabilizar una infraestructura libre cual no se puede usar para desventaja de la sociedad civil.

Hemos estudiado el proyecto argentino AlterMundi y un número de redes comunitarias que este ha ayudado a impulsar.
Nos hemos enfocado en la controversia de la soberania de comunicación cual hemos analizado desde la Teoría del Actor-Red y la teoría de bienes comunes.

Y hemos llegado a la conclusión que..

\begin{comment}
[PHPHLT2016]
"As new technologies offer new ways of engaging with emergent research
environments, our actual practices as ethnographers also shift." --> man kann darüber meta reflektieren, welche konsequenzen das für meine arbeit hat

"how inequality is extended, reproduced or complicated by digital media technologies" --> kann leider nicht befriedigend beurteilen aus der ferne, aber ich würde eher für das gegenteil argumentieren beim projekt: es ist ein versuch, the digital devide zu schließen (bridging the digital devide);
\end{comment}
