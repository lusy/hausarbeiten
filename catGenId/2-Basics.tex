\section{Fonaments teòrics}

\subsection{La immigració a Catalunya al llarg dels anys}

``L'últim patriarca'' és, entre altres coses, la història d'una família d'immigrants.
Per això, repassarem breument el desenvolupament de la immigració a Catalunya a l'útlim segle.
Hem de notar que Catalunya sempre ha sigut una terra d'acollida d'immigrants.
Només al segle XX hi havia multiples onades d'immigració, començant amb la primera als anys '20, passant per la segona, als anys '50-'60, la tercera a les '80 i arribant a l'actualitat [quelle!],~\autocite{TarPaGa2013}.
Totes aquestes onades tenen trets diferents -- més prominent, per exemple, l'origen de la nova població o les raons per immigrar, però totes tenen també similituds.
Una d'elles és la recepció dels i de les recentment arribats.
Com assenyala Fransesc Candel al seu llibre ``Els altres catalans'', cada onada està rebuda per una mena de desconfiança, prejuicis i franca hostilitat~\autocite[17]{Candel1965}.
La població vella (syn) té por del desconegut i de l'incert i sovint reacciona amb discursos i comportaments racistes:
``\ldots heu vingut a menjar-vos el pa dels catalans\ldots''~\autocite[17]{Candel1965},
``Aquests nou[sic!] catalans no coneixen Catalunya: la seva tradició, la seva història, el seu art, la seva cultura, la seva literatura, els seus costums, el seu folklore\ldots''~\autocite[17]{Candel1965},
``Parlen un català gruixut, groller i vulgar\ldots Desconeixen la gramàtica catalana. No saben llegir en aquest idioma. Escriure'l, encara menys.''~\autocite[18]{Candel1965}

La immigració de la família Driouch descrita a la novel·la té com a fons un fet/periode històric:
les diverses onades migratòries del nord d'Àfrica cap Espanya que es produexien a partir de la dècada de les '80~\autocite{TarPaGa2013}.
Segons aquest mateix informe, a l'any 2013 els marroquins són amb 20.4\% el més gran grup d'immigrants a Catalunya.
Els autors confirmen:
``El col·lectiu procedent del Marroc ha estat el més important des dels inicis de la immigració estrangera recent a Catalunya, tant pel nombre com per la seva distribució en el territori.''~\autocite{TarPaGa2013}

La mateixa autora de ``L'útlim patriarca'' va néixer a Marroc, quan el seu pare ja vivia a Catalunya, i va immigrar a Vic amb la resta de la seva família quan tenia vuit anys~\autocite{Vidal2012}.
Llavors, una lectura de la novel·la com obra amb elements autobiogràfics (llevat del grad de proximitat entre ficció i realitat) ocorre de manera automàtica.



--compare with interviews!!
Com s'adapta la gent?


\begin{comment}

---------

Entrevistes: lectura autobiogràfica?

  http://www.belletrista.com/2011/Issue12/features_2.php
--> die Fragmente können vlt mit zum Patriarchat?

  --naming things!!!
  "N. By learning the language, by learning the words, and reading the dictionary, she starts to name what happens around her, and the first step to get out of an abusive situation is to know what's happening and have the ability to name it. So that's very important. If you think about therapy, when we go to therapy the only thing we do is name the things that are happening around us and inside us. So, if you don't have enough words to name the things, you are not able to be free. And literature is a way to see different lives—lives that can be deeper and end happily. "

  "We don't talk a lot about violence and sometimes there is something very dangerous about people that are violent, and that is the silence. Silence makes you part of this violence."

   N: Some of the things that happen around him justify what he does. I was very surprised when I was writing the book that women in the family were so important in making him become the monster he becomes.

   L. So their behavior allows him to become that way?

   N. Yes, and you can see that. His sisters seem to coddle and protect him, his mother a little less so, and less so as she gets older. I think that wasn't very helpful for him. It's something that's really shocking because they're women, but I've seen in that area of Morocco that sometimes you receive more compassion from men than from women. Or from the older women in the families because they are becoming more powerful as they get older. 

"L'àvia sempre va justificar el comportament poc usual del seu fill amb aquesta història [la bufetada] [...] Sí, els sobresalts se't fiquen a dins i es van transformant en la pitjor part de tots nosaltres, però ja ho saps, filla, que en el fons el teu pare és de bona fusta i no faria mai mal. És només això, que els espants no li han acabat de marxar mai del cos i això l'ha fet algú diferent." (p.18)
-- justificacions; niemand zieht den zur Rechenschaft; vgl Laurie Penny, the men are never responsible for their bad life: the women or minorities are the
guilty ones; that's what society teaches us

"Mimoun s'havia anat acostumant que, per a ell, les normes s'exceptuaven sempre." (p.51)
Laurie Penny?

"A Mimoun no és que li semblés la conducta més lògica del món, però tenint en compte que en aquell pais les coses funcionaven tant del revés i que els cristians no tenien cap sentit ni de l'honor ni del que ell considerava dignitat, l'explicació podia ser prefectament plausible." (p.89)
-- bei so was rege ich mich extrem auf; diese Figur ist so krass ueberzogen dargestellt, um gewisse Wirkung zu erzielen/Ideen zu vermitteln; welche?

"Era així com Mimoun aconseguia sempre que les dones de la seva vida l'anessin convertint en patriarca." (p.99)
--Kommentar bzgl. die Schwester kuemmert sich um alles, nachdem er von Spanien abgeschoben wird; wird von allen in seinen Schwachsinn bekraeftigt, die ihm alles erlauben und immer andere Schuldige finden

Soumisha sobre M:
"És un bon home, creu-me, tot el que li passa no és voluntat seva. I li explicava allò del filtre d'amor a la manera de Curial e Güelfa." (p.214)
-- es ist bemerkenswert, wie ihn alle entschuldigen; es sind immer die anderen schuld. die frauen und die minderheiten

 http://www.elcritic.cat/entrevistes/najat-el-hachmi-tothom-esta-sorpres-que-jo-escrigui-sobre-sexe-per-que-perque-soc-marroquina-3627
 P: A ‘L’últim patriarca’ sembla que la responsabilitat de l’honor i de la vergonya recaigui sobre la dona.
A: Això és un dels pilars bàsics del patriarcat. Crec que la majoria de religions monoteistes ho són; el control de la sexualitat femenina és bàsic.

--vgl Laurie Penny

el moral doble és increïble:
"Si ell es divorciés, la seva dona li hauria de continuar sent fidel fins a la mort; per alguna cosa havia estat ell el primer de tenir-la." (p.151)
--> quina intenció té aquest relat? solidifica estereotips..

A banda d’això, sí, la veritat és que al món musulmà segueix molt vigent la idea que l’home té uns instints naturals que no pot controlar i que és la dona qui s’ha d’encarregar de protegir-se.

sobre Jaume:
"No, era un home, n'estava segur, però no patia la disminució natural del seu gènere a l'hora de fer les tasques de la casa." (p.131)

altra violació:
"Però el Manel tenia aquell mena d'instint de caçador que han de tenir per força els que estan destinats  a ser gran patriarques i no entenia que era no." (.91)
-- no means no

Es diu que Catalunya és terra d’acollida i no és veritat: és terra d’immigrants; no existiria sense la immigració. Ara ha sortit un llibre molt interessant que es diu ‘Catalunya al mirall de la immigració’, d’Andreu Domingo, i editat per L’Avenç.


[Vidal2012]

quote El Hachmi, Najat (2004). Jo també sóc catalana. Barcelona: Columna.
"En definitiva: has de fer els malabarismes que calguin per anar decent per a l’estètica nord-africana i alhora no semblar una pobra noia reprimida davant dels originaris de Catalunya."
--> interseccionalitat?
--> o identitat cultural?

"Para ella, hay muchos tipos de discriminación: la del racista que golpea de frente, pero también la del paternalista, que dice cosas como que, aceptando al inmigrante, nuestra cultura se enriquece."
-- identitat cultural?

quote Interview El Hachmi:
"Los pornógrafos de la etnicidad acentúan rasgos de ti que en tu país encontrarías ridículos."
"El inmigrante no quiere pertenecer a una asociación de inmigrantes, sino a una de vecinos..."
"Cada mujer lleva el velo por motivos diferentes. Y no se
puede pretender salvar, de manera paternalista, a las pobres moritas del yugo de
sus maridos.
-El yugo existe.
—Como en otras culturas."
"El problema es que la mayoría tiene permiso de residencia sin permiso de trabajo. La ley de extranjería las condena a la clandestinidad laboral y eso hace que dependan del marido. Les corta la vía de emancipación."

"Treu-te això del cap, que em fas passar vergonya. I ella que no, que em sentiré despullada, que no. Mira que aquí les coses són diferents i a mi em coneix molta gent i tinc una empresa i no hi ha cap nexessitat de portar aquests draps." (p. 183)
-- der stellt sich sehr befreit und so dar; und spaeter findet er gut, dass seine tochter nen hijab traegt..

"Me'l posava per resar, primer. Després per estar per casa. FIns que vaig sentir que era imprescindible, que no podria viure mai més passant amb el cap descobert davant de ningú. Me'l vaig posar per anar a comprar i vaig sentir les mirades estranyades de les botigueres que em coneixien [...] Vaig sortir així un parell de vegades i un dia el pare em va veure. [...] Au, no surtis més amb aquest drap al cap." (p.228)
-- warum gucken sie die Verkaeuferinnen komisch an? ist sie etwa nicht dieselbe person? was sehen sie im hijab? ein symbol der anderen? ein symbol der opression?

"La mare em va fer anar a ca la Soumisha a buscar alguna cosa i jo vaig posar-me el mocador, [...] Sembles un àngel, m'havia dit ella, segur que entraràs al cel directament, per la porta gran. [...] Els nostres ulls es van trobar i allà mateix ho vaig saber, que no hauria hagut de posar-me el mocador. [..] no sé com no vaig caure. Ell no deia res, però jo ja el sentia derrere meu i quan va dir para, para o encara serà pitjor, jo ja no sé si vaig córrer o em vaig aturar, però em recordo a terra, amorrada a la clavaguera i ell vinga donar-me puntades de peu. No recordo els cops, no recordo si em va picar a la cara, a l'estòmac. [...] I llavors vaig mirar tot al meu voltant i vaig veure els clients del bar de davant de casa amb la beguda a la mà que no deien res i els que passaven pel costat que no deien res i els que ens coneixien i no deien res i allò era estar sola." (p.228-229)
--estar sola
-- la indiferència de la gent davant "els_les altres"

----------

La població estrangera a Catalunya (APUNTS de prospectiva territorial, numero 2)

"El marroquins, segon grup de la província, presenta una migració més masculina en els grups de més edat (majors de 35 anys), però equilibrada en les edat inferiors. Destaca especialment l'important volum de població infantil: el 14,6\% dels marroquins tenen menys de 5 anys."

------------------------

[Candel1965]

Els altres catalans

"On hi ha feina, hi són ells" (p.17)
"A la terra dels seus pares no hi ha feina" (p.17)
"Se senten conquistats per Catalunya; no del tot, és clar [...] si els diuen murcians o gallecs s'enfaden; si els diuen catalans, no" (p.17)
"Davant la persistent i discutida qüestió: "...heu vingut a menjar-vos el pa dels catalans..." es posen furiosos, naturalmen! i aleshores, només aleshores, malparlen de Catalunya. A ells no els el regala ningú, el pa; ells se'l suen;" (p.17)

"I de vegades ells mateixos, amb la mateixa espasa flamígera esmentada de "heu vingut a menjar-vos el pa.." etcètera, ataquen els darrers immigrants que denigren Catalunya. Punt." (p.17)

""Aquests nou[sic!] catalans parles català, el van aprendre sense adonar-se'n. Molts el parles de manera natural i quotidiana, perquè sí, i altres per afany de sentir-se catalans de debò"" (p.17)

""Tota aquesta gent no s'adona de la seva aclimatació. "Són" catalans fins a cert punt. "No" són catalans, també fins a cert punt. No és una qüestió d'honor ni de principis."" (p.19)

"Aquells immigrants murcians no havien arribat a Catalunya com a colonitzadors. Tampoc, o molt febement, com a invasors o peoners. Havien vingut a treballar i a menjar, senzillament, perquè a la seva terra es morien de gana [...] Avui dia, tots s'han integrat, i alguns, ultrapassant o sobrepujant aquesta integració, s'han tornat furibunds catalanistes. [...] Aquesta esperiència pot demostrar o permetre d'esperar que amb els immigrants d'ara passarà el mateixo poc més o menys." (p.32)

"el que passa és que no acabem de decidir-nos a anomenar catalans els qui han nascut aquí de pares de fora" (p.34)

\end{comment}

\subsection{Interseccionalitat}
En l'any 1989 la professora de dret i advogada de drets civils Kimberle Crenshaw fa servir el terme ``interseccionalitat'' per primera vedaga~\autocite{Crenshaw1989}.
Amb aquesta noció l'acadèmica vol descriure l'opressió multiple que sofreixen persones que pertanyen a la vegada a diversos grups marginalitzats.
Ella critica que fins aquest moment els discursos acadèmics, legals i activistes analitzen la discriminació al llarg d'un únic eix,
tractant l'opressions de gènere i racistes com dos fenòmens mútuament excloents,
concentrant-se en els membres privilegiats de cada grup i d'aquesta manera amaguen la multidimensionalitat d'experiències de les persones més vulnerables, com per exemple les dones negres.
Segons Crenshaw, la interseccionalitat denomina un tipus de discriminació que consta de més que la suma de les seves parts.
Això vol dir que les dones negres per exemple experimenten una discriminació molt particular que és diferent de la combinació d'experiències de les persones negres i les dones en general (que no significa que les dones negres no siguin discriminades com dones o com negres, però que també experimenten una opressió particular com a ``dones negres'').

Des de llavors, aquest concepte s'ha fet servir en contextos diferents per analitzar situacions i col·lectius que pateixen d'una opressió multiple.

\subsection{Cultura, llengua, traducció i identitat}
Com adverteix el fundador de les ciències postcolonials Edward Said en el seu llibre ``Culture and Imperialism'', avui dia cap persona pot ser definida amb una sola etiqueta
i identitats (atribuïdes o seleccionades per a una mateixa) com ``dona'', o ``musulmà'', o ``americà'' serveixen només per a una primera orientació~\autocite{Vidal2012}.
A més, aquestes categories són subjectives i existeixen sempre dins un narratiu particular que les ha creat, afirma la professora de ciències de traducció Mona Baker, citada en~\autocite{Vidal2012}.
Said planteja també la idea que tothom construeix la seva pròpia identitat cultural i ètnica.
Ja no tenim cultures i identitats ``pures'' i ``estables''~\autocite{Vidal2012}.
Però encara, amb la mateixa força, l'altre, el diferent, l'estrany està percebut com una amenaça per l'ordre establert i per això es rep amb por i prejuicis.

Un aspecte fundamental/fundador de la nostra identitat i de la nostra cultura (llevat de que entenem sota aquests dos termes) és/són la/les llengüa/es a través de les quals ens comuniquem.
A més a més, a través del llenguatge les cultures diferents es poden entendre, que segons Vidal Claramonte és crucial, ja que avui dia més que mai gran quantitats de persones no viuen en la seva llengua materna.~\autocite{Vidal2012}.
O, més específicament, viuen en més que una llengua i això produeix constantement conflictes i trastorns i fa necessari la reorganització de les esquemesque organitzan la seva vida~\autocite{Albin2005}.

Deleuze i Guattari, citats per Vidal Claramonte, posen una pregunta interessant:
``How many people today live in a language that is not their own? Or no longer, or not yet, even know their own and now poorly the major language that they are forced to serve?''~\autocite{Vidal2012}
i elaboren que aquest problema sigui molt propi dels i de les immigrants i els/les seves filles.

En el seu treball sobre l'obra de Najat El Hachmi María Carmen África Vidal Claramonte destaca la traducció com l'element bàsic de ``L'últim patriarca''.
Explica que la traducció té lloc a nivels multiples: podem llegir no només una traducció entre dues llengües, sinó també entre dues cultures i fins i tot entre dues generacions~\autocite{Vidal2012}.
Tenim per això exemples multiples:
la filla qui tradueix els preus dels aliments a la seva mare, no només canviant la llengüa, sinó també convertint les pesetes en duros;
la filla qui ajuda a la seva mare embarassada de parlar amb la ginecòloga:
\begin{quote}
``Hi havia coses que no sabia passar d'un idioma a l'altre, que no volia passar d'un idioma a l'altre. Contiuanva sense entendre per què tantes dones per tot el món m'explicaven coses d'aquelles a mi. Quina va ser l'última vegada que li va venir la regla a la mare? I jo ja sabia què era allò de la regla, però no n'havia parlat mai amb ella. La primera vegada que li va venir? [\ldots] La primera vegada que va tenir relacions sexuals? Déu, Déu, volia fugir corrents de tot allò, jo no les vull saber, totes aquestes coses, i encara menys traduir-les a un idioma on no hi havia cap paraula que jo conegués per a relacions sexuals que no fossin paraulotes [\ldots] au, va, pregunta-li-ho. La mare em mirava i deia què, què t'ha preguntat, i jo hauria volgut desaparèixer així, de cop, i que elles mateixes s'entenguessin. No podia dir follar, no [\ldots] Vaig intentar de buscar un eufemisme. Quants anys tenies quan vas dormir amb el pare per primera vegada? I no la vaig mirar als ulls mentre li ho deia; ella va dir, també ben de pressa, ens vam casar que jo tenia divuit anys. Això és tot.''~\autocite[217]{ElHachmi2008}
\end{quote}
En aquest fragment la traducció multiple és especialment interessant.
Tenim les dificultats de la filla de trobar paraules adequades per relacions sexuals en una llengua, en la qual encara existeix el tabu de parlar de manera lliure sobre aquestes coses,
i en la qual ningú mai li ha parlat d'això d'una manera que no sigui vulgar.
Tenim també la seva assumpció ``natural'' qua el pare sigui el primer home amb qui la seva mare n'ha tingut--una traducció de la pregunta de la metgessa.
I tenim encara la resposta de la mare, que és una nova traducció: ``ens vam casar que jo tenia divuit anys''~\autocite[217]{ElHachmi2008}--traducció generacional, perquè ``casar-se'' substitueix ``dormir amb el pare'' i traducció llingüística, ja que la filla passa la frase a català~\autocite{Vidal2012}.

Un objecte important per a la filla, al qual es fa referència al llarg del llibre, és el diccionari de la llengua catalana.
El diccionari i, en conseqüència, la llengua, són un refugi per a la noia de les contradiccions de la vida quotidiana.
Les consultacions amb el/les lectures del diccionari serveixen també com un fil que ens ajuda seguir la història.

Hem de recordar el valor simbólic dels diccionaris d'una llengua.
El diccionari és el primer llibre d'una nació, cita Verónica Albin en una entrevista amb l'intelectual Ilan Stavans~\autocite{Albin2005}.
També, és important notar que el mític objecte ``el diccionari'' no existeix.
Cada llengua disposa de diccionaris multiples, cada un d'aquests té lxs sevxs redactorxs, persones amb ideologies i circunstàncies concretes que naturalment influeixen les seves obres. [quote!]
Com ho formula Henri Meschonnic, citat per Albin, els diccionaris reflecteixen ``les conflits, les masquages des
conflits, les clichés qui font l'album de famille d'une culture''~\autocite{Albin2005}.
Stavans es posa d'acord amb aquesta noció i subratlla que encara que siguin xenòfobs, els diccionaris serveixen per definir l'univers d'una persona,
perquè ``The limits of our language are the limits of our worldview.''~\autocite{Albin2005}.

\begin{comment}
  Quotes
  ------
"L'àvia havia pres sang d'eriçó, s'havia banyat amb aigua on havia diluït l'esperma del seu home; s'havia fet fumejar l'entrecuix amb la barreja que cremava al foc, feta de sofre, gallarets esmicolats i excrements de colom secs.
  Tots els remeis que les àvies de l'època li havien recomanat." (p.12)
-- exotisme? aber so bisschen ins laecherliche gezogen; vlt sich lustig machen ueber rueckstaendigkeit? vlt ist es auch gar nicht so gemeint, kann man aber so lesen; però per altra banda no parla "d'un altre lloc", sino de "l'època".. doncs la història sóna molt propera...

"Sobretot quan el pare va tornar a parlar i a nosaltres ens deia allò de digueu a la puta de la vostra mare que... digueu a la porca de la vostra mare que... digueu a aquella gossa que... Nosaltres només li deiem mama ell diu que... Allà vam començar a fer de traductors." (p. 177)
-- el concepte del traductor apareix multiples vegades també; només d'una llengua a l'altra, sinò entre cultures.

"Per escapar del <texit>poltergeist</>, si no tens na senyora cridanera i baixeta com Tangina Barronsm has de riure molt, fins a sentir que tens les costelles a punt de petar, o has de plorar molt, fins a sentir que t'has buidat, o has de tenir un orgasme, que de fet i fet, també és buidar-se. Jo encara no en sabia, de tenir orgasmes, al pare no li agradava que ningú plorés i a la mare no li agradava que ningú rigués. De manera que vaig començar a llegir, paraula per paraula, aquell diccionari de la llengua catalana." (p.181)
--això del poltergeist es repeteix també
--el diccionari te un paper central; roter Faden; paper de la llengua??

"No devien entendre gaire res i les filles del veí de davant deien a la mare, au, fes-lo fora de casa, si vols, nosaltres t'ajudarem. Ella no les entenia, somreia i deia sí, sí, però elles ja devien veure que no en traurien l'aigua clara. Llavors es psssaven els dits entre els cabells de permament i em feien servir de traductora. Em feien sentir frases que jo no volia sentir i em feien dir coses que jo no volia dir. Què dieun?, deia la mare, què diu?, deien elles. Jo hauria cridat res, res, res, calleu totes si no us enteneu." (p.196)

"Llàgrimes sense soroll, i jo vaig acabar a traduir: vés-te'n  no tornis mai més. I vaig interpretar una mica el paper." (p.223-224)
-- traductora

\end{comment}

\begin{comment}
[Albin2005]
On Dictionaries

"VA: In the early 19 th century, Constantin François de Chasseboeuf, Comte de Volney, said that the
first book of a nation is a dictionary of its language,"

prescriptivism vs descriptivism

"VA: Regarding "love," you mentioned in Dictionary Days that Acadians, Caldeans, Phoenicians,
Sumerians, Babylonians, Egyptians, Normans, Toltecs, Vikings, and Quechuas didn't have a word for
it. Knowing that images are an important part of how you see the world, what would you have done
had you been born speaking Latin, that according to linguists doesn't have a lexeme for gray or
brown, or born to that of the Dani of New Guinea, whose only color words are for black and white, or
speaking a 4-color language like Hanunóo that has words only for black, white, green and red?

"VA: Living in two or more cultures, two or more languages, produces some rifts and upheavals; it
requires a constant rearranging of schemata."

"IS: Furthermore, translation always involves wonderment and surprise: what is the speaker really saying?
Is there a way to convey the message in my own language? Is it possible to avoid becoming a
falsifier? The answer to the last question, obviously, is no. Every translation is a misrepresentation."

-----------------
[Vidal2012]

"Fins que la mare es va cansar de tot allò i va dir, aquest any, les notes les vaig a
buscar jo, fins i tot les teves... Jo feia de traductora, com sempre. La mare deia
digues-li que és una mala puta i que deixi estar el meu marit d’una vegada, i jo
somreia i deia la mare diu que com que és ella qui passa tant de temps amb els
fills, que és millor que sigui qui et vingui a buscar les notes i, a més, que ja tenia
moltes ganes de conèixer-te. Doncs jo preferiria parlar amb el teu pare directa-
ment, que és una mica estrany que tu tradueixis l’informe a la teva mare, no et
sembla? Ja t’agradaria, ja, que hagués vingut ell, deia la mare sense haver esperat
la meva traducció, malparida, no et molestes ni a dissimular-ho. Diu que el pare té
molta feina i no li anava bé de venir, però que ella ja es refia de mi. Notable,
excel·lent, notable, excel·lent, mostra interès, tot allò no tenia traducció i jo deia
res, que diu que tot ha anat bé. Només un bé de gimnàstica i li aniria bé de fer
alguna activitat fora de l’escola, sobretot anglès, que aquí no en fem i ella té faci-
litat. La mare va dir val, val, i volia dir que ni pensar-ho només perquè era ella qui
ho havia proposat. (El Hachmi 2008: 263)"
"De nuevo, la vida cotidiana como traducción. La madre es aquí doblemente
subalterna: con respecto al esposo y fuera del espacio privado, en el público, por
ser diferente."
"Gayatri Spivak publi-
có en 1988 en el que denuncia que la subalterna, no es que no pueda hablar, sino
que no se la escucha, que su discurso no está validado por las instituciones."

quote Spivak in Segarra
"cualquier intento de ayudar a los subalternos tropieza con problemas éticos impo-
sibles de soslayar: la tendencia a considerarlos como una masa homogénea, en
lugar de fijarse en su singularidad heterogénea y, en especial, la intención benevo-
lente de querer hablar por ellos, lo cual significa un acto de apropiación, y no con
ellos. (Segarra 2006)"
Segarra, Marta (2006). «Más allá del poscolonialismo. Contra la subalternidad», La Van-
guardia, 1-3-2006.

«All speaking, even seemingly the most
immediate, entails a distanced decipherment by another, which is, at best,
an interception.» (Spivak 1999: 309).

"El lenguaje se utiliza en este
ejemplo como un instrumento de poder que ya no se basa en un ideal armónico
de comunidad en el que prevalece el principio de cooperación."


\end{comment}

\subsection{Patriarcat}
L'última noció teòrica que definirem abans de començar l'anàlisi de l'obra de Najat El Hachmi és el patriarcat.
Com senyala la periodista i activista Laurie Penny, aquest terme és refereix al comandament/regiment d'una elit d'homes sobre la resta de la societat.
Els homes que no disposen de poder polític, tenen al menys la satisfacció de manar els membres de la seva família com a compensació per la falta de control sobre la resta de les seves vides, destaca l'autora~\autocite[69-70]{Penny2014}.

\begin{comment}
Quotes
------
"Aquesta és la història de Mimoun, fill de Driouch, fill d'Allal, fill de Mohamed, fill de Mohand, fill de Bouziane, i que nosaltres anomenarem, simplement, Mimoun." (p.7)
-- die Wichtigkeit, das Mystische, sein Stammbaum, seine Geschichte wird ihm entzogen?

"És la seva història i la història de l'últim dels grans patriarques que formen la llarga cadena dels avantpassants de Driouch. Cadascun d'ells havia viscut, actuat i influït en la vida de tots els que els envoltaven amb la fermessa de les gran figures bíbliques." (p.7)

"Aquesta és l'única veritat que us volem explicar, la d'un pare que ha d'afrontar la frustració de no veure acomplert el seu destí, la d'una filla que, sense haver-s'ho prposat, va canviar la història dels Driouch per sempre." (p.7)
-- el destí té un paper central; es menciona bastant sovint

"El pare deia mira, el teu germà és molt menys ploraner que tu [...] I què faràs quan t'hi barallis, qui serà el vencedor, tu o ell, que és més petit? Si vols que t'acabi respectant i et digui Azizi\footnote{Apel.latiu que els germans petits donen als més grans com a senyal de respecte}, ja pots imposar-t'hi." (p.21)
grandioese Erziehungsmethoden, die fuer geschwisterliche Liebe sorgen und bescheuerte Geschlechterstereotypen auf keinen Fall fortsetzen.
iwann bringt er mit 3 seinen kleinen Bruder um. So viel zum Erfolg davon

altra violació:
"Però el Manel tenia aquell mena d'instint de caçador que han de tenir per força els que estan destinats  a ser gran patriarques i no entenia que era no." (.91)
-- no means no
\end{comment}

\begin{comment}
[Penny2014]

"How are men supposed to cope with this loss of power in a society that still insists that the only way to be a man is to grab as much power as possible, to be rich, to be capable of extreme violence, to dominate other men physically and to dominate women sexually and emotionally? The received wisdom is that they're not supposed to cope. Without power over others, particularly over women, men are supposed to crumble, to lash out, to collapse in an extravagant welter of identity implosion that leaves a suspicious mess on the carpet." (p.64)

"'Patriarchy' does not mean 'the rule of men'. It means 'the rule of fathers' - literally, the rule of powerful heads of household over everybody else in society. Men further down the social chain were expected to be content with having power over women in order to make up for their lack of control over the rest of their lives.
[...]
Most individual men do not rule very much, and they never have. Most individual men don't have a lot of power, and now the small amount of social and sexual superiority they held over women is being questioned." (p.69-70)

"There are two big secrets about 'traditional masculine power' that mainstream culture does not want us to discuss, and it is imperative that we discuss them honestly [...]
The first big secret is this: most men have never really been powerful. Throughout human history, the vast majority of men have had almost no structural power, except over women and children. In fact, the power over women and children - technical and physical dominance within the sphere of one's own home - has been the sop offered to men who had almost no power outside of it." (p.75)

"Thus, a poor man working a job he hated could once expect to feel, at the very least, superior to his wife and children, to be master of his home even if he was treated like a slave outside it." (p.76)
\end{comment}

\subsection{On? Resum del llibre}
``L'últim patriarca'' és la història d'una família marroqiuna entre Marroc i Catalunya.
És la història d'un pare i marit extremadament abusiu, qui mai de la vida ha posat els interessos i el benestar de cap altra persona davant dels seus.
És la història de la seva esposa, una dona excepcional segons les normes marroquines, qui pateix en silenci les escapades del seu marit i mai troba el seu lloc a la societat catalana.
I, sobre tot, ``L'últim patriarca'' és la història de la seva filla, qui, sobre tot, només vol poder ser ella mateixa.
El llibre està escrit en un llenguatge molt particular.
Tenim tota la història explicada des de la perspectiva de la filla i encara una instància narrativa auctorial.
L'estil directe al llarg del llibre no està marcat de manera gràfica en ningún lloc; el discurs s'assembla més a un corrent de consciència.
És curiós també, que el nom de la filla no es menciona ni una sola vegada, que crea una atmosfera de universalitat.

* referències a la Mercè Rodoreda: vinculació amb la cultura/tradició literària catalana --> soll das hier hin?
"La mare a vegades semblava la Colometa en comptes de la Mila, de tant com havia netejat els excrements secs de damunt els taulons de fusta sota les deules d'uralita." (p.199)
--refèrencia Mercè Rodoreda
--també la primera i una de les poques vegades on surt el nom de la mare

"Jo no era Mercè Rodoreda, però havia d'acabar amn l'ordre que ja feia temps que em perseguia." (p.331)
"Va dir o la deixes, o et deixo. Jo no em vaig creure el que sentia, però era la meva mare que parlava, era la Mila que s'havia afartat de netejar capelles i relíquies, la Colometa que fugia de tot per trobar-se." (p.222)
"Una estona i jo vaig admirar la mare per ser més que una Mila, més que una Colometa, per ser de debò." (p.223)

* Sandra Cisneros: "The house on Mango Street"
cap 15 (segona part): "Una casa en un passatge, no pas a Mango Street"
-- referència a Sandra Cisneros! (look up)
-- vinculació a altres obres romàniques
"Tot i que mudar és canviar o transformar, el que nosaltres vam fer va ser mudar-nos, canviar de casa sense transformar-nos gaire." (p.230)

"La nostra casa a Mango Street però sense Lucy ni <textit>chicanos</>."

"De vegades passa allò que la mort et fa pensar en la vida i aquell primer estiu al nostre particular Mango Street, el pare va tenir un instant d'aquells de lucidesa."(p.235)
* al Petit Princep (els lligams)
Referències al "Petit Princep":
"... però Mimoun s'havia fet gran, havia començat a anar a escola i, el més important, havia començat a exercitar-se en el difícil art de domesticar les persones que l'envoltaven, de crear lligams, que deia la guineu." (p.24)
--> einer der roten Faeden der Geschichte

\begin{comment}
Weitere interessante Anmerkungen
--------------------------------

* es wird nie klar wie die Tochter heißt; Universalität? Kann sich jede* damit identifizieren? (insgesamt werden ganz selten Namen erwaehnt. Aus irgendeinem Grund werden die Liebhaber*innen Mimouns meist mit Namen erwaehnt; der Name der Mutter kommt 2 mal vor)
* interessante Erzählweise: eine Figur erzählt in 1. Person, trotzdem auktoriale Erzählung
* referències a la Mercè Rodoreda: vinculació amb la cultura/tradició literària catalana
* keine graphisch markierte direkte Rede: alles ist ein Textfluss; stream of consciousness
* estereotips: die Mutter kann und tut nur putzen und kochen und ihren Mann gehorchen obwohl er sich ungeheuerlich verhaelt; der Vater ist richtig ueberzogen als Arsch dargestellt, der seine Familie schlaegt, rumvoegelt und absurde Vorstellungen von Ehre, Ehe und Familie hat;
* Identitaetssuche (vgl. versuchen den eigenen Platz zu finden; die Freund*innen angucken; in die Religion; )
* eine Möglichkeit: biographisches Lesen

\end{comment}

\begin{comment}
  Quotes (unklar wohin damit, kommen mir aber wichtig vor)
  --------------
La mare
"Ja començava a pensar que aquell tampoc no havia de ser el seu destí quan la va conèixer a ella." (p.87)
-- el destí ist ein der anderen roten Faeden; spaeter wirds auch fuer die Tochter erwaehnt

"La mare era massa tossuda per ser la dona de Mimoun. Ell neceissitava una dona que es deixés domesticar per a tot i ella en les coses que i eren importants [versteht sich putzen und kochen], no en sabia, de cedir." (p.108)
-- am Ende wird sie auch gebrochen..

"Mimoun pensava que aquella separació posaria a prova els lligams que havia creat amb la seva dona i es veuria si ja l'havia domesticada prou." (p.124)

"Jo deia a la mare au, va, anem a mercat, que ell no hi és, o anem a aquella botiga de teles i te les tries tu mateixa per fer-te els vestits, anem a fer el volt o a veure alguna amiga teva. Ella que no, que no, que ell no hi és però ho sap gairebé tot. Allà vaig començar a entendre fins a quin punt estava domesticada i que potser aquell lligam ha era per a tota la vida." (p.237)

"... I to demostren, les nenes, et demostren que t'estimen facis el que facis i el seu amor és sempre incondicional.
  Jo ja vaig néixer amb aquest deure afectiu, amb una mare esquerpa domesticada des del principi del seu casament i un pare que no veuria gaire sovint; amb aquesta herència havia d'acomplir el meu deure afectiu." (p.147)
-- gender stereotypes

el moral doble és increïble:
"Si ell es divorciés, la seva dona li hauria de continuar sent fidel fins a la mort; per alguna cosa havia estat ell el primer de tenir-la." (p.151)
--> quina intenció té aquest relat? solidifica estereotips..

sehr interessanter Ehren-Begriff vgl "Feminism is the radical notion that women are people." also bei ihm trifft das definitiv nicht zu.
(Isabel):
"Se li oferia sempre que ell ho necessitava i allò era còmode, però ella ja havia estat d'altres homes abans i Mimoun no havia de preservar cap seu honor perquè considerava que havia nascut sense." (p.156)

"Ella no li havia estat mai fidel, ja havia estat amb altres homes abans que amb ell, perquè li havia de ser fidel ell, doncs?" (p.156)
-- wieder sehr selektive Interpretation von Sachen

"No sabem si a aquelles altures Isabel sabia res de l'existència de la mare, dels seus fills ni de mi mateixa. El que sí que sabem és que la seva existència va ser prou coneguda per tots nosaltres.
  Van passar anys en què les notícies sobre Mimoun eren només que estava amb Isabel i que les cristianes, ja se sap, quan s'enganxen a un home ja no el deixen mai." (p.158)
  -- ja genau, die Frauen sind an alles schuld; und die *anderen*

"Que aquell seu fill no havia donat el divorci a la seva esposa, que ella quedava lligada a ell per sempre sense poder canviar cap circumstància del seu destí. Abandonada, però lligada a ell, allò anava en contra de totes les lleis, tant dels àrabs com les nostres." (p.160)
-- destí: eine der Sachen, die wiederholt werden
-- dels àrabs, com les nostres -- la filla no es compta a ella mateixa com part dels àrabs??
-- o qui és nosaltres aquí? contrast entre la cultura àrab i la cultura amazigh?

"L'avi va dir vesteix-te, dona, que véns amb mi, i ella que no, que no, que Mimoun em mataria si ho sabés, que no. Mimoun ja t'està matant, li havia respost l'avi i a mi em va quedar la frase per sempre." (p.164)

"Jo no ho recordo gaire bé, però es veu que Déu em va il.luminar i vaig fer servir la meva veueta de nena per arreglar els problemes de tota aquella família. O potser tenia el moment de lucidesa més gran que he tingut mai a la vida. Sé que la frase que li vaig dir va ser aquesta, perquè se'n va parlar molt, d'allò. Vaig passar a formar part del recull de llegendes dels Driouch. Per què no deixes d'una vegada aquesta meuca cristiana i fas el favor d'encarregar-te de nosaltres? No et sembla que ja és l'hora que pensis en la teva família?" (p.164-165)

"Em van treure l'aparell de les mans per impertinent i es van escandalitzar, però aviat van estar tots molt orgullosos de mi. Molt." (p.165)

"Jo hi volia anar, a veure Isabel [...] i així podria saber quina cara feia una dona com aquella. Lletja, segur. Havia de ser lletja i pudent, com havia dit la mare tantes vegades que eren les dones que mengen porc." (p.185)
--estereotips

"Sí que era lletja, feia cara de dolenta de pel.lícula, de les dolentes que no sedueixen, que només maquinen [...] Bruixa!, vaig pensar. Que la mare és més guapa, que ho sàpigues, que és més bonica." (p.186)
-- keine solidaritaet zwischen den "opfern", die werden gegeneinander ausgespielt; die wichtigste characteristik der frauen wird ihr aussehen

"Quan la mare sentia la clau dins el pany, deia, marxo que ja ve i feia veure que no hi havia parlat mai, amb la Soumisha. El pare no havia dit mai que no hi pogués parlar, però valia més no fer-li saber massa coses." (p.213)
--erneut die Solidaritaet zwischen frauen wird teilweise vorgebeugt

"La mare deia surts massa i començava aquells discursos de jo a la teva edat ja feia... Jo a la seva edat no hauria sabut què fer perquè ella només netejava i netejava i no sabia com anar al metge sense el pare, com havia de comprar sense el pare, com havia de viure sense el pare." (p.188)
-- interseccionalitat? idees de les diferents societats com ha de ser una nena?
-- conflicte amb la mare (conflicte generacional)

"Jo em sentia heroïna, havia de salvar la meva familia. La mare sempre deia que jo era més responsable que els meus germans, més treballadora, més estudiosa, més de tot, però em penso que l'única cosa que jo era més que ells era nena.
  M'hauria anat bé la capa de <textit>supermana</> en anar bar per bar quan encara era tan d'hora" (p.191)
--wieder die stereotype, wie maedchen sind..
-- "wir sind halt anders gut"

  "La mare, estesa al llit, em va dir no podies haver donat roba neta als teus germans, que la porten tota tacada de tomàquet d'ahir? Jo m'havia canviat, rentat la cara com cada matí..." (p.220)
  -- schon wieder die Frau, die allen hinterher rennen muss...

"La mare tornarà i necessitarà que la cuidis, només et té a tu i ja ets prou grandeta per fer algunes coses. Jo volia ser prou grandeta per a altres coses, no volia passar-me els dies netejant perquè els altres embrutessin, encara que potser no era aquella la manera en què ho pensava perquè només devia tenir deu o onze anys." (p.221)
-- interessant ist auch, dass nie genau die zeit angegeben wird, obwohl zb wann der Bruder geboren wird eigentlich ziemlich genau zu bestimmen waere..
-- schon wieder sollen sich die Frauen, Altersunabhaengig, um die anderen kuemmern, immer die anderen (vol dir els homes) nach vorne stellen etc.

"Ella es deia Rosa i la mare no la sabia anomenar pel nom. De tan baixeta i rodona com era, tothom li va començar a dir bombona de butà, tot i no ser taronja. Només veure-la es podia entendre que la decisió del pare havia da ser involuntària, per força." (p.194)
-- jaja, genau, der arme hat ueberhaupt keine wahl gehabt, kein freier willen und keine moeglichkeit nein zu sagen; und die frauen sind ueberhaupt nicht auf ihr aeusseres reduziert

"No hi teníem res que no fossin les veïnes de davant, que deien denuncieu-lo, que tothom ja ho veu el que està fent i que si voleu us acompanyem a serveis socials. La mare deia no, jo mai he demanat caritat i aqeusta no serà la primera vegada i estirava els estalvis tant com podia." (p.211)
-- aquest concepte d'estar pobre és vergonyós; isolación, patir en silenci
   falscher Stolz

"Jo tenia ganes de dir a aquella senyora de cabells ben negres tenyits que la mare ja havia estat mare tres cops seguits i que no li havia passat mai res, sense test d'O'Sulivan ni gimnàstica prepart." (p.217)
--desafiament a l'ordre en Espanya: no és l'única manera possible de viure les coses.

Kapitel Estima Déu i ell t'estimarà
- Identitaetssuche
- Ohne vernuenftige Vorbilder + Erklaerung, ohne Sinn und Verstand
- Escaipismus
- zu Extremhaltungen neigend
- sich aufgehoben fuehlen
"Jo em vaig proposar de ser una bona musulmana, la millor." (p.226)

"Déu meu, fes que el pare torni al bon camí, però ho deia en la llengua de la capital de comarca perquè en la llegua dels musulmans jo no hauria sabut com dir-ho. S'hi valia: en l'última part de l'oració, on demanes alguna cosa directament a Déu, podies fer servir la llenuga que et fos més còmoda." (p.226)
-- llengua-idenitat


\end{comment}
