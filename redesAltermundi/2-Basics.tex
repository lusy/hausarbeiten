\section{Marco teórico}

En este capitulo repasaremos brevemente nociones y teorías claves que nos ayudarán a analizar el proyecto de redes comunitarias Altermundi.

\subsection{Commons/bienes comunes y redes comunitarias}

% insert Def outline

% das kommt spaeter in die Analyse, wenn ich aus der Ecke argumentieren moechte
En cierto sentido, las redes comunitarias se pueden ubicar en la lógica de los bienes comunes (en inglés ``commons'').

No estoy segura de si está arreglado de manera formal, pero probablemente podemos decir, que partes de la técnica partenecen a personas privadas mientras otras partes son más una propiedad comunal.
Aunqué la tecnología que utilizan las redes comunitarias es bastante robusta (vease Cap. X), está en el interés común de arreglar problemas que usuarixs particuladxs tienen.

\begin{comment}
Elinor Ostrom, una de las científicas más reconocidas por su trabajo en este campo, define los bienes comunes (en inglés ``commons'') como bienes ``que se producen, se heredan o se transmiten en una situación de comunidad'' (\cite[][]{Ostrom1990}).
Los bienes comunes surgen entonces como resultado de procesos comunales y su estatuto puede cambiar en cada momento, tal como cambian las relaciones en la comunidad (\cite[][]{Harvey2012}).
Los bienes comunes se han de diferenciar de los bienes privados, han de estar fuera de la lógica del mercado.

Según la teoría convencional, los y las consumindores/as de algunos bienes comunes están en concurrencia entre sí.
Si una persona usa un recurso particular, esto disminuye la posibilidad de usar este recurso para otras personas (\cite[][p.85]{Helfrich2012})
(por ejemplo si alguien se lleva dos kilos de manzanas de un árbol en el parque local, el árbol tiene dos kilo menos para todas las otras personas a que les gustaría coger manzanas).
Eso no está válido para bienes como información o conocimiento: si una persona escucha un programa de radio, esto no disminuye el programa o la experiencia para el resto de sus oyentes.

También, se suele postular que los bienes comunes son non-exclusivos (\cite[][p.86]{Helfrich2012}).
Eso quiere decir que todas las personas pueden gozar, o bien, tienen el derecho de gozar de ellos de manera libre y no restringida.
Aunque sea justo y lógico desde un punto de vista normativo/moral, este principio se puede violar fácilmente.
Usando los avances tecnológicos, las leyes o otros mecanismos sociales o económicos es bastante fácil (y se pone en práctica muy a menudo) limitar el uso de ciertos recursos o bienes para personas o grupos de personas específicos o excluirlas del uso completamente.
A veces se va aún más allá, creando la impresión de que este orden, esta exclusión, que son productos de procesos sociales, sean naturales e inalterables.
[Rieder2012]
## "community" --> redes *comunitarias*

> The term “community” has played a central role in American affective politics for a long time,
> and as Cherry Schrecker (2006) argues, this thoroughly positive connotation carries, with ex-
> tremely few exceptions, through most of Anglo-Saxon sociology.

* en alemán: más ambiguo (los nacis y la "Volksgemeinschaft")
* español? ("comunidad", "comunitari@");
\end{comment}

\subsection{Actor Network Theory}

La Teoría del Actor-Red (Actor Network Theory o ANT) se atribuye al antropólogo y filósofo francés Bruno Latour~\autocite{Latour2010}.
Es un concepto bastante abstracto y a veces difícil de captar.
En esencia se trata de un modelo de redistribución de acciones y poder.
Un actor puede ser todo o cada unx que hace algo: actores son no solamente personas, sino también organizaciones, otros seres o incluso objetos inanimados~\autocite{Venturini2010b}.
Todos los actores estan interconectados, no existen actores en aislamiento.
En suma, los actores son tales porque inter-actuan.
Lo interesante y específico es que cada actor se puede descomponer en una red suelta y cada red puede ser comprimida en un actor:
podemos hacer un tipo de liquido zooming in and out, according to nuestro foco de investigación.

La ANT está muy apta para el análisis de fenómenos sociales y controversias ya que explicita dependencias.


\subsection{Instituciones y jerarquías}

Bernhard Rieder habla en su artículo ``Institutionalizing without Institutions'' de dos distintos tipos de poder~\autocite{Rieder2012}.
Los llama ``statutory'' y ``capillary power''.
Lo que llama ``poder legal'' (``statutory power'') es el poder que viene de instituciones formales y sus regulaciones.
En contraste, el poder capilar (``capillary power'') se debe a prácticas y convenciones sociales, a normas no escritas.
Entonces, aun cuando/comoquiera que no existen instituciones en un contexto social, siempre hay prácticas sociales y dinámicas establecidas, entonces un espacio social nunca está vacio de relaciones de poder.

La científica feminista Jo Freeman ha observado en el contexto de la segunda onda del movimiento feminista, un hecho similar:
el movimiento solía(??) estar formado casi exclusivamente por grupos sin estructura formal.
Weiterhin, la falta de instituciones o jerarquías formales no significa que tales no existen, sino que operan de modo encubierto y por lo tanto estan aun más difíciles de controlar, desafíar y cambiar~\autocite{Freeman1970}.

Esta aversión hacía las estructuras formales y las jerarquías está muy típica para todos los proyectos de base de la izquierda.
Y aunque Freeman entiende la desconfianza?(mistrust), voice (esp)/enuncia una crítica profundamente válida:
"structurelessness" se convierte en meta en si, también en casos donde estructura formal será necesaria o útil para lograr las metas de los grupos in question (en cuestión?).
Para Freeman esto es el caso para todos los objetivos que van más allá del intercambio de experiencias y "consciuousness-raising".
La estructura formal y explícita es necesaria para que todxs puedan participar, también personas en posiciones marginalizadas, con pocos recursos (como tiempo, ..) y que no pertenezcan? al círculo interno? \autocite{Freeman1970}.

Su postura se puede resumir así: lo que cada grupo de activistas tiene que hacer es pensar en qué grado de estructura y cuál está necesario para el éxito de sus actividades.

\subsection{Ciberoptimismo}

El ciberoptimismo es otro concepto mencionado por Bernhard Rieder en su artículo ``Institutionalizing without Institutions''~\autocite{Rieder2012}.
Lo describe como...
El autor toma una postura crítica antes ello.
El argumento central es que la tecnología sóla no puede resolver problemas políticos y que, al final, necesitamos para ellos soluciones políticas.

% Otros autores/aportaciones al ciberoptimismo

\begin{comment}
[Rieder2012]

"The central point that I have tried to make over these pages is that the major fault of contempo-
rary cyber-optimism is perhaps not simply its technological determinism, but a tendency towards
an essentialist view of both technology and democracy that eschews the complexities and deep
contradictions that characterize both." (Schlussfolgerung)

"One of the most common claims frames the Internet as a force of democratization. Appearing
recently in conjunction with the “Web 2.0” phenomenon, it portrays network technology as an
agent of decentralization that will bring an end to cultural hierarchies"
Idealerweise: Bildungsauftrag, Empowerment, aber in der Realitaet wenige verstehen wirklich was da so geht und deshalb bleibt die Arbeit auch an einigen wenigen Menschen haengen.

"In this “cyber-optimistic” viewpoint, the Internet is the agent of a
“capillary revolution” that is set to bring decentralization, equality, and democracy."

"the Internet allows capillary configurations
of power – local initiatives, ad-hoc pressure groups, fan cultures, “issue publics” – to challenge
the statutory powers that be. "
Allerdings:
Wird argumentiert, dass alle mitmachen koennen, das stimmt aber nicth so direkt.
Koloniale Zusammenhaenge bestehen;
Geographische Schwierigkeiten;
Oekonom. Probleme (oft in Kombi mit schwierigem Terrain); --> Kommerzielle Provider haben kein Interesse
Staatliche Repression --> dem Staat passt nicht dass X oder Y kommunizieren kann und eine oeffentliche Plattform hat

"while for communitarians the digital world mir-
rors the values of egalitarian forms of direct democracy and grassroots networking.” (Norris
2001, p. 232)"

"the convergence of this mutating counterculture with laissez-faire capitalism"
--> vlt kritische Perspektive: community networks as counterculture, aber andererseits spielen sie dem Staat teilweise in die Haende (laissez-faire capitalism, vgl Interview Gui)

"In a second step, an essentialist understanding of the “network” concept as a force of “decen-
tralization”, “flexibility”, “complexity”, etc. served as the intellectual vehicle that explains why
social systems, in the networked future, will inevitably liberate themselves from systemic road-
blocks such as governments in order to finally rejoin an optimal state of flow. In this narrative,
technology will unshackle capillary self-organization from the suffocating embrace of statutory
institutions."

"The logical enemy, in this equation, is the state"
"The libertarian ethos that characterizes the Silicon Valley brand of techno-utopism indeed takes
many cues from a cultural heritage that builds around the (strong) individual and favors small-
scale community governance where the individual is not submerged in the structures of complex
bureaucratic systems. Naturally, there is a strong mistrust of the state – and any large bureaucratic
structure for that matter"


"Democracy rests on the idea that, except for technical details for which experts
may be useful, the important decisions of society are within the capability of ordinary citizens.
Not only can ordinary people make decisions about these issues, but they ought to, (Zinn 2003)"
--> vgl Altermundi: Experts: tech. Details; las redes: toman decisiones de manera autónoma

    * Kritische Betrachtung dieses Texts ist auch angesagt, mehrere der Quellen, die zitiert werden, werden gar nicht im Literaturverzeichnis aufgefuehrt

* Rieder, Bernhard "Institutionalizing without Institutions? Web 2.0 and the Conundrum of Democracy" In F. Massit-Folléa, C. Méadel, & L. Monnoyer-Smith (Eds.), Normative experience in internet politics (pp. 157-186). (Collection Sciences sociales). Paris: Transvalor-Presses des Mines, 2012
  ** ciberoptimismo - análisis crítico
  ** se necesitan soluciones políticas para problemas políticos
  ** statutory vs capillary power
  ** conotación positiva de "community"
## statutory power

"statutory": formal establishments
--> das Projekt (alle Community Network Projekte) als Versuch, sich institutionalisierter Macht zu entziehen

## capillary power
y "capillary power": social mechanisms
"And these two levels do indeed evoke different means of coercion, different tech-
niques of control, a different praxis of power."

> Even if there are no institutions (as formal establishments) regulating behavior,
> there are always institutions in the sense of mechanisms,
> rules, and established dynamics.
--> die am Ende dazu fuehren, dass Arbeit doch bei einigen wenigen haengen bleibt; dass Konflikte entstehen und gar nicht so einfach ist, diese zu loesen

% Análisis
> If technology won’t deliver us from the conundrums of
> governance, negotiation, and struggle, we
> may as well reengage politics proper[ly].
--> Altermundi es explicitamente un proyecto político?
Ich glaub, it always boils down to thinking critically about what one's doing and assuming responsibility for it.

> As Jens Jessen, has recently pointed out 21 in
> Die Zeit, it is democracy that guarantees a
> free Internet and not the other way around.
--> Siehe China, Iran, Cuba etc;
In dem Kontext verorten sich auch solche Projekte wie Altermundi

"We are creating a world that all may enter without privilege or prejudice
accorded by race, economic power, military force, or station of birth."
“We are creating a world where anyone, anywhere may express his or her beliefs, no matter how singu-
lar, without fear of being coerced into silence or conformity.” (Barlow 1996)
--> oder auch nicht --> siehe oben; community networks versuchen das aber wirklich umzusetzen

"Especially in American research and critical comment, there
seems to be a widely shared view according to which the Internet allows capillary configurations
of power – local initiatives, ad-hoc pressure groups, fan cultures, “issue publics” – to challenge
the statutory powers that be." (p.4)
--> aber um diese ueberhaupt moeglich zu sein, brauchen die Leute erstmal einen Zugang zu einer physischen Infrastruktur, der gar nicht so selbsverstaendlich gegeben ist.
--> koloniale Zusammenhaenge und Logiken werden reproduziert; aufgrund von geographischen (schwieriges Terrain), oekonomischen (Anbindung lohnt sich fuer kommerzielle Anbierter*innen nicht), politischen (der Staat oder X, dass sich Y Verhoer verschaffen kann) Gruende, haben Menschen keinen Zugriff zum Internet

"In this narrative,
technology will unshackle capillary self-organization from the suffocating embrace of statutory
institutions." (p.6)

"The term “community” has played a central role in American affective politics for a long time,
and as Cherry Schrecker (2006) argues, this thoroughly positive connotation carries, with ex-
tremely few exceptions, through most of Anglo-Saxon sociology." (p.9)
// im Gegensatz zu Deutsch (Nazis und "Volksgemeinschaft")
// ich weiss nicht wie es auf Spanisch ist ("comunidad", "comunitari@"); aber im Kontext von Community Netzwerke spiegelt sich diese positive Definition wieder

"In many of the “self-organized” sys-
tems that make up Web 2.0, we find that a small group dominates structures of visibility." (p.15)

"This mediation has to rely on at least some communality – Rawls
speaks of “reasonable” plurality – but the challenge is indeed to manage difference." (p.16)
// Also Altermundis Probleme sind nicht technischer sondern sozialer Natur --> passt zur Schlussfolgerung des Texts


"The central point that I have tried to make over these pages is that the major fault of contempo-
rary cyber-optimism is perhaps not simply its technological determinism, but a tendency towards
an essentialist view of both technology and democracy that eschews the complexities and deep
contradictions that characterize both." (s.19)


"But technology focused anti-statism
might actually be pulling the carpet from under its own feet. The hope for magical technological
solutions to the messy realities is counterproductive if it leads to an attitude that disengages tradi-
tional political process to simply “route around it” " --> wir brauchen politische Loesungen fuer politische Probleme

"This is perhaps the argument I really wanted to arrive at after these long meanderings. Instead
of asking whether the Internet is an agent of democratization, we could be asking what kind of
democracies we need to deal with the deep social and cultural transformations that currently play
out on a global scale."



\end{comment}

\begin{comment}
%ANT
[Latour2010] ANT

-- ich weiß noch nicht wie hilfreich das ist, aber paar Gedanken/Kernzitate vom Text

"whenever you wish to define an entity (an agent, an actant, an actor) you have to deploy its attributes, that is, its network"
"the technology.. /digital technology .. makes networks material and explicit" (infrastruktur ist präsent und sichtbar im öffentlichen Raum) --> wem gehört die Infrastruktur? wer kontrolliert die? Wer hat (physischen) Zugang dazu?
"the expansion of digitality has
enormously increased the material dimension of networks: the more digital, the less
virtual and the more material a given activity becomes." <-- das ist das Exaktzitat

"In its simplest but also in its deepest sense, the notion of network is of use
whenever action is to be redistributed."

Eigenschaften eines Netzwerks:
* distribution
* visibility/Explizierung von dependencies

Think about modelling the case study through ANT: kind of a liquid zooming in and out
Kern der Theorie:
"any entity can be seized
either as an actor (a corpuscle) or as a network (a wave). It is in this complete
reversibility—an actor is nothing but a network, except that a network is nothing but
actors—that resides the main originality of this theory. Here again, network is the
concept that helps you redistribute and reallocate action."

"philosophers have been carried out by the verb to be and its
problem of identity and not by the verb to have and the range of its properties and
avidities. But the web is changing all of that and fast: “to have” (friends, relations,
profiles...) is quickly becoming a stronger definition of oneself than “to be."" --> Sachen kriegen Identität durch ihre Beziehung zu einander

"subversion it introduces in the notion of distance (the adjectives “close” and “far” are
made dependant on the presence of conduits, bridges, and hubs)," <-- physical network

"it dissolves entirely the individual versus society conundrum"

Generelle Kritik: "Since the information is here why not use it?" --> und spricht gar nicht ethical concerns an

%------------------------

[Venturini2010b]

"In controversies, any actor can decompose in a loose
network and any network, not matter how heterogeneous, can coagulate to function as
an actor."

"To understand how social phenomena are built it is not
enough to observe the actors alone nor is it enough to observe social networks once they are
stabilized. What should be observed are the actors-networks—that is to say, the fleeting con-
figurations where actors are renegotiating the ties of old networks and the emergence of new
networks is redefining the identity of actors."

"an actor is anything doing something. "

"there is no such thing as an isolated actor. Actors are always composed
by and components of networks."

"Actors are such because they inter-act, shaping relations"

\end{comment}
