\section{Introducción}

\epigraph{``It is democracy that guarantees free Internet and not the other way around.''}{Jens Jessen in~\autocite{Rieder2012}}
\epigraph{``All bits are created equal.''}{Quelle?? Laurie Penny~\autocite[69]{Penny2014}}

% Introducción general

Cunado hablamos de ``redes'' en el contexto digital cada unx de nosotrxs tiene una noción diferente/se imagina una cosa diferente.
Una gran parte se imagina seguramente las herramientas de redes sociales, las aplicaciones como Facebook, Twitter, Instagram, Google+, Snapchat, etc.
Otrxs piensan en la misma gente que se comunica a través de estas herramientas y relaciones entre estas personas.
En un contexto científico más específico habría seguramente también lxs que recurren a la Actor Network Theory de Bruno Latour.
Y aún otro grupo de personas comprende ``redes'' sobre todo en un sentido físico: la infraestructura compuesta por cables, routers, antenas y computadoras, cual es la premisa para la existencia de las aplicaciones mencionadas arriba.

% ubicación científica

El tema de redes se ubica en el entrecruzamiento de las ciencias de tecnología, la sociología y la informática.

\begin{comment}
por qué es importante/interesante?
	\item Erläutern Sie kurz, in welchem Themenbereich Ihre Arbeit angesiedelt ist. Wo werden Sie einen Beitrag leisten?
	\item Das Ziel sollte es sein, den groben Kontext Ihrer Arbeit darzustellen.

Política digital <-- Gabriella Coleman heranziehen
\end{comment}


% Motivación

Para la investigación actual nos interesarán sobre todo las redes en el sentido de infraestructura física, pero también las redes sociales de personas que utilizan, construyen y mantienen esta infraestructura.
Esta imbricación entraña un potencial conflictivo inmenso, ya que lxs que controlan la infraestructura, controlan al final la comunicación y tienen un poder enorme.
Justo en este ámbito se ubica el proyecto argentino Altermundi que es el foco del presente análisis.


% Objetivo
\begin{comment}
\noindent \emph{Welche Ziele werden mit der Arbeit verfolgt? Welche zentralen Fragen lassen sich daraus ableiten?}
	Die Ziele sollten so spezifisch wie möglich sein. Das hilft Ihnen im Verlauf der Umsetzung zu prüfen, ob Sie Ihre Ziele erreichen konnten. Bitte achten Sie darauf, dass die gesetzten Ziele realistisch sind und das Sie in der Lage sind, das erfolgreiche Erreichen dieser Ziele im Bereich Evaluation zu prüfen.
  \end{comment}

% Pregunta / hipótesis concreta para el trabajo

\begin{comment}
* análisis: ciberoptimismo al nivel físico de las redes/al nivel social de la organización de los grupos
\end{comment}


\begin{comment}
La noción ``Redes'' tiene sentidos multiples:
* redes sociales, en el sentido de grupos de gente y las relaciones entre ellxs
* redes sociales, en el sentido de herramientas y aplicaciones digitales que permiten que la gente se relacione de cierta manera en el ámbito digital
* redes en el sentido de la ANT: un modelo abstracto para observar interacción y redistribución de relaciones/recursos/poder; permiten hacer zoom in and out
* redes como infraestructura física: cables, computadoras, antenas, routers <-- un requisito para las demás capas/la mayoría de los sentidos
** también redes sociales de gente (en un sentido amplio: gente organizada/estructurada de cierta manera) que decide/construye y mantiene la infraestructura física.
--> potencial conflictivo enorme: lxs que/quienes controlan la infraestructura, controlan la comunicación y tienen un poder enorme
\end{comment}

\subsection{Trabajos previos}

En los últimos años se han realizado varias investigaciones que analizan redes comunitarias desde perspectivas diferentes: técnica, sociológica, ...
Sin embargo, hasta hoy hay pocos trabajos que se dedican concretamente a la realidad latinoamericana con sus ideosíncracias y desafíos particulares.
%stimmt das??

El libro ``Wireless Networking in the Developing World'' por ejemplo incluye un case study (esp!) venezolano~\autocite[437-452]{WNDW2013}, no obstante, el trabajo es más un manual para el despliegue de redes inalámbricas con los recursos disponibles en comunidades locales y very littel budget, explicando en detalle las tecnologías necesarias, entonces bastante técnico, y no una investigación desde la sociología o las ciencias de tecnología.

Vamos a mencionar aquí aquellas investigaciones que se han estudiado y que nos han de interesar como que mencionan uno o varios aspectos que intentaremos a analizar.

Uno de los trabajos que analiza redes comunitarias desde una perspectiva social es el artículo ``Expanding the Internet Commons: The Subversive Potential of Wireless Community Networks'' de Primavera de Fillippi y Félix Tréguer~\autocite{FiTre2015}.
Lxs investigadorxs se centran en el contexto europeo y nos dan una overview sobre las diferentes motivaciones, formas de organización y regulaciones a las que los diferentes proyectos están sotmesos (esp?).
...
Concluyen su investigación con una lista de propuestas para las legislaciones nacionales que suma (summarize?) cómo ellas pueden facilitar el trabajo de los proyectos de redes comunitarias.


\subsection{Estructura de la monografía}

Este trabajo está organizado de la siguiente manera.
Primero, presentaremos el marco teórico que nos ayudará analizar el proyecto Altermundi en el contexto de las ciencias de tecnopolítica.
Seguiremos observando el concepto de redes comunitarias wifi y la historia/génesis de Altermundi.
Después, intentaremos un análisis del proyecto y su entorno a través de la Teoría del Actor-Red (Actor Network Theory o ANT) y ...
En conclusión, ...

