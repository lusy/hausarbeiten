\section{Tu risa, Poema XV und ... in Re-vision}

In diesem Kapitel möchten wir gezielt alternative Lektüren der Gedichte ``Tu risa'' und ``Poema XV'' (``Me gustas cuando callas\ldots'') anbieten.
Diese wurden unterm anderen deswegen ausgewählt, weil sie aus zwei verschiedenen Gedichtsbändern kommen, die in ihrer Verfassung fast 30 Jahre auseinander liegen.
(Poema XV gehört zu ``Veinte poemas de amor y una canción desesperada'', die 1924 veröffentlicht wurden, und Tu risa zu ``Los versos del capitán'', dessen Erstveröffentlichung im Jahr 1952 war.)
Das lässt uns vermuten, dass sich der Stil, Bilder, Anschauung des Autors gewandelt haben und wir auch recht unterschiedliche Liebesgedichte vorfinden würden.
Im Weiteren gilt, die Frauen- (und Männer-)bilder, die durch diese zwei Gedichte evoziiert werden, zu analysieren und zu diskutieren, in wie fern eine Entwicklung dieser stattgefunden hat.

\begin{comment}
  Zu berücksichtigen:
  * figurale, räumliche und zeitliche Deixis
  * bei Lyrik gibts immer eine Metareflexionsebene (also die Form?)
  * Kunst befindet sich auch immer mit anderer Kunst in Dialog
  * Bei Liebeskram: Petrarca! Shakespeare?? (davon hab ich wohl leider keine Ahnung)

  Make a claim about how the poem works/What the poet is doing (your thesis!):
  Overall effect of the poem's craft elements?
  Where does the poem take us (emotionally, intellectually, narratively, etc?) ---> vgl Richards': "help us organize our minds"

% Was ist mein Projekt/Meine These??
Ist es möglich, die Gedichte so zu lesen, als wäre das lyr. Ich weiblich?

Male bias in der Produktion und Rezeption von Inhalten (vgl auch "male gaze")

% Allgemeine Gedanken. Am Anfang oder ans Ende stellen; oder Zwischendurch als Zwischenschlussvolgerungen; (Aus dem Handout)
4. Exemplarische Re-Vision einzelner Gedichte
   Tu risa, Poema XV
   * bestimmte Frauen- (und Männer-)bilder werden von den Gedichten konstruiert --> einseitig, problematisch, Macht
     ** abgesprochene Agenz; Frauen sind stumm, ohne eigenes Wesen; Ansammlung von (erotisierten) Körperteilen
   * Heteronormativität, Monogamie (2er-Beziehungen) als Teilungsstrategie des Kapitalismus
   % hier oder in der Rezeption??
   * Duncan weist darauf hin, dass Kritiker Nerudas Liebeslyrik oftmals nur vom Standpunkt des männlichen Sprechers im Text aus betrachten, anstatt die Position der Geliebten in Betracht zu ziehen. Daraus resultiert eine einseitige und repetitive Verbreitung aus männlicher Perspektive, welche schon in den Gedichten selbst zum Vorschein kommt. Die weibliche Perspektive wurde bis auf wenige Ausnahmen systematisch ignoriert.
   * Der sexuellen Dialektik, welche sich unter der Oberfläche dieser Gedichte verbirgt, wurde bisher
   kaum Aufmerksamkeit gewidmet. Interessant: Sexismus kommt nicht so sehr in der Bildsprache
   Nerudas vor als in dem Lesen dieser Bilder ("Sexism does not reside so much in the actual images as in
   the reading of those images.") > Warum wird die passive und stille (oder still gehaltene) Frau in einer
   Machtposition gesehen, wenn doch ihre eigene Existenz von der ihres männlichen Liebhabers abhängt?
   Warum wird sie als dem Mann gleichgestellt betrachtet, wenn sie in den Texten als seine Untergebene
   porträtiert wird? Warum sprechen Kritiker von einem Dialog zwischen den Liebhabern, wenn der
   weibliche Teil nie versprachlicht wird (keine Stimme erhält)?"") %% tun die Kritiker*innen das? Wenn ja ist es spannend! Einbeziehen!
   * Duncan erklärt weiterhin, dass unter dem Vorwand der Lobpreisung des Frauseins und dessen Einzigartigkeit, diese Einschätzungen über Nerudas Liebeslyrik textuelle Widersprüche vorenthalten, welche symptomatisch für die Unterdrückung der Frau in einer patriarchalen Kultur seien. Nur dadurch, dass die Frau zu einem Liebesobjekt wird, erwacht sie zu Leben. Ohne ihren männlichen Liebhaber ist sie “vacía, sin substancia“. Diese Darstellung der Frau in den Texten kontrastiert drastisch neben der des männlichen Sprechers, wessen Existenz nicht von der physischen Präsenz oder der weiblichen Liebe abhängig ist. Ohne ihre Liebe mag er sich verloren fühlen, aber seine Identität gerät nie in eine Krise, er sagt, ohne zu zögern, “todos saben quién soy“.
   * Das "Maskuline" wird als universell dargestellt
   * In Nerudas Gedichten hat die Stimme des männlichen Sprechers Autorität, da dessen Standpunkt/Betrachtung selten oder gar nicht in Frage gestellt wird, weil seine Stimme als die einzig mögliche Perspektive dargestellt wird. Referenz zum male gaze (Kaplan, E., Ann). // Universalisierung der männlichen Perspektive
   * auch wenn es schwierig ist, das lyr. Ich und du genau zu erfassen, bzw. als männlich oder weiblich zu kennzeichnen, allein aufgrund der Sprache, haben die Lesenden einen bestimmten Erwartungshorizont (auch durch Einfluss eines biografischen Lesens):
    das lyr. Ich: männlich
    das lyr. Du: weiblich
    -> es werden gewisse Klischee-Bilder für Weiblichkeit und Männlichkeit evoziert; einseitig, der Komplexität der Realität nicht gerecht
   * problematische Frauenbilder: schwach, still, kein gleichwertiges Gegenüber
   * Mann: besitzergreifend, oberflächlich (nur an Körper interessiert)

   * selbst wenn man in die Gedichte den polit. Diskurs der Zeit reininterpretiert und das lyr. Du vlt als die Heimat liest, trotzdem problematisch: warum wird dieser Diskurs mit sexualisierten Bildern weiblichen Körpers ausgetragen?

   %evtl nach 3 Schieben
* Male gaze
  ** das ist auch was die Menschen von der Kunst und Gesellschaft im Allgemeinen kennen:
     dass Kunst aus männlicher Perspektive dargestellt und rezipiert wird; dass Frauen erotisiert werden
  ** In Nerudas Gedichten hat die Stimme des männlichen Sprechers Autorität, da dessen Standpunkt/Betrachtung selten oder gar nicht in Frage gestellt wird, weil seine Stimme als die einzig mögliche Perspektive dargestellt wird.
  ** Filmtheorie: Übereinstimmung von 3 Blicken: Kamera, (männl.) Protagonist und Zuschauer (vgl Laura Mulvey)
     -> Objektifizierung der Frau, Frau als passive Rezipientin sexueller Aufmerksamkeit
     -> Verfestigung patriarchalischen Normen

* "als nah, natürlich wahrgenommen", die Leser*innen können sich damit identifizieren
  ** Menschen können sich direkt mit den Figuren identifizieren, dadurch dass es nicht extradiagetisch ist
  ** Duncan spricht von "real", "natural/beautiful expression of male/female relationships", die Menschen können sich damit identifizieren, sie sehen die Gedichte als Reflexion der gesellschaftlichen Idealen, die sie gelernt haben anzustreben
  ** Duncan:  "a sort of manual on the workings of modern romantic love and have come  to be regarded by many, as a standard agains which real life  relationships can be judged"
  ** Duncan: ".. in fact it seems extraordinarily "real" and "natural", as if it were not a literary construct at all, but, rather a reflection of real-life experiences. Readers who turn to these poems to learn "what love is supposed to be like", "what men are like", and "what women are like", ultimately receive a skewed message told from the traditional dominant male position"

* Letztendlich wechselt die Rolle der Frau in Nerudas Liebeslyrik, wobei all diese Rollen männliche
Fantasien involvieren: Manchmal ist sie die umsorgende, geerdete Mutter zu seiner jungenhaft
verlorenen Person, andere Male ist sie die gefährliche Verführerin, welche ihn seiner Seele zu berauben
droht. Andere Male ist sie die unbewegliche/unantastbare Ikone, ein erstarrtes Bild von Schönheit oder
Sinnlichkeit, und andere Male wiederum ist sie die beschäftigte, kleine Hausfrau. Durch die
Verstummung der weiblichen Stimme kann der männliche Liebhaber die Frau zu allem werden lassen,
wohin ihn seine Vorstellungskraft in diesem Moment treibt. “If she is unique in any way, it is only
because she is his special creation“.
Auch wird Sie als Spiegelung seiner Seele gesehen und ebenso als seine Gefangene, welche sich zu
fügen hat und sich glücklich schätzen darf, unter all den anderen auserwählt worden zu sein. (Bsp.
hierfür p.434 rechts mittig bis unten, diese Denkweise ist bis heute u.a. in lateinamerikanischen
Ländern nicht ganz unüblich.)

\end{comment}

\begin{comment}
  % Tu risa
  %% images
  * risa
  * pan - aire - luz - primavera  --> womit das Lächeln verglichen wird; Synonyme/symbole fürs Leben, lebenswichtige Attribute
  * rosa - flor --> was lebendiges, schönes, delikates
  * ola de plata
  * lanza -espada --> kamp, krieg, konflikt
  * puertas de la vida --> die Toren des Paradies? (auch cielo) --> der Mann als Gottes gleich, Schöpfer, gibt der Frau erst ihre Identität (das letzte taucht in dem Gedicht vlt nicht direkt auf)
  * opposition: tierra - cielo (die Frau ist unten auf der Erde) --> Hierarchie
  * mi sangre mancha las piedras de la calle --> violencia
  * hora oscura
  * el mar en otoño -- was ist dafür charakteristisch?
    ** mar -- chile
    ** opposition: otoño -- primavera --> das Ende bzw Anfang des Lebens
  * Isotopie(? image field): cascada de espuma - mar
  * patria -- calles torcidas de la isla --> Chile/Isla negra? (seit wann hat er das Haus da?)
  * noche - día
  * noche - luna
  * el muchacho torpe

  %% comparisons/metaphors/... why are they effective? (clear/strong/unusual connection between the 2)
  * risa como espada fresca: strong, giving strength, armas

  %% addressed subject(s)
  * love

  %% larger context of the poem
  * ``Los versos del capitán'': fast 30 Jahre später; Neruda hat eine persönliche Entwicklung durchlaufen
  * vlt sinnlichere/fleischlichere/präsentere Bilder vom lyr. Du; aber ist es wirklich anderes Modell für Weiblichkeit, das hier den Lesenden angeboten wird?
  * immer noch keine aktive, autonom handelnde Frauenfiguren

  %% mode of the poem
  * lyric: associative, vivid language

  %% aus dem Handout/Ideen notizen
    Los versos del capitán
      * Duncan dazu: In dem Gedichtband Los versos del capitán müht sich der männliche Sprecher wieder in ähnlicher Weise mit ihrer Stille ab. Laut Duncan fühlt sich der männliche Sprecher außer Stande ihr Schweigen oder ihre Andersartigkeit zu durchdringen und sieht sich deshalb gezwungen, ihren Körper mit Gewalt zu durchdringen, um seine Kontrolle über sie geltend zu machen, und somit auch ihre Sexualität, welche für ihn eine Bedrohung birgt, abzuwehren.

    * Dies lässt sich besonders deutlich in der Sektion “Deseo“ in Los versos del capitán ablesen.
    * Othering
    * Glorifizierung der Gewalt gegen Frauen

    Tu risa
      Tu risa (Los versos del capitán) -> die Welt bleibt vor der Tür stehen; Das Private ist *nicht* politisch
      das schöne Leben zu hause als Ausgleich für den Scheiß, der auf der Welt passiert

      Laurie Penny's Beziehungskonzept


  % Poema 15
  %% images
  * ausente - voz de lejos - silencio - callas
  3 semantische Felder: silencio -- noche-oscuridad -- mariposa
  * un beso te cerrara la boca
  * llena del alma mía
  * mariposa de sueño
  * mariposa en arrullo
  * lámpara
  * anillo
  * la noche callada, constelada
  * estrella

  %% addressed subject(s)
  * love
  * vlt einsamkeit?

  %% larger context of the poem
  * Teil von "Veinte poemas de amor.." (also auch in Zusammenhang mit dem Rest hier zu betrachten; wohl die Passivität und so des lyr. Du ist recht charakteristisch für alle Gedichte)
  * early poetry; young Neruda

  %% mode of the poem
  * lyric: associative, vivid language
  * aber vlt auch elegy? (laments/remembers)


  % allgemein
  %% Who is the speaker
  das männliche lyr. Ich; nicht zwangsläufig grammatikalisch markiert
  aber in binärer Opposition zum weibl. lyr Du konstruiert; Klischeebilder;
  durch die Beschreibungen/zugewiesenen Position/evozierten Bilder eindeutig als männlich/weiblich bestimmbar
  unterbewusste Parallele zw. lyr. Ich und Autor gezogen
  in "Tu risa" : "este torpe muchacho que te quiere" = speaker?
  (In "El tigre"/"El condor": der tiger bzw kondor)

  %% mode of the poem:
  * Bella: lyric (associative, vivid language)
  * El tigre/El condor: dramatic lyric (associative, vivid language + tells a story)

  %% poet's diction? (colloquial/formal/elaborate/...)?
  vgl Handout Cortázar: die alltägliche Sprache der einfachen Menschen; Kontrastierend zu der vergeistigten abstrakten europ. Liebeslyrik
  sein Freund Julio Cortázar dazu: “Muy pocos conocían a
  Pablo Neruda, a ese poeta que bruscamente nos devolvía lo nuestro, nos arrancaba a la vaga teoría de
  las amadas y las musas europeas para echarnos en los brazos una mujer inmediata y tangible, para
  enseñarnos que un amor de poeta latinoamericano podía darse y escribirse hic et nunc, con las simples
  palabras del día, con los olores nuestras calles, con la simplicidad del que descubre la belleza sin el
  asentimiento de los grandes helitropos y la divina proporción.“
  http://www.neruda.uchile.cl/critica/jcortazar.html ("Neruda entre nosotros")

  %% aus dem Handout:
    Poema 15
      lyr. Du:
      * weiblich ("llena", "dolorosa")
      * stumm, Agens abgesprochen
      * abwesend
      * leblos
      * es werden vom lyr. Ich Eigenschaften reinprojiziert

      lyr. Ich:
      * träumerisch?
      * aber auch über das Du bestimmend

      * Frau in die Opferrolle gesteckt, passiv, Gefäß, welches man füllen kann (=Projektionsfläche)

      * Ihre Stille ist laut Duncan ein zweischneidiges Schwert, da es ihm zum einen erlaubt, seine Fantasie
        auf sie zu projizieren, ihr hingegen ermöglicht es sich von ihm zu distanzieren und “the other“ zu
        bleiben, trotz seinen Versuchen diese Differenz oder Barriere zwischen ihnen zu löschen. Unzählige
        Bsp. Auf S. 435 links mittig mit verschiedenen Ansätzen des lyrischen Ich, sich mit diesem “silencio“
        auseinander zu setzen.

%%%%%%%
  % Bella

  %% addressed subject(s)
  * love
  * weibliche Schönheit
\end{comment}

\begin{comment}
  % Kommentar zu "Tu risa"
* was für romantische Beziehungskonzepte werden in unserer Gesellschaft akzeptiert (Laurie Penny) (die romantische Zweierbeziehung als einziges gesellschaftlich anerkanntes Model + deren Funktion als Ausgleich für strukturell-polit. Probleme/Missstände)
  ** hoher Stellenwert der 2er heterosexuellen Beziehung
  ** für Frau: die ultimate (einzig mögliche?) persönliche Erfüllung?
  ** Duncan: "the position of the woman in the texts is an enviable one"
  ** Duncan: "Neruda's women  possess the qualities we have been taught to covet: they are beautiful,  sensual, desirable, and eminently agreeable.[..] They want what all  women have been conditioned to want in the way of self-fulfillment:  romantic love. They may have no identity, no voice, no sense of purpose,  but, in exchange, they are promised the reward of man's eternal  devotion if they agree to play their role properly. [...] The message to  female readers is clear: woman bears the responsibility of attracting,  nurturing, and keeping man's love alive. He is not obliged to love her  and, as he often reminds her, he can fare better without her than she  can without him. Without love, he is still a man, but she, without love,  is nothing."
  ** Adrienne Rich: "the word "love" in itself is in need of re-vision"
  ** Adrienne Rich: "the myth of the special woman" and its destructiveness: "We seem to be special women here, we have liked to think of ourselves as special, and we have known that men would tolerate, even romanticize us as special, as long as our words and actions didn't threaten their privilege of tolerating or rejecting us according to their ideas of what a special woman ought to be [..] how divisive and ultimately destructive is this myth of the special woman"
  ** Molly Haskell (in Duncan's text): "the idea of woman's inferiority, a lie so deeply ingrained in our social behavior that merely to recognize it is to risk unraveling the entire fabric of civilization"
  ** Laurie Penny: "Mit dem Posen der romantischen Liebe, besonders der heterosexuellen romantischen Liebe in der Ehe, meinen wir uns der bitteren Realität von Arbeit und Tod verweigern zu können - dabei präparieren sie uns genau dafür, verkuppeln uns und stecken uns in lauter kleine Schubladen aus Leid und Leidenschaft: Du und ich gegen den Rest der Welt, Baby. Wir verlieben uns, weil das leichter ist als wenn wir lernen müssten, uns freizuschwimmen."
  ** Laurie Penny: "Man kann Menschen in antagonistischen Paaren zusammenspannen und vor ihrer Umwelt isolieren, damit sie sich gegenseitig die strukturelle Herzlosigkeit dieser Welt in die Schuhe schieben." (S.226)
  ** Laurie Penny: "Man kann die Suche nach einer einfachen Bindung zu einem elenden, ermüdenden RItual machen, das eine strenge Geschlechtskonformität voraussetzt und den menschlichen Geist unterdrückt."
  ** Laurie Penny: "Frauen aller Gesellschaftsschichten wird beigebracht, dass sie sich zuallererst um die Liebe von Männern bemühen müssen, um darüber ihren Wert zu taxieren, Männer auf sich aufmerksam zu machen. Und in allen Gesellschaftsschichten wird die romantische Demütigung dafür genutzt, Frauen klein zu machen. Jeder heterosexuelle Mann, mit dem ich mich je über Partnersuche unterhalten habe, dass Frauen in romantischen Dingen alle Macht haben, auch die ultimative Macht, auf die sexuelle Avancen eines Mannes einzugehen oder sie zurückzuweisen und einen Mann in den Kreis ihrer "Freunde" einzureihen, was dieser als völlig schwachsinnig empfindet, denn natürlich will kein echter Mann für eine Frau nur ein Freund sein. Die Macht, nein zum Sex zu sagen, finden Männer ungeheuerlich und erachten es daher nur als gerecht, dass Frauen und Mädchen alle anderen Spielarten der Macht seit vielen trostlosen Generationen vorenthalten werden. Das hätte etwas für sich, wenn die Macht, nein zum Sex zu sagen, in der Praxis respektiert würde." (S.225)

\end{comment}
