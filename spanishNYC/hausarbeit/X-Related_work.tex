Not sure where to put it:

may be part of intro?
and then part of argumentation?

%%%%%%%%%%%%%%%%%%%%%%%%%%%%%%%%%%%%%%%%%%%%%%%%%%%%%%%%%%%

%[Mahootian05]
Spannende Sachen:

%Research question
``the questions to be answered are (1) to what extent are code choice
and codeswitching governed by social factors?''
* style/social variables govern language choice
``where this variety is used intentionally, it is meant to emphasize the speaker’s bilingual identity''

-->``codeswitching may be used intentionally to index the “bilingual identity.”''

%Datensatz:
Untersucht codeswitchings in Latina Magazine;
(Articles: Mainly in English/ mixed code)
all are translated to Spanish
``Spanish version are often shorter, abbreviated versions of
the English texts,'' --> assumed English dominance

``The primary target audience of this
publication is American Hispanic women between the ages of 15 – 40 who can function
in both English and Spanish.'' (``The publication information about Latina is from a phone interview with the advertising sales manager for the
Midwest.'')

``Writers are not directed to use or not use codeswitching''

``435 codeswitches found
in two issues of Latina: February, 1999 and May, 1999''

%Categories for the switchings
provided by:
* Two Spanish-English bilingual graduate students (who put the data into a database)
* emailed to three Spanish–English bilinguals of Hispanic descent (had to justify random sample of switches)

categories:
(1) idiomatic, which included examples such as la musica tejana as well as food names, titles of Spanish songs, movies,
and so forth,
(2) attention-getting, such as y cuales in example (1), repeated below as example (14)
(3) emotionally/culturally evocative/bonding.
(linguistic necessity: is it the same as idiomatic? e.g. there is a simpler shorter term in Spanish for some specific notion)

* consensus on idiomatic usage
* general consensus on evocative switches
* mixed responses on attention getters: ``mixed between, “Draws your
attention because it’s not English” to “I’m not sure why.”''

``Only a small percentage of
the switched utterances are words where there is no single equivalent lexical item. The
majority of the switches have common, simple English counterparts.''

%Social motivations for code switchings
* emphasize group indentity
* create unity
* psycological impact : ``hermanos has a greater psychological
impact and more emotional appeal to ethnic identity. The word hermanos captures the
same sense of solidarity''
* solidarity
* create intimate domain
* ``emotions are better expressed in Spanish, because words sound more
powerful in Spanish,''

``mujer, familia, raza, hermanos, una carta de amor''

--> ich glaube switches in die Richtung (Rahmen Englisch, Switches to Spanisch) sind einfacher zu argumentieren als anders rum^^

%on conscious choice
``only some of which can be termed socially meaningful,
conscious choices. Some switches may very well be devoid of any social motivation,
even when “chosen” consciously.''

socially motivated: ``(1) even if you are the kind of MUJER who thinks...'' (``conjures up ethnic identity, the image of a woman of Latina heritage'')
not socially motivated: ``(2) It helps to know which items are worth the sticker shock, Y CUALES not.'' (begründet den Strukturspaß)

``It has also been noted (Gumperz, 1982; Romaine, 1995) that when political ideology
changes and a group becomes more conscious of their ethnicity, attitudes toward code
mixing change.''

``Romaine95: With the rise in ethnic
consciousness, these speech styles have become symbolic of Chicano ethnicity.''

``It is a way for speakers to underscore their ethnicity,
their connection to their heritage and to others who share that heritage and the values
associated with it, within the majority culture and language.''

% printed media!:
conscious choice of codeswitchings;
seem to be accepted as official code; ``is a variety with a distinct communicative function which
has achieved “official” status.''
``Use of codeswitching in a mainstream national publication is evidence of its
acceptance and propagation.''

``I analyze code choice as it appears in a conventionalized format, printed media,
where language choice is made consciously: copy is written, proofread and approved
by a number of people before it is set to print.''

Mass media and language: ``Bell (1984) observes that the language of mass media draws from the norms of
the population with which it wishes to identify.''

``One may conclude then, that if language-mixed
texts are found in institutionalized publications such as journals and magazines, it is a
reflection of a community norm which has found acceptance.''

%%%%%%%%%%%%%%%%%%%%%%%%%%%%%%%%%%%%%%%%%

% https://blogs.commons.georgetown.edu/ket37/2012/01/02/code-switching-in-13-years-of-latina-magazine-covers/
% TODO: add reference
(wenn ich mich richtig erinnere, war der Artikel nicht so der Knaller, aber man kann ihm trotzdem erwähnen, zusammen mit dem Kommentar, dass man den nicht so gut fand)

bzw, das kann man auch dem entnehmen, dass er sonst nirgends publiziert wurde;
aber vlt gibts mittlerweile auch andere arbeiten zum thema?
