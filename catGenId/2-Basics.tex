\section{Fonaments teòrics}

\subsection{La immigració a Catalunya al llarg dels anys}

``L'últim patriarca'' és, entre altres coses, la història d'una família d'immigrants.
Per això, repassarem breument el desenvolupament de la immigració a Catalunya a l'útlim segle.
Hem de notar que Catalunya sempre ha sigut una terra d'acollida d'immigrants.
Només al segle XX hi havia multiples onades d'immigració, començant amb la primera als anys '20, passant per la segona, als anys '50-'60, la tercera a les '80 i arribant a l'actualitat [quelle!],~\autocite{TarPaGa2013}.
Totes aquestes onades tenen trets diferents -- més prominent, per exemple, l'origen de la nova població o les raons per immigrar, però totes tenen també similituds.
Una d'elles és la recepció dels i de les recentment arribats.
Com assenyala Fransesc Candel al seu llibre ``Els altres catalans'', cada onada està rebuda per una mena de desconfiança, prejuicis i franca hostilitat~\autocite[17]{Candel1965}.
La població vella (syn) té por del desconegut i de l'incert i sovint reacciona amb discursos i comportaments racistes:
``\ldots heu vingut a menjar-vos el pa dels catalans\ldots''~\autocite[17]{Candel1965},
``Aquests nou[sic!] catalans no coneixen Catalunya: la seva tradició, la seva història, el seu art, la seva cultura, la seva literatura, els seus costums, el seu folklore\ldots''~\autocite[17]{Candel1965},
``Parlen un català gruixut, groller i vulgar\ldots Desconeixen la gramàtica catalana. No saben llegir en aquest idioma. Escriure'l, encara menys.''~\autocite[18]{Candel1965}

La immigració de la família Driouch descrita a la novel·la té com a fons un fet/periode històric:
les diverses onades migratòries del nord d'Àfrica cap Espanya que es produexien a partir de la dècada de les '80~\autocite{TarPaGa2013}.
Segons aquest mateix informe, a l'any 2013 els marroquins són amb 20.4\% el més gran grup d'immigrants a Catalunya.
Els autors confirmen:
``El col·lectiu procedent del Marroc ha estat el més important des dels inicis de la immigració estrangera recent a Catalunya, tant pel nombre com per la seva distribució en el territori.''~\autocite{TarPaGa2013}

Com s'adapta la gent?


\begin{comment}

La població estrangera a Catalunya (APUNTS de prospectiva territorial, numero 2)

--> discuteix la població estrangera i el seu desenvolupament en el periode de 2000-2013
analitza la distribució territorial

"a partir de la dècada de 1980 es produexien diverses onades migratòries del nord d'Àfrica"
--> ich glaube, mein Roman faellt hier rein; nach meinen Berechnungen sind sie Anfang der 90ern nach Catalunya gekommen

"A Catalunya [...] El primer contigent són els africans (27.5\%), la gran majoria marroquins (20.4\% del total d'immigrants a Catalunya)."

"El col·lectiu procedent del Marroc ha estat el més important des dels inicis de la immigració estrangera recent a Catalunya, tant pel nombre com per la seva distribució en el territori."

"El marroquins, segon grup de la província, presenta una migració més masculina en els grups de més edat (majors de 35 anys), però equilibrada en les edat inferiors. Destaca especialment l'important volum de població infantil: el 14,6\% dels marroquins tenen menys de 5 anys."

relevància per a mi:
trets/dades històrics serveixen com a base per la història

------------------------

[Candel1965]

Els altres catalans

"On hi ha feina, hi són ells" (p.17)
"A la terra dels seus pares no hi ha feina" (p.17)
"Se senten conquistats per Catalunya; no del tot, és clar [...] si els diuen murcians o gallecs s'enfaden; si els diuen catalans, no" (p.17)
"Davant la persistent i discutida qüestió: "...heu vingut a menjar-vos el pa dels catalans..." es posen furiosos, naturalmen! i aleshores, només aleshores, malparlen de Catalunya. A ells no els el regala ningú, el pa; ells se'l suen;" (p.17)

"I de vegades ells mateixos, amb la mateixa espasa flamígera esmentada de "heu vingut a menjar-vos el pa.." etcètera, ataquen els darrers immigrants que denigren Catalunya. Punt." (p.17)

""Aquests nou[sic!] catalans no coneixen Catalunya: la seva tradició, la seva història, el seu art, la seva cultura, la seva literatura, els seus costums, el seu folklore... Tampoc en aquest aspece no coneicen la regió d'on procedeixen. Però jo pregutno: quants catalans hi ha que coneguin tot això que he dit? Del poble, cap. De les classes privilegiades, alguns. Però tot això ja és qüestió de cultura"" (p.17)

""Aquests nou[sic!] catalans parles català, el van aprendre sense adonar-se'n. Molts el parles de manera natural i quotidiana, perquè sí, i altres per afany de sentir-se catalans de debò"" (p.17)

""Parlen un català gruixut, groller i vulgar[...] Desconeixen la gramàtica catalana. No saben llegir en aquest idioma. Escriure'l, encara menys. Però no n'hi ha per a escandaltzar-se'n. Infinitat de catalans d'origen, que s'expressen en català, que parlen en català i que viuen en català, llegiexen en castellà i escriuen les seves cartes en castellà."" (p.18)

""Tota aquesta gent no s'adona de la seva aclimatació. "Són" catalans fins a cert punt. "No" són catalans, també fins a cert punt. No és una qüestió d'honor ni de principis."" (p.19)

"Aquells immigrants murcians no havien arribat a Catalunya com a colonitzadors. Tampoc, o molt febement, com a invasors o peoners. Havien vingut a treballar i a menjar, senzillament, perquè a la seva terra es morien de gana [...] Avui dia, tots s'han integrat, i alguns, ultrapassant o sobrepujant aquesta integració, s'han tornat furibunds catalanistes. [...] Aquesta esperiència pot demostrar o permetre d'esperar que amb els immigrants d'ara passarà el mateixo poc més o menys." (p.32)

"el que passa és que no acabem de decidir-nos a anomenar catalans els qui han nascut aquí de pares de fora" (p.34)

\end{comment}

\subsection{Interseccionalitat}
En l'any 1989 la professora de dret i advogada de drets civils Kimberle Crenshaw fa servir el terme ``interseccionalitat'' per primera vedaga~\autocite{Crenshaw1989}.
Amb aquesta noció l'acadèmica vol descriure l'opressió multiple que sofreixen persones que pertanyen a la vegada a diversos grups marginalitzats.
Ella critica que fins aquest moment els discursos acadèmics, legals i activistes analitzen la discriminació al llarg d'un únic eix,
concentrant-se en els membres privilegiats de cada grup i d'aquesta manera amaguen la multidimensionalitat d'experiències de les persones més vulnerables, com per exemple les dones negres.

\subsection{Llengua, traducció i identitat}

\subsection{Patriarcat}
L'última noció teòrica que definirem abans de començar l'anàlisi de l'obra de Najat El Hachmi és el patriarcat.
Com senyala la periodista i activista Laurie Penny, aquest terme és refereix al comandament/regiment d'una elit d'homes sobre la resta de la societat.
Els homes que no disposen de poder polític, tenen al menys la satisfacció de manar els membres de la seva família com a compensació per la falta de control sobre la resta de les seves vides~\autocite[69-70]{Penny2014}.


\begin{comment}
[Penny2014]

"How are men supposed to cope with this loss of power in a society that still insists that the only way to be a man is to grab as much power as possible, to be rich, to be capable of extreme violence, to dominate other men physically and to dominate women sexually and emotionally? The received wisdom is that they're not supposed to cope. Without power over others, particularly over women, men are supposed to crumble, to lash out, to collapse in an extravagant welter of identity implosion that leaves a suspicious mess on the carpet." (p.64)

"'Patriarchy' does not mean 'the rule of men'. It means 'the rule of fathers' - literally, the rule of powerful heads of household over everybody else in society. Men further down the social chain were expected to be content with having power over women in order to make up for their lack of control over the rest of their lives.
[...]
Most individual men do not rule very much, and they never have. Most individual men don't have a lot of power, and now the small amount of social and sexual superiority they held over women is being questioned." (p.69-70)

"There are two big secrets about 'traditional masculine power' that mainstream culture does not want us to discuss, and it is imperative that we discuss them honestly [...]
The first big secret is this: most men have never really been powerful. Throughout human history, the vast majority of men have had almost no structural power, except over women and children. In fact, the power over women and children - technical and physical dominance within the sphere of one's own home - has been the sop offered to men who had almost no power outside of it." (p.75)

"Thus, a poor man working a job he hated could once expect to feel, at the very least, superior to his wife and children, to be master of his home even if he was treated like a slave outside it." (p.76)
\end{comment}
