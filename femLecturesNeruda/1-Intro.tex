\section{Einleitung}

Pablo Neruda, der Nationalpoet Chiles, ist ein der bekanntesten und beliebtesten, wenn nicht der bekannteste überhaupt, Lyriker*innen Lateinamerikas.
Seine Werke werden heute noch rege rezipiert und sind Tausendmal(syn!) von der Kritik kommentiert worden.

%Liebeslyrik

Neruda hat uns viele prägende (syn!) Werke hinterlassen und jedoch kann uns seine Liebeslyrik stutzen lassen.
Von vielen sehr positiv wahrgenommen und als Wiedergabe universellen menschlichen Gefühle angepriesen, können wir in dieser, wenn aus einem bestimmten Standpunkt betrachtet, beunruhigende Ideologien und Frauenbilder entdecken.

In dieser Arbeit möchten wir Nerudas Liebeslyrik aus einer feministischen Perspektive lesen.
Das Ziel dieser Arbeit(syn!) ist es nicht, die ``richtige'' Interpretation Nerudas Liebeslyrik auszuarbeiten, sondern viel mehr eine Pluralität der Lektüren (in Anlehnung an Annette Kolodny (quote!)) anzuregen, die Leseweisen ermöglicht, die im androzentrischen Literaturkritikdiskurs verdeckt bleiben.

% Feminismus-Begriff

% These??
\begin{comment}
  * literatur/sprache shape our minds
  * wachsamkeit/Sensibilität schaffen fürs Erkennen patriarchalischen Projekte, die mittels "universelle Gefühle", "schöne Liebeslyrik" vermittelt werden/dafür verkauft werden

  immer noch zu schwammig..
\end{comment}

% Weitere Werkzeuge

Durch das Close Reading exemplarisch ausgewählten Gedichte soll die Poetisierung bestimmten heteronormativen Perspektiven aufgezeigt und in Frage gestellt werden.

% Aufbau der Arbeit!
Desweiteren ist diese Arbeit wie folgt aufgebaut: im zweiten Kapitel werden zunächst etablierte Rezeptionen Nerudas Liebeslyrik dargestellt/zusammengefasst.
In Kapitel 3 gehen wir auf einige Begriffe bzw. Methoden ein, die für unsere weitere Analyse eine bedeutende Rolle spielen werden.
Darauf aufbauend bieten wir examplarisch (syn!) alternative Leseweisen von den Liebesgedichten ``Tu risa'' (``Los versos del capitán'') und ``Poema XV'' (``Veinte poemas de amor'') an.
Schließlich/schlussendlich/Anschließend/ausblickend wird die Arbeit nochmal zusammengefasst und ein Ausblick ...

\begin{comment}
1. Intro
  * Ziel von Feministischen Lektüren: zugrunde liegende Machtstrukturen in Werken und deren Rezeption aufzudecken
    ** androzentrische Perspektive der Literatur:
       *** Männer in Mittelpunkt (als Figuren)
       *** von Männern gemacht
       *** an Männer gerichtet
  * Wie erreicht? Durch eine Pluralität der Lektüren und Close Reading
ohne das ouevre Nerudas nicht als ganzes in Frage stellen
exemplarische Lektüren: Poetisierung bestimmter Heteronormativen Perspektiven
gehört historisiert; nicht als zeitlos darzustellen

* durch Sprache (und Literatur dann) bestimmte identitäre Zuschreibung geprägt: Sprache formt unser Gehirn, Gedanken, Weltanschauung, Denkweise, Kathegorien --> im Kopf behalten wenn man mit "Lektüren" umgeht
\end{comment}
