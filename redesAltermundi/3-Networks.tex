\section{Redes comunitarias}

Para las metas del trabajo actual, cuando hablamos de redes comunitarias vamos a entedender redes comunitarias inalámbricas (computer networks).


Para reflexionar sobre la infraestructura técnica, hemos primeramente de entender su funcionamiento.

En su forma convencional las redes de comunicación estan, más a menudo que no, organizadas de una manera jerárquica.
En la informática hablamos de una topología estrella (Bild!): todos los nodos estan conectados a través de un nodo central,
que tiene un papel clave.
Si este nodo falla, toda la red se desmantela y nadie puede comunicarse.

Sin embargo, existen también otras topologías/maneras de organizar una red.
Las redes comunitarias emplean frecuentamente la que se llama mesh (en malla) (Bild):
cada nodo está conectado con más de un otro, en el caso ideal cada nodo está conectado directamente con cada otro (full mesh).
No obstante la full mesh no está adecuada para redes más grandes, sobre todo inalámbricas, donde una directa linea de vista(?) se necesita para establecer la conexión.
En este caso tenemos más bien una mesh entre vecinxs, y, a veces, enlaces punto a punto a distancias más largas.
La mesh tiene la ventaja de estar mucho más perdurable.
Todos los nodos en la mesh son iguales y si uno de ellos se quema, el resto sigue existiendo y puede comunicarse como antes.
Ningún nodo tiene este papel clave que hemos visto en la otra topología y por eso no puede ejercer control sobre la comunicación de los demás~\autocite{FiTre2015}.

Ya hemos señalado que las redes comunitarias usan muy a menudo redes mesh.
¿Pero cuáles son sus otras características?
En general, se trata de proyectos de base locales con una exigencia política.
La idea clave es construir una infraestructura de comunicación decentralizada, creada, mantenida y poseada por la comunidad misma de lxs usuarixs.
Así, ni el estado, ni empresas grandes tienen el control sobre la red y pueden cerrarla de manera fácil.
% casos extremos
Redes comunitarias son especialmente interesantes en contextos de régimenes políticos autocráticos o en casos de catástrofes naturales~\autocite{Mabb2014}, ~\autocite{FiTre2015}.
Para los últimos está relevante sobre todo la tecnología de la mesh, que ante otras, garantiza servicio en condiciones extraordinarias.
% otras características

% motivaciones
No obstante, hay varias otras motivaciones para la construcción(syn) de tales redes~\autocite{Vaseva2016b}.
Desde ... , hasta el disfruto de areglar problemas técnicos y aprender como funciona la tecnología.
...
Muchos de estos proyectos han surgido desde la necesidad de conección, que proveedores convencionales no querían satisfacer ya que no veían oportunidad de lucro.
Se trata aquí sobre todo de poblaciones pequeñas, ajenas de ciudades grandes y a menudo con terreno difícil.
Así nacieron los proyectos Altermundi, guifi.net, freifunk..



\begin{comment}

"
https://pillku.org/article/altermundi-y-las-redes-comunitarias-digitales/
AlterMundi y las redes comunitarias, historia y perspectivas
Entrevista a Nicolás Echániz sobre el trabajo de AlterMundi en relación a las redes comunitarias digitales.Definiciones, desafíos, estrategias y propuestas
7 de febrero de 2015

Decimos que una Red Libre tiene estas características:

* libre uso: puede ser utilizada por sus participantes para ofrecer y recibir cualquier tipo de servicio que no afecte su buen funcionamiento
* neutralidad: no inspecciona ni modifica los flujos de datos dentro de la red más allá de lo necesario para su operación
* libre interconexión: permite, de forma libre y gratuita, el flujo de datos con otras redes que respeten las mismas condiciones
* libre tránsito: provee a otras redes libres acceso a las redes con las que mantiene acuerdos voluntarios de libre interconexión.

Una Red Comunitaria es una Red Libre en la que, además de las características ya enunciadas, se presentan otras:

* propiedad colectiva: su infraestructura es propiedad de la comunidad que la despliega;
* gestión social: la red es gestionada por la misma comunidad;
* diseño accesible: la información sobre cómo funciona la red y sus componentes es pública y accesible;
* participación abierta: cualquiera puede extender la red, respetando su diseño y sus principios.

Nuestro trabajo se centra principalmente en el desarrollo de un modelo de red específicamente adaptado a la realidad de pequeñas poblaciones digitalmente excluidas por el modelo comercial tradicional.
La misión de las redes comunitarias puede variar de un sitio a otro, pero en general tienden a empoderar al pueblo en el uso y comprensión de la tecnología y a promover un acceso más inclusivo a los recursos digitales, locales y externos.


El modelo de red MiniMaxi se compone de:

* una referencia de hardware que comprende tanto el router y las antenas a utilizar como los materiales para la adaptación a la intemperie y la instalación
* un firmware (sistema operativo) para instalar en los routers, que permite su auto-configuración
* un número de herramientas de software necesarias para la personalización y el monitoreo de estado de la red 

Otra experiencia que resulta digna de mención es la interacción que se ha conseguido entre las redes del Valle de Paravachasca con la Universidad Nacional de Córdoba, a través del Laboratorio de Redes y Comunicación y de la Prosecretaría Informática. Desde este año, gracias a un enlace propio de 50Km, las redes comunitarias han establecido interconexión con la red de la Universidad. La UNC, aparte de comenzar a fomentar la investigación y el estudio académico de este tipo de redes, también se ha ido estableciendo como un importante centro de convergencia de las redes regionales, alojando tanto al NAP de CABASE en Córdoba como al futuro nodo de Arsat.
Para nuestras redes comunitarias, lograr este tipo de interconexiones representa un grado de madurez que no se había conseguido previamente en el país y me animaría a decir en el continente.

sigue: descripción de QuintanaLibre, más sobre la infraestructura física

José de la Quintana es un pueblito serrano, con muy baja densidad de población, que tampoco es destino turístico y por lo tanto para las empresas tradicionales es muy poco atractivo como mercado. Los celulares sólo funcionan en las lomas altas y hay muy pocas líneas de telefonía fija, que están más tiempo fuera de servicio que funcionando. Los proveedores de Internet (inalámbrica) existentes antes de la llegada de QuintanaLibre ofrecían conexiones de 512Kbps (“medio mega”) a quienes tenían la suerte de tener visibilidad con sus torres, por valores inaccesibles para mucha gente.

Hoy en día contamos, gracias a la UNC y a la empresa Silica Networks, con un enlace propio dedicado de 20Mbps (veinte “megas”) simétricos, que se distribuye entre más de cuarenta familias, a las que se suma también la escuela secundaria, la radio comunitaria y los espacios públicos del pueblo, donde es muy común ver chicos usando la red con sus netbooks de Conectar Igualdad. La situación de conectividad hoy es comparable a la de ciudades medianas, por unos costos ínfimos que socializamos para mantener en funcionamiento la red.


"
----------


redes convencionales: organizadas de manera jerárquica (Telematik heranziehen?):
topología centralizada: el nodo en el centro tiene control/papel clave: si falla, toda la red se desmantela.

redes mesh: cada nodo está conectado con más de 1 otro: topología más perdurable/estable; todos los nodos son iguales;
full mesh: cada nodo está conectado directamente con cada otro: la red falla si todos los nodos fallan --> grande capacidad de recuperación

redes comunitarias: son en general redes mesh con una pretención política
características:
* infraestructura de comunicación decentralizada, creada y mantenida por la comunidad de lxs usuarixs (el estado/empresas grandes no pueden cerrarla tan facilmente) --> interesante sobre todo también para contextos autócratos (regimenes políticos antidemocráticos, vease Iran, China, Cuba, .. )
* garantizar acceso libre a información
* garantizar la libertad de expresión
* ..
* el firmware: software libre

      \item infraestructura de comunicación abierta, accesible para tod@s
      \item infraestructura creada y mantenida por la comunidad de l@s usuari@s
      \item ejemplos:
        \begin{itemize}
          \item Freifunk (Alemania)
          \item guifi.net (España/Cataluña)
          \item ninux (Italia)
          \item Funkfeuer (Austria)
        \end{itemize}

  \begin{itemize}
    \item conectar a comunidades excluidas por los proveedores convencionales de servicios Internet
    \item garantizar acceso libre a información
    \item garantizar la libertad de expreción
    \item profundizar los propios conocimientos técnicos, experimentar
    \item educar y concienciar a más gente
  \end{itemize}

anti sistemas autocratas:
"Anlässe für dieses neuerliche Interesse sind etwa die Versuche in autoritären
Staaten, das Internet komplett abzuschalten, um den Informationsaustausch
zu verhindern."[p.12][Mabb2014]

auch fuer Katastrophenfaelle relevant
"Ein weiteres Einsatzfeld sind Mesh-Netze im Katastrophenfall, besonders nach
Naturkatastrophen. So nutzten etwa nach dem Hurrikan Sandy Bürger im Brook-
lyner Viertel Red Hook ein solches, bereits vorhandenes und weiter funktionsfä-
higes Netz. Es wurde um einen Dienst erweitert, über den Bewohner Schäden ­
melden und lokale Informationen austauschen konnten. Zusammen mit der
Katastrophenschutzbehörde wurde das lokale Netz provisorisch per Satellit
ans Internet angebunden. Der dort eingesetzte technische Werkzeugkasten
des „Commotion Wireless Project“ teilt viele Komponenten mit dem deutschen
Freifunk."[p.12][Mabb2014]

bridging the digital devide!

surgen de necesidades locales:
por ejemplo en el caso de Altermundi: pequeños pueblos, poblaciones en las altas cumbres; localidades donde los proveedores convencionales no ven oportunidad de lucro y por eso no prestan servicio

se ubican en el contexto del ciberoptimismo (DEF)

reflexión crítica: está la solución fiable en gran escala: vgl papel del estado (IV Gui); última milla

[FiTre2015]
"rather binary antagonism between decentralization/freedom and centralization/control in
communications resources."
-- vlt bisschen uebertrieben, das so schwarz-weiss zu sehen, gewisser grad an zentralitaet ist unabdingbar, e.g. wenn das lokale Netzwerk ans Internet angeschlossen werden sollte

% Motivationen fuer WCN
%% Surveilance/Privacy
"privacy threats raised by centralized communications architectures"
"regulatory incentives to filter online content under the pressure of public officials"
"dominant telecom operators might undermine users autonomy is through their collaboration with intelligence agencies
for surveillance purposes"
"light-touch approach to logging users’ communications"

%% Net neutrality/freedom of speech
"aspiration to preserve network neutrality and civil liberties online to an eagerness to counteract
the growing concentration of power in the hands of a few large ISPs;"
"As opposed to commercial ISPs blocking certain
ports and censoring websites or content, most community networks are intended to protect net neutrality"

%% self-governed communications infrastructure, that allows for all that
"political drive to successfully roll-out and maintain a citizen-owned
telecom infrastructure."

%% need for communication infrastructure, since region not lucrative for commercial ISPs
"need to support undeserved areas lacking broadband connectivity to the will to provide a more diversified (and often
cheaper) means to access the Internet;"
"lack of affordable or high-quality Internet access"
"many provide connectivity
to places that traditional, commercial ISPs neglect. These are often undeserved areas or poor neighborhoods, whether in rural or urban settings."

%% interest in tech tinkering and understanding how networks work
"desire to learn and experiment with telecommunication
technologies, to the satisfaction of being part of a collective of like-minded individuals."

%% Bildungsauftrag -- central fuer Altermundi
"users’ lack of technical skills is sometimes one of the most challenging problems, and can lead projects to fail"
"all the WCNs we have interviewed, active and skilled volunteers are in charge of training new-
comers and neophytes, helping them, for instance, to set up and manage their routers and Wi-Fi antennas"
"Accordingly, community networks often promote the use of free software, decentralized
online services and end-to-end encryption techniques."

% Def community networks
"As opposed to more larger and centralized network infrastructures owned and managed by powerful third parties (such as the
state or large, highly capitalized Internet Service Providers (ISPs)), grassroots community networks are deployed by the community and for the
community at the local or regional level"
Auch A. Medosch
Und IV Nico Pillku
"WCN constitute, essentially, a political choice"
"all items are typically made publicly available."
"many
centralized WCNs have made the logical interface for administrating the network available to all members who wish to access it,"
-- public maps with the nodes are available
-- ich weiss nicht, ob es auch fuer finanzberichte gilt
-- aber auf jeden Fall hat altermundi auf der seite recht extensive doku
it's an ongoing work -- the infrastructure crumples if people cease to maintain it and provide broadband

% redes comunitarias -- tech description
-- nicht alle community networks sind wlans! (rhizomatica; manche sind nicht nur wireless --> guifi; und nicht alle sind mesh --> auch guifi;)
"many community networks do not rely on radio technologies"
"rely solely and exclusively on
free-to-use airwaves (or ‘spectrum commons’), WCN are to some extent more independent from incumbent ISPs than landline community"

% mesh
"simultaneously be both a client and a relay node for other users"
"it automatically reconfigures itself according to the availability and
proximity of bandwidth or storage." (das ist eher ad-hoc)
"to grow organically with minimal coordination and give them maximum resiliency: with mesh topology, there is
theoretically no sensitive points (or single points of failure) to jeopardize the functioning of the local network"
"Even if a particular node is down, dynamic connections between nodes enable packets to travel"
"to the extent that the network is dense enough and that many users operate as
relay nodes, the only way to shut down the network is to shut down every single node it is made of"

% Free networks
"Such questions are being addressed by the Free Network Foundation (FNF)—a nonprofit organization created to support ‘free networks’
—defined as any network that equitably grants the following freedoms to all: ‘Freedom to communicate for any purpose, without discrimination,
interference, or interception; freedom to grow, improve, communicate across, and connect to the whole network; freedom to study, use, remix,
and share any network communication mechanisms, in their most reusable forms.’ In conjunction with this definition and labeling effort, the FNF
seeks to create a license for interconnection agreements"

% list of projects in Europa
For our study, we focused on a handful of groups, and in particular FreiFunk (Germany), Wlan
Slovenija (Slovenia), Guifi.net (Spain) and Tetaneutral.net in Toulouse (France) —the latter is also a member of the FFDN, a federation of French
grassroots networks initially spearheaded by the landline community network FDN. Other European WCN include Ninux (Italy), Funfeuer
(Austria), the Athens Wireless Metropolitan Network (Greece), Djurslands.net (Denmark) and Czfree.net (Czech Republic).
% otros ejemplos
rhizomatica en México; VillageTelco? in South Africa; Peoplesopen.net in California; Altermundi in Argentina; Mesh Bogota? in Colombia; Brazil?

"while the political values attached to decentralization might have driven the launch of this
initiative, such motives are not in and of themselves sufficient for the network to scale up beyond a restrained community of highly engaged
individuals with strong ideological values."
-- naja, in ARG existiert auf jeden fall "real need"; die Frage ist ob man in dem Kontext wirklich von grossem Skalieren reden kann, ist aber vlt auch voellig egal
"To grow, these community networks must also provide a service that is considered at least as good
and preferably better than that of mainstream ISPs. "

% Security
"autonomous local networks" -- das ist auch interessant wegen soberania comunicativa
"Local mesh networks also enable users to
escape from the ubiquitous and pervasive surveillance that is occurring on the global Internet"
"‘Devices operating in any wireless network—including mesh networks—use a radio transmitter that can always be located by
triangulation’,"
"even with highly distributed networks, traffic can always be monitored."
"in no way can local community networks replace proper encryption techniques."
"Their primary advantage in times of crisis
is the fact that they provide community with the means to communicate independently from the central command of governments and traditional
operators. They enable citizen to organize (politically or otherwise) even in the eventuality that the established powers activate the so-called ‘kill-
switch’ and shut down communications networks in a given area (Hasan et al.)"
"even through they do not have to give their identity to log into our mesh network, they are not anonymous toward the
authorities or other entities due to hardware and software profiles of their devices and other metadata’"

\end{comment}


\begin{comment}
[Rieder2012]

% Redes
> As Jens Jessen, has recently pointed out in
> Die Zeit, it is democracy that guarantees a
> free Internet and not the other way around.
--> Siehe China, Iran, Cuba etc;
In dem Kontext verorten sich auch solche Projekte wie Altermundi

"We are creating a world that all may enter without privilege or prejudice
accorded by race, economic power, military force, or station of birth."
“We are creating a world where anyone, anywhere may express his or her beliefs, no matter how singu-
lar, without fear of being coerced into silence or conformity.” (Barlow 1996)
--> oder auch nicht --> siehe oben; community networks versuchen das aber wirklich umzusetzen

"Especially in American research and critical comment, there
seems to be a widely shared view according to which the Internet allows capillary configurations
of power – local initiatives, ad-hoc pressure groups, fan cultures, “issue publics” – to challenge
the statutory powers that be." (p.4)
--> aber um diese ueberhaupt moeglich zu sein, brauchen die Leute erstmal einen Zugang zu einer physischen Infrastruktur, der gar nicht so selbsverstaendlich gegeben ist.
--> koloniale Zusammenhaenge und Logiken werden reproduziert; aufgrund von geographischen (schwieriges Terrain), oekonomischen (Anbindung lohnt sich fuer kommerzielle Anbierter*innen nicht), politischen (der Staat oder X, dass sich Y Verhoer verschaffen kann) Gruende, haben Menschen keinen Zugriff zum Internet

\end{comment}
