\section{Nerudas Liebeslyrik: etablierte Rezeption}

% TODO Evtl move this passage to intro
Meine erste umfangreichere Begegnung mit Nerudas Werk fand während meines Studiums der Spanischen Philologie im Hauptseminar zu Pablo Neruda im Sommersemester 2016 statt, im Rahmen dessen die vorliegende Arbeit entsteht.
In der Sitzung, die sich der Liebeslyrik widmete, sprach eine Komilitonin von ``universellem menschlichem fleischlichem Begehren''.
Unter den mehr als 30 Seminarteilnehmenden scheinte niemand mit dieser Aussage ein Problem zu haben bis auf die Profesorin und 2-3 Komilitoninnen (mich inklusive), die etwas älter als das Durchschnittsalter von 20 waren.

Die Liebeslyrik von Pablo Neruda wird durch eine (fast absolute) Mehrheit von Lesenden und Literaturkritiker*innen ausschließlich positiv rezipiert.
Prominente Akademiker*innen wie Ilan Stavans sprechen von einem ``poeta bisexual, hermafrodita, poesía erótica, acceptada y leída tanto por hombres como por mujeres'' (quelle??)
und nennt die Poesie ``eternal''\cite{Stavans2014}.
%Stavans zitieren und polemisch einsteigen: "poeta bisexual, hermafrodita, poesía erótica, accpetada y leída tanto por hombres como por mujeres"
%hat er das im Seminar bei uns gesagt?

Stavans ist in dieser Haltung keineswegs alleine.
%TODO Quellen! (und remove language communities, von denen ich am Ende nichts zitiere)
Zahlreiche Kritiker*innen im Spanisch, Deutsch und Englisch-sprachigen Raum (und bestimmt auch darüber hinaus) lesen die Werke in den Bändern \textit{Veinte poemas de amor y una canción desesperada}, \textit{Los versos del capitán} und \textit{Cien sonetos de amor} als natürliche Manifestation von Liebesgefühlen, als beneidenswerte und musterhafte Beispiele für romantische Liebe.
Laut Keith Ellis ist \textit{Veinte poemas} ``tal vez el libro de poesía más popular del mundo'' (11)
und René de Costa nennt die Auswirkung der Bänder auf jüngere Spanisch-sprachigen Liebesgedichte ``extraordinary'' (35). %Aus Duncan1992
Antonio González Montes beschreibt die Liebe in Nerudas Werk als ``una experiencia fundamental, profunda y total, compromete y envuelve de modo absoluto y absorbente a los seres que lo comparten'' (25).
Guillermo Araya betrachtet die Frau in den Gedichten als ``homenajeada'' (179) und den Raum, den sie in den Text annimmt, als priviligiert, da sie ``the object of the male speaker's unrelenting, adoring gaze (151)'' ist~\cite{Duncan1992}.
Diese Haltung wird nicht nur von männlichen Kritikern vertreten.
Belén Lagos schreibt dem weiblichen lyrischen Du Macht zu und dem Mann eine Position von ``cierta sutil impotencia''.
Der Mann wird laut ihr zu ``una figura frágil, anorante, acosada por la presencia infinita de la mujer'', sobald dieser mit der idealisierten Frau seiner Träume konfrontiert wird.

In ihrer kritischen Analyse bestätigt Cynthia Duncan, dass eine Mehrheit der Kritiker*innen und Leser*innen, die Gedichte als ``natural and beautiful expression of male/female love relationships'' lesen~\cite{Duncan1992}.
Sie nennt diese sogar %find bessere Formulierung
``a sort of manual on the workings of modern romantic love and have come to be regarded, by many, as a standard agains which real-life relationships can be judged''~\cite{Duncan1992}.

\begin{comment}
%TODO: check these people! (and others)
``Others who have treated Neruda's love poetry from the male perspective include: Aguirre, Alazraki, Alonso, Durán, Pérez, and Pimentel. While I do not disparage the work of these or other critics of Neruda's poetry, I feel that they have contributed to the notion that the texts are "natural" reflections of male/female relationships and that the position of the woman in the texts is an enviable one. To my knowledge, the only critics to date who have acknowledged that the asence of a feminine perspective in Neruda's poetry is problematic are John Felstiner and Marjorie Agosin.''

\end{comment}

Wie bereits Duncan beobachtet, ist es wichtig, die Widersprüche in diesen Ansichten aufzudecken
und sich klar zu machen, dass diese immer aus einer männlichen Perspektive auf die Lyrik kommen (unabhängig davon welches Gender die Kritiker*innen tatsächliche haben).
Liebe wird von der Kritik fast ausschließlich aus der Perspektive des männlichen Sprechers und nie aus der Perspektive der weiblichen Geliebten wahrgenommen und interpretiert~\cite{Duncan1992}.
%``it is important to note that such views of Neruda's love poetry are almost always fraught with contradiction and reflect (despite the actual sex of the critic) with a distinctly male bias.
%That is to say, critics have tended to view love from the position of the male speaker in the texts rather than from the position of the female beloved.''~\cite{Duncan1992}.
Im restlichen Teil dieser Arbeit möchte ich diese Leseweisen aufbrechen und die ``Natürlichkeit'' dieser Männer-zentrierten Wahrnehmung von Liebe dekonstruieren.

\begin{comment}
%Duncan's project, ist im Grunde auch was ich mache:
``There is no doubt that Neruda is a gifted poet; whether he is guilty of sexism as a writer is not a point I wish to debate here.
But, I do take issue with the critical readings of Neruda which have drawn attention to the repression, subjugation, and silencing of women in his poetry only to dismiss these factors as natural conditions which are in and of themselves praiseworthy.''
"Sexims does not reside so much in the actual images as in the readingof those images: why is the passive, silenced woman seen in a position of power when her very existence depends on her male lover? Why is she viewed as man's equal when she is portrayed in the texts as his subordinate? Why do critics speak about a dialogue between the lovers when the woman's part is never voiced?
In The guise of honoring womanhood and its uniqueness, these readings of Neruda's love poertry conceal textual contradictions that are symptomatic of the way women are repressed in patriarchal culture."
\end{comment}

