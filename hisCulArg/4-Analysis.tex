\section{Evita: \emph{un instrumento de la construcción de la nación durante el peronismo}}

Un ejemplo particular del proyecto de consolidación de la imagen nacional argentina creada por el peronismo es el libro de primer grado infantil \textit{Evita}~\autocite{Albornoz1952}.

Nos dedicaremos en este análisis breve a las preguntas siguientes:
¿quiénes actorxs/protagonistas están dibujadxs?
¿está reflejada la realidad social en su complejidad?
¿la heterogeneidad cultural?
¿a quién se dirige?
¿cuál es la imagen de Argentina que pinta el libro?
¿cómo estan construidos las imágenes centrales de Eva y Juán Perón?

\subsection{Protagonistas}

Tenemos en el libro dos tipos de protagonistas: los personajes anónimos y las personas nombradas.

\subsubsection{Lxs trabajadorxs anónimxs}
Lxs protagonistas anónimxs respresentan el pueblo: la base popular que apoya el gobierno peronista y que está apoyada por ello.
Se trata aquí de un grupo sin nada de particular, caracterizado solamente por su amor de la patria, de los valores peronistas y de lxs líderes del regimen.
El rasgo común de todxs es la risa permamente: que nos señala que lxs protagonistas están contentxs, no puedan pedir nada más, viven en un paradiso en la tierra.
Aunqué el peronismo afirma lxs criollxs del interior como su base, como ya hemos resumido en capítulo..., su discurso se construye sobre todo en términos de clase.
La compleja realidad étnica del país no se incluye de manera directa en las políticas o el discurso.
También el libro está vacío de características étnicas (con una excepción: personas en vestidos étnicos en página 52).

Por otro lado, las referencias a la clase trabajadora son varias y muy claras:
la figura del \textit{descamisado} en página 59, los capítulos ``Día de Trabajo'' (p.35), ``Derechos del Trabajador Hoy'' (p.44-45), ``El buey'' (p.49), ``Hombre de campo'' (p.66), ``Todos deben producir'' (p.39),
con su homilía:
``La Patria pide a los niños argentinos que produzcan lo más posible. Así colaboran al bienestar común. Perón dijo:`Hay que producir, producir y producir'.''(p.39).

Lo que también llama la atención es la falta de agencia por parte de estxs protagonistas.
Son siempre lxs beneficiarixs pasivxs de los regalos, afectos y la bondad, sobre todo de la figura de Eva, y en menor grado de Juan Perón, presentadas en el libro, a lxs que miran con rostros vueltos hacia arriba y miradas de adoración (p.65).

\subsubsection{Lxs niñxs}
Otro grupo dibujado con freqüencia son lxs niñxs, así que se trata, después de todo, de un libro infantil.
Interesante aquí es notar que mucha vez tienen nombres: el libro habla de Yolanda, Inés, Zoilo y Ubaldo (p.16), Horacio (p.17), Cholita (p.21),
probablemente con el objetivo de facilitar la identificación entre lxs lectorxs jóvenes y lxs protagonistas y sus valores.

%Se transmite la impresión que lxs niñxs y el pueblo se pueden comprar
%* wiederkehrende Motive
%  * "juguetes para mí" (p.76)

\subsubsection{Eva y Juan Perón, otrxs actorxs nombradxs}
El resto de los personajes mencionados por su nombre que aparecen en el libro son exclusivamente personajes claves para la formación del Estado-nación argentino.
Domingo Faustino Sarmiento, Manuel Belgrano, José de San Martín todos tienen una página dedicada.
Y, supuestamente aparecen también Eva y Juan Perón.

No podemos hojear tres o cuatro páginas sin que aparezca al menos una vez unx de ellxs o ambxs.
Y cada vez están destacadas sus logros, sus virtudes y el amor incondicional que merecen y que el pueblo le da.
Tanto a Eva como a Juan Perón se refiere con un conjunto/campo/red de epítetos repetidos.
Eva es ``la mamá'' (p.18), ``nuestra Madre Espiritual'' (p.10), la ``madre espiritual de los niños'' (p.75), ``jefa espiritual de la nación'' (p.71).
Su imagen esta estilizada en términos de maternidad y espiritualidad religiosa.
Las características de maternidad no son arbitrarias: la madre es la persona que da nacimiento (fíjase en la relación con ``nación''), una de las personas más importantes para la formación de lxs niñxs.
Fijámosnos también en el comentario del \textit{Diccionario de estudios culturales latinoamericanos}: ``Sobre todo las mujeres en tanto madres, primero, y en su papel de maestras, posteriormente, han contribuido a la divulgación de los discursos hegemónicos nacionales y pedagógicos pese a que no fueron los sujetos que los formularan.''~\autocite[190-191]{SzIr2009}.
De hecho, Eva y Juan Perón están varias veces representadxs como padres figurativxs de cada uno y cada una, de toda Argentina:
en página 3--``Mi mamá, mi papá, Perón, Evita'', ``Mamá y papá me aman. Perón y Evita, nos aman.'' (p.7), ``Mi hermanita y yo, amamos a mamá, papá, Perón y Evita.'' (p.8).
Lxs dos representan una de las ``parejas heterosexuales ideales'' a las que se refiere el \textit{Diccionario} que ``evoc[an] la unidad nacional más allá de diferencias raciales y sociales''~\autocite[192-193]{SzIr2009}.

Al respecto a los retratos con conotación religiosa, el culto a la personalidad y las semejanzas de Eva Perón con las representaciones de la Virgen María son tan fuertes, que es fácil de comprender las inquietudes de la iglesia católica al respeto~\autocite{Chamosa2010}.
% obviamente después del fallacimiento de Eva Perón, se le dedica una página; la muerte ha convertido el culto de persona más fuerte? quizá no, se nota lo mismo también para Juán Perón
Además, se le atribuyen cualidades con conotación univocamente femenina.
Ella es ``buena'' (p.15), ``cariñosa'' (p.15), ``querida'' (p.67), ``amó a todos'' (p.13).

Juan Perón, por otra parte, es el líder, presidente, general, buen gobernante.
Tiene un papel más ``fuerte'', más activo, típico ``masculino''.
No podemos dejar de notar los papeles estereotípos de género que operan aquí.
Mientras Eva es la compasiva, la madre, Juan Perón es el líder activo que se ocupa de actividades ``masculinas'': el ejercito, las deudas, etc.
El deminutivo ``Evita'' con el que está denominada con freqüencia y era conocida en Argentina es aún otro rasgo en apoyo de este argumento.
En la vida pública, a mujeres a menudo se refiere por su nombre o un nombre diminutivo, mientras que los hombres están llamado más bien por su apellido o combinación de nombre y apellido (pensamos en nombres que aparecen juntos como ``Dilma y Temer'' o ``Hilary and Trump'').
Aún unx puede argumentar que a través del deminutivo se crea una intimidad entre Eva Perón y el pueblo (que ciertamente no es falso),
esta denominación también invoca asociasiones con una niña, alguien infantil, con una persona que no puede ser tomada en serio.

Sin embargo, lo que es nuevo es que Eva Perón como mujer no esta relegada al privado, sino tiene un papel activo político/público;
La vemos en representaciones de manifestaciones y discursos públicos (p.60,71), muy típicos por el movimiento, se destaca también su trabajo para el voto femenino.

En las representaciones de ambxs es interesante también la referencia repetida al trabajo:
Evita es ``Mártir del trabajo'' (p.22),
Juan Perón es ``Primer Trabajador Argentino'' (p.35).
Esto quiere destacar su vinculación con la clase trabajadora, se representan como una parte de ella.


\subsection{Imagen de la nación}

La imagen peronista de la nación en \textit{Evita} se transmite no sólo por lxs protagonistas del libro, sino a través de relatos inmediatos de Argentina.

%La nación argentina tiene un papel activo
Comenzando con el mapa en la página 29, podemos descubrir varias representaciones directas del Estado.
Los símbolos estatales de ``la Nueva Argentina'': la bandera, el escudo, la escarapela y el himno, también están dibujados explicitamente (p.18-19).

Como aborda en su análisis de Patagonia Gabriela Nouzeilles, el poder estatal se manifiesta, entre otras cosas, en su capacidad para someter la naturaleza y configurar el espacio~\autocite{Nou1999}.
La fetichisación de la naturaleza, su sumisión al servicio de la gente y su valor símbolico para la imagen del estado encontramos en \textit{Evita} en los capítulos dedicados a las ``bellezas de la patria'' (p.56-57) y en las referencias a ``tierras ajenas'' como las Islas Malvinas o la Antártida, que forman parte del imaginario de la nación (p.24-25).

Demás, el libro forja la imagen de una Argentina moderna, industrializada.
Proyectos y artefactos de la infraestructura están representados varias veces:
tenemos el capítulo ``Caminos'' (p.50), así como dibujos múltiples de ferrocarriles(p.48,72), carros, coches, del teléfono (p.58), avión (p.72), barcos (p.61,72).

Otro elemento fundamental de la figuración de la nación es el calendario de los días festivos~\autocite{SzIr2009}.
También estos encontramos en \textit{Evita}:
páginas se dedican al
``Día de la Libertad'' (p.22),
a la ``Independencia Política e Independencia Económica'' (p.30-31),
al ``Día del Trabajo'' (p.35),
y al ``Día de la Recuperación Nacional'' (p.48).
Además, hay al final del libro una lista de las ``Fechas que conviene recordar'' (p.79-80).


\subsection{Destinatarixs}

En primer lugar, lxs destinatarixs de este libro son lxs niñxs de escuela primaria, lxs niñxs que aprenden a leer y escribir.
El lenguaje es simple, repetitivo, con accento en lo más importante (las figuras centrales de Perón y Evita, la imagen de Argentina).
Las temas son una mezcla entre lo que suposadamente ha de interesar niñxs: juquetes, amor hacia la familia, .. y propaganda política.
El mensaje para ellxs es claro: Argentina es parte de su familia, Juan y Eva Perón son prácticamente una segunda pareja de padres.

Aunqué el libro está claramente dirigido a lxs seguidorxs del peronismo,
hecho señalado por el lenguaje y los temas abordados,
es interesante observar que esto no se nota necesariamente en los dibujos de las figuras.
Con pocas excepciones (las figuras en vestidos étnicos en página 52 y el \textit{descamisado} dibujado en p.59),
las figuras (syn!) suelen mostrar un aspecto más burgués que de clase trabajadora.


\begin{comment}

* "Policlínico Evita" (p.43), "Fundación Eva Perón"(p.43), "clubes infantiles Evita" (donde se practica futbol^^)(p.55)
* se rescatan los éxitos (el voto femenino, el pago de las deudas, p.72...)

* p.10: oraciones, rezar, catolicismo; pero también aquí "nunca olvido a Eva Perón, nuestra Madre Espiritual"
  * gottesgleichheit (hat anscheinend die kirche etwas verstoert^^ -- Chamosa o FF?)
  * kann alle wuensche verwirklichen
\end{comment}
