\section{Introducció}

Das Ziel dieser Arbeit ist es nicht, die ``richtige'' Interpretation Nerudas Liebeslyrik anzubieten, sondern viel mehr eine Pluralität der Lektüren (in Anlehnung an Annette Kolodny) anzuregen, die Leseweisen ermöglicht, die im androzentristischen Literaturkritikdiskurs verdeckt bleiben.


\begin{comment}
1. Intro
  * Ziel von Feministischen Lektüren: zugrunde liegende Machtstrukturen in Werken und deren Rezeption aufzudecken
    ** androzentrische Perspektive der Literatur:
       *** Männer in Mittelpunkt (als Figuren)
       *** von Männern gemacht
       *** an Männer gerichtet
  * Wie erreicht? Durch eine Pluralität der Lektüren und Close Reading
ohne das ouevre Nerudas nicht als ganzes in Frage stellen
exemplarische Lektüren: Poetisierung bestimmter Heteronormativen Perspektiven
gehört historisiert; nicht als zeitlos darzustellen
\end{comment}
