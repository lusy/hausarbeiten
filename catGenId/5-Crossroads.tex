\section{Una noia a l'encreuament. Lectura interseccional}

  interseccionalitat -- Crenshaw

En l'any 1989 la professora de dret i advogada de drets civils Kimberle Crenshaw fa servir el terme ``interseccionalitat'' per primera vedaga~\autocite{Crenshaw1989}.
Amb aquesta noció l'acadèmica vol descriure l'opressió multiple que sofreixen persones que pertanyen a la vegada a diversos grups marginalitzats.
Ella critica que fins aquest moment els discursos acadèmics, legals i activistes analitzen la discriminació al llarg d'un únic eix,
tractant l'opressions de gènere i racistes com dos fenòmens mútuament excloents,
concentrant-se en els membres privilegiats de cada grup i d'aquesta manera amaguen la multidimensionalitat d'experiències de les persones més vulnerables, com per exemple les dones negres.
Segons Crenshaw, la interseccionalitat denomina un tipus de discriminació que consta de més que la suma de les seves parts.
Això vol dir que les dones negres per exemple experimenten una discriminació molt particular que és diferent de la combinació d'experiències de les persones negres i les dones en general (que no significa que les dones negres no siguin discriminades com dones o com negres, però que també experimenten una opressió particular com a ``dones negres'').

Des de llavors, aquest concepte s'ha fet servir en contextos diferents per analitzar situacions i col·lectius que pateixen d'una opressió multiple.

\begin{comment}
\subsubsection{Quotes}
"Jo vaig anar a aportar la matrícula, ja aterrida de tants passadissos i tantes aules i si no sé ni trobar les oficines, com m'ho faré per trobar la meva classe.
 El primer dia ens van fer anar a la sala d'actes i allà van dir les llistes de cada grup. Tothom va riure quan van dir el meu nom, que el van dir tan diferent que jo no sabi ni que fos jo. És clar, en aquell lloc no hi estaven acostumats, a gent com jo. Era l'única de la classe que feia batxillerat, tota sola sense ni el noi dels ulls crema que havia de ser amb mi sempre"
 -- anders sein faellt ins Auge;
    Menschen machen sich lustig;
    Migrant*innenkids werden rausgeekelt oder von ihren Eltern aus der Schule rausgeholt

"[...] i jo que havia trencat lleis no escrites i havia decidit que no volia ser ni auxiliar d'infermeria ni administrativa de grau u ni mecànic ni electricista.
  Pesaven força espases de Dàmocles damunt meu: que si jo a la teva edat ja estava casada, que si en la teva cultura ja se sap que no val la pena, que us acaben casant tard o d'hora, la d'aquest és l'últim curs i alguna altra que tenia el pare al cap, com allò de les dones que no traeixen mai els pares però que sí que acaben traint els homes.
  Tot això duia jo a la motxilla, però ningú se'n va adonar. Al principi l'institut va ser un espai d'angoixa, que tot funcionés tan diferent" (p.273)

\end{comment}

\subsection{Entre dues cultures}

\begin{comment}
  1) la cultura d'origen representada per
     * el pare
     * la mare
     * la família amb els avis i les tietes
     * els moros de Catalunya
  2) la cultura d'acollida, representada per
     * l'escola
     * les amigues
     * les amants del pare

+ stereotips mútuus de cada grup per l'altre
\end{comment}

\subsection{Entre el desig d'autoidentificació i les atribucions del masclisme}

\begin{comment}
  1) autoidentificació
  buscant la seva identitat
  * religió
  * literatura

  2) les atribucions
  * com han de ser les nenes/les dones?
    ** alleine fuer die ganze care arbeit zustaendig, wird als natuerliche neigung inszeniert
  * les dones i les "altres" són culpables per tot (vgl Laurie Penny)
    --> keine Solidaritaet zwischen den Frauen (zb la mare i les amants) moeglich, obwohl sie alle Opfer sind; sie werden gegen einander ausgespielt
  * frauen werden auf ihr aeusseres reduziert
  * die frauen wuerden nie genuegen;
\end{comment}

L'última noció teòrica que definirem abans de començar l'anàlisi de l'obra de Najat El Hachmi és el patriarcat.
Com senyala la periodista i activista Laurie Penny, aquest terme és refereix al comandament/regiment d'una elit d'homes sobre la resta de la societat.
Els homes que no disposen de poder polític, tenen al menys la satisfacció de manar els membres de la seva família com a compensació per la falta de control sobre la resta de les seves vides, destaca l'autora~\autocite[69-70]{Penny2014}.

\begin{comment}
\subsubsection{Quotes}
%------
"Aquesta és la història de Mimoun, fill de Driouch, fill d'Allal, fill de Mohamed, fill de Mohand, fill de Bouziane, i que nosaltres anomenarem, simplement, Mimoun." (p.7)
-- die Wichtigkeit, das Mystische, sein Stammbaum, seine Geschichte wird ihm entzogen?

"És la seva història i la història de l'últim dels grans patriarques que formen la llarga cadena dels avantpassants de Driouch. Cadascun d'ells havia viscut, actuat i influït en la vida de tots els que els envoltaven amb la fermessa de les gran figures bíbliques." (p.7)

"Aquesta és l'única veritat que us volem explicar, la d'un pare que ha d'afrontar la frustració de no veure acomplert el seu destí, la d'una filla que, sense haver-s'ho prposat, va canviar la història dels Driouch per sempre." (p.7)
-- el destí té un paper central; es menciona bastant sovint

"El pare deia mira, el teu germà és molt menys ploraner que tu [...] I què faràs quan t'hi barallis, qui serà el vencedor, tu o ell, que és més petit? Si vols que t'acabi respectant i et digui Azizi\footnote{Apel.latiu que els germans petits donen als més grans com a senyal de respecte}, ja pots imposar-t'hi." (p.21)
grandioese Erziehungsmethoden, die fuer geschwisterliche Liebe sorgen und bescheuerte Geschlechterstereotypen auf keinen Fall fortsetzen.
iwann bringt er mit 3 seinen kleinen Bruder um. So viel zum Erfolg davon

altra violació:
"Però el Manel tenia aquell mena d'instint de caçador que han de tenir per força els que estan destinats  a ser gran patriarques i no entenia que era no." (.91)
-- no means no
\end{comment}

\begin{comment}

\subsubsection{Notes:[Penny2014]}

"How are men supposed to cope with this loss of power in a society that still insists that the only way to be a man is to grab as much power as possible, to be rich, to be capable of extreme violence, to dominate other men physically and to dominate women sexually and emotionally? The received wisdom is that they're not supposed to cope. Without power over others, particularly over women, men are supposed to crumble, to lash out, to collapse in an extravagant welter of identity implosion that leaves a suspicious mess on the carpet." (p.64)

"'Patriarchy' does not mean 'the rule of men'. It means 'the rule of fathers' - literally, the rule of powerful heads of household over everybody else in society. Men further down the social chain were expected to be content with having power over women in order to make up for their lack of control over the rest of their lives.
[...]
Most individual men do not rule very much, and they never have. Most individual men don't have a lot of power, and now the small amount of social and sexual superiority they held over women is being questioned." (p.69-70)

"There are two big secrets about 'traditional masculine power' that mainstream culture does not want us to discuss, and it is imperative that we discuss them honestly [...]
The first big secret is this: most men have never really been powerful. Throughout human history, the vast majority of men have had almost no structural power, except over women and children. In fact, the power over women and children - technical and physical dominance within the sphere of one's own home - has been the sop offered to men who had almost no power outside of it." (p.75)

"Thus, a poor man working a job he hated could once expect to feel, at the very least, superior to his wife and children, to be master of his home even if he was treated like a slave outside it." (p.76)
\end{comment}

