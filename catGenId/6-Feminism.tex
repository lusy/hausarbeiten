\section{Entre el desig d'autoidentificació i les atribucions del masclisme. Lectura feminista}

\begin{comment}
  1) autoidentificació
  buscant la seva identitat
  * religió
  * literatura

  2) les atribucions
  * com han de ser les nenes/les dones?
    ** alleine fuer die ganze care arbeit zustaendig, wird als natuerliche neigung inszeniert
  * les dones i les "altres" són culpables per tot (vgl Laurie Penny)
    --> keine Solidaritaet zwischen den Frauen (zb la mare i les amants) moeglich, obwohl sie alle Opfer sind; sie werden gegen einander ausgespielt
  * frauen werden auf ihr aeusseres reduziert
  * die frauen wuerden nie genuegen;
\end{comment}

Una altra noció teòrica central per a l'obra és el patriarcat.
Com assenyala la periodista i activista Laurie Penny, aquest terme es refereix al regiment d'una elit d'homes sobre la resta de la societat.
I historicament, aquesta elit és bastant petita.
Llavors, els homes que no disposen de poder polític, tenen al menys la satisfacció de manar els membres de la seva família com a compensació per la falta de control sobre la resta de les seves vides, destaca l'autora~\autocite[69-70]{Penny2014}.
Aquestes són les estructures del patriarcat, i exactament aquesta descripció trobarem molt aplicable per a la novel·la:
``És la seva història i la història de l'últim dels grans patriarques [\ldots]. Cadascun d'ells havia viscut, actuat i influït en la vida de tots els que els envoltaven amb la fermessa de les gran figures bíbliques.''~\autocite[7]{ElHachmi2008}.

Una idea interessant aquí és que les dones tenen un paper bastant important per a la reproducció de l'ordre patriarcal, ja que elles són en la majoria dels casos encara les persones que primerament s'ocupen de l'educació dels fills i de les filles i llavors aquelles que perpetuen les normes opressives.
``Some of the things that happen around him justify what he does. I was very surprised when I was writing the book that women in the family were so important in making him become the monster he becomes'' assenyala Najat El Hachmi en una entrevista, referint-se a Mimoun~\autocite{HaAM2011}.
Aquest concepte d'excuses i justificacions es recolza també en el text de la novel·la: ``però ja ho saps, filla, que en el fons el teu pare és de bona fusta i no faria mai mal. És només això, que els espants no li han acabat de marxar mai del cos i això l'ha fet algú diferent''~\autocite[18]{ElHachmi2008}.
Doncs, Mimoun creix a creure que per a ell tot està permès.
``Era així com Mimoun aconseguia sempre que les dones de la seva vida l'anessin convertint en patriarca.''~\autocite[99]{ElHachmi2008}.

Les eines del patriarcat i del masclisme per manar les dones i la sexualitat femenina són multiples.

%reducció de les dones en la seva aparença física
Una de les més classiques és la reducció de les dones en la seva aparença física, que es converteix en l'única característica relevant per valuar una dona.
El que és encara més alarmant i trist aquí és que aquesta postura es manté també per dones en les seves relacions entre elles i fins i tot en l'evaluació de si mateixes.
Observem els pensaments de la filla sobre una de les amants del seu pare:
``Sí que era lletja, feia cara de dolenta de pel·lícula, de les dolentes que no sedueixen, que només maquinen [\ldots] Bruixa!, vaig pensar. Que la mare és més guapa, que ho sàpigues, que és més bonica''~\autocite[186]{ElHachmi2008}
o sobre una putativa amant de la seva parella:
``Un cul més gros que el meu no faria malbé la meva llibertat sostinguda.''~\autocite[308]{ElHachmi2008}.
Una dona està aquí reduïda a una part del cos i el verdader culpable per la situació s'ofusca.
%"La roba sempre havia estat un problema. Des que havies fet el canvi [...] Que mira si els malucs et creixen més del compte què voleu que hi faci i no teniu més talles que quaranta. Una quaranta-dos en una botiga de roba moderna i per a joves no s'estilava. Si ets jove es dóne per fet que estàs prima" (p.284)
%--körperimage, die einer aufgedrungen wird
%"Encara ara tinc pànic als emprovadors de les botigues i em fa mandra anar a comprar roba per no haver-m'hi de ficar." (p.285)
%-- keine solidaritaet zwischen den "opfern", die werden gegeneinander ausgespielt;
Les possibilitats per solidaritat i suport mutu entre dones es distrueixen.

% atribucions: com són les nenes
Una altra eina són les atribucions assídues de qualitats i capacitats que tingui una nena o una dona.
Perquè si tothom et diu continuament que siguis així o d'aquesta altra manera, al final també acabes creient-lo.
I les atribucions són, naturalment, no de qualitats vinculades amb control i poder, sinó tot el contrari: de qualitats submisses, suaus, agradables, no poderoses i quals no canvien el món.
Sobre les dones cau la responsabilitat per totes les tasques domèstiques, tota la feina de cura i afecte i això, per descomptat, sense cap agraïment, sense res a canvi.
La filla està sotmesa de prescripcions vàries:
``[i] t'ho demostren, les nenes, et demostren que t'estimen facis el que facis i el seu amor és sempre incondicional. Jo ja vaig néixer amb aquest deure afectiu''~\autocite[147]{ElHachmi2008},
``[l]a mare sempre deia que jo era més responsable que els meus germans, més treballadora, més estudiosa, més de tot, però em penso que l'única cosa que jo era més que ells era nena''~\autocite[191]{ElHachmi2008},
``[l]a mare, estesa al llit, em va dir no podies haver donat roba neta als teus germans, que la porten tota tacada de tomàquet d'ahir?''~\autocite[220]{ElHachmi2008}.
% la dona qui satisfà les necessitats dels homes
Llavors, el patriarcat tracta les dones com una propietat dels homes, com si la seva única funció fos satisfer les seves necessitats, ser una combinació d'amant sumbissa i dona per fer feines.
Trobem aquestes relacions no només entre la mare i Mimoun, entre la filla i Mimoun, sinó també entre la filla i el seu (futur) marit.
``Jo no tenia prou temps per excitar-me i aquell fica treu només li anava bé a ell, però jo feia ah, uh, mm, ui, ai.''~\autocite[306]{ElHachmi2008}.

% destruir l'autoestima: una dona sense un home...; nie genuegen
\begin{comment}
Fuehlt sich nicht zugehoerig; die Frauen, die nie genuegen
----------------------------------------------------------
"La Laia havia dit: em penso que li agrades, a l'Arumí. Et mira d'una manera, cada cop que passem davant seu. Jo tenia clar que no podia agradar a ningú i encara menys a un d'aquí de tota la vida. Hi havia una premissa que m'explicava el món a pesar de les seves incongruències: als moros els agraden totes les dones, però especialment les mores. Als d'aquí, en canvi, no els havien d'agradar mai les mores. Era contranatural. Si no com s'explica que el pare amagués la seva dona de totes les mirades que no fossin cristianes?" (p.243)
-- fuehlt sich nicht zugehoerig

"Hi havia altres motius per creure que jo no podia agrada a ningú: 1) No havia tingut mai cap noviet a classe, cosa que havia passat amb la majoria de companyes a aquelles edats. 2) [...] ningú em feia mai cap petó [...] 3) La mare sempre em feia portar una trena llarga que ja semblava part del meu cos, els cabells ben enrere, duia ulleres i havia fet el canvi tan de pressa que semblava una geganta al costat dels meus companys d'escola, la mare de tots." (p.243-244)

"[...] havia dit a la Laia no em presentes la teva amiga, tan guapa que és? Jo vaig dir te'n fots o què? No, no, ho dic de debò, però sempre em va quedar el dubte perquè ho deia amb aquell mig somriure que fa ell." (p.244)
--fuehlt sich verarscht wenn sie andere menschen huebsch nennen; die frauen sind nie genug: genug jung, schoen, wuerdig fuer liebe

"Escolta, va dir un dia, vols sortir amb mi? Mira que n'ets, de cruel, li vaig dir i ell que no devia entendre res i jo encara vaig entendre-ho menys." (p.245)

"He portat la teva mare perquè he pensat que si t'havia trobar amb un home aquí potser et mataria i ella sempre impedeix que faci aquesta mena de coses. Un home? He vist que eren les quatre de la matinada i encara tenies els llums oberts, segur que hi ha algun home amb tu, què ha de fer, si no, una dona que viu sola?" (p.328)
-- jaja genau, wenn da kein penis ist, sollte sie sich hinlegen und sterben, ist klar dass es so nicht weiter gehen kann -\_-
-- una dona sense un home no és una persona integra i saludable (vgl "Una dona sense un home és com un peix sense bicicleta")
-- moral doble, per a un home (sol) està acceptat tenir relacions amoroses, per a una dona - no (putes)
\end{comment}

% putes/control sobre la sexualitat femenina
Una eina més del patriarcat per subjugar les dones és el control sobre la sexualitat femenina.
No es suposa que les dones poden tenir desitjos sexuals, la seva sexualitat ha de ser una cosa que els homes impugnen.
Doncs l'única arma femenina consisteix en la negació de l'acte sexual.
Dir que no seria una arma bastant potent (i molts homes s'enfuriren sobre ella) si en realitat fos respetada~\autocite{Penny2014}.
Però la trista realitat és que el ``no'' femení no val gaire.
Assalts sexuals i violència de gènere són lamentablement una part de la normalitat del món de avuí, representacions d'això trobem també en ``L'últim patriarca''.
En aquest context, el que és encara més exasperant i irònic és que s'espera que les dones protegeixin la seva honor i són les úniques portadores de la vergonya.
I els càstitjos socials són molt forts per aquelles que no aconseguexin protegir-se o, per no esmentar, aquelles que si li agradi tenir relacions sexuals.
La societat patriarcal les etiqueta com ``putes'' i les sotmet a escarn i isolació.
La paraula serveix també com a advertència per totes les altres, les demostra com acabaren si surten del marc predefinit de les normes i prescripcions masclistes:
``Ell havia dit no et vull veure mai més parlant amb un home al mig del carrer, que no sigui dit que la filla del Driouch és una puta qualsevol.''~\autocite[269]{ElHachmi2008}.
La filla també nota la hipocresia i l'absurditat de la denominació:
``oficialment no podia entrar a cap local on seure i prendre alguna cosa, que una norma no escrita es veu que deia que allò era de putes. Jo començava d'estar-ne farta, de la paraula, i que totes les dones del món fossin la mateixa cosa''~\autocite[285]{ElHachmi2008}.

En aquesta situació de confusió profunda, la filla busca el seu lloc/possibilitats per autoidentificació.

Malgrat tot, la filla és un personatge fort qui ``sense haver-s'ho prposat, va canviar la història dels Driouch per sempre''~\autocite[7]{ElHachmi2008}.
Ella és la persona amb la ``capa de supermana'' qui salva la família en ocasions múltiples.

Autoidentificació?
-------------------
la mestra amiga
"Jo li havia parlat de crisis, de crisis que encara era incapaç de reconèixer com a indentitàries, de pits que creixen massa, de la mare que no volia que em depilés i de com m'havia llençat els tampons per por que jo perdés la virginitat, així, sense ni parlar-ne ni res, havia vist el dibuix de les instruccions i els havia llençat a les escombraries." (p.268)

Autoidentificació - recerca de la identitat
"El metge va dir això són atacs d'ansietat i sonava tan greu que encara em vaig espantar més. Tens motivus per estar així, algun problema personal? No, doctor, no, la meva vida és perfecta, volia dir-li, com al de qualsevol adolescent que ha de fer-se gran i no sap com fer-ho. COm tots, suposo, li vaig dir, i em va donar aquells tranquil·litzants que m'havia de posar sota la llengua si em venia allò una altra vegada." (p.302)

Autopercepció/autoestima
-----------------------
(ell a ella crec)
"I tu suposo que deus ser d'aquests immigrants que viuen sols i tal. No t'ho creus ni tu, devia pensar, i vam caminar fins al Club que hi havia més amunt, on mig a les fosques ja em va dir que estava boig per mi. Què? Si no em coneixes de res. Et conec i t'he seguit durant els últims sis mesos, t'estimo. T'estimo, t'estimo, sonava dins del meu cap i jo només vaig poder riure. No em pots estimar si no saps com sóc. I és clar que puc. Només digues-me que em donaràs una oportunitat. En sentir allò qualsevol altra amb l'autoestima equilibrada hauria fugit corrents i hauria notat que anava més cremat del que solen anar els homes a la seva edat, que aquell no era el meu tipus ni de molt lluny. Però encara tenia assumit que si un home em mirava era perquè tenia alguna cosa a la cara[...]"(p.288-289)
--stalker
--nein sagen lernen
--autoestima

"No vaig trigar gaire a veure-m'hi estirada, per què havia d'anar tot tan de pressa? [...] que ja estava només en calces i sostenidor davant d'ell [...] Per què no vaig dir no, encara no, jo no ho vull, això. Volia demostrar que tenia tanta pressa com ell, que a pesar de la diferència d'edat, jo sabia molt bé el que em feia. I no en tenia ni idea." (p.293)
-- immer noch nicht in der lage, nein zu sagen; schaemt sich weil sie keine erfahrung hat


"Si anava o no a l'institut depenia de tants factors i cap d'ells tenia a veure amb si jo em portava bé o no, o amb si jo treia bones notes o no, o amb si jo feia cas o no. Estranyes desaparicions havien tingut lloc en els últims dos anys a l'escola i encara gràcies que no m'havia tocat a mi. Desaparicions de noies com jo que venien d'un lloc semblant al lloc on jo vaig néixer però que potser eren molt diferents de mi o devien tenir molta menys sort que jo. Noies que ara tenen tres o quatre fills" (p.271)
-- entre culture i masclisme

"A mi em tocava desaparèixer de l'escenari escolar i encara no sé com no va passar. Un factor era l'avi, que era l'únic que em preguntava què, com han anat els exàmens, ho has aprovat tot?" (p.271)
"La mare ja m'ho havia dit, el teu pare diu que aquest és l'últim any que vas a escola, i era una cantarella que es repetia cada final de curs. Aquest ñes l'últim, i jo deia val, però sabia que no seria així. Potser l'altre factor va ser la professora massa amiga del pare, que en alguna cosa devia influir-lo, que deia la teva filla ha d'estudiar una carrera" (p.271--272)
-- entre culture i masclisme


%religió
refugi també en la religió.

\begin{comment}
* Identitaetssuche (vgl. versuchen den eigenen Platz zu finden; die Freund*innen angucken; in die Religion; )

Món mascliste: les dones percebudes com a propietat dels homes
--------------------------------------------------------------
"[...] ell que va començar a dir-me que millor que no fos de cara al públic, no? És qu les feines de bars i restaurants, creu-me, són molt pesades, la gent no és respectuosa i a mi no m'agradaria que et vinguessin a dir res, que ets la meva dona, ara." (p.319)

"Jo encara rumiava en aquell parell d'opinions, encara dubtava si no era que volia començar aquella mena de control subtil que solen fer els amrits, quan ja va trucar el pare, quan ja va trucar el pare. La teva mare vol parlar amb tu. Que diu que us ha vist en un bar i que tu anaves destapada i sense mocador i ensenyant el cul a tothom. Filla, pots verstir com una dona casada, si et plau?" (p.320)
-- control de la família segueix

"Vaig penjar i li ho vaig explicar a ell. Va callar, massa que va callar. Potser t'ho hauries de pensar. Què, portar mocador? Vaig riure només d'imaginar-me a mi mateixa d'aquella manera, no hi sabria ni caminar, vestida així. No, el mocador no, però una mica més tapadeta, que la gent ho ha de saber, que estàs casada." (p.320)

"Casada, casada, les dones casades van així, no estudien, no treballen, cuinen molt bé i tenen les coses totes ben endreçades." (p.320)
-- vlt interseccionalitat amb cultura?

"No va venir a picar a la meva porta, no va demanar-me de genolls que tornés, diuen que va desaparèixer. Sï que havia trucat i havia dit allò de mira que si et veig amb un altre jo no ho podré suportar, mira que to ets meva i de ningú més. Jo li vaig dir que mirava massa aquella mena de programes de televisió i que pobre d'ell que se m'acostés, que ja no hi havia res entre ell i jo. No et donaré el divorci, et quedaràs sempre més penjada i no podràs tornar-te a casar. Ni em vull casar ni estaré gaire més temps casada amb tu, que jo ja he damanat el divorci als jutjats d'aquí" (p.326)
-- moral doble

Autoidentificació - religió
---------------------------
Kapitel Estima Déu i ell t'estimarà
- Identitaetssuche
- Ohne vernuenftige Vorbilder + Erklaerung, ohne Sinn und Verstand
- Escaipismus
- zu Extremhaltungen neigend
- sich aufgehoben fuehlen
"Jo em vaig proposar de ser una bona musulmana, la millor." (p.226)

"Déu meu, fes que el pare torni al bon camí, però ho deia en la llengua de la capital de comarca perquè en la llegua dels musulmans jo no hauria sabut com dir-ho. S'hi valia: en l'última part de l'oració, on demanes alguna cosa directament a Déu, podies fer servir la llenuga que et fos més còmoda." (p.226)
-- llengua-idenitat

Hijab
-----
quote Interview El Hachmi:
"Los pornógrafos de la etnicidad acentúan rasgos de ti que en tu país encontrarías ridículos."
"El inmigrante no quiere pertenecer a una asociación de inmigrantes, sino a una de vecinos..."
"Cada mujer lleva el velo por motivos diferentes. Y no se
puede pretender salvar, de manera paternalista, a las pobres moritas del yugo de
sus maridos.
-El yugo existe.
—Como en otras culturas."
"El problema es que la mayoría tiene permiso de residencia sin permiso de trabajo. La ley de extranjería las condena a la clandestinidad laboral y eso hace que dependan del marido. Les corta la vía de emancipación."

"Treu-te això del cap, que em fas passar vergonya. I ella que no, que em sentiré despullada, que no. Mira que aquí les coses són diferents i a mi em coneix molta gent i tinc una empresa i no hi ha cap nexessitat de portar aquests draps." (p. 183)
-- der stellt sich sehr befreit und so dar; und spaeter findet er gut, dass seine tochter nen hijab traegt..

"Me'l posava per resar, primer. Després per estar per casa. FIns que vaig sentir que era imprescindible, que no podria viure mai més passant amb el cap descobert davant de ningú. Me'l vaig posar per anar a comprar i vaig sentir les mirades estranyades de les botigueres que em coneixien [...] Vaig sortir així un parell de vegades i un dia el pare em va veure. [...] Au, no surtis més amb aquest drap al cap." (p.228)
-- warum gucken sie die Verkaeuferinnen komisch an? ist sie etwa nicht dieselbe person? was sehen sie im hijab? ein symbol der anderen? ein symbol der opression?

"La mare em va fer anar a ca la Soumisha a buscar alguna cosa i jo vaig posar-me el mocador, [...] Sembles un àngel, m'havia dit ella, segur que entraràs al cel directament, per la porta gran. [...] Els nostres ulls es van trobar i allà mateix ho vaig saber, que no hauria hagut de posar-me el mocador. [..] no sé com no vaig caure. Ell no deia res, però jo ja el sentia derrere meu i quan va dir para, para o encara serà pitjor, jo ja no sé si vaig córrer o em vaig aturar, però em recordo a terra, amorrada a la clavaguera i ell vinga donar-me puntades de peu. No recordo els cops, no recordo si em va picar a la cara, a l'estòmac. [...] I llavors vaig mirar tot al meu voltant i vaig veure els clients del bar de davant de casa amb la beguda a la mà que no deien res i els que passaven pel costat que no deien res i els que ens coneixien i no deien res i allò era estar sola." (p.228-229)
--estar sola
-- la indiferència de la gent davant "els\_les altres"

Hijab (Kübra bei der Friedrich Ebert Stiftung)
---------------------------------------------
frauen mit kopftuch wird die legitimation abgesprochen
- kann sie überhaupt sprechen? (nicht nur deutsch, sondern überhaupt)
- darf sie überhaupt etwas sagen?
- kann man ihr vertrauen?

Andere sprechen für sie? Was waren wohl ihre Beweggründe ein Kopftuch zu tragen?

Persönliche Erfahrungen ins Licht gestellt, hinterfragt, das Mensch-Sein dieser Person wird in Frage gestellt

https://www.youtube.com/watch?v=tcGaDPUSJL0


vlt als eine Art Schlusswort:
-- die Tochter ist die ganze Zeit, trotz dem ganzen shit, eine starke figur
"Jo no ho recordo gaire bé, però es veu que Déu em va il.luminar i vaig fer servir la meva veueta de nena per arreglar els problemes de tota aquella família. O potser tenia el moment de lucidesa més gran que he tingut mai a la vida. Sé que la frase que li vaig dir va ser aquesta, perquè se'n va parlar molt, d'allò. Vaig passar a formar part del recull de llegendes dels Driouch. Per què no deixes d'una vegada aquesta meuca cristiana i fas el favor d'encarregar-te de nosaltres? No et sembla que ja és l'hora que pensis en la teva família?" (p.164-165)

"Em van treure l'aparell de les mans per impertinent i es van escandalitzar, però aviat van estar tots molt orgullosos de mi. Molt." (p.165)

+ la capa de supermana
+ revenge am Ende

"Si jo sóc la mena de persona que sóc és només per culpa seva, jo era ben normal, abans. No és veritat, havia dit jo, i havia mirat la mare. Què? Que no, que no és veritat el que expliques, que tota la història aquesta te la vas inventar tu i si no de què estaries tan preocupat per una filla que no és ni teva? Però ella ho va confessar, va dir que va ser el teu oncle. Ella et va dir que havia estat ell per salvar la vida, per no deixar els seus tres fills sols amb un boig com tu, va ser això el que va passar. [...] però tu ho vas transformar tot per tenir una excusa per anar-te'n de putes, [...] per fer sempre tot el gue t'ha donat la gana." (p.317)
-- sie gibt ihrem vater die meinung
\end{comment}

