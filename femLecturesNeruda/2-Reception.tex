\section{Nerudas Liebeslyrik: etablierte Rezeption}

Die Liebeslyrik von Pablo Neruda wird durch eine (fast absolute) Mehrheit von Lesenden und Literaturkritiker*innen ausschließlich positiv rezipiert.
Prominente Akademiker*innen wie Ilan Stavans sprechen von einem ``poeta bisexual, hermafrodita, poesía erótica, accpetada y leída tanto por hombres como por mujeres'' (quelle??).

\begin{comment}
Stavans zitieren und polemisch einsteigen: "poeta bisexual, hermafrodita, poesía erótica, accpetada y leída tanto por hombres como por mujeres"
hat er das im Seminar bei uns gesagt?
http://america.aljazeera.com/articles/2014/6/20/the-unknown-neruda.html
"Likewise, the inspiration of some of the “Twenty Love Poems and a Song of Despair” (1924), including poems XV (“Me gustas cuando callas porque estás como ausente”) and XX (“Puedo escribir los versos más tristes esta noche”), seem — there is no better way to put it! — eternal."
Kann man machen
etablierte Rezepzionsperspektive
ist weit nicht der Einzige
\end{comment}
