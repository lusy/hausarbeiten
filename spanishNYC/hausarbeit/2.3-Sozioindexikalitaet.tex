%Style/Socioindexicality

%%%%%%%%%%%%%%%%%%%%%%%%%%%%%%%%%%%%%%%%%%%%%%%%%%%%%%%%%%%%%%%%%
%
% Not sure what to do with it
%

*style/socioindexicality: adopt linguistic forms of groups one sympathises with/has intense
relationships with
*``Although overt norms favor standard speech, powerful covert norms encourage
group members to remain faithful to group codes, linguistic and otherwise.'' (Zentella 2007)
** Dominicans keep their low prestige variety, because ``overcorrectness'' is associated with
``feminine''

Stichwörter:
Ideologie, Sprecher\_innenidentität

``Peter Trudgill
(1972), for instance, called upon the perceived toughness of working-class men
as a motive for middle-class men to adopt local working-class sound changes,
accounting for the upward spread of change''

``The
pejoration of many English words referring to females is a perfect example of the
systematic absorption of ideology into the lexicon.''

``(1) the term was used
repeatedly in negative utterances about specific women or categories of women;
and (2) the utterances of those who said such negative things were registered
disproportionately.''

``This very local construction of meaning in variation, the recruiting of
a vowel as part of a local ideological struggle, suggested that variation can be a
resource for the construction of meaning and an integral part of social change. But
this power of variation was lost in the large-scale survey studies of sound change
in progress in the years that followed, as social meaning came to be confused
with the demographic correlations that point to it.''

schemata of speaking (Piaget 1954): we notice differences and attribute meaning to them

not answered: is stylistic meaning compositional?

%`The connection between the sound of rhotacization and oiliness and
between oiliness and a specific persona is a particularly striking example
of iconization''

``At the same time,
this distancing process reinscribes the old types by creating a new space
in the social map in opposition to them. Meanwhile, yuppies’ adoption
of a non-Beijing feature, full tone, projects them out into transnational
space.''

``distinct styles associated with different communities of practice''
--> aber nach der Definition der Communities of Practice würde ich sagen, sind sie für mich eher unbrauchbar..


%%%%%%%%%%%%%%%%%%%%%%%%%%%%%%%%%%%%%%%%%%%%%%%%%%%%%%%%%%%%%%%
%
%Eingearbeitet
%

Kategorien: ``Ich räume alles auf''
Menschen sortieren Sachen in Kategorien, damit sie kognitiv damit umgehen können
Kleidungsstil: indexikalische Bedeutung -> vermittelt einen bestimmten Hintergrund (zB Sakko->Prof)

``Sociolinguists generally think of styles as different ways of saying the same thing.''
``style is not a surface manifestation, but originates in content.''
``Different ways of saying things are intended to signal different ways of being,
which includes different potential things to say.''

%Sozioindexikalität
%1st order
gegeben: Gruppenzuweisung
nicht durch die Sprecher\_innen selbst bestimmt
``association by social actors of a linguistic form/variety with some meaningful social group (female, spanish, working class..)'' %reference???
-> Manche features eignet man sich an oder gewöhnt sich ab aus ideologischen Gründen
-> unklar wo das zitat herkommt
%1st order
``A first-order index simply indexes membership in a population''


%2nd order
describing, noticing.. of the 1st order indexicality;
variaties are differently notices, rationalized, ... from land to land and community to community
ideological analysis: social categories are locally created by social actors and discoverable by analysis rather than given; %auch quelle unklar
wie sich Menschen projizieren, durch Stil selber kreieren, Performance, bewusster, intenationaler Einsatz/Gebrauch

Kurios: Theaterspieler\_innen, Schriftsteller\_innen führen Figuren durch sekundäre Indexikalität ein -> die werden mit Attributen ausgestattet

früher: alle möglichen Kategorien erstellt: black/white/old/young/hispanic/...
aber nicht berücksichtigt, was die Individuen machen, um sich selber darzustellen
Individuen nicht als Akteure berücksichtigt
%2nd and higher order
``But the social evaluation of a population
is always available to become associated with the index and to be internalized
in speakers’ own dialectal variability to index specific elements of character. 2 At
that point, the linguistic form becomes a marker, a second-order index,''

``Participation in discourse involves a continual
interpretation of forms in context, an in-the-moment assigning of indexical
values to linguistic forms.''

``an nth order usage, is always available for reinterpretation – for the
acquisition of an n + 1st value.''

``fluid and ever-changing ideological field.''
``The emergence of an n + 1st indexical value is the result of an ideological move,''


%indexical field - definition
``a constellation of meanings that are ideologically linked'' [Eckert08]
``an embodiment of ideology in linguistic form'' [Eckert08]
``not a static structure'' [Eckert08]

``Variables have indexical fields rather than fixed meanings because speakers use
variables not simply to reflect or reassert their particular pre-ordained place
on the social map but to make ideological moves.''
``The use of a variable is not
simply an invocation of a pre-existing indexical value but an indexical claim
which may either invoke a pre-existing value or stake a claim to a new value.''

und dann kommt Wertung dazu:
``a negative evaluation of a speaker using
the apical variant might be that the speaker is inarticulate or lazy, a favorable
evaluation might be that he or she is unpretentious or easygoing''
--> interpretation depends on:
* perspective of the hearer
* style in which it is embedded

``Since the same variable will be used to make ideological moves by different
people, in different situations, and to different purposes, its meaning in practice
will not be uniform across the population.''

field of potential meanings -> indexical field

``Thus variation constitutes an indexical system
that embeds ideology in language and that is in turn part and parcel of the
construction of ideology.''

``the variation (and the entire linguistic) enterprise must
be integrated into a more comprehensive understanding of language as social
practice''

Objective of the paper:
``propose an approach to the study of social meaning in variation that builds
upon linguistic-anthropological theories of indexicality, and most particularly
Michael Silverstein’s (2003) notion of indexical order.''
``the meanings
of variables are not precise or fixed but rather constitute a field of potential
meanings – an indexical field, or constellation of ideologically related meanings,''
``The field is fluid, and each new activation has the potential to change the field by
building on ideological connections''



``Thus variation constitutes an indexical system
that embeds ideology in language and that is in turn part and parcel of the
construction of ideology.''

``the variation (and the entire linguistic) enterprise must
be integrated into a more comprehensive understanding of language as social
practice''

agency and ideology: Why do speakers do what they do?

``Speakers’ agency in the use of variables has
been viewed as limited to making claims about their place in social space by
either emphasizing or downplaying their category membership through the
quantitative manipulation of markers.''
--> The old view: speakers' agency is ignored

``This generalization says nothing about the kinds of behaviors and ideologies that
underlie these patterns, what kinds of meaning people attach to the conservative
and innovative variant, who does and does not fit the pattern and why.''

``variables index demographic categories not directly but indirectly (Silverstein
1985), through their association with qualities and stances that enter into the
construction of categories.''

%Stil
``Style has a similar function in everyday language,
picking out locations in the social landscape such as Valley girls, cholos,''

variables: components of styles

%Persona style
``at this level that we connect linguistic styles with other stylistic systems
such as clothing and other commoditized signs and with the kinds of ideological
constructions that speakers share and interpret''

``Ideology is at the center of stylistic practice:''
``every stylistic move is the result of an interpretation of the social
world and of the meanings of elements within it''

``styles and stylistic moves can be quite local, ultimately they connect the
linguistic sign systematically to the political economy and more specifically to
the demographic categories''

``By stylistic practice, I mean both the interpretation and the production of
styles, for the two take place constantly and iteratively''

``the noticing of the style and the noticing of the group or individual that
uses it are mutually reinforcing, and the meaning of the style and its users
are reciprocal.''

``an ideological link is constructed between the linguistic and the social.''

``This process of selection is made
against a background of previous experience of styles and features; a stylistic
agent may be more attuned to particular kinds of differences''

``Once the agent
isolates and attributes significance to a feature, that feature becomes a resource
that he or she can incorporate or not into his or her own style.''
``The occurrence
of that resource in a new style will change the meaning both of the resource and
of the original style''


%%%%%%%%%%%%%%%%%%%%%%%%%%%%%%%%%%%%%%%%%%%%%%%%%%%%%%%%%%%%%%%%%%%%%
%
% Noch nicht eingearbeitet
%

Stil (Eckert?)

[Eckert08]

``they segmented the new wave style into meaningful elements''
``these girls’ stylistic moves are local,
they deal with fundamental issues related to gender and adolescence: innocence
and independence.''

``making pegging available for segmentation and (re)interpretation.''
``the resources they were using
were available and salient because they had been established at a much more
generalized cultural level.''

``linguistic style is rarely constructed in as intentional a fashion
as clothing style, it is similarly a process of bricolage.''

``But just as women are not making direct gender claims when they use female-
led changes, burnouts are not making direct urban claims when they use
urban-led changes.''
``what they associated with urban life and urban kids.''
-> indirect connections between social categories and linguistic features

``The urban kids that they identified with were white kids who knew
how to cope in the dangerous urban environment – kids they saw as autonomous,
tough, and street-smart. Presumably in adopting urban forms, suburban kids
were affiliating with those qualities, not claiming to be urban.''

```‘acts of identity’ (Le Page and Tabouret-Keller 
1985) are not primarily a matter of claiming membership in this or that group
or category as opposed to another, but smaller acts that involve perceptions of
individuals or categories''

``systematically related to the macrosociologist’s categories
and embedded in the practices that produce and reproduce them. It is in the
links between the individual and the macrosociological category that we must
seek the social practices in which people fashion their ways of speaking, moving
their styles''


