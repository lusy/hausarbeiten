\section{Fazit}

\begin{comment}
    * Einleitung und Fazit müssen zusammenpassen.
    * sind die Erkenntnisse im Fazit aus der Arbeit ableitbar?
    * Vorgehensweise zusammenfassen
\end{comment}

Kritische Zusammenfassung
\begin{itemize}
    \item was war in den Texten nicht so gelungen?
    \item welche Fragen sind offen geblieben?
    \item in welche Richtung kann noch weiter geforsch werden?
    \item was sind gegnerische Meinungen zum Thema?
\end{itemize}

Es werden 2 Ansätze vorgestellt, was (Zweit)Spracherwerb angeht: sequentieller Spracherwerb und Erwerb eines Netz von Konstruktionen

Gegnerische Meinungen:

\cite{Cook93} - generative Perspektive auf Zweitspracherwerb
``A person who knows two languages has been through the acquisition process twice.
Second Language research must explain the means by which the mind can acquire more than one grammar.''

``One goal of second language research is to describe grammars of more than one language simultaneously existing in the same person.''

``The importance of second language research lise not in its account of the knowledge and acquisition of the L2 in isolation, but its account of the second language present and acquired in a mind that already knows a first [...]''



