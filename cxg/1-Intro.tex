\section{Intro}

\begin{comment}
* ca 3/4 Seiten
* Untersuchungsgegenstand
* Erkenntnisinteresse
* Forschungsstand
* Vorgehensweise: also Section 2 macht blabla, Section 3 blablup, ....
* Ergebnisse können/sollen angedeuten werden
\end{comment}

Idee:
\begin{itemize}
    \item Beschreiben des theoretischen Rahmens der Konstruktiongrammatik/Gebrauchsbasierten Perspektive/Kognitiven Linguistik/lala + Zweitspracherwerb;
    \item Beschreiben der historischen Entwicklung von cross linguistic Transfertheorien;
    \item Beschreiben der CLD/CLLD Studie Spanisch-Englisch
    \item Putting last 2 things in perspective bzgl des Rahmens vom 1.
\end{itemize}

Es gibt im Zweitspracherwerb Phänomene wie Transfer aus der Erstsprache (L1) und Übergeneralisierung.
Diese lassen sich in einem kognitiv linguistischen/konstruktionsgrammatischen theoretischen Rahmen erklären.
Beispielhaft wird das auf eine Studie der Contrast-Left-Dislocation (CLD) und Clitic-Left-Dislocation (CLLD) Konstruktionen im Spanischen und Englischen demonstriert.
