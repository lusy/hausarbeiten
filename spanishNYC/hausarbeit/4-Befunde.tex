section{Befunde}

% vlt anderer Titel, wenn man genau weiß, was die These ist
%\subsection{Hauptthese}
%Bzw vlt neue Struktur zum gesamten Kapitel überlegen

Die automatisch erkannten Gebrauche von Englischen Floskeln wurden per Hand angeguckt und in Kategorien unterteilt.
Insgesamt wurden in den 1419 Artikeln 3271 unique? Wörter identifiziert.
Das heißt wenn das selbe Wort in einem Artikel öfter vorkam, wurde es hierfür nur einmal gezählt.

\begin{itemize}
  \item Die Artikel: Eine Art Performance, die auf ein Zielpublikum zugeschnitten ist.
  \item Sucht eine bestimmte Wirkung (welche?)
  \item Was hat die Tatsache, dass es sich um Code-Switching in geschriebener Sprache handelt, für eine Bedeutung?
\end{itemize}

%------------------------------------------
\subsection{Ergebnisse beschreiben}
(vlt exemplarisch; wobei es macht evtl sinn, die alle mal zu erwähnen zumindest)

Kategorien:
\begin{itemize}
  \item Named Entities
  \item Kochrezepte/Essen
  \item ``look''

\end{itemize}



%--------------------------------------
\subsection{Interpretation}

\begin{itemize}
  \item English is the more prestigious language;
  \item the dominant (both in numbers and power) group of people in the USA speaks English
  \item serves to avoid misunderstandings/clarify (zb cooking recipes)
  \item es wird ein bestimmter Lebensstil kreiert/bedient
\end{itemize}

\begin{comment}
  %Language Shift
Language Shift (Spanish -> English) : Speech community of a language shifts to speaking another language (source? Thomason?)
Can we talk about language shift in this context?
Who is shifting? The editors of the magazine? Or do they induce language shift?
Should that be a focus at all? Or focus more on communities of practice/style projection through language/sozioindexikalität, etc.

Ein Shift Spanisch->Englisch ist nicht nur mit Prestige zu erklären, sondern auch mit allgemeinen Anpassungs/Integrationsprozessen:
Wer in den USA Englisch kann, hat bessere Chancen einen Job zu finden, etc. na, aber das hat wohl auch mit Prestige zu tun gewissermaßen

%Language Prestige
English is the more prestigious language;
the dominant(both in numbers and power) group of people in the USA speaks English
serves to avoid misunderstandings/clarify (zb cooking recipes)
--> das kommt alles zur Interpretation!


\end{comment}

