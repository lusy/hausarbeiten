\section{Llengua i traducció. Lectura llingüística}

El llibre està escrit en un llenguatge molt particular.
Tenim tota la història explicada des de la perspectiva de la filla i encara una instància narrativa auctorial.
L'estil directe al llarg del llibre no està marcat de manera gràfica en ningún lloc; el discurs s'assembla més a un corrent de consciència.

Com adverteix el fundador de les ciències postcolonials Edward Said en el seu llibre ``Culture and Imperialism'', avui dia cap persona pot ser definida amb una sola etiqueta
i identitats (atribuïdes o seleccionades per a una mateixa) com ``dona'', o ``musulmà'', o ``americà'' serveixen només per a una primera orientació~\autocite{Vidal2012}.
A més, aquestes categories són subjectives i existeixen sempre dins un narratiu particular que les ha creat, afirma la professora de ciències de traducció Mona Baker, citada en~\autocite{Vidal2012}.
Said planteja també la idea que tothom construeix la seva pròpia identitat cultural i ètnica.
Ja no tenim cultures i identitats ``pures'' i ``estables''~\autocite{Vidal2012}.
Però encara, amb la mateixa força, l'altre, el diferent, l'estrany està percebut com una amenaça per l'ordre establert i per això es rep amb por i prejuicis.

Un aspecte fundamental/fundador de la nostra identitat i de la nostra cultura (llevat de que entenem sota aquests dos termes) és/són la/les llengüa/es a través de les quals ens comuniquem.
A més a més, a través del llenguatge les cultures diferents es poden entendre, que segons Vidal Claramonte és crucial, ja que avui dia més que mai gran quantitats de persones no viuen en la seva llengua materna.~\autocite{Vidal2012}.
O, més específicament, viuen en més que una llengua i això produeix constantement conflictes i trastorns i fa necessari la reorganització de les esquemesque organitzan la seva vida~\autocite{Albin2005}.

Deleuze i Guattari, citats per Vidal Claramonte, posen una pregunta interessant:
``How many people today live in a language that is not their own? Or no longer, or not yet, even know their own and now poorly the major language that they are forced to serve?''~\autocite{Vidal2012}
i elaboren que aquest problema sigui molt propi dels i de les immigrants i els/les seves filles.

En el seu treball sobre l'obra de Najat El Hachmi María Carmen África Vidal Claramonte destaca la traducció com l'element bàsic de ``L'últim patriarca''.
Explica que la traducció té lloc a nivels multiples: podem llegir no només una traducció entre dues llengües, sinó també entre dues cultures i fins i tot entre dues generacions~\autocite{Vidal2012}.
Tenim per això exemples multiples:
la filla qui tradueix els preus dels aliments a la seva mare, no només canviant la llengüa, sinó també convertint les pesetes en duros;
la filla qui ajuda a la seva mare embarassada de parlar amb la ginecòloga:
\begin{quote}
``Hi havia coses que no sabia passar d'un idioma a l'altre, que no volia passar d'un idioma a l'altre. Contiuanva sense entendre per què tantes dones per tot el món m'explicaven coses d'aquelles a mi. Quina va ser l'última vegada que li va venir la regla a la mare? I jo ja sabia què era allò de la regla, però no n'havia parlat mai amb ella. La primera vegada que li va venir? [\ldots] La primera vegada que va tenir relacions sexuals? Déu, Déu, volia fugir corrents de tot allò, jo no les vull saber, totes aquestes coses, i encara menys traduir-les a un idioma on no hi havia cap paraula que jo conegués per a relacions sexuals que no fossin paraulotes [\ldots] au, va, pregunta-li-ho. La mare em mirava i deia què, què t'ha preguntat, i jo hauria volgut desaparèixer així, de cop, i que elles mateixes s'entenguessin. No podia dir follar, no [\ldots] Vaig intentar de buscar un eufemisme. Quants anys tenies quan vas dormir amb el pare per primera vegada? I no la vaig mirar als ulls mentre li ho deia; ella va dir, també ben de pressa, ens vam casar que jo tenia divuit anys. Això és tot.''~\autocite[217]{ElHachmi2008}
\end{quote}
En aquest fragment la traducció multiple és especialment interessant.
Tenim les dificultats de la filla de trobar paraules adequades per relacions sexuals en una llengua, en la qual encara existeix el tabu de parlar de manera lliure sobre aquestes coses,
i en la qual ningú mai li ha parlat d'això d'una manera que no sigui vulgar.
Tenim també la seva assumpció ``natural'' qua el pare sigui el primer home amb qui la seva mare n'ha tingut--una traducció de la pregunta de la metgessa.
I tenim encara la resposta de la mare, que és una nova traducció: ``ens vam casar que jo tenia divuit anys''~\autocite[217]{ElHachmi2008}--traducció generacional, perquè ``casar-se'' substitueix ``dormir amb el pare'' i traducció llingüística, ja que la filla passa la frase a català~\autocite{Vidal2012}.

Un objecte important per a la filla, al qual es fa referència al llarg del llibre, és el diccionari de la llengua catalana.
El diccionari i, en conseqüència, la llengua, són un refugi per a la noia de les contradiccions de la vida quotidiana.
Les consultacions amb el/les lectures del diccionari serveixen també com un fil que ens ajuda seguir la història.

Hem de recordar el valor simbólic dels diccionaris d'una llengua.
El diccionari és el primer llibre d'una nació, cita Verónica Albin en una entrevista amb l'intelectual Ilan Stavans~\autocite{Albin2005}.
També, és important notar que el mític objecte ``el diccionari'' no existeix.
Cada llengua disposa de diccionaris multiples, cada un d'aquests té lxs sevxs redactorxs, persones amb ideologies i circunstàncies concretes que naturalment influeixen les seves obres. [quote!]
Com ho formula Henri Meschonnic, citat per Albin, els diccionaris reflecteixen ``les conflits, les masquages des
conflits, les clichés qui font l'album de famille d'une culture''~\autocite{Albin2005}.
Stavans es posa d'acord amb aquesta noció i subratlla que encara que siguin xenòfobs, els diccionaris serveixen per definir l'univers d'una persona,
perquè ``The limits of our language are the limits of our worldview.''~\autocite{Albin2005}.

%naming things
En una entrevista Najat El Hachmi aborda un aspecte clau del llenguatge--la possibilitat d'anomenar persones, emocions, esdeveniments i relacions al voltant nostre.
Clau, perquè si no podem parlar dels fets i els problemes que ens envolten de manera adequada, tampoc podem elaborar solucions.
En particular, l'autora es refereix aquí a la violència extrema que comet el personatge de Mimoun.
El Hachmi assenyala que si no parlem de la violència, ens convertim en seus còmplices.
Per tant, destaca ella, les lectures del diccionari per part de la filla són el primer pas cap a estant en control de la situació, cap a un alliberament.
D'altra banda, la literatura és una manera d'ampliar els propis horitzons, d'experimentar vides diferents, de descobrir que altres maneres de viure són possibles~\autocite{HaAM2011}.

\begin{comment}
  Quotes (moved)
  ---------------
\end{comment}

\begin{comment}
\subsubsection{Quotes}
%  ------
"L'àvia havia pres sang d'eriçó, s'havia banyat amb aigua on havia diluït l'esperma del seu home; s'havia fet fumejar l'entrecuix amb la barreja que cremava al foc, feta de sofre, gallarets esmicolats i excrements de colom secs.
  Tots els remeis que les àvies de l'època li havien recomanat." (p.12)
-- exotisme? aber so bisschen ins laecherliche gezogen; vlt sich lustig machen ueber rueckstaendigkeit? vlt ist es auch gar nicht so gemeint, kann man aber so lesen; però per altra banda no parla "d'un altre lloc", sino de "l'època".. doncs la història sóna molt propera...

"Sobretot quan el pare va tornar a parlar i a nosaltres ens deia allò de digueu a la puta de la vostra mare que... digueu a la porca de la vostra mare que... digueu a aquella gossa que... Nosaltres només li deiem mama ell diu que... Allà vam començar a fer de traductors." (p. 177)
-- el concepte del traductor apareix multiples vegades també; només d'una llengua a l'altra, sinò entre cultures.

"Per escapar del <texit>poltergeist</>, si no tens na senyora cridanera i baixeta com Tangina Barronsm has de riure molt, fins a sentir que tens les costelles a punt de petar, o has de plorar molt, fins a sentir que t'has buidat, o has de tenir un orgasme, que de fet i fet, també és buidar-se. Jo encara no en sabia, de tenir orgasmes, al pare no li agradava que ningú plorés i a la mare no li agradava que ningú rigués. De manera que vaig començar a llegir, paraula per paraula, aquell diccionari de la llengua catalana." (p.181)
--això del poltergeist es repeteix també
--el diccionari te un paper central; roter Faden; paper de la llengua??

"No devien entendre gaire res i les filles del veí de davant deien a la mare, au, fes-lo fora de casa, si vols, nosaltres t'ajudarem. Ella no les entenia, somreia i deia sí, sí, però elles ja devien veure que no en traurien l'aigua clara. Llavors es psssaven els dits entre els cabells de permament i em feien servir de traductora. Em feien sentir frases que jo no volia sentir i em feien dir coses que jo no volia dir. Què dieun?, deia la mare, què diu?, deien elles. Jo hauria cridat res, res, res, calleu totes si no us enteneu." (p.196)

"Llàgrimes sense soroll, i jo vaig acabar a traduir: vés-te'n  no tornis mai més. I vaig interpretar una mica el paper." (p.223-224)
-- traductora

"A la mare no li deixaven fer res i ella es vestia amb caftans bonics, que així era com havien de vestir les dones dels homes importants i rics. Perquè es veu que érem rics, nosaltres. A mi tot allò em fer gràcia al principi, però aviat ja vaig tornar a tenir ganes d'anar a casa." (p.249)

"Si anava o no a l'institut depenia de tants factors i cap d'ells tenia a veure amb si jo em portava bé o no, o amb si jo treia bones notes o no, o amb si jo feia cas o no. Estranyes desaparicions havien tingut lloc en els últims dos anys a l'escola i encara gràcies que no m'havia tocat a mi. Desaparicions de noies com jo que venien d'un lloc semblant al lloc on jo vaig néixer però que potser eren molt diferents de mi o devien tenir molta menys sort que jo. Noies que ara tenen tres o quatre fills" (p.271)

"A mi em tocava desaparèixer de l'escenari escolar i encara no sé com no va passar. Un factor era l'avi, que era l'únic que em preguntava què, com han anat els exàmens, ho has aprovat tot?" (p.271)
"La mare ja m'ho havia dit, el teu pare diu que aquest és l'últim any que vas a escola, i era una cantarella que es repetia cada final de curs. Aquest ñes l'últim, i jo deia val, però sabia que no seria així. Potser l'altre factor va ser la professora massa amiga del pare, que en alguna cosa devia influir-lo, que deia la teva filla ha d'estudiar una carrera" (p.271--272)

\end{comment}

\begin{comment}
\subsubsection{Comments}
[Albin2005]
On Dictionaries

"VA: In the early 19 th century, Constantin François de Chasseboeuf, Comte de Volney, said that the
first book of a nation is a dictionary of its language,"

prescriptivism vs descriptivism

"VA: Regarding "love," you mentioned in Dictionary Days that Acadians, Caldeans, Phoenicians,
Sumerians, Babylonians, Egyptians, Normans, Toltecs, Vikings, and Quechuas didn't have a word for
it. Knowing that images are an important part of how you see the world, what would you have done
had you been born speaking Latin, that according to linguists doesn't have a lexeme for gray or
brown, or born to that of the Dani of New Guinea, whose only color words are for black and white, or
speaking a 4-color language like Hanunóo that has words only for black, white, green and red?

"VA: Living in two or more cultures, two or more languages, produces some rifts and upheavals; it
requires a constant rearranging of schemata."

"IS: Furthermore, translation always involves wonderment and surprise: what is the speaker really saying?
Is there a way to convey the message in my own language? Is it possible to avoid becoming a
falsifier? The answer to the last question, obviously, is no. Every translation is a misrepresentation."

-----------------
[Vidal2012]

"Fins que la mare es va cansar de tot allò i va dir, aquest any, les notes les vaig a
buscar jo, fins i tot les teves... Jo feia de traductora, com sempre. La mare deia
digues-li que és una mala puta i que deixi estar el meu marit d’una vegada, i jo
somreia i deia la mare diu que com que és ella qui passa tant de temps amb els
fills, que és millor que sigui qui et vingui a buscar les notes i, a més, que ja tenia
moltes ganes de conèixer-te. Doncs jo preferiria parlar amb el teu pare directa-
ment, que és una mica estrany que tu tradueixis l’informe a la teva mare, no et
sembla? Ja t’agradaria, ja, que hagués vingut ell, deia la mare sense haver esperat
la meva traducció, malparida, no et molestes ni a dissimular-ho. Diu que el pare té
molta feina i no li anava bé de venir, però que ella ja es refia de mi. Notable,
excel·lent, notable, excel·lent, mostra interès, tot allò no tenia traducció i jo deia
res, que diu que tot ha anat bé. Només un bé de gimnàstica i li aniria bé de fer
alguna activitat fora de l’escola, sobretot anglès, que aquí no en fem i ella té faci-
litat. La mare va dir val, val, i volia dir que ni pensar-ho només perquè era ella qui
ho havia proposat. (El Hachmi 2008: 263)"
"De nuevo, la vida cotidiana como traducción. La madre es aquí doblemente
subalterna: con respecto al esposo y fuera del espacio privado, en el público, por
ser diferente."
"Gayatri Spivak publi-
có en 1988 en el que denuncia que la subalterna, no es que no pueda hablar, sino
que no se la escucha, que su discurso no está validado por las instituciones."

quote Spivak in Segarra
"cualquier intento de ayudar a los subalternos tropieza con problemas éticos impo-
sibles de soslayar: la tendencia a considerarlos como una masa homogénea, en
lugar de fijarse en su singularidad heterogénea y, en especial, la intención benevo-
lente de querer hablar por ellos, lo cual significa un acto de apropiación, y no con
ellos. (Segarra 2006)"
Segarra, Marta (2006). «Más allá del poscolonialismo. Contra la subalternidad», La Van-
guardia, 1-3-2006.

«All speaking, even seemingly the most
immediate, entails a distanced decipherment by another, which is, at best,
an interception.» (Spivak 1999: 309).

"El lenguaje se utiliza en este
ejemplo como un instrumento de poder que ya no se basa en un ideal armónico
de comunidad en el que prevalece el principio de cooperación."


\end{comment}

