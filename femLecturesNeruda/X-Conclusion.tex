\section{Fazit}

  ** Adrienne Rich: "the word "love" in itself is in need of re-vision"
  -> bell hooks: we need a definition of love.
  love is....
  Und wenn wir die Definition zugrunde legen ist es direkt sichtbar, dass es sich bei den Gedichten nicht um diese Liebe handelt.
  Dafür bräuchte man eine vollwertige Person.
% Umgangsstrategie: Re-writing?
% subversive Strategien, um Probleme/Absurditäten, die sich in der Gesellschaft verfestigt haben aufzuzeigen
% diversion bzw diversifizierung: für andere Bilder, Diskurse, Modelle sorgen
\begin{comment}
    [Kolodny1980]

"What makes it so exciting, of course, is
that it can be constantly relearned and refined, so as to provide
either an individual or an entire reading community, over time,
with infinite variations of the same text. It can provide that, but,
I must add, too often it does not"
Quote Komilitonin mit: "universell menschliches fleischliches Begehren" -> und argue that we need critical feminist thinking education in order to shift perspectives in such readings.
Fast der ganze Kurs mit Ausnahme der Professorin und ein paar Studentinnen, die über das Durschnittsalter von 22 waren, scheinten mit der Leseweise kein Problem zu haben.
Ich frage mich was die selbe Person heute darüber sagen würde, wenn sie die Gedichte nochmal liest.

\end{comment}

\begin{comment}
    [Duncan1992]
"As John Felstiner observes, it is difficult to attack ``the sexism intrinsic to Neruda's view of man and woman,'' because he is ``a writer who produced so much genuinly compelling verse; a writer whose long career spoke for so many in Latin America, and whose death seemed to follow inevitably on the tragic seizure of Chile's government in September 1973'' (91)."

Vlt move this to 4
"Neruda's women possess the qualities we have been taught to covet: they are beautiful, sensual, desirable, and eminently agreeable."
"In short, they strike us as being ``real'', rathre than the carefully chosen literary constructs they are."
"They want what all women have been conditione to want in the way of self-fulfillment: romantic love"
-> vgl Laurie Penny; Liv Stromquist
"They may have no identity, no voice, no sense of puprose, but, in exchange, they are promised the reward of man's eternal devotion if they agree to play their role properly."

"The message to female readers is clear: woman bears the responsibility of atttacting, nurturing, and keeping man's love alive. He is not obliged to love her and, as he often reminds her, he can fare better without her than she can without him. Without love, he is still a man, but she, without love, is nothing."
"Neruda's women are familiar images, women ``as they should be''. His men are ``real men''. Together, they function to reassure us that everything is as it has always been, and that all is well with the world. But, is it? Are men and women destined to repeat, generation after generation, the same patterns? Are they eternally locked in their gender-specific roles?"
    --> Vlt als Fazit: beauty ins not in subjugation of women; learn to view beauty (and love, vgl bell hooks) in new/different ways;
--> learn to question the familiar

Molly Haskell:
"the idea of woman's inferiority, a lie so deeply ingrained in our social behavior that merely to recognize it is to risk unraveling the entire fabric of civilization"

  [Rich1972]
  ``We need to know the writing of the past, and know it differently than we have ever known it; not to pass on
  a tradition but to break its hold over us.''

  [Kolodny1980]

"What unites and repeatedly invigorates feminist literary criti-
cism, then, is neither dogma nor method but, as I have indicated
earlier, an acute and impassioned attentiveness to the ways in
which primarily male structures of power are inscribed (or en-
coded) within our literary inheritance; the consequences of that
encoding for women-as characters, as readers, and as writers;
and, with that, a shared analytic concern for the implications of
that encoding not only for a better understanding of the past, but
also for an improved reordering of the present and future as well."
-> Das ist meine Perspektive für die "feministischen Lektüren"
\end{comment}
\begin{comment}
1. Intro
  * Ziel von Feministischen Lektüren: zugrunde liegende Machtstrukturen in Werken und deren Rezeption aufzudecken
    ** androzentrische Perspektive der Literatur:
       *** Männer in Mittelpunkt (als Figuren)
       *** von Männern gemacht
       *** an Männer gerichtet
  * Wie erreicht? Durch eine Pluralität der Lektüren und Close Reading
ohne das ouevre Nerudas nicht als ganzes in Frage stellen
exemplarische Lektüren: Poetisierung bestimmter Heteronormativen Perspektiven
gehört historisiert; nicht als zeitlos darzustellen

* durch Sprache (und Literatur dann) bestimmte identitäre Zuschreibung geprägt: Sprache formt unser Gehirn, Gedanken, Weltanschauung, Denkweise, Kathegorien --> im Kopf behalten wenn man mit "Lektüren" umgeht

Worüber geschrieben wird und welche Schrifte überliefert werden, prägt auch welche Themen als wichtig angesehen werden.

[Kolodny1980]
"What was at stake was not so much literature
or criticism as such, but the historical, social, and ethical conse-
quences of women's participation in, or exclusion from, either
enterprise"
-> use in conclusion?
\end{comment}
