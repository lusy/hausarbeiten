\section{El proyecto Altermundi}

\subsection{redes comunitarias}
\begin{comment}
* infraestructura de comunicación decentralizada creada y mantenida por la comunidad de lxs usuarixs
* garantizar acceso libre a información
* garantizar la libertad de expresión
\end{comment}

\subsection{Altermundi: fundación, objetivo, estructura, desarrollo histórico}
\begin{comment}
* existe en su forma actual desde 2012
* objetivo: facilitar la construcción y el mantenimiento de redes comunitarias inalámbricas
* grupo básico de activistas + redes locales autónomas
* educar y concienciar: talleres, empoderamiento, el grupo central en papel consultativo
* la comunidad decide
\end{comment}


\subsection{Análisis}
* community network projects: are to be found in the context of (2nd wave) cyber optimism: "decetralization", "distributed control", "self-governance", "non-hierarchical organization"
an attempt at a capillary revolution -> ist es erfolgreich?
\begin{comment}
> framing the Internet alternately as lawless, anarchic,
> free, “a world where anyone, anywhere may express his or her beliefs, no matter how singular,
> without fear of being coerced into silence or conformity” (Barlow 1996) (p.1)

* la infraestructura está prerequisito para participación

## Conclusión del texto [Rieder2012]

> If technology won’t deliver us from the conundrums of
> governance, negotiation, and struggle, we
> may as well reengage politics proper[ly].

* sobre todo problemas sociales/políticos: organización de grupos (no tanto técnicos): ¿cómo decidimos como grupo? ¿quién hace qué? ¿quién está responsable?
* soluciones políticas para problemas políticos

\end{comment}

