\section{Introducció}

\epigraph{``Every translation is a misrepresentation.''}{Ilan Stavans in~\autocite{Albin2005}}
\epigraph{``'Patriarchy' does not mean 'the rule of men'. It means 'the rule of fathers'--literally, the rule of powerful heads of household over everybody else in society. Men further down the social chain were expected to be content with having power over women in order to make up for their lack of control over the rest of their lives.''}{Laurie Penny~\autocite[69]{Penny2014}}

Avui dia a la gent li agrada pensar que vivim a un món postfeminista.
Les dones podem treballar (en paper) en totes les professions, hem aconseguit el vot i en molts països del món la violació dins del matrimoni ja està prohibida per la llei (encara que no des de fa gaire temps).
Doncs, la lluita feminista s'ha acabat.
Què més volem?
Ja ho hem atès tot, ho podem tenir tot.
Sóm tots i totes iguals.

Però, és realment així?
Estadistiques mostren una diferència gran en el pagament de dones i homes amb les mateixes qualificaccions a les mateixes posicions laborals~\autocite{EU2014},~\autocite{MDBGH2012}.
L'abortament i, amb ell, la capacitat de la dona de decidir sobre el seu propi cos, sobre la seva vida sexual i social està encara prohibit (i castigat) a molts països~\autocite{UN2013}.
Moltes dones i persones que no es defineixen dins de les normes (xifres?) per tot el món encara experimenten una objectificació sexual i violencia de gènere.
Resumint, les dones encara estan sotmeses a estàndards dobles i normes ètiques, socials i religioses absurdes.

I si hi afegim les opressions racistes i classistes que sufreix una gran part de la població feminina mundial, l'imatge es torna encara pitjor.

Llavors, no, aparentament no sóm tots i totes iguals i encara tenim un camí llarg davant nostre.
Precisament per això és essencial parlar d'aquests assumptes, indicar injustícies i estàndards dobles, discutir-los i organitzar-se.
I just part d'aquest debat forma també el llibre ``L'últim patriarca'' de l'escriptora catalana Najat El Hachmi.

``L'últim patriarca'' és la història d'una família marroqiuna entre Marroc i Catalunya.
És la història d'un pare i marit extremadament abusiu, qui mai de la vida ha posat els interessos i el benestar de cap altra persona davant dels seus.
És la història de la seva esposa, una dona excepcional segons les normes marroquines, qui pateix en silenci les escapades del seu marit i mai troba el seu lloc a la societat catalana.
I, sobre tot, ``L'últim patriarca'' és la història de la filla, qui, sobre tot, només vol poder ser ella mateixa.

El llibre posa preguntes difícils i importants com:
Qui és català? Qui decideix?
Com s'ha de comportar una dona entre dues cultures i dues societats?
Quin paper té la llengua per la formació de la identitat?

Al seu assaig ``Dancing through the minefield'' la crítica literària feminista Annette Kolodny defensa/advoca per a una pluralitat de les lectures~\autocite{Kolodny1980}.
Partint d'aquesta idea, intentarem a analitzar l'obra d'El Hachmi des d'una multitud de punts de vista diferents.

\subsection{Estructura de la monografia}
Aquest treball està organitzat de la següent manera.
Primer, presentarem el marc teòric dins el qual volem analitzar el roman ``L'últim patriarca''.
A continuació, discutirem multiples facetes de l'obra i proposarem lectures diferents.
En conclusió, ...

\begin{comment}
Fragestellung:
??
verschiedene readings anbieten:
* interseccionalitat
* translation/language
* laurie penny

--> adrienne rich/anette kolodny in die intro nehmen
Annette Kolodny: "In my view, our purpose is not and should not be the formulation of
any single reading method or potentially procrustean set of critical procedures[...] Instead, as I see it, our
task is to initiate nothing less than a playful pluralism, responsice to the possibilities of multiple critical
schools and methods, but captive of none.."

Leitfragen/themen:
* Qui és català? Qui decideix?
* La dona entre dues cultures i dues societats: expectacions, sentiments,... | interseccionalitat
* La dona en una societat machista                                           |
* coming-of-age
* el paper de la llengua

------

Motto candidates:

“The greatest masterpiece in literature is only a dictionary out of order.”
― Jean Cocteau
VA: Jean Cocteau once quipped that even the greatest masterpieces of literature are nothing but a
dictionary out of order.
(Stavans: Dictionaries Interview)

"IS: Every translation is a misrepresentation."
(Stavans: Dictionaries Interview)
\end{comment}
