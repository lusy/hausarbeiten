\section{Tu risa, Poema XV und Soneto XX in Re-vision}

\begin{comment}
4. Exemplarische Re-Vision einzelner Gedichte
   La risa, Soneto XX, Poema XV
   * bestimmte Frauen- (und Männer-)bilder werden von den Gedichten konstruiert --> einseitig, problematisch, Macht
     ** abgesprochene Agenz; Frauen sind stumm, ohne eigenes Wesen; Ansammlung von (erotisierten) Körperteilen
   * Heteronormativität, Monogamie (2er-Beziehungen) als Teilungsstrategie des Kapitalismus
   * Das "Maskuline" wird als universell dargestellt
   * Duncan weist darauf hin, dass Kritiker Nerudas Liebeslyrik oftmals nur vom Standpunkt des männlichen Sprechers im Text aus betrachten, anstatt die Position der Geliebten in Betracht zu ziehen. Daraus resultiert eine einseitige und repetitive Verbreitung aus männlicher Perspektive, welche schon in den Gedichten selbst zum Vorschein kommt. Die weibliche Perspektive wurde bis auf wenige Ausnahmen systematisch ignoriert.
   * Duncan erklärt weiterhin, dass unter dem Vorwand der Lobpreisung des Frauseins und dessen Einzigartigkeit, diese Einschätzungen über Nerudas Liebeslyrik textuelle Widersprüche vorenthalten, welche symptomatisch für die Unterdrückung der Frau in einer patriarchalen Kultur seien. Nur dadurch, dass die Frau zu einem Liebesobjekt wird, erwacht sie zu Leben. Ohne ihren männlichen Liebhaber ist sie “vacía, sin substancia“. Diese Darstellung der Frau in den Texten kontrastiert drastisch neben der des männlichen Sprechers, wessen Existenz nicht von der physischen Präsenz oder der weiblichen Liebe abhängig ist. Ohne ihre Liebe mag er sich verloren fühlen, aber seine Identität gerät nie in eine Krise, er sagt, ohne zu zögern, “todos saben quién soy“.
   * auch wenn es schwierig ist, das lyr. Ich und du genau zu erfassen, bzw. als männlich oder weiblich zu kennzeichnen, allein aufgrund der Sprache, haben die Lesenden einen bestimmten Erwartungshorizont (auch durch Einfluss eines biografischen Lesens):
    das lyr. Ich: männlich
    das lyr. Du: weiblich
    -> es werden gewisse Klischee-Bilder für Weiblichkeit und Männlichkeit evoziert; einseitig, der Komplexität der Realität nicht gerecht
   * problematische Frauenbilder: schwach, still, kein gleichwertiges Gegenüber
   * Mann: besitzergreifend, oberflächlich (nur an Körper interessiert)

   * selbst wenn man in die Gedichte den polit. Diskurs der Zeit reininterpretiert und das lyr. Du vlt als die Heimat liest, trotzdem problematisch: warum wird dieser Diskurs mit sexualisierten Bildern weiblichen Körpers ausgetragen?

   Soneto XX
      lyr. Ich:
      - kann rein grammatikalisch nicht als männliches lyr. Ich identifiziert werden
      - Urteilt - Cynthia Duncan: "the significance of the woman's body depends on the male speaker's perception of her: she is alternately "mi fea" and "mi bella" depending on his mood"
      - es geht um seine Interpretation und Wünsche (Wiederholung des Possessivs "mi")
      lyr. Du:
      - weiblich ("fea", "bella")
      - sehr körperlich dargestellt, Fetischisierung des weibl. Körpers
      - das Wesen fehlt
      - in Einzelteile zerlegter Körper ("he contado tu cuerpo")
      - Vergleiche weibl. Körper - Natur

      Zeile 12-14: politische Lektüre/Aussage/Diskurs über sexualisierte Bilder

    Poema 15
      lyr. Du:
      * weiblich ("llena", "dolorosa")
      * stumm, Agens abgesprochen
      * abwesend
      * leblos
      * es werden vom lyr. Ich Eigenschaften reinprojiziert

      lyr. Ich:
      * träumerisch?
      * aber auch über das Du bestimmend

      * Frau in die Opferrolle gesteckt, passiv, Gefäß, welches man füllen kann (=Projektionsfläche)

    Los versos del capitán
      * Duncan dazu: In dem Gedichtband Los versos del capitán müht sich der männliche Sprecher wieder in ähnlicher Weise mit ihrer Stille ab. Laut Duncan fühlt sich der männliche Sprecher außer Stande ihr Schweigen oder ihre Andersartigkeit zu durchdringen und sieht sich deshalb gezwungen, ihren Körper mit Gewalt zu durchdringen, um seine Kontrolle über sie geltend zu machen, und somit auch ihre Sexualität, welche für ihn eine Bedrohung birgt, abzuwehren.

      * Dies lässt sich besonders deutlich in der Sektion “Deseo“ in Los versos del capitán ablesen.
      * Othering

    La risa
      La risa (Los versos del capitán) -> die Welt bleibt vor der Tür stehen; Das Private ist *nicht* politisch
      das schöne Leben zu hause als Ausgleich für den Scheiß, der auf der Welt passiert


brigitte vasallo: El pensamiento monógamo genera identidades cerradas que operan con violencia
* el capitalismo nos está dividiendo
* en casa/nuestras relaciones: espacio de seguredad
* la policía del género, de la monogamia la llevamos a dentro
  ** la verguenza cuando nos enamoramos de dos personas al mismo tiempo

* Male gaze
  ** das ist auch was die Menschen von der Kunst und Gesellschaft im Allgemeinen kennen:
     dass Kunst aus männlicher Perspektive dargestellt und rezipiert wird; dass Frauen erotisiert werden
  ** In Nerudas Gedichten hat die Stimme des männlichen Sprechers Autorität, da dessen Standpunkt/Betrachtung selten oder gar nicht in Frage gestellt wird, weil seine Stimme als die einzig mögliche Perspektive dargestellt wird.
  ** Filmtheorie: Übereinstimmung von 3 Blicken: Kamera, (männl.) Protagonist und Zuschauer (vgl Laura Mulvey)
     -> Objektifizierung der Frau, Frau als passive Rezipientin sexueller Aufmerksamkeit
     -> Verfestigung patriarchalischen Normen

* "als nah, natürlich wahrgenommen", die Leser*innen können sich damit identifizieren
  ** Menschen können sich direkt mit den Figuren identifizieren, dadurch dass es nicht extradiagetisch ist
  ** Duncan spricht von "real", "natural/beautiful expression of male/female relationships", die Menschen können sich damit identifizieren, sie sehen die Gedichte als Reflexion der gesellschaftlichen Idealen, die sie gelernt haben anzustreben
  ** Duncan:  "a sort of manual on the workings of modern romantic love and have come  to be regarded by many, as a standard agains which real life  relationships can be judged"
  ** Duncan: ".. in fact it seems extraordinarily "real" and "natural", as if it were not a literary construct at all, but, rather a reflection of real-life experiences. Readers who turn to these poems to learn "what love is supposed to be like", "what men are like", and "what women are like", ultimately receive a skewed message told from the traditional dominant male position"

* was für romantische Beziehungskonzepte werden in unserer Gesellschaft akzeptiert (Laurie Penny) (die romantische Zweierbeziehung als einziges gesellschaftlich anerkanntes Model + deren Funktion als Ausgleich für strukturell-polit. Probleme/Missstände)
  ** hoher Stellenwert der 2er heterosexuellen Beziehung
  ** für Frau: die ultimate (einzig mögliche?) persönliche Erfüllung?
  ** Duncan: "the position of the woman in the texts is an enviable one"
  ** Duncan: "Neruda's women  possess the qualities we have been taught to covet: they are beautiful,  sensual, desirable, and eminently agreeable.[..] They want what all  women have been conditioned to want in the way of self-fulfillment:  romantic love. They may have no identity, no voice, no sense of purpose,  but, in exchange, they are promised the reward of man's eternal  devotion if they agree to play their role properly. [...] The message to  female readers is clear: woman bears the responsibility of attracting,  nurturing, and keeping man's love alive. He is not obliged to love her  and, as he often reminds her, he can fare better without her than she  can without him. Without love, he is still a man, but she, without love,  is nothing."
  ** Adrienne Rich: "the word "love" in itself is in need of re-vision"
  ** Adrienne Rich: "the myth of the special woman" and its destructiveness: "We seem to be special women here, we have liked to think of ourselves as special, and we have known that men would tolerate, even romanticize us as special, as long as our words and actions didn't threaten their privilege of tolerating or rejecting us according to their ideas of what a special woman ought to be [..] how divisive and ultimately destructive is this myth of the special woman"
  ** Molly Haskell (in Duncan's text): "the idea of woman's inferiority, a lie so deeply ingrained in our social behavior that merely to recognize it is to risk unraveling the entire fabric of civilization"
  ** Laurie Penny: "Mit dem Posen der romantischen Liebe, besonders der heterosexuellen romantischen Liebe in der Ehe, meinen wir uns der bitteren Realität von Arbeit und Tod verweigern zu können - dabei präparieren sie uns genau dafür, verkuppeln uns und stecken uns in lauter kleine Schubladen aus Leid und Leidenschaft: Du und ich gegen den Rest der Welt, Baby. Wir verlieben uns, weil das leichter ist als wenn wir lernen müssten, uns freizuschwimmen."
  ** Laurie Penny: "Man kann Menschen in antagonistischen Paaren zusammenspannen und vor ihrer Umwelt isolieren, damit sie sich gegenseitig die strukturelle Herzlosigkeit dieser Welt in die Schuhe schieben." (S.226)
  ** Laurie Penny: "Man kann die Suche nach einer einfachen Bindung zu einem elenden, ermüdenden RItual machen, das eine strenge Geschlechtskonformität voraussetzt und den menschlichen Geist unterdrückt."
  ** Laurie Penny: "Frauen aller Gesellschaftsschichten wird beigebracht, dass sie sich zuallererst um die Liebe von Männern bemühen müssen, um darüber ihren Wert zu taxieren, Männer auf sich aufmerksam zu machen. Und in allen Gesellschaftsschichten wird die romantische Demütigung dafür genutzt, Frauen klein zu machen. Jeder heterosexuelle Mann, mit dem ich mich je über Partnersuche unterhalten habe, dass Frauen in romantischen Dingen alle Macht haben, auch die ultimative Macht, auf die sexuelle Avancen eines Mannes einzugehen oder sie zurückzuweisen und einen Mann in den Kreis ihrer "Freunde" einzureihen, was dieser als völlig schwachsinnig empfindet, denn natürlich will kein echter Mann für eine Frau nur ein Freund sein. Die Macht, nein zum Sex zu sagen, finden Männer ungeheuerlich und erachten es daher nur als gerecht, dass Frauen und Mädchen alle anderen Spielarten der Macht seit vielen trostlosen Generationen vorenthalten werden. Das hätte etwas für sich, wenn die Macht, nein zum Sex zu sagen, in der Praxis respektiert würde." (S.225)

hist. Hintergrund:

 * Adrienne Rich: "historically men and women have played very different parts in each other' lives. Where woman has been a luxury for man, and has served as the painter's model and the poet's muse, but also as comforter, nurse, cook, bearer of his seed, secretarial assistant and copyist of manuscripts, man has played a quite different role for the female artist."
\end{comment}
