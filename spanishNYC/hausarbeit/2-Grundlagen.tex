\section{Grundlagen}

Grundlagen
----------
Language Shift (Spanish -> English) : Speech community of a language shifts to speaking another language (source?)
Can we talk about language shift in this context?
Who is shifting? The editors of the magazine? Or do they induce language shift?
Should that be a focus at all? Or focus more on communities of practice/style projection through language/sozioindexikalität, etc.

%Communities of practice
can we say that the target reader group of these magazines is a community of practice?
--> bisschen weit hergeholt; passt meiner Meinung nach nicht zu der Definition von Community of Practice

English is the more prestigious language;
the dominant(both in numbers and power) group of people in the USA speaks English
serves to avoid misunderstandings/clarify (zb cooking recipes)

Speech patterns (Vortrag Spanish in NYC)
* bilingualism: 63\% census respondents (2007) who speak Spanish at home also report that they
speak English well or very well
*shift from Spanish to English after 2nd generation migrants
* Spanish speaking community survives due to new migrations

*style/socioindexicality: adopt linguistic forms of groups one sympathises with/has intense
relationships with
*``Although overt norms favor standard speech, powerful covert norms encourage
group members to remain faithful to group codes, linguistic and otherwise.'' (Zentella 2007)
** Dominicans keep their low prestige variety, because ``overcorrectness'' is associated with
``feminine''
* media: promotes wide spread/conservative varieties

-> Zentellas Konzept ``linguistic insecurity'': ``consider their dialect inferior to others''[Zentella07]


* Spanglish /Code switching!
``Powerful socioeconomic and cultural forces stimulate borrowing whenever two cultures are in contact;
the borrowing by the subordinate group's language from that of the dominant group is always
significantly greater'' [Zentella90]
Factors, which facilitate borrowings:[Zentella90]
% 1) Most items reflect a cultural reality that is new or different, e.g., \[kei/keike/keiki\]
%``cake'', \[bobipín\] ``bobby pin'', \[ŷins/bluŷines\] ``(blue)jeans.''
% ``la boila'' (boiler) [Zentella97]
 2) Others may be facilitated or even triggered by a similarity in the phonological and/or morphological
 structure of a word in the dominant language that makes it sound like a possible subordinate word,
 matre(s) ``mattress'' is similar to that of madre(s) ``mother(s)''
 ZB ``libreria'' (Buchladen) für Bibliothek (library)
 3) reduction in the number of syllables, e.g., \[beis\] replaces florero for ``vase''
 4) prestige
 5) Anglicisms can play the role of neutralizer between competing dialectal variants because the
 prestigious outside language acts as the lingua franca that resolves the conflict without favoring one
 group at the expense of the other.

``but it seems that the longer a group has been in the U.S., the less they use the homeland's term''
``second generation Cubans incorporated more loans in their speech than members of the first generation.''
``Language proficiency appears to be a more relevant factor than age'' [Zentella90]

* influence of the media: ``they do contribute to popular acceptance and use of some new vocabulary'' [Zentella90]
(siehe oben bei Spanglish)
-> Spielen auch für leveling eine Rolle: versuchen neutrale Varianten zu nutzen, um möglichst mehr Menschen zu erreichen;

Stichwörter:
Ideologie, Sprecher\_innenidentität

Mögliche Änderungen:
* feature addtion
* feature replacement
* feature loss

Das Problem ist:
* wir wissen nicht wer genau die Artikel schreibt;
* nur wer die Zielgruppe ist (mehr oder weniger)
* wir können schwer Theorien anwenden, die Sprecher\in-zentriert sind, da es unklar ist, wer das geschrieben hat; (und auch irrelevant)
* also eher schwer zb Gender/Social networks mit ein zu beziehen
* Aber was schein eine Rolle zu spielen sind: Code Switching; Borrowings; Language Shift; Sozioindexikalität (Style)


\begin{comment}
%%%%%%%%%%%%%%%%%%%%%%%%%%%%%%%%%%%%%%%%%%%%
Allgemeine Theorie

Linguistische Wandelprozesse kann man durch
* Interferenz
* allgemeine Demarkierung (zu erwarten in Kontaktsituationen)
erklären

Sprachliche Prozesse:
* transparenz fördernde --> Hörer orientiert --> bei extremer Mehrsprachigkeit
* Ökonomie fördernde --> Sprecher orientiert --> eher bei größeren Einsprachigkeit


%Stil (Eckert?)

Kategorien: ``Ich räume alles auf''
Menschen sortieren Sachen in Kategorien, damit sie kognitiv damit umgehen können
Kleidungsstil: indexikalische Bedeutung -> vermittelt einen bestimmten Hintergrund (zB Sakko->Prof)

%Sozioindexikalität

1st order
gegeben
``association by social actors of a linguistic form/variety with some meaningful social group (female, spanish, working class..)''
-> Manche features eignet man sich an oder gewöhnt sich ab aus ideologischen Gründen

2nd order
describing, noticing.. of the 1st order indexicality;
variaties are differently notices, rationalized, ... from land to land and community to community
ideological analysis: social categories are locally created by social actors and discoverable by analysis rather than given;
wie sich Menschen projizieren, durch Stil selber kreieren, Performance, bewusster, intenationaler Einsatz

Kurios: Theaterspieler\_innen, Schriftsteller\_innen führen Figuren durch sekundäre Indexikalität ein -> die werden mit Attributen ausgestattet

früher: alle möglichen Kategorien erstellt: black/white/old/young/hispanic/...
aber nicht berücksichtigt, was die Individuen machen, um sich selber darzustellen
Individuen nicht als Akteure berücksichtigt

%%%%%%%%%%%%%%%%%%%%%%%%%%%%%%%%%%%%%
[Zentella07]

%Intro

the defining role of language networks in identity, i.e., “identity is defined as the linguistic con-
struction of membership in one or more social groups or categories” (Kroskrity 2001: 107)
Latina/o identity in the USA is often linked to Spanish, presumed
to be the heritage language of more than 40 million people with roots in 20
Spanish-speaking Latin American nations, including Puerto Rico.
[Zentella07]

distinct ways of being Latina/o are shaped by the dominant language ideology
that equates working-class Spanish speakers with poverty and academic failure,
[Zentella07]
--> also eine Erklärung, warum sie zu Language Shift tendieren würden

The fate and form of the languages spoken by US
Latinas/os will be determined in part by the ways in which they respond to the
construction of their linguistic identities as a group and as members of distinct
speech communities, and those responses in turn can have a significant impact
on Latina/o unity.

%Latina/o Linguistic Capital

In
the USA, where race has been remapped from biology onto language because
public racist remarks are censored, comments about the inferiority and/or
unintelligibility of regional, class, and racial dialects of Spanish and English
substitute for abusive remarks about color, hair, lips, noses, and body parts, with
the same effect.

--> noch eine Erklärung für Language Shift

no one expects you to be able to change your color, but you are expected to
change the way you speak radically to earn respect (Urciuoli 1996).

These attitudes are communicated in everyday conversations and promoted by
the media and public institutions, but some groups of Latinas/os are more affected
than others.

%%%%%%%%%%%%%%%%%%%%%%%%%%%%%%%%%%%%%%%%%%%%%%%%%%%%%%%%%%%%%%%

[Thomason03]

lexical borrowing - both form and content are new to the borrowing language

feature addition
        replacement
        deletion

--> sind meine Sätze Beispiel für einen dieser Prozesse?

When fluent speakers of language A incorporate features into A
from another language, B, the first and most common interference features
will be non-basic lexical items, followed (if contact is sufficiently intense) by
structural features and perhaps also basic vocabulary.

(By contrast, if people who are not fluent speakers of A introduce features
into A from another language, B, the first interference features (and usually
the most common ones overall) will not be lexical, but rather phonological and
syntactic.By contrast, if people who are not fluent speakers of A introduce features
into A from another language, B, the first interference features (and usually
the most common ones overall) will not be lexical, but rather phonological and
syntactic.) --> curious fact, but not really useful in the context

%borrowing vs shift-induced interference
These two types of interference were characterized in Thomason and
Kaufman (1988) as borrowing, in which features are incorporated into A by
native (L1) speakers of A, versus shift-induced interference, in which a group of
L2 learners of A carry over features from B (their L1) into A during a process
of shift from B to A.

--> i would say, we have to do with borrowings

Linguistic factors in linguistic interference:
* universal markedness
* typological distance between source and recipient language

Social factors:
* imperfect learning
* intensity of contact
* speaker attitudes (prestige, ..)

Features that
are deeply embedded in elaborate interlocking structures are in general less
likely to be borrowed, because they are less likely to fit into the recipient
language’s structures; that is why the lexicon, which for all its structure is less
highly organized than other grammatical subsystems, is borrowed first, and it
is why inflectional morphology tends to be borrowed last. But highly integrated
features may be borrowed readily between systems that are typologically very
similar;

And when contact is intense enough, there appear to be no absolute linguistic
barriers at all to borrowing

% 2. Mechanisms of Intereference

Mechanisms of contact-induced change fall into four categories.
* one set of mechanisms comes into play when the implementers of a
change are bilingual in both source and recipient language
* while the other set comprises second language
acquisition strategies
* A third category, “negotiation,” seems to
overlap with both of these types
* and the fourth category has to
do with more or less conscious and deliberate decisions by speakers to imple-
ment language change

%code switching
Code-switching, as used in this section, includes both intrasentential switching
(sometimes called code mixing) and intersentential switching


Code-switching is a (perhaps the) major route by which loanwords enter a  %loanwords
language. It surely plays a role in at least some kinds of structural borrowing
as well, although the more dramatic kinds of structural interference are prob-
ably likely to result from code alternation instead (see section 2.2 below).

I believe that is impossible in principle and in practice to draw an
absolute boundary between code-switching and borrowing. % das ist mir auch zu feine Unterscheidung, ich würde damit nichts anfangen

A code-switched word or other morpheme becomes a borrowing if
it is used more and more frequently – with or without phonological adaptation
– until it is a regular part of the recipient language, learned as such by new
learners.

%borrowing and new word inventions
The addition by borrowing of a new word for a new
concept, like bok choy in English, must begin with a single use and continue
with increasing usage by the innovating speaker(s) and by other speakers, and
the addition by invention of a new word, like photocopy, must follow the same
path.

These parallels are hardly surprising, given that all speakers draw on a
variety of repertoires, typically characterized as styles, registers, and dialects,
in using a single language. If there is no evidence to the contrary – and certainly
no convincing evidence has been presented – then it is surely most reasonable
to assume that speakers whose repertoires include more than one language
will employ the same strategies in deploying their linguistic resources. Change
resulting from code-switching between different languages does, of course,
differ from change via diffusion from another register and dialect borrowing;
but, as noted at the start of this chapter, the differences are a matter of degree,
not of kind.

%code-switching vs shift-induced interference
lexical items predominate in code-switching, while
phonological and syntactic features predominate in shift-induced interference.

%code alternation
%e.g. using one language at home, another at work
streng situationell --> funktional

In many cases the two languages
are used by the same speaker with different interlocutors, often monolinguals;

the unconscious and involuntary incorporation of foreign structural features into
one of a bilingual’s languages fits very well with psycholinguists’ finding that
“bilinguals rarely deactivate the other language totally”

morphemes, both lexical and grammatical, would be introduced directly via
code-switching, changing from code-switches to borrowings through increas-
ingly frequent usage by code-switching speakers and then (if not all members
of the speech community engage in code-switching) by adoption by other
speakers;

%passive familiarity
Sometimes interference features are introduced by speakers whose competence
in the source language is strictly passive – that is, a speaker may borrow a
feature from a dialect or language that she or he does not speak actively at all.
(e.g. from African American Vernacular English)

%Deliberate change
Theories of language change rarely or never allow for the possibility of deliber-
ate change, except in such trivial cases as the conscious adoption of loanwords
and even new sounds in words of foreign origin

types of deliberate decisions:
* two-language mixtures created by bilinguals,
apparently to serve as a symbol of a new ethnic identity; (Chicana Spanish, "Kanakendeutsch"? meine Bsp)
* of speaker groups deliberately withholding their
“real” language from outsiders, using instead a distorted and simplified for-
eigner talk version that, in some cases, forms the basis for a trade pidgin
* creation of a secret language, either by phonological distortion [...] or lexical replacement.
* the motive for making a particular change has to do with emphasizing
in-group status, or differentness from other groups;

% definitions for language attrition, language death
% not sure it has something to do with the current topic
as the “loans to loss” model, extensive
borrowing leads eventually to language loss;

the adoption of lexical
and structural features from the dominant language – that is, convergence
toward the dominant language

in general a change is more likely to occur if independ-
ent forces are pushing in the same direction.

Bewegende Macht: Sprecher\_innen-Kreativität

%%%%%%%%%%%%%%%%%%%%%%%%%%%%%%%%%%%%%%%%%%%%%%%%%%%%%%%%%%%%%%%
[Klein92]

beim L1-Erwerb muss Kognition + soz. Identität dazu erworben werden
beim L2-Erwerb nicht; aber zieht wohl auch soz. Entwicklung mit sich

%%%%%%%%%%%%%%%%%%%%%%%%%%%%%%%%%%%%%%%%%%%%%%%%%%%%%%%%%%%%%
[Krefeld04]

Modelierung migratorischen Kommunikationsräume

Der Raum ist "Produkt seiner interagierenden Bewohner"

"Welt in aktueller" und "Welt in potentieller Reichweite" --> individuell, bei jeder Person anders

"Identische Räume darf man daher grundsätzlich niemals voraussetzen; vielmehr ergeben sich mehr oder weniger große gemeinschaftliche Teilräume aus dem Gebrauch bzw. aus der Entwicklung gemeinsam verfügbarer Idiome und Varietäten"

"System räumlicher Gliederungen [...] Differenzierung der Intimität und Anonymität, der Fremdheit und Vertrautheit, der sozialen Nähe und Distanz" [Schütz/Luckmann 1979, 68]

Dimensionen des kommunikativen Raums:
* Sprache
* Sprecher\_in
* Sprechen

Räumlichkeit der Sprache:
* Umgebung und umgebungsspezifische Varietäten (Arealität) (z.B. Land/Stadt, West/Ost)
* Territorium und seine Sprache(n) (Territorialität) (Dialekten? Sprachen?)

Arealität: "Bindung sprachlicher, in der Regel dialektarel Merkmale an einen spezifischen Ort"
Territorialität: "staatlich garantierte und nicht selten juristisch sanktionierte Geltung einer Staatssprache in einem administrativ scharf begrenzten Gebiet"

Räumlichkeit der Sprecherin (und der Hörerin):
* Provenienz (wo kommt sie her?)
* Mobilität (wohin/wie hat sie sich bewegt)
* Repertoire

Räumlichkeit des Sprechens:
* situative Positionalität
* Interaktion

"Relationen zwischen Provinenz und sozialer Hierarchie, d.h. die Formen sozialer Mobilität zu beachten"

Positionalität des Sprechers:
relativen Nähe/Distanz --> "Dabei ist die soziale Nähe, d.h. die Vernetzung mit unmittelbar erreichbaren Kommunikationspartnern (aus peer-groups, bzw. Familien), von der pragmatischen Nähe, der Situationalität, zu unterscheiden"

Aufgabe der Raumlinguistik: "wie die Sprecher selbst ihren (gemeinsamen) Raum konstruiert haben und wie sie ihn selbst wahrnehmen"

Glossotop: "Analog dazu bezeichne ich den Ort einer mehrsprachigen Kommunikationsgemeinschaft als 'Glossotop'"
--> "Gesamtheit der Regularitäten (...) die den lokalen Gebrauchder sprachlichen Varietäten in einer bestimmten lebensweltlichen Gruppe (zum Beispiel einer Familie, einer Nachbarschaft, einer peer-group etc.) steuern."

"wer spricht mit wem in welcher Varietät/Sprache?" --> soziale Netze ([Immacolata Tempesta 2000])

%%%%%%%%%%%%%%%%%%%%%%%%%%%%%%%%%%%%%%%%%%%%%%%%%%%%%%%%%%
[Milroy93]

--> introduces the network concentp; the strength of weak ties; gender variable

Untersucht: impact of speakers' variables social class, social network and gender on linguistic variability and change
``discuss the way in which these extralinguistic variables are interrelated''

``gender difference is often prior to social class in accounting for sociolinguistic variation''

selten jemand hinterfragt welche Variablen werden bei Untersuchungen gewählt und warum
bis dato (1993): class is primary variable <- critism

``what are usually called low-prestige varieties can be maintained over generations as flourishing vernaculars.''
also kann Class und prestige nicht die einzige Erklärung für Language Shift und Loss sein

``strong informal social ties within
communities provide the mechanisms that enable speakers to maintain non-
standard dialects, rural or urban, despite intense pressure from the standard
language through routes such as the educational system and the media.''
kann wohl eher schwierig als Argumentation für mich benutzt werden

``The form which linguistic gender-marking has commonly been interpreted as
taking is for women to approximate more commonly than men of similar s t a t u s
to the (so-called) prestige norm. But as Coates (1986) has pointed out, no satis-
factory explanation has emerged of why women should be more oriented than
men to a prestige norm.''
naja, mir fällt eine ziemlich lange Erklärung dazu ein.
aber die hier argumentieren gleich, dass es anders rum funktioniert...

``and the relation of gender to social class and to prestige patterns is by no means consistent
or predictable from the usual assumptions about females preferring higher social-class norms.''

``The generalization suggested is not that females
favour prestige forms, but that they create them; i.e., if females favour certain
forms, they become prestige forms. In these developments, both class and
gender are implicated, but gender is prior to class.''
eine Erklärung dafür wäre, dass Frauen den Hauptteil der Care-Arbeit leisten,
also sind sie diejenigen, die auch die Sprache weiter geben

``Close-
knit networks, which will of course vary in actual levels of density and multi-
plexity, are assumed to have the capacity to maintain and even enforce local
conventions and norms - including linguistic norms. It is, after all, remarkable
that stigmatized linguistic forms and low-status vernaculars can persist over
centuries in the face of powerful national policies for diffusing and imposing
standard languages,''
spannend, aber vermutlich nicht sehr nützlich in meinem Kontext

``close-knit network structure is associated with
language maintenance; the corollary to this is that a loose-knit network struc-
ture is associated with language change. We have argued in detail elsewhere
that where ties are relatively loose-knit, communities will be susceptible to
change originating from outside localized networks. These changes are not
necessarily in the direction of the standard''

``loose-knit structures with
socially and geographically mobile middle class speakers, and close-knit ties
with lower and upper class speakers.''
-->vlt spannend

``innovators are likely to be persons weakly linked to
such a group. Susceptibility to outside influence is likely to increase in inverse
proportion to the strength of the tie with the group''
vlt nochmal ``The strength of weak ties'' lesen.. wobei bin mir nicht sicher was es
bei der Arbeit helfen würde.. man kann eh nicht zuverlässig behaupten ob es da ein 
Netzwerk gibt und wie stark genau die Connections sind

``A ‘weak tie’ model of change thus seems able to account for some
instances of variation and change which are difficult to explain in terms of the
usual unqualified assumption that linguistic change is encouraged by frequency
of contact and relatively open channels of communication''

``Some
linguists have apparently assumed that ‘social network’ is concerned only with
strong ties and is roughly synonymous with ‘peer-group’,''

``some linguists have seemed to believe that
individuals may o r may not possess a ‘social network’, when i n fact all individ-
uals ar e embedded in networks.''

%%%%%%%%%%%%%%%%%%%%%%%%%%%%%%%%%%%%%%%%%%%%%%%%%%%%%%%%%%%%%

Media
-----
This is especially true regarding language, because, as Trudgill notes, «the media... have almost no effect at all in
phonological or grammatical change» (1984:61), although they do contribute to popular acceptance and
use of some new vocabulary. [Zentella90]

Marshall McLuhan: hot and cold media
hot media: ``Hot media do not leave so much to be filled in by audience.
Hot media are, therefore, low in participation or completion by audience.'' [Willie79]

(http://www.cliffsnotes.com/sciences/sociology/contemporary-mass-media/the-role-and-influence-of-mass-media)

``Communities and individuals are bombarded constantly with messages from a multitude of sources including TV, billboards, and magazines, to name a few. These messages promote not only products, but moods, attitudes, and a sense of what is and is not important. Mass media makes possible the concept of celebrity: without the ability of movies, magazines, and news media to reach across thousands of miles, people could not become famous. In fact, only political and business leaders, as well as the few notorious outlaws, were famous in the past. Only in recent times have actors, singers, and other social elites become celebrities or “stars.” '' (http://www.cliffsnotes.com/sciences/sociology/contemporary-mass-media/the-role-and-influence-of-mass-media)

%The limited effects theory
The limited‐effects theory argues that because people generally choose what to watch or read based on what they already believe, media exerts a negligible influence.
%Criticism
First, they claim that limited‐effects theory ignores the media's role in framing and limiting the discussion and debate of issues. How media frames the debate and what questions members of the media ask change the outcome of the discussion and the possible conclusions people may draw. Second, this theory came into existence when the availability and dominance of media was far less widespread.

%The class-dominant theory
The class‐dominant theory argues that the media reflects and projects the view of a minority elite, which controls it.
For example, owners can easily avoid or silence stories that expose unethical corporate behavior or hold corporations responsible for their actions.
The issue of sponsorship adds to this problem. Advertising dollars fund most media.
elevision networks receiving millions of dollars in advertising from companies like Nike and other textile manufacturers were slow to run stories on their news shows about possible human‐rights violations by these companies in foreign countries.

%The culturalist theory
The culturalist theory, developed in the 1980s and 1990s, combines the other two theories and claims that people interact with media to create their own meanings out of the images and messages they receive.
This theory sees audiences as playing an active rather than passive role in relation to mass media.
Therefore, culturalist theorists claim that, while a few elite in large corporations may exert significant control over what information media produces and distributes, personal perspective plays a more powerful role in how the audience members interpret those messages.

%http://www.pbs.org/speak/ahead/mediapower/media/#talk
% Jack Chambers
% “Talk the talk?” [author’s title “TV and Your Language.”]. Website “Do You Speak American?”
% McNeil - Lehrer Productions. http://www.pbs.org/speak/ahead/mediapower/media
% 2005
Second, the lasting power of words that spread via the mass media has nothing to do with the various media themselves. Punk’d lasted just two seasons (2002-03). While it lasted, its name was raised into common parlance. What are the chances a word will persist for another five to 10 years? Not good. In buzzwords as in outré attire, there is a direct relationship between the height of the craze and the decline into oblivion. Fads mark their users as members of an in-group. The faster fads spread, the more pressure there is to find a new marker. Only your mother, if she was a beatnik, thinks rimless specs are groovy. Only your grandmother, if she was a gate, thinks black horn-rims are crazy.

A final common assumption is that the media leads language changes. In fact, it belatedly reflects the changes.

The same fallacy seems to underlie the casual assumption that the mass media drives all kinds of language changes.

If the mass media can popularize words and expressions, then “presumably” it can also spread other kinds of linguistic changes. We generalize from one limited effect to a host of others.

Finally, we should note that high mobility has even greater social significance than the media explosion.

%More stuff zu Media
%\cite[vgl.][s.20]{Tomasello06}
\end{comment}
