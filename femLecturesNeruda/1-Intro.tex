\section{Einleitung}

Pablo Neruda, der Nationalpoet Chiles, ist einer der bekanntesten und beliebtesten, wenn nicht der bekannteste überhaupt, Lyriker*innen Lateinamerikas.
Seine Werke werden heute noch rege rezipiert und sind ausgiebig von der Literaturkritik kommentiert worden.

% Liebeslyrik

Nerudas Werk ist umfangreich und vielfältig, der Poet hat sich mit vielen Aspekten des menschlichen Lebens und dem aktuellen Tagesgeschehnis befasst.
Er bearbeitet in seinen Gedichte seine Erlebnisse während des Spanischen Bürgerkriegs, seine Rezeption anderer Poeten, seine Begeisterung für den Kommunismus, seine Bewunderung für die Lateinamerikanische Natur, sowie ``universellere'' Themen wie Liebe, ...
Unter den vielen prägenden Werken sind auch einige, die uns heutzutage stutzen lassen.
Dabei beziehe ich mich nicht nur auf die Oda an Stalin oder ???, die aus heutiger Sicht zweifellos umstritten sind, sondern auch teilweise auf Nerudas Liebeslyrik.
Von vielen sehr positiv wahrgenommen und als Wiedergabe universellen menschlichen Gefühle angepriesen, %TODO quote
können wir in dieser auch beunruhigende Ideologien und Frauenbilder entdecken.

% Ziel der Arbeit

Ausgehend von der Annahme, dass Sprache und Literatur nicht nur die Welt widerspiegeln, sondern unsere Weltanschauung, Gedanken und Umgang mit der uns umgebenden Welt aktiv mitgestalten~\cite{Kolodny1980},~\cite{North2013},
möchte ich in dieser Arbeit Nerudas Liebeslyrik aus einer feministischen Perspektive lesen.
Das Ziel dieser Arbeit(syn!)/Aufsatzes? ist es nicht, die ``richtige'' Interpretation Nerudas Liebeslyrik (syn!) auszuarbeiten, sondern, in Anlehnung an Annette Kolodny, viel mehr eine Pluralität der Lektüren anzuregen~\cite{Kolodny1980}, Leseweisen ermöglichen, die im androzentrischen Literaturkritikdiskurs verdeckt bleiben.
Durch das Close Reading exemplarisch ausgewählten Gedichte aus zwei verschiedenen Gedichtsbändern soll die Poetisierung und Universalisierung bestimmten heteronormativen und männerzentrierten Perspektiven aufgezeigt und in Frage gestellt werden.

  % iwie ist mein Projekt mehr oder weniger das gleiche wie das von Frau Duncan: ``There is no doubt that Neruda is a gifted poet; whether or not he is guilty of sexism as a writer is not a point I wish to debate here. But, I do take issue with the critical readings of Neruda which have drawn attention toth repression, subjugation, and silencing of women in his poetry only to dismiss these factors as natural conditions which are in and of themselves praiseworthy.''\cite{Duncan1992}

% Aufbau der Arbeit!
Im Weiteren ist diese Arbeit wie folgt aufgebaut:
Im zweiten Kapitel werden zunächst etablierte Rezeptionen Nerudas Liebeslyrik vorgestellt.
In Kapitel 3 gehe ich auf einige Begriffe bzw. Methoden ein, die für die weitere Analyse eine bedeutende Rolle spielen werden.
Darauf aufbauend biete ich in Kapitel 4 exemplarisch alternative Leseweisen von den Liebesgedichten ``Poema XV'' (\textit{Veinte poemas de amor}), ``Tu risa'' (\textit{Los versos del capitán}) und ``El tigre'' (\textit{Los versos del capitán}) an.
Abschließend wird die Arbeit nochmal zusammengefasst und ein Ausblick wird umrissen.

