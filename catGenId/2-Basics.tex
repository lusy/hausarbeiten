\section{Fonaments teòrics}

\subsection{La immigració a Catalunya al llarg dels anys}
\subsection{Interseccionalitat}
\subsection{Cultura, llengua, traducció i identitat}
\subsection{Patriarcat}
\subsection{On? Discusió de l'estil, llenguatge, etc}


"La nostra casa a Mango Street però sense Lucy ni <textit>chicanos</>."


\begin{comment}
\subsection{Weitere interessante Anmerkungen}
%--------------------------------

* es wird nie klar wie die Tochter heißt; Universalität? Kann sich jede* damit identifizieren? (insgesamt werden ganz selten Namen erwaehnt. Aus irgendeinem Grund werden die Liebhaber*innen Mimouns meist mit Namen erwaehnt; der Name der Mutter kommt 2 mal vor)
* interessante Erzählweise: eine Figur erzählt in 1. Person, trotzdem auktoriale Erzählung
* referències a la Mercè Rodoreda: vinculació amb la cultura/tradició literària catalana
* keine graphisch markierte direkte Rede: alles ist ein Textfluss; stream of consciousness
* eine Möglichkeit: biographisches Lesen

\end{comment}

\begin{comment}
\subsection{Quotes (unklar wohin damit, kommen mir aber wichtig vor)}
%  --------------

"Que aquell seu fill no havia donat el divorci a la seva esposa, que ella quedava lligada a ell per sempre sense poder canviar cap circumstància del seu destí. Abandonada, però lligada a ell, allò anava en contra de totes les lleis, tant dels àrabs com les nostres." (p.160)
-- destí: eine der Sachen, die wiederholt werden
-- dels àrabs, com les nostres -- la filla no es compta a ella mateixa com part dels àrabs??
-- o qui és nosaltres aquí? contrast entre la cultura àrab i la cultura amazigh?

"Quan la mare sentia la clau dins el pany, deia, marxo que ja ve i feia veure que no hi havia parlat mai, amb la Soumisha. El pare no havia dit mai que no hi pogués parlar, però valia més no fer-li saber massa coses." (p.213)
--erneut die Solidaritaet zwischen frauen wird teilweise vorgebeugt

"Ell no sabia que eren els pantalons els que es feien petits i no pas jo que els triava cenyits. Què hi havia de fer si el meu cul creixia i creixia? Res, buscar talles més grans; [...] Però per aquella època tot el que era llarg era de iaia i jo hauria preferit morir que presentar-me així a l'escola." (p.253)


"A la classe em deien pilota perquè era l'única alumna que anava a passejar amb una professora, però no sabien que si no hagués estat per tot el que ella m'aportava, pels horitzons nous que m'oferia, jo m'hauria mort, potser no per fora, però per dins sí." (p.254)

"Fins que va passar allò del camió i jo ja vaig començár a pensar que aquell no podia ser el meu destí, ni el nostre destí ni el destí de ningú." (p.258)
--leitmotiv destí

"Agafava la porta i se n'anava, així, pam, i jo ja començava a pensar que aquell no era el meu destí." (p.324)

"Ho estava qua ell havia dit que em portaria al metge, que em preparés que aniria perquè em fessin una revisió i em diguessin si era verge o no. En condicions normals el meu cap hauria funcionat, hauria sabut perfectament que cap metge li faria allò que demanava, un certificat de virginitat emès per la seguretat social. Però a mi el cap ja no m'anava i era tot un esgotament tan enorme que només volia dormir[...] Encara tenia les pastilles per posar sota la llengua i me'n vaig començar a empassar, una, dues, tres, quatre, fins a perdre'n el compte [...]" (p.314)

"M'havia protegit sense protegir-me i m'havia defensat sense defensar-me, que devien ser les úniques coses que jo necessitava. I amb pressions del pare, sense saber com, ell va anar cedint, va, dona, no siguis així, va dona, siguis d'aquesta manera, és el teu pare i un pare és un pare." (p.322)

"I mira, no vull problemes amb el teu pare, que sempre m'està trucant. Posa't el mocador i ja està, si no passa res, la meva mare l'ha dut, la teva mare l'ha dut i no s'han pas mort, no? Estarem sempre junts, t'ho prometo, li havia dit jo, i ell m'havia dit, però què és una promesa?" (p.323)

\end{comment}
