\section{Introducción}
El peronismo es uno de los movimientos políticos más discutidos de Argentina, Latinoamérica y el mundo entero.
Se trata de un fenómeno complejo al análisis del cual se han dedicado muchas disciplinas: desde la historia y las ciencias políticas hasta la antropología y las ciencias culturales.
Abordar todas las discuciones sobre este asunto no es posible ni es el objetivo de este texto.

% Was ist mein Ziel: Nación o Evita?
El objetivo de trabajo actual es más bien analizar el tropo/mito/la imagen de la nación imaginada por el peronismo reflejada en el libro de lectura para primer grado interior \textit{Evita} del editorial Luis Lassere, editado por Graciela Albornoz de Videla en año 1952.

Se trata de un libro que facilita la enseñanza y aprendizaje de lectura y escritura dirigido a niñxs de la escuela primaria del periodo.
Como ya el título señala, el libro está vinculado muy estrechamente con los ideales y las imágenes del regimen político en vigor.
Aunqué se dirije a lxs más jóvenes, \textit{Evita} no falla de abordar los temas y símbolos claves del peronismo, apoyando la idea de un aparato de propaganda estatal extenso/transcedental (weitreichend).
\textit{Evita} forma parte del proyecto de crear un mito homogeneo de la nación argentina
y de consolidar la autoridad de lxs líderes peronistas: Juan y Eva Perón.


- nationenbildung
- momento clave para Argentina
- cantidad enorme de comentarios academicos y otros
- aparato propagandista grande


A continuación, este trabajo está organizado de la manera siguiente.
Primero, investigaremos el periodo histórico en el cual se ubica dicha publicación.
Después, presentaremos el marco teórico que nos ayudará analizar los rasgos claves de \textit{Evita}.
Seguiremos con el análisis del libro infantil.
En conclusión, resumiremos otra vez los resultados y plantaremos algunas ideas para una discusión más profunda.
