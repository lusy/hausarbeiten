\section{Einleitung}

\begin{comment}
* ca 3/4 Seiten
* Untersuchungsgegenstand
* Erkenntnisinteresse
* Forschungsstand
* Vorgehensweise: also Section 2 macht blabla, Section 3 blablup, ....
* Ergebnisse können/sollen angedeuten werden
\end{comment}


%TODO
%Generall Intro fehlt!
% code switching, borrowing als Phänomene, die einen Language Shift andeuten;
% dafür kann es verschiedene Gründe geben
% interessant herauszufinden, welche davon in konkreten Sprachkontaktsituationen zutreffen
% Prognosen darüber, ob eine Sprache (an einem bestimmten Ort/Gemeinschaft) erhalten bleibt oder verloren geht

Sprachwandel ist einer der zentralen Phänomene, die die Soziolinguistik untersucht.
Fast immer(?) ist Sprachkontakt eine Voraussetzung für Wandel.
In welchen Kontaktsituationen kommt es jedoch zu Wandel und in welchen etabliert sich ein "stabiler" Bilinguismus?
Welche Faktoren begünstigen den Wandel?
Welche Indizien weisen auf Wandelprozesse hin?

%TODO
%Notiz zum generischen Femininum

Das Ziel dieser Arbeit ist eine Analyse der Code-Switches und Entlehnungen zwischen dem Spanischen und dem Englischen in Artikeln der US-amerikanischen Zeitschrift ``Siempre mujer'' aus einer soziolinguistischen Perspektive.
Es gilt herauszufinden, ob diese Code-Switches in bestimmten Kontexten passieren oder bestimmten rekurenten Mustern unterliegen.
Wenn diese Erwartung erfüllt wird, wird eine sozioindexikalische Erklärung versucht.

Die Ausgangshypothese ist, dass dies sehr wohl der Fall ist.
Im hauptsächlich spanischen Text werden in speziellen Fällen englische Ausdrücke erwartet:
z.B. wenn es um technische Begriffe oder um Idiome/Redewendungen geht, für die keine genaue spanische Übersetzung existiert.
Weitere Fälle, in denen wir Englisch erwarten können, sind region-/kulturspezifische Begriffe: Essen, Tiere und Pflanzen, Bräuche, etc.

Es wäre weiterhin interessant zu untersuchen, in wie fern bewusst ins Englische geswitcht wird, da diese Sprache als prestigeträchtiger wahrgenommen wird oder um einen bestimmten Lifestyle(?) zu inszenieren, wodurch dann auch z.B. versucht wird gewisse Produkte zu verkaufen.

Um diese Analyse zu realisieren, wurden Artikel aus der elektronischen Ausgabe der Zeitschrift heruntergeladen und Code-Switches ins Englische extrahiert.


\begin{comment}
Pregunta
--------
Vull investigar si l'altre idioma es fa servir en contextos específics.
La suposició inicial és que sí, que si tenim un text escrit principalment en castellà, trobarem expresions en anglès en casos especials,
per exemple per designar termes tècnincs o frases fixes que no en tenen una traducció exacta en castellà.
Esperem també que observarem la tendecia inversa: que les frases fixes del castellà es mantenen en textos angleses.
Un altre cas de mantenir l'idioma "original" seria si es tracte de coses típiques d'una regió que no es poden trobar als llocs on es parla l'altre idioma (per exemple plantes, animals, menjar, etc.).
A més a més, seria interessant observar si frases de "l'altra llengua" s'utilitzan a causa de tenir un estatut més "alt" o amb l'objectiu de vender un producte.

Vinculació a la investigació existent
------------------------------------
El tema del billingüísme castellà-anglès als Estats Units no és nou i ja existeixen bastants treballs que investigan aquest fenomen.
Per exemple la investigació d'Ana Zentella i Ricardo Otheguy es va concentrar massa al billingüisme i l'ensenyament,
als diferents processos que ocorren en casos de coexistència de més d'una llengua.
Tanmateix, seria interessant examinar de quina manera els mitjans de comunicació (i en aquest cas concret, les revistes feminines) utilitzan dos codis i,
d'aixó intentar a concloure quin és l'objectiu que persegueixin.

------------------------------------

\end{comment}

\subsection{Verwandte Arbeiten}
(Forschungsstand)

Das Thema der Zweisprachigkeit Spanisch-Englisch in den USA ist nicht neu.
Es existieren bereits zahlreiche Arbeiten, die sich diesem Thema widmen.
Ana Zentella und Ricardo Otheguy sind nur zwei der Forschenden(syn?), die mit ihrer Forschung zu Zweisprachigkeit und Spracherwerb/Sprachunterricht einen erheblichen Beitrag auf dem Gebiet geleistet haben (konkreter!).

Der Fokus dieser Arbeit liegt jedoch anderswo.
Medientheorie?

\begin{comment}
Was soll dann hier rein?
Ich hab keine Ahnung ob jemand schon so was analysiert hat..
Eine oberflächliche Internetsuche sagt ja :)

Soll das in die Intro? Oder eigener Kapitel? Oder in 2.?
Ich glaub, ich machs in die Intro:
[Ticknor12]

[Mahootian05]
Die Autorin untersucht die Beziehung zwischen Code Choice (Sprachwahl?), zweisprachige Identität und Sprachwandel.
Sie stellt sich folgende Fragen:
* in wie fern sind stilistische und soziale Variablen, inklusive Identität verantwortlich für die Wahl des sprachlichen Codes?
"The first asks the extent to which stylistic and social variables, including
identity, govern code choice."
* was ist die Beziehung zwischen Wahl des sprachlichen Codes und Sprachwandel?
Dabei interpretiert sie die Benutzung von gemischtem Code in Druckmedien bereits als ein Zeichen von Sprachwandel.

Besonders relevant, dass es sich um Druchmedien handelt: die Codeauswahl hier passiert bewusst, die Artikel werden nicht nur von ihrer Autorin gelesen, sondern unterlaufen weitere Redigierstufen bevor sie abgedruckt werden.
Also einerseits Merkmal der Akzeptanz: Wenn mixed Code Diskurs in den Mainstream Medien angekommen ist (vgl die Debatte wegen Sprachverunglimpfung und bla und die abwertende Wahrnehmung von \textit{pocho} und \textit{caló} Sprecher*innen).
Erklärungsversuche:
* Schaffen einer Intimität, Nähe zwischen Herausgeber*innen und Leserinnen
* "Linguistic necessity" (das genaue Wort existiert auf Englisch nicht)
* Manche Ausdrücke wurden durch Muttersprachler*innen als "emotional stärker" auf Spanisch beurteilt
* Zugleich Bewusstsein dafür, dass Englisch die "prestige trächtigere" Sprache ist, dass sie auf Englisch angewiesen sind, um an der Wirtschaft teilhaben zu können/für ihre Existenzgrundlagen sorgen zu können;
Andererseits auch Versuch, ihre (hispanische?) linguistische und kulturelle Identität zu bewahren.

Ähnlich
Shahrzad Mahootian führt eine soziolinguistische Analyse des Magazins \textit{Latina} durch.
Da liegen die Artikel anscheinend in 3 Versionen vor: Spanisch, Englisch und Englisch mit Codeswitching in Spanisch.
Dabei scheint die spanische Version in der Regel eine kürzere Zusammenfassung der Englischen zu sein.

Bzw. die Autorin macht folgende Kategorien auf:
(1) idiomatic: \textit{la musica tejana}, food names, titles of Spanish songs, movies,..
(2) attention-getting
(3) emotionally/culturally evocative/bonding

Und befragt 3 Muttersprachler*innen dazu, wie sie 56 ausgewählte Beispiele, wo Code-Switches vorkommen, kategorisieren würden.

"Language is both coconstructor and
a reflection of social identity."

"I’ve shown that codemixed discourse is itself a speech variety and that codemixed
discourse and codeswitching may be used intentionally to index the “bilingual identity.”"

"Mixed Spanish-English code is another means for Latinos in the
U.S.A. (a) to show unity among themselves as a subset of all Hispanic-Americans, (b)
to identify themselves as a group separate from their predecessors’ generation, and (c)
to continue to maintain strong emotional ties with their heritage."

\end{comment}

\subsection{Aufbau der Arbeit}

Im Weiteren ist diese Arbeit wie folgt aufgebaut:
Kapitel 2 gibt einen Überblick über die theoretischen Grundlagen, auf denen die vorliegende Analyse aufbaut.
In Kaptel 3 stelle ich das untersuchte Korpus und die benutzte Methodik vor.
Die Ergebnisse der Untersuchung werden in Kapitel 4 diskutiert.
Kapitel 5 gibt einen Ausblick auf Verbesserungen, Anregungen und Fragen für zukünftige Untersuchungen und fasst nochmal abschließend die Arbeit zusammen.

%\cite[vgl.][s.20]{Tomasello06}
