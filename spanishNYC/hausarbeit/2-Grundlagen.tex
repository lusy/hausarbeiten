\section{Grundlagen}

\begin{comment}
Begriffe:
* code switching
(* borrowings)
* language shift
* Sozioindexikalität
* Indexical field?
* Medientheorie?
\end{comment}

%Try to keep short, it should clear the notions, is not the main part of the paper!

Dieser Kapitel stellt kurz die Grundlagen vor, auf denen sich die spätere Analyse stützt.
Im Folgenden werden die Begriffe Code-Switching und Borrowing (Entlehnung), Language Shift (Sprachwechsel),
Sozioindexikalität und indexikalische Felder erläutert.

\subsection{Code-Switching und Borrowing}
Der Begriff Code-Switching beschreibt die alternierende Benutzung von zwei oder mehreren Codes (Sprachen, Varietäten, Dialekten) innerhalb einer und der selben Äußerung. %cite needed vlt Pfaff \cite[]{Pfaff79}

Code Switching kann sogar innerhalb derselben Sprache erfolgen, wenn Menschen sich je nach Situation verschiedener Register bedienen.

Borrowing oder Entlehnung ist die Benutzung der Elemente einer Sprache in einer anderen Sprache.
Meistens meinen wir damit Entlehnungen auf der Lexikon-Ebene, jedoch können auch Elemente anderer Systeme entlehnt werden (was durchaus seltener passiert)~\cite{Thomason03}.% \cite[vgl.][]{Thomason03}.
Wichtig dabei ist, dass Entlehnungen in einer Sprache A durch Muttersprachlerinnen dieser Sprache eingesetzt werden~\cite{Thomason03}. % \cite[vgl.][]{Thomason03}.
Wenn die Personen, die die Elemente aus Sprache B in Sprache A übertragen, Muttersprachlerinnen der Sprache B sind, die die Sprache A gerade erlernen, reden wir von einer ``shift-induced interference''~\cite{Thomason03}. % \cite[vgl.][]{Thomason03}.

Laut Sarah Thomason ist Code-Switching vermutlich der häufigste Weg, auf dem Entlehnungen Teil einer Sprache werden~\cite{Thomason03}. % \cite[vgl.][]{Thomason03}.
In der Praxis ist es oft schwierig zu bestimmen, ob es sich bei einem konkreten Gebrauch um ein Code-Switch oder um eine Entlehnung handelt.
Code-Switches und Borrowings können nicht scharf von einander abgegrenzt werden und der Unterschied zwischen den Beiden ist eher quantitativer als qualitativer Natur.
Wenn ein Begriff vereinzelt und im Sprachgebrauch einer einzelnen Sprecherin vorkommt, würden wir das als Code-Switching bezeichnen.
Wenn jedoch derselbe Begriff immer wieder und von mehreren Sprecherinnen gebraucht wird, reden wir von einer Entlehnung~\cite{Thomason03}. % \cite[vgl.][]{Thomason03}.

Interessant ist ferner die Motivation für das Vorkommen beider Phänomene.
Ana Zentella spricht von ``powerful socioeconomic and cultural forces'', die Borrowings in Kontaktsituationen zwischen zwei Kulturen fördern~\cite{Zentella90}. %\cite[vgl.][]{Zentella90}.
Laut der Wissenschaftlerin, folgende Faktoren ermöglichen Entlehnungen:
\begin{itemize}
  \item Begriffe, die kulturelle Phänomene beschreiben, die in der Muttersprache und der dazugehörigen Kultur in der Form nicht existieren. Beispiele hierfür wären [kei/keike/keiki] ``cake'', [\^yins/blu\^yines] ``(blue)jeans''~\cite{Zentella90} oder ``la boila'' (boiler). %\cite[vgl.][]{Zentella97} wo kommt dieses letzte bsp her? --> buch wieder ausleihen!
  \item Ähnlichkeit in der phonologischen oder morphologischen Struktur eines Wortes.
    Deshalb wird das spanische Wort ``libreria'' (Buchladen) auf einmal im Spanischen auch für Bibliothek (auf Englisch ``library'', auf Spanisch ``biblioteca'') benutzt. %wo kommt das bsp her? vermutlich aus der selben quelle da oben
  \item einer der universellen Prinzipien des Sprachwandels -- Sparsamkeit: das entlehnte Wort ist kürzer. Deshalb wird zum Beispiel ``florero'' durch ``vase'' ersetzt~\cite{Zentella90}.% \cite[vgl.][]{Zentella90}.
  \item Prestige: die Sprache oder Varietät, aus der entlehnt wird, wird als prestigeträchtiger angesehen.
  \item im konkreten Fall von Entlehnungen zwischen dem Spanischen und dem Englischen in den USA können Anglizismen als neutralisierende Begriffe zwischen den verschiedenen Varietäten des Spanischen dienen.
\end{itemize}

Zentella beobachtet ferner die Tendenz, dass Sprecherinnen der 2. Generation mehr Entlehnungen benutzen als die der 1. Generation~\cite{Zentella90}.% \cite[vgl.][]{Zentella90}.


\subsection{Language Shift}
%TODO: die Frage ist, werde ich in die Richtung argumentieren? wenn nicht, dann kann man evtl den Teil rausnehmen
%kann man die Floskel als ein Shift von Spanisch->Englisch begreifen?
%Warum? Warum nicht?
%Who is shifting? the editors of the magazine?or do they induce language shift?
%should this be a focus at all? or more on style/socioindexicality?

% brauch ich den kapitel überhaupt?
% Potowski schreibt auch noch zu den Faktoren, die Shift begünstigen..

Mit \textit{Language Shift} bezeichnen wir den Prozess, bei dem eine Sprache, die als Kommunikations- und Sozialisationsmittel für eine Gemeinschaft gedient hat, durch eine andere ersetzt wird~\cite{Potowski13}. % \cite[vgl.][]{Potowski13}.
Wenn die Weitergabe einer Sprache von einer Generation an der nächsten aufhört, können wir sagen, dass die Gemeinschaft den Shift vollendet hat.

Mehrere Faktoren können einen Language Shift begünstigen.
Ähnlich wie beim Code-Switching ist eine häufige Motivation hier die Prestige. %cite?
Mit der Zeit wird zu der prestigeträchigere Sprache oder Varietät übergegangen.
Die Personen werden besser angesehen, haben bessere Aussichten auf dem Arbeitsmarkt, etc.

%Fix me: bisschen chaotisch, komisch formuliert und ohne wirklich ein Ziel zu haben;
In Situationen der Migration beobachten wir häufig einen Language Shift.
In der 3. Generation Migrantinnen ist es schwierig, die Herkunftssprache weiter zu behalten, wenn keine starken Netzwerke dafür sorgen, dass sie weiter verwendet wird.
Tendenziell ist der Erwerb der Landessprache ein Zeichen für gute Integration in die neue Gesellschaft und ermöglicht einer den Zugang zu mehreren Ressourcen (Bildung, Arbeitsmarkt, Freizeitangeboten, etc.)

\subsection{Sozioindexikalität, indexikalische Felder und Stil}
\label{chap:sozioindexicality}

Menschen sortieren alle Sachen, die sie umgeben, mehr oder weniger bewusst in Kategorien, damit sie kognitiv damit umgehen können.
%macht dieser Satz als Eröffnung Sinn? es wird nur indirekt wieder aufgegriffen. und ich hab dafür erstmal keine quelle; Sofies Welt^^

Primäre Sozioindexikalität bezeichnet die Zugehörigkeit/Zuweisung einer Person zu bestimmten sozialen/Bevölkerungsgruppen (z.B. Frau, spanisch sprachiger Herkunft, Arbeiterin). %unklar welche die Quelle ist. Zentella97??
Diese passiert nach ``objektiven Kriterien'' und wird nicht durch die Sprecherin selbst bestimmt und gesteuert.
Mit anderen Worten: ``A first-order index simply indexes membership in a population.''~\cite{Eckert08}.% \cite[vgl.][]{Eckert08}

%sekundäre Indexikalität
Jedoch ist für Penelope Eckert diese Kategoriezugehörigkeit nicht statisch (was der klassischen soziolinguistischen Interpretation entsprechen würde). %reformulate sentence.; reference zu classical sociolinguistics?
Sprecherinnen sind Akteurinnen, die entsprechend ihrer Ideologie und Ziele ihre Sprache und ihr allgemeines Auftreten aktiv selbst mitgestalten.
Die Interpretation, bzw. die bewusste Wahl und Aneignung (sprachlicher) Elemente in einem bestimmten Kontext und der damit assoziirten Werte und Eigenschaften werden wir als sekundäre Sozioindexikalität bezeichnen.
Eckert erklärt, dass die linguistischen Merkmale zu sekundären Indizien werden: sie zeigen nicht nur die Zugehörigkeit einer Person zu einer bestimmten Gruppe, sondern verwandeln sich in Indizien/Symbole für bestimmte Charaktereigenschaften, Ideologien, etc.~\cite{Eckert08}. % \cite[vgl.][]{Eckert08}.
Die sekundäre Sozioindexikalität ist somit eine ideologische Analyse (sprachlicher) Merkmale: die sozialen Kategorien und die Zugehörigkeit zu denen werden nicht von außen zugeschrieben sondern durch soziale Akteurinnen geschaffen und interpretiert.
Es ist zudem immer möglich ein Index höherer Ordnung zu erschaffen: in dem Moment, in dem ein Index n-ter Ordnung idelogisch reinterpretiert wird, entsteht ein Index der (n+1)-ter Ordnung.
%-> die Indizien sind nicht linear, auch wenn das so klingt

%Dieser Begriff ist eng an den Begriff ``Stil'' gekoppelt und in mehreren Zusammenhänge/für mehrere Zwecke können Beide synonym verwendet werden.

%Index. Feld?
Eckert bezeichnet ein indexikalisches Feld als ``a constellation of meanings that are ideologically linked''~\cite{Eckert08}. % \cite[vgl.][]{Eckert08}.
Der Wissenschaftlerin nach haben linguistische Variablen / Merkmale keine festen Bedeutungen, sondern viel mehr damit assoziierte indexikalische Felder.
Wenn eine Variable/ein Merkmal benutzt wird, wird dann nicht eine statische Bedeutung aktiviert, sondern, je nach Kontext und Ideologie der Gesprächspartnerin, eine von mehreren möglichen Bedeutungen hervorgerufen, bzw. es wird möglicherweise auch eine komplett neue geschaffen~\cite{Eckert08}. % \cite[vgl.][]{Eckert08}.
Die indexikalischen Felder sind ferner ``fluid'': jede neu geschaffene Bedeutung kann potenziell das gesamte Feld verändern, in dem sie die ideologischen Verbindungen innerhalb des Feldes neu definiert. %vlt nach dem nächsten Satz erst?
%Eckert stellt fest, dass die Bedeutung einer und der selben Variable/linguistischen Merkmals nicht uniform durch die Gesamtbevölkerung sein kann, da diese von verschiedenen Menschen in unterschiedlichen Situationen und zu verschiedenen Zwecken benutzt werden kann~\cite{Eckert08}. % \cite[vgl.][]{Eckert08}.
%``Since the same variable will be used to make ideological moves by different people, in different situations, and to different purposes, its meaning in practice will not be uniform across the population.'' \cite[vgl.][S.466]{Eckert08}

%Stil
Der Begriff ``Stil'' ist eng am Begriff Sozioindexikalität gekoppelt und beide können in mehreren Zusammenhängen synonym verwendet werden.
Eckert erklärt, dass sie, entgegen der weit verbreiteten Ansicht, nicht die Meinung vertritt, ``Stil'' wären verschiedene Weisen, die selben Sachen zu sagen.
Diese verschiedenen Weisen gehen aus den verschiedenen Identitäten und Ideologien der Sprecherinnen hervor und bedeuten, dass die Sprecherinnen, potenziell, auch ganz verschiedene Sachen zu sagen haben können~\cite{Eckert08}. % \cite[vgl.][]{Eckert08}.
``Stil'' ist eigentlich viel breiter gefasst: dadrunter werden nicht nur die gewählten linguistischen Merkmale verstanden, sondern diese werden mit weiteren stilistischen Systemen kombiniert (z.B. Kleidung etc.)
Zugrunde der stilistischen Praxen (es sind sowohl die Produktion als auch die Interpretation von Stils gemeint) liegen Ideologien, erläutert Eckert.
Jede stilistische Entscheidung ist das Ergebnis der Interpretation der sozialen Welt und der Bedeutungen deren Elemente~\cite{Eckert08}. %\cite[vgl.][]{Eckert08}.
Die Auswahl bestimmter Merkmale für die eigene stilistische Representation hängt von der bisherigen Erfahrung mit stilistischen Systemen ab.
Je nach Erfahrung nehmen Menschen gewisse Unterschiede stärker wahr als andere.
Wenn ein Merkmal als solches wahrgenommen und mit Bedeutung ausgestattet wird, können sich Sprecherinnen entscheiden, dieses in die eigene stilistische Representation einzubauen oder eben nicht.
Die Entscheidung für oder gegen bestimmte linguistische Features verläuft in der Regel weniger bewusst als die Entscheidung für oder gegen Elemente anderer stylistischen Systemen (z.B Kleidung).
Jedoch verläuft die Bedeutungszuschreibung an Elementen und die Wahl und Kombinieren dieser Elemente ähnlich für alle stilistischen Systeme.

\begin{comment}

[Eckert08]

``Peter Trudgill
(1972), for instance, called upon the perceived toughness of working-class men
as a motive for middle-class men to adopt local working-class sound changes,
accounting for the upward spread of change''

agency and ideology: Why do speakers do what they do?

``Speakers’ agency in the use of variables has
been viewed as limited to making claims about their place in social space by
either emphasizing or downplaying their category membership through the
quantitative manipulation of markers.''
--> The old view: speakers' agency is ignored

``This generalization says nothing about the kinds of behaviors and ideologies that
underlie these patterns, what kinds of meaning people attach to the conservative
and innovative variant, who does and does not fit the pattern and why.''

``The
pejoration of many English words referring to females is a perfect example of the
systematic absorption of ideology into the lexicon.''

%%%%%%%%%%%%%%%%%%%%%%%%%%%%%%%%%%%%%%%%%%%%%%%
% spannend, aber für jetzt für irrelevant befunden
[Milroy93]

--> introduces the network concept; the strength of weak ties; gender variable
the strength of weak ties? (ein bisschen abseits)

Untersucht: impact of speakers' variables social class, social network and gender on linguistic variability and change
``discuss the way in which these extralinguistic variables are interrelated''

``gender difference is often prior to social class in accounting for sociolinguistic variation''

selten jemand hinterfragt welche Variablen werden bei Untersuchungen gewählt und warum
bis dato (1993): class is primary variable <- critism

fokus auf gender der sprecher\_innen;
kann ich nicht wirklich benutzen, weil ich nicht weiß, wer das geschrieben hat; aber die Zielgruppe ist definitiv gegendert..

``what are usually called low-prestige varieties can be maintained over generations as flourishing vernaculars.''
also kann Class und prestige nicht die einzige Erklärung für Language Shift und Loss sein
%spannender Gedanke, ich weiß jedoch nicht wo wie und in welcher Form ich das benutzen könnte

``strong informal social ties within
communities provide the mechanisms that enable speakers to maintain non-
standard dialects, rural or urban, despite intense pressure from the standard
language through routes such as the educational system and the media.''
kann wohl eher schwierig als Argumentation für mich benutzt werden

%vlt eher nicht; ich kann schwer sagen, was für sozialen Netzwerken die Leser_innen dieses Magazins angehören
``Close-
knit networks, which will of course vary in actual levels of density and multi-
plexity, are assumed to have the capacity to maintain and even enforce local
conventions and norms - including linguistic norms. It is, after all, remarkable
that stigmatized linguistic forms and low-status vernaculars can persist over
centuries in the face of powerful national policies for diffusing and imposing
standard languages,''
spannend, aber vermutlich nicht sehr nützlich in meinem Kontext

``close-knit network structure is associated with
language maintenance; the corollary to this is that a loose-knit network struc-
ture is associated with language change. We have argued in detail elsewhere
that where ties are relatively loose-knit, communities will be susceptible to
change originating from outside localized networks. These changes are not
necessarily in the direction of the standard''

``loose-knit structures with
socially and geographically mobile middle class speakers, and close-knit ties
with lower and upper class speakers.''
-->vlt spannend

``innovators are likely to be persons weakly linked to
such a group. Susceptibility to outside influence is likely to increase in inverse
proportion to the strength of the tie with the group''
vlt nochmal ``The strength of weak ties'' lesen.. wobei bin mir nicht sicher was es
bei der Arbeit helfen würde.. man kann eh nicht zuverlässig behaupten ob es da ein 
Netzwerk gibt und wie stark genau die Connections sind

``A ‘weak tie’ model of change thus seems able to account for some
instances of variation and change which are difficult to explain in terms of the
usual unqualified assumption that linguistic change is encouraged by frequency
of contact and relatively open channels of communication''

``The form which linguistic gender-marking has commonly been interpreted as
taking is for women to approximate more commonly than men of similar s t a t u s
to the (so-called) prestige norm. But as Coates (1986) has pointed out, no satis-
factory explanation has emerged of why women should be more oriented than
men to a prestige norm.''
naja, mir fällt eine ziemlich lange Erklärung dazu ein.
aber die hier argumentieren gleich, dass es anders rum funktioniert...

``and the relation of gender to social class and to prestige patterns is by no means consistent
or predictable from the usual assumptions about females preferring higher social-class norms.''

``The generalization suggested is not that females
favour prestige forms, but that they create them; i.e., if females favour certain
forms, they become prestige forms. In these developments, both class and
gender are implicated, but gender is prior to class.''
eine Erklärung dafür wäre, dass Frauen den Hauptteil der Care-Arbeit leisten,
also sind sie diejenigen, die auch die Sprache weiter geben

``Some
linguists have apparently assumed that ‘social network’ is concerned only with
strong ties and is roughly synonymous with ‘peer-group’,''

``some linguists have seemed to believe that
individuals may o r may not possess a ‘social network’, when i n fact all individ-
uals ar e embedded in networks.''

%\cite[vgl.][s.20]{Tomasello06}
\end{comment}
